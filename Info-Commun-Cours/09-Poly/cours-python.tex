\documentclass[book]{magnolia}

\magtex{tex_driver={pdftex},
        tex_packages={float,caption,titling,epigraph,minitoc,slashbox,tabularx,cancel,pgfplots,xypic,nicefrac},
        tex_pstricks={pstricks,pst-plot}}
\magfiche{document_nom={Cours de Sup},
          auteur_nom={François Fayard},
          auteur_mail={fayard.prof@gmail.com}}
\magcours{cours_matiere={maths},
          cours_niveau={mpsi},
          cours_chapitre_numero={11},
          cours_chapitre={Dérivation}}
\magmisenpage{misenpage_presentation={tikzvelvia},
              misenpage_format={a4},
              misenpage_nbcolonnes={1},
              misenpage_preuve={non},
              misenpage_sol={non}}
\maglieudiff{lieu_lycee={Aux Lazaristes},
             lieu_classe={MPSI 1},
             lieu_annee={2020--2021}}
\magprocess

\usepackage[scale=3]{ccicons}
\title{{\Huge\bf Cours d'Informatique Commune}\\\vspace{1cm}
%       \textbf{\Huge 2021--2022}\\
       \textbf{\Huge Maths Sup}\\\vspace{1cm}
       \textsc{F. Fayard}\\\vspace{1cm}
       \includegraphics[width=8cm]{../../Commun/Images/lazos-bde-2024.png}\\%\vspace{1cm}
       \includegraphics[width=8cm]{../../Commun/Images/lazos.png}}
% \author{{\sc François Fayard}, {\sc Victor Lambert}}
% \date{3 Août 2020 -- 1a20aa}

% \pretitle{%
%   \begin{center}
%   \includegraphics[width=12cm]{/Users/fayard/Desktop/lazos.png}\\[\bigskipamount]
% }
% \posttitle{\end{center}}

\mtcsettitle{minitoc}{}
\mtcsettitle{secttoc}{Table des matières}
\mtcsettitle{parttoc}{Table des matières}

\usetikzlibrary{positioning}
% \lstset{%
%   frame            = tb,    % draw frame at top and bottom of code block
%   tabsize          = 1,     % tab space width
%   numbers          = none,  % display line numbers on the left
%   framesep         = 3pt,   % expand outward
%   framerule        = 0.4pt, % expand outward 
%   commentstyle     = \color{green},      % comment color
%   keywordstyle     = \color{blue},       % keyword color
%   stringstyle      = \color{blue},    % string color
%   backgroundcolor  = \color{colorLazoBlue1Light}, % backgroundcolor color showstringspaces = false,              % do not mark spaces in strings
%   basicstyle       = \ttfamily,
%   breaklines       = true
% }

\input{vc.tex}

\begin{document}

\maketitle

La version de ce document est la \textsc{\GITAbrHash}.\\

% Je remercie Jean-Baptiste Bianquis du lycée du Parc qui a écrit
% certains chapitres sur lesquels je me suis basé, Jean-Pierre Becirspahic à qui j'ai emprunté de nombreux TPs
% qu'il a écrit lorsqu'il était professeur à Louis-Le-Grand, Émeric Tourniaire du lycée Henri IV à qui j'ai
% aussi emprunté un TP et Victor Lambert des Lazaristes qui a partagé le TP sur le cryptage de César. Le chapitre
% sur les graphes est en grande partie emprunté au livre de NSI de terminale de Jean-Christophe Filliâtre.\\

Merci à tous les élèves des Lazaristes pour leurs remarques et corrections. Nous remercions particulièrement
\nom{Arthur Buis}, \nom{Raphaël Des Boscs}, \nom{Sacha Evrard}, \nom{Titouan Francheteau}, \nom{Hélène Ghaleb}, \nom{Cyprien Mas}, \nom{Alessandro Morales}, \nom{Antoine Nippert}, \nom{Martin Palandre}, \nom{Léa Sulpice} et bien entendu \nom{Carole Vacherand} pour leurs
nombreuses corrections apportées.\\

\vfill


\begin{center}
  \ccbysa\\
  \vspace{2ex}
  This work is licensed under a Creative Commons\\
  Attribution-ShareAlike 4.0 International License.\\
  \url{https://creativecommons.org/licenses/by-sa/4.0/legalcode.fr}\\
  \vspace{2ex}
  La dernière version de ce document ainsi que\\
  les sources \LaTeX{} sont disponibles à l'adresse\\
  \url{https://github.com/FayardProf/Maths-MPSI-MP2I}
  \end{center}
  \vspace{2ex}
  \begin{center}
  \textbf{Vous êtes autorisés à~:}
  \end{center}
  \vspace{2ex}
  \begin{itemize}
  \item \textbf{Partager}~: copier, distribuer et communiquer le matériel par tous les moyens et sous tous formats.
  \item \textbf{Adapter}~: remixer, transformer et créer à partir du matériel
  pour toute utilisation, y compris commerciale.
  \end{itemize}
  \vspace{2ex}
  \begin{center}
  \textbf{Selon les conditions suivantes~:}
  \end{center}
  \vspace{2ex}
  \begin{itemize}
    \item \textbf{Attribution}~: Vous devez créditer l'œuvre, intégrer un lien vers la licence et indiquer si des modifications ont été effectuées à l'œuvre. Vous devez indiquer ces informations par tous les moyens raisonnables, sans toutefois suggérer que l'offrant vous soutient ou soutient la façon dont vous avez utilisé son œuvre.
    \item \textbf{Partage dans les mêmes conditions}~: Dans le cas où vous effectuez un remix, que vous transformez, ou créez à partir du matériel composant l'œuvre originale, vous devez diffuser l'œuvre modifiée dans les même conditions, c'est à dire avec la même licence avec laquelle l'œuvre originale a été diffusée.
    \item \textbf{Pas de restrictions complémentaires}~: Vous n'êtes pas autorisé à appliquer des conditions légales ou des mesures techniques qui restreindraient légalement autrui à utiliser l'œuvre dans les conditions décrites par la licence.
    \end{itemize}

% This work is licensed under a Creative Commons license.
% \begin{center}
% \includegraphics[width=0.2\textwidth]{../../Commun/Images/cc-by-sa.png}
% \end{center}

\tableofcontents

\part{Cours}

\chapter{Valeur, type, variable}
\setcounter{numeroexercicecours}{1}
\input{cours-valeur_type_variable}
\section{Exercices}
\setcounter{numeroexercice}{1}
\documentclass{magnolia}

\magtex{tex_driver={pdftex}}
\magfiche{document_nom={Valeur, type, variable},
					auteur_nom={François Fayard},
					auteur_mail={francois.fayard@auxlazaristeslasalle.fr}}
\magexos{exos_matiere={maths},
         exos_niveau={mpsi},
         exos_chapitre_numero={1},
         exos_theme={Valeur, type, variable}}
\magmisenpage{}
\maglieudiff{}
\magprocess

\begin{document}

%BEGIN_BOOK
\magsection{Valeur, type}

\magsubsection{Nombre entier}

\exercice{nom={Évaluer une expression}}
Déterminer la valeur et le type de chacune des expressions suivantes~:
\begin{pythoncode}
In [1]: 5 * 2 + 1 ** 2
In [2]: 5 * (2 + 1) ** 2
In [3]: -16 // 5
In [4]: 8 / 2
\end{pythoncode}

\exercice{nom={Les oeufs}}
On suppose que la variable \verb_n_ contient le nombre d'oeufs dont on dispose et on souhaite
calculer le nombre \verb_b_ de boites de 6 oeufs nécessaires à leur transport.
\begin{questions}
\question Pour quelles valeurs de $n$ l'expression \verb_n // 6_ donne-t-elle la bonne
  réponse~?
\question Trouver une expression donnant la bonne réponse.
\end{questions}
\begin{sol}
\begin{questions}
\question c'est la bonne réponse si $n$ est multiple de 6.
\question \verb_(n + 5) // 6_.
\end{questions}
\end{sol}

\magsubsection{Nombre flottant}

% \exercice{nom={Entiers et flottants}}
% Évaluer les expressions suivantes en déterminant celles qui renvoient des résultats de
% type \verb!int!.
% \begin{pythoncode}
% In [1]: 4 + 2
% In [2]: 1 - 6 * 7
% In [3]: 18 // 7
% In [4]: 42 / 6
% In [5]: 22 / (16 - 2 * 8)
% \end{pythoncode}

\exercice{nom={Évaluer une expression}}
Déterminer la valeur et le type de chacune des expressions suivantes, d'abord sans utiliser
Python, puis en l'utilisant.
\begin{pythoncode}
In [1]: 2 ** 3.0 + 4
In [2]: int(8.6) + 2
In [3]: float(2) ** 3
\end{pythoncode}

% \exercice{nom={Conversion}}
% \begin{questions}
% \question Les expressions suivantes renvoient-elles la même valeur~?
% \begin{pythoncode}
% In [1]: 8.5 / 2.1
% In [2]: int(8.5) / int(2.1)
% In [3]: int(8.5 / 2.1)
% \end{pythoncode}
% \question Que dire des expressions suivantes~?
% \begin{pythoncode}
% In [1]: float(8 * 2)
% In [2]: 8 * 2
% In [3]: 8. * 2.
% \end{pythoncode}
% \end{questions}

% \exercice{nom={Fonctions usuelles}}
% Comment calculer les nombres suivants~?
% \[\e^2\qsep \sqrt{13}\qsep \cos\frac{\pi}{5}\qsep \e^{\sqrt{5}},\]
% \[\ln 2\qsep \ln 10\qsep \log_{2} 10\qsep \tan\frac{\pi}{2}.\]

\magsubsection{Chaine de caractères}

\magsubsection{Booléen}

% \exercice{nom={Évaluer une expression}}
% Déterminer la valeur et le type de chacune des expressions suivantes, d'abord sans utiliser
% Python, puis en l'utilisant.
% \begin{pythoncode}
% In [1]: 2 == 1 + 1
% In [2]: 2 == 1 + 1 + 1
% In [3]: (2 == 1 + 1 + 1) and (2 == 1 + 1)
% In [4]: 2 == 1 + 1 + 1 or 2 == 1 + 1
% In [5]: 1 == 0 // 0
% In [6]: (not (0 == 0)) and (1 == 0 // 0)
% In [7]: 2 >= 3 or 2 ** 4 * 5 ** 2 // 20 == 20
% In [8]: True or 4 > 3 and 3 > 4
% In [9]: not False and False
% \end{pythoncode}

\exercice{nom={Année bissextile}}
Une année est bissextile dans les deux cas suivants.
\begin{itemize}
\item Si l'année est divisible par 4 et non divisible par 100.
\item Si l'année est divisible par 400.
\end{itemize}
On suppose que la variable $n$ contient l'année qui nous intéresse. Donner une expression
Python qui s'évalue en \verb_True_ si l'année est bissextile et en \verb_False_
sinon. 

\magsubsection{Tuple}

\magsection{Programmation impérative}

\magsubsection{Variable}

\magsubsection{État du système}

% \exercice{nom={Changement d'état}}
% \begin{questions}
% \question Quelle est la valeur affichée par l'interprète après la séquence d'instructions
%   suivante~?
% \begin{pythoncode}
% a = 3
% a = 4
% a = a + 2
% a
% \end{pythoncode}
% \question Quelle est la valeur affichée par l'interprète après la séquence d'instructions
%   suivante~?
% \begin{pythoncode}
% a = 2
% b = a * a
% b = a * b
% b = b * b
% b
% \end{pythoncode}
% \end{questions}

\exercice{nom={Suites d'affectations}}
% \begin{questions}
% \question Quelles sont les valeurs de \verb!a! et \verb!b! après les instructions suivantes~?
% \begin{pythoncode}
% a = 7
% b = a
% a = 9
% \end{pythoncode}
% \question Quelles sont les valeurs de \verb!x!, \verb!y!, \verb!z! après les instructions suivantes~?
% \begin{pythoncode}
% x = 23
% y = 18
% z = x 
% x = y 
% y = z 
% \end{pythoncode}
% \question
Quelles sont les valeurs de \verb!x!, \verb!y! après les instructions suivantes~?
\begin{pythoncode}
In [1]: x = 23
In [2]: y = 18
In [3]: x = x + y 
In [4]: y = x - y 
In [5]: x = x - y 
\end{pythoncode}
% \end{questions}

\exercice{nom={Quésako}}
On suppose que les variables \verb_a_ et \verb_b_ contiennent initialement les valeurs
$a_0$ et $b_0$. Quelles sont les valeurs contenues par \verb_a_ et \verb_b_ après les
instructions suivantes~?
\begin{pythoncode}
In [1]: a = a + b
In [2]: b = a - b
In [3]: a = a - b
\end{pythoncode}

% \exercice{nom={Échange}}
% S'il n'était pas possible d'effectuer plusieurs affectations simultanément, comment
% procéderiez-vous pour échanger le contenu de deux variables \verb_a_ et \verb_b_.

\magsubsection{Entrée, sortie}

\exercice{nom={Entrée, sortie}}
Donner l'état du shell après chacune des commandes suivantes.
\begin{pythoncode}
In [1]: 1 + 2
In [2]: print(1 + 2)
In [3]: print(print(1 + 2))
In [4]: print(1) + 2
In [5]: print(1) + print(2)
\end{pythoncode}


% \magsubsection{Logo}

% \exercice{nom={Noeud papillon}}

% Tracez en \textsc{logo} le noeud papillon suivant. La hauteur du noeud papillon est de
% 100 et sa largeur est de 200. On utilisera la fonction \verb_atan(x)_ du module
% \verb_math_ qui calcule la mesure $\theta\in\intero{-\pi/2}{\pi/2}$ en radian de
% l'unique angle dont la tangente vaut $x$.
% \begin{center}
% \includegraphics[width=0.3\textwidth]{../../Commun/Images/python-exos-val-1.pdf}
% \end{center}

%END_BOOK

\end{document}


\chapter{Flot d'exécution}
\setcounter{numeroexercicecours}{1}
\input{cours-flot_execution}
\section{Exercices}
\setcounter{numeroexercice}{1}
\documentclass{magnolia}

\magtex{tex_driver={pdftex}}
\magfiche{document_nom={Fonctions, flot d'execution},
          auteur_nom={François Fayard},
          auteur_mail={francois.fayard@auxlazaristeslasalle.fr}}
\magexos{exos_matiere={maths},
         exos_niveau={mpsi},
         exos_chapitre_numero={2},
         exos_theme={Flot d'exécution}}
\magmisenpage{misenpage_presentation={tikzvelvia},
         misenpage_format={a4},
         misenpage_nbcolonnes={1},
         misenpage_preuve={non},
         misenpage_sol={non}}
\maglieudiff{}
\magprocess

\begin{document}

%BEGIN_BOOK

\magsection{Programmation procédurale}
\magsubsection{Fonction}

\exercice{nom={Convertir l'heure}}
Écrivez une fonction \verb!conversion_heure(n)! prenant en entrée un entier donnant le
nombre de secondes qui s'est écoulé depuis minuit et renvoyant un \verb!tuple! donnant
l'heure au format heure, minute, seconde. Par exemple \verb!conversion_heure(4567)! devra
renvoyer le tuple \verb_(1, 16, 7)_.

\magsubsection{Liste}
\magsubsection{Ordre d'évaluation}
\magsection{Programmation structurée}
\magsubsection{Branchement}

\exercice{nom={Trier 3 éléments}}
Écrire une fonction \verb!tri(a, b, c)! qui prend en argument trois nombres réels $a$, $b$ et
$c$ et qui renvoie le triplet formé de cet 3 éléments, triés par ordre croissant.

\exercice{nom={Chevauchement}}
Écrire une fonction \verb_chevauche(a, b, c, d)_ prenant en entrée quatre entiers $a$, $b$, $c$ et $d$
et renvoyant \verb_True_ si les segments d'extremités $a$ et $b$ d'une part et $c$ et $d$ d'autre part
ont une intersection non vide.


% \exercice{nom={Moyenne}}
% Écrire un programme initialisant une variable \texttt{note} qui correspond à une moyenne au baccalauréat et renvoyant ensuite la mention s'il y a lieu, le succès ou l'échec à l'examen.

\magsubsection{Boucle for}





% \exercice{nom={Énumérations}}
% \begin{questions}
% \question Écrire une fonction \verb!entiers(i, j)! prenant en entrée deux entiers $i$ et
%   $j$ tels que $i\leq j$ et qui affiche sur une même ligne les entiers de l'intervalle
% 	$\intere{i}{j}$ séparés par le caractère \verb!-!.
% \question Modifier la fonction précédente pour qu'elle affiche désormais les entiers de
%   l'intervalle $\intere{i}{j}$ qui ne sont pas des multiples de 7.
% \end{questions}

\exercice{nom={Graphisme en console}}
\begin{questions}
\question Définir une fonction \verb!triangle1(n)! qui prend en argument un entier $n$
  et qui dessine dans le shell un triangle sur $n$ lignes
\begin{pythoncode}
In [1]: triangle1(5)
(*@\textcolor{purple}{*}@*)
(*@\textcolor{purple}{**}@*)
(*@\textcolor{purple}{***}@*)
(*@\textcolor{purple}{****}@*)
(*@\textcolor{purple}{*****}@*)
\end{pythoncode}
\question Définir une fonction \verb!triangle2(n)! qui dessine ce même triangle mais dans
  l'autre sens.
\begin{pythoncode}
In [2]: triangle2(5)
(*@\textcolor{purple}{*****}@*)
(*@\textcolor{purple}{****}@*)
(*@\textcolor{purple}{***}@*)
(*@\textcolor{purple}{**}@*)
(*@\textcolor{purple}{*}@*)
\end{pythoncode}
\question Définir une fonction \verb!pyramide1(n)! qui dessine une pyramide sur $2n-1$ lignes.
\begin{pythoncode}
In [3]: pyramide1(5)
(*@\textcolor{purple}{*}@*)
(*@\textcolor{purple}{**}@*)
(*@\textcolor{purple}{***}@*)
(*@\textcolor{purple}{****}@*)
(*@\textcolor{purple}{*****}@*)
(*@\textcolor{purple}{****}@*)
(*@\textcolor{purple}{***}@*)
(*@\textcolor{purple}{**}@*)
(*@\textcolor{purple}{*}@*)
\end{pythoncode}
\question Définir une fonction \verb!pyramide2(n)! qui dessine une pyramide de la manière
  suivante.
\begin{pythoncode}
In [4]: pyramide2(5)
    *
   * *
  * * *
 * * * *
* * * * *
\end{pythoncode}
\end{questions}

% \exercice{nom={Le \textsc{Talkhys}}}
% Le \textsc{Talkhys} est un traité d'arithmétique d'\textsc{Ibn Albanna}, mathématicien
% marocain de la première moitié du $13^{\text{e}}$ siècle. On y trouve un certain nombre
% d'identités remarquables.
% \begin{questions}
% \question Écrire un programme reproduisant cette table dans la console.
% \begin{pythoncode}
%         1 x 1         =         1
%        11 x 11        =        121
%       111 x 111       =       12321
%      1111 x 1111      =      1234321
%     11111 x 11111     =     123454321
%    111111 x 111111    =    12345654321
%   1111111 x 1111111   =   1234567654321
%  11111111 x 11111111  =  123456787654321
% 111111111 x 111111111 = 12345678987654321
% \end{pythoncode}
% \question Faire de même avec la table suivante.
% \begin{pythoncode}
% 8 x 1         + 1 = 9
% 8 x 12        + 2 = 98
% 8 x 123       + 3 = 987
% 8 x 1234      + 4 = 9876
% 8 x 12345     + 5 = 98765
% 8 x 123456    + 6 = 987654
% 8 x 1234567   + 7 = 9876543
% 8 x 12345678  + 8 = 98765432
% 8 x 123456789 + 9 = 987654321
% \end{pythoncode}
% \end{questions}

% \exercice{nom={Palindrome}}
% Un palindrome est un mot ou une phrase qui peut se lire indifféremment de la gauche vers
% la droite tels \verb!kayak! ou \verb!eluparcettecrapule!. Écrire une fonction
% \verb!palindrome(s)! qui prend en entrée une chaine de caractères et qui
% renvoie \verb!True! si ce mot est un palindromme et qui renvoie \verb!False! sinon.

\exercice{nom={Le Rot13}}
Le \textsc{Rot13} (rotate by 13 places) est un cas particulier du chiffrage de \textsc{César}.
Comme son nom l'indique, il s'agit d'un décalage de 13 caractères de chaque lettre du texte à
chiffrer~: a devient n, b devient o, c devient p, etc. Son principal aspect pratique est que
le codage et le décodage se réalisent exactement de la même manière puisque notre
alphabet comporte 26 lettres. \textsc{Rot13} est parfois utilisé dans les forums en ligne
comme un moyen de masquer la réponse à une énigme, un spoiler, ou encore une expression
grossière.
\begin{questions}
\question Écrire une fonction \verb!decale(c)! prenant en entrée un caractère et renvoyant
  ce même caractère codé en \textsc{Rot13} si $c$ est un caractère entre \verb!a! et \verb!z!
  et qui renvoie $c$ sinon. On pourra utiliser les fonctions \verb!ord! et \verb!chr!.
\question Définir une fonctin \verb!rot13(s)! prenant en entrée une chaine de caractères et
  renvoyant cette même chaine de caractères codée en \textsc{Rot13}.
\question Utilisez cette fonction pour connaitre la réponse à l'énigme suivante~: Quelle
  est la différence entre un informaticien et une personne normale~?
  \begin{center}
\verb!har crefbaar abeznyr crafr dh'ha xvyb-bpgrg rfg étny à 1000 bpgrgf, ha vasbezngvpvra!
\verb!rfg pbainvaph dh'ha xvybzèger rfg étny à 1024 zègerf!.
  \end{center}
\end{questions}

\exercice{nom={Plus grand plateau}}
On considère une liste \verb!a! dont les éléments sont égaux aux entiers 0 ou 1. Rédiger une fonction
\verb!pg_plateau(a)! calculant le nombre maximal de 0 consécutifs présents dans cette liste. Par exemple,
pour la liste suivante, la fonction devra renvoyer la valeur 7.
\begin{center}
\includegraphics[width=0.8\textwidth]{../../Commun/Images/python-exo-plateau}\\
\end{center}

\begin{sol}
La fonction suivante répond à la question
\begin{pythoncodeline}
def pg_plateau(a):
    width = 0
    max_width = 0
    for k in range(len(a)):
        if a[k] == 0:
            width = width + 1
            if width > max_width:
                max_width = width
        else:
            width = 0
    return max_width
\end{pythoncodeline}
\end{sol}

\magsubsection{Réduction}

\exercice{nom={Somme}}
Écrire une fonction calculant la somme de tous les entiers inférieurs ou égaux à $n$ inclus
qui sont multiples de $3$ ou de $5$.

\exercice{nom={Suite}}
Écrire une fonction permettant de calculer le $n$-ième terme de la suite définie par
\[\forall n\in\N \qsep u_n\defeq\sum_{k=0}^n \frac{1}{k!}.\]
% On souhaite avoir un affichage de la forme :
% \begin{pythoncode}
% u_10 = ...
% u_11 = ...
% u_12 = ...
% ...
% \end{pythoncode}
% \begin{sol}
% \begin{pythoncode}
% >>> n = 21
%     u = 0
%     for k in range(n):
%         u = u + 1 / math.factorial(k)
%         if k>=10:
%             print("u_" + str(k) + " =", u)
% \end{pythoncode}
% \end{sol}

\exercice{nom={Moyenne, variance}}
 Dans cet exercice, on souhaite calculer la moyenne et la variance d'une liste
  de nombres flottants.
  \begin{questions}
  \question Écrire une fonction \verb!moyenne(t)! qui renvoie la
    moyenne de la liste \verb!t!.
  \question La variance d'une famille finie $t\defeq (t_0,\dots,t_{n-1})$ est donnée par
      \[\mathbb{V}(t)\defeq\frac{1}{n}\sum_{k=0}^{n-1} \p{t_k-\overline{t}}^2\] 
  où $\overline{t}$ est la moyenne de $t$.
  \begin{questions}
  \question Écrire une fonction \verb!variance1(t)! calculant
    la variance de la liste \verb!t!.
  \enonce Cette formule nécessite deux parcours de la liste : un pour calculer $\overline{t}$, et l'autre pour calculer $\mathbb{V}(t)$. Pour calculer $\mathbb{V}(t)$, on peut aussi utiliser la formule de Koenig-Huygens
  \[\mathbb{V}(t)=\cro{\frac{1}{n}\sum_{k=0}^{n-1} t_k^2} -\overline{t}^2.\]
  \question Prouver cette formule.
  \question S'en servir pour écrire une fonction \verb!variance2(t)! qui n'effectue qu'un seul parcours de la liste.
  \end{questions}
  \end{questions}
  \begin{sol}
  \begin{pythoncode}
  def somme(L):
      s=0
      for x in L:
          s=s+x
      return s
  \end{pythoncode}
  \begin{pythoncode}
  def moy(L):
      s=0
      for x in L:
          s=s+x
      return s/len(L)
  \end{pythoncode}
  \begin{pythoncode}
  def var(L):
      n=len(L)
      s,c = 0,0
      for x in L:
          s = s + x
          c = c + x*x
      return c/n - (s/n)*(s/n)
  \end{pythoncode}
  \end{sol}

\exercice{nom={Palindrome}}
Un \emph{palindrome} est un mot pouvant se lire dans les deux sens comme~: radar, rotor, kayak.
Écrire une fonction \verb_palindrome(c)_ qui prend en entrée une chaine de caractères \verb_c_ et qui renvoie le booléen \verb_True_ si cette chaine est un palindrome et \verb_False_ sinon. On rappelle que si \verb_s_ est une chaine de
caractères, on accède au caractère d'indice $k$ grâce à \verb_s[k]_.

\exercice{nom={Monotonie}}
\begin{questions}
\question Écrire une fonction \verb!est_croissante(a)! prenant en entrée une liste d'entiers \verb!a! et renvoyant
  \verb!True! si cette liste est triée dans l'ordre croissant et \verb_False_ sinon.
\question Écrire une fonction \verb!est_monotone(a)! prenant en entrée une liste d'entiers \verb!a! et renvoyant
  \verb!True! si cette liste est triée dans l'ordre croissant ou décroissant et \verb_False_ sinon.
\question Écrivez une fonction répondant à la question précédente mais ne parcourant qu'une seule fois la liste.
\end{questions}



\magsubsection{Boucle while}

% \exercice{nom={Racine carrée entière}}
% La racine carrée entière d'un entier $n\in\N$ est l'unique entier $p$ vérifiant $p^2\leq n<(p+1)^2$.
% \begin{questions}
% \question Rédiger une fonction \verb_isqrt(n)_ qui calcule la racine carrée entière de $n$.
% \question Écrire une deuxième version de cette fonction en ne s'autorisant cette fois que des additions.
% \end{questions}

\exercice{nom={Constante d'Euler}}
On note pour tout $n \in \Ns$
\[S_n\defeq \sum_{k=1}^{n} \frac{1}{k},\qquad
	u_n\defeq S_n-\ln(n) \et v_n\defeq u_n-\frac{1}{n}.\] 
On admet que $(u_n)$ et $(v_n)$ tendent vers la même limite $\gamma$ appelée constante
d'Euler et que l'on a
\[\forall n \in\Ns \qsep v_n \leq \gamma \leq u_n.\]
Écrire une fonction qui calcule un encadrement de $\gamma$ de largeur inférieure à $\epsilon$.
Notre fonction renverra un tuple des deux réels encadrant $\gamma$.



\exercice{nom={Nombre univers}}
On appelle \emph{nombre univers} (en base 10) un nombre réel dont la partie décimale contient
n'importe quelle succession de chiffres de longueur finie. Un exemple simple de nombre
univers en base 10 est la constante de \textsc{Champernowme} $0.123456789101112131415161718192021\ldots$ On pense que $\pi$ est un nombre univers mais
personne n'a pour le moment réussi à le démontrer. De même, on appelle \emph{suite univers} (en base 10) une suite de nombres entiers elle que n'importe quelle succession de chiffres
de longueur finie se trouve dans l'un des termes de cette suite. Il a été prouvé que la
suite des puissances de 2 est une suite univers.
\begin{questions}
\question Écrire une fonction \verb!univers(s)! prenant en entrée une chaine de caractères
  ne comportant que des chiffres et renvoyant la plus petite valeur de $n$ pour laquelle
	\verb!s! est présent dans l'écriture décimale de $2^n$.
\question Déterminer la plus petite puissance de 2 contenant votre date de naissance
  au format \textsc{jjmmaaaa}.
\end{questions}

\exercice{nom={Nombres premiers}}
Le but de cet exercice est de déterminer les 1000 plus petits nombres premiers. Pour
déterminer si un nombre est premier, on utilise le critère suivant~: un entier $p$ est
premier si et seulement si $p\geq 2$ et lorsqu'il n'est divisible par aucun entier
$k$ tel que $2\leq k\leq p$.
\begin{questions}
\question Écrire une fonction \verb!premier(p)! prenant en paramètre un entier $p$ et qui
  renvoie le booléen \verb!True! lorsque $p$ est premier et le booléen \verb!False!
	dans le cas contraire.
\question En remarquant qu'il suffit de montrer que $p$ n'admet aucun diviseur $k$
  tel que $2\leq k$ et $k^2\leq p$ pour montrer que $p$ est premier, écrire une
  nouvelle fonction \verb!premier_bis(p)! plus efficace.
\question Utiliser cette fonction pour afficher les mille plus petits nombres premiers.
\question La conjecture de Goldbach postule que tout entier pair supérieur à 3
  peut s'écrire comme somme de deux nombres premiers (éventuellement égaux). Vérifier
	cette conjecture pour tout entier inférieur où égal à 1000.
\question À contrario, montrer que la conjecture suivante est fausse~: tout nombre impair
  est la somme d'une puissance de 2 et d'un nombre premier.
\end{questions}

\exercice{nom={Suite de Conway}}
Les premiers termes de la suite de Conway sont $1, 11, 21, 1211, 111221,\ldots$
chaque terme étant obtenu en lisant à haute voix le terme précédent. C'est pourquoi
Conway avait baptisé cette suite \emph{look and say}. Par exemple, le terme
1211 se lit \og un 1, un 2, deux 1\fg donc le terme suivant est 111221.
\begin{questions}
\question Écrire une fonction \verb!lookandsay(s)! prenant en paramètre une chaine de
  caractères représentant un entier et renvoyant la chaine de caractère représentant
	l'entier suivant dans la suite de Conway.
\question À l'aide de cette fonction, afficher les 20 premiers termes de la suite de
  Conway.
\question Il a été démontré que si on note $u_n$ le nombre de chiffres du $n$-ième
  nombre de Conway, le rapport $u_{n+1}/u_n$ admet une limite finie $l$.
	Donnez une valeur approchée de $l$.
\question Une autre propriété de cette limite est qu'elle ne dépend pas de la
  valeur initiale (excepté 22). Le vérifier expérimentalement.
\question Démontrer que dans a suite de Conway, ne peuvent apparaitre que les chiffres 1, 2 et 3.
\end{questions}

\magsubsection{Boucles imbriquées}



\exercice{nom={Doublon}}
Écrire une fonction \verb!doublon(a)! prenant en entrée une liste \verb!a! et renvoyant \verb_True_
si \verb_a_ possède un doublon et \verb_False_ sinon. Par exemple, \verb_doublon([3, 4, 7, 3, 2])_
devra renvoyer \verb_True_ car 3 est présent deux fois dans la liste.

\exercice{nom={Somme}}
Écrire une fonction \verb!somme(a, s)! prenant en valeur une liste d'entiers \verb!a! et un entier \verb!s!
et qui renvoie \verb_True_ si $s$ est la somme de deux entiers de la liste $a$ et \verb!False! sinon.
Par exemple \verb!somme([1, 7, 2, 4], 11)! devra répondre \verb!True! car $7+4=11$ et
\verb!somme([1, 7, 2, 4], 14)! devra répondre \verb!False!.

% \exercice{nom={Correction, terminaison}}
% Dans l'exercice qui suit, on pourra pour chacun des programmes écrits justifier sa correction et sa terminaison à l'aide de variants et d'invariants de boucles.
% \begin{questions}
% \question
% Soit la suite $(u_n)$ définie par : $u_0=0$ et, pour tout $n\in\N,u_{n+1}=\sqrt{3+u_n}$.

% \'Ecrire un programme permettant de calculer $u_{100}$.
% \begin{sol}
% \begin{pythoncode}
% import math as m
% u = 0
% for i in range(100):
%     # u contient u_i (vrai pour i = 0)
%     u = m.sqrt(3 + u)
%     # u contient u_{i+1}
% #\`A la fin du dernier tour de boucle, i = 99 donc u contient u_{100}
% print(u)
% \end{pythoncode}
% Boucle for... pas besoin de variants...
% \end{sol}
% \question Soit la suite $(u_n)$ dite de Fibonacci, définie par:
% $\begin{cases}
% u_0=0,u_1=1\\u_{n+2}=u_{n+1}+u_n,\text{ pour }n\in\N
% \end{cases}
% $\\
% \'Ecrire un programme permettant de calculer $u_{100}$.

% \begin{sol}
% \begin{pythoncode}
% u,v=0,1
% for i in range(100):
%     #  u contient u_{i} et v contient u_{i+1} (vrai pour i = 0)
%     u, v = v, u + v
%     #  u contient u_{i+1} et v contient u_{i+2}
% #\`A la fin du dernier tour, i = 99 donc u contient u_{100}
% print(u)
% \end{pythoncode}

% Boucle for... pas besoin de variants...
% \end{sol}
% \question \'Ecrire un programme qui détermine le plus petit entier $n$ tel que $1+2+\dots+n$ dépasse strictement $1000$.
% \begin{sol}
% \begin{pythoncode}
% s = 0
% n = 0
% while s <= 1000:
%     #  s contient 0+...+n (vrai pour n = 0)}
%     n = n + 1
%     #  donc s contient 0+...+(n-1)
%     s = s + n
%     #  s contient 0+...+(n-1)+n
% # ici s > 1000 pour la première fois et s contient $0+\dots+n$
% print(n)
% \end{pythoncode}
% \end{sol}
% \end{questions}


% \exercice{nom={La tortue bourrée}}



% \magsection{Quelques compléments}

% \magsubsection{Fonction}



% \exercice{nom={Paramètres optionnels}}
% Lire l'aide de la fonction \verb_print_ afin de comprendre le rôle des paramètres
% optionnels \verb_sep_ et \verb_end_. En déduire le résultat de l'execution du
% script suivant.
% \begin{pythoncode}
% print(1, 2, 3, sep='+', end='=')
% print(6, 5, 4, sep='\n', end='*')
% \end{pythoncode}

% \magsubsection{Indice, slice}



% \exercice{nom={Mélange de \textsc{Monge}}}
% Le mélange de \textsc{Monge} d'un paquet de cartes numérotées de 1 à $2n$ consiste à
% démarrer un nouveau paquet avec la carte 1, à placer la carte 2 au dessus de ce paquet,
% puis la carte 3 au dessous du nouveau paquet et ainsi de suite en plaçant les cartes
% paires au dessus du nouveau paquet et les cartes impaires au dessous. Autrement dit, si
% le mélange initial est représenté par la suite $(1,2,\ldots,2n)$, son mélange de
% \textsc{Monge} sera représenté par la suite $(2n,2n-2,\ldots,4,2,1,3,5,\ldots,2n-3,2n-1)$.
% Réaliser, à l'aide d'une ligne de \textsc{Python}, le calcul d'un mélange de 
% \textsc{Monge} d'une chaine de caractères \verb_s_.

%END_BOOK

\end{document}


\chapter{Fonction}
\setcounter{numeroexercicecours}{1}
\input{cours-fonction}
\section{Exercices}
\setcounter{numeroexercice}{1}
\documentclass{magnolia}

\magtex{tex_driver={pdftex}}
\magfiche{document_nom={Fonction},
          auteur_nom={Victor Lambert},
          auteur_mail={victor.lambert@auxlazaristeslasalle.fr}}
\magexos{exos_matiere={maths},
         exos_niveau={mpsi},
         exos_chapitre_numero={3},
         exos_theme={Fonction}}
\magmisenpage{}
\maglieudiff{}
\magprocess

\begin{document}

%BEGIN_BOOK

\magsection{Fonction}

\magsubsection{Fonction}
\magsubsection{Les fonctions comme valeurs}

\exercice{nom={Dérivée numérique}}
On se donne une fonction numérique $f$ dont on souhaite obtenir une approximation de la dérivée $f'(x)$. Pour cela, on utilise les deux approximations suivantes
\[f'(x)\approx\frac{f(x+\epsilon)-f(x)}{\epsilon} \quad\text{et}\quad f'(x)\approx\frac{f(x+\epsilon)-f(x-\epsilon)}{2\epsilon}.\]
\begin{questions}
\question Écrire les fonctions \verb!derive_1(f: function, x: float, eps: float) -> float! et \verb!derive_2!, de
même signature, qui prennent en entrée $f$, $x$ et $\epsilon$ et qui renvoient respectivement l'approximation de $f'(x)$ donnée par la première et la seconde formule.
\question Testez la fonction \verb!deriv_1! avec $f(x)\defeq x^2$ puis $f(x)\defeq \ln x$ pour $x=1$ et différentes valeurs de
  $\epsilon$ de la forme $10^{-n}$. Pour quelle valeur de $n$ l'approximation est-elle la meilleure~?
% \question Étant donné une fonction $f$, observez les premières valeurs de la suite
% de terme général \[u_n=-\log_{10}\abs{\frac{\phi_n(x) - f'(x)}{f'(x)}}\] o\`u $\phi_n(x)$ est le nombre flottant renvoyé par la fonction \verb!derive_1! appelée avec $x=1$ et $\epsilon=10^{-n}$. La fonction $f'$ sera implémentée en utilisant l'expression de la dérivée de $f$. 
% Donnez la signification de $u_n$ et expliquez les variations des premiers termes de la suite $(u_n)$. Estimez, en fonction de $u\approx 10^{-16}$, la valeur de $\epsilon$ qui semble minimiser l'erreur commise en utilisant la fonction \verb!derive_1! pour obtenir une approximation de $f'(x)$.
\question Répétez l'expérience avec la fonction \verb!derive_2!. Quelle conclusion pouvez-vous en tirer~?
% \question Quelle méthode recommanderiez-vous à une personne souhaitant calculer numériquement $f'(x)$ avec la meilleure précision si on a seulement accès à une fonction calculant $f(x)$~?
\end{questions}

\magsubsection{Sortie anticipée}

\exercice{nom={Doublon}}
Écrire une fonction \verb!doublon(a)! prenant en entrée une liste \verb!a! et renvoyant \verb_True_
si \verb_a_ possède un doublon et \verb_False_ sinon. Par exemple, \verb_doublon([3, 4, 7, 3, 2])_
devra renvoyer \verb_True_ car 3 est présent deux fois dans la liste. Notre fonction devra
sortir dès qu'un doublon est trouvé.

\exercice{nom={Mêmes éléments}}
Écrire une fonction \verb!memes_elements(u: list[int], v: list[int]) -> bool! déterminant si les
listes $u$ et $v$ possèdent les mêmes éléments, peu importe l'ordre et leur nombre d'occurences.
Par exemple
\begin{center}
  \verb!memes_elements([7, 3, 5, 3], [3, 7, 5])!
\end{center}
  devra répondre \verb!True! et
\verb!memes_elements([1, 7, 2], [2, 1])! devra répondre \verb!False! car 7 est présent dans la
première liste mais pas dans la seconde.

\exercice{nom={Chaine bien parenthésée}}
On dit qu'une chaine de caractères $s$ constituée de \verb_'('_ et de \verb_')'_ est bien parenthésée lorsque
chaque parenthèse ouvrante est correctement fermée. Par exemple
\verb_"(()())"_ est bien parenthésée alors que \verb_"())("_ et \verb_"(()"_ ne le sont pas.\\

En utilisant
un compteur qui compte le nombre de parenthèses ouvrantes que l'on a vu jusqu'à présent et qui ne sont pas fermées,
écrire une fonction \verb!bien_parenthesee(s: str) -> bool! nous indiquant si la chaine caractère \verb!s!,
que l'on supposera constituée uniquement de \verb_'('_ et de \verb_')'_, est bien parenthésée.



\magsubsection{Assertion, test unitaire}
\exercice{nom={Chasse aux bugs}}
Les fonctions des questions suivantes sont buguées. Pour chaque fonction, le but est de proposer un test unitaire
qui n'est pas passé par la fonction puis de corriger le bug.
\begin{questions}
\question On définit la suite de Fibonacci par
  \[F_0\defeq 0\qsep F_1\defeq 1\qsep \text{et}\quad \forall n\in\N\qsep F_{n+2}\defeq F_{n+1}+F_n.\]
  Donner un test unitaire qui n'est pas passé par la fonction suivante devant calculer le $n$-ième terme
  de la suite de Fibonacci
\begin{pythoncode}
def fibo(n):
    """fibo(n: int) -> int"""
    a = 0
    b = 1
    for _ in range(n - 1):
        a, b = b, a + b
    return b
\end{pythoncode}
  puis corriger la fonction pour qu'elle devienne valide.
\end{questions}
\magsubsection{Sortie anticipée}

\magsection{Variable locale et globale}
\magsubsection{Variable locale}
\magsubsection{Variable globale}
\magsubsection{Composition de fonctions}

\magsection{Programmation récursive}
\magsubsection{Fonction récursive pure}



\exercice{nom={Ensemble des parties}}
On suppose les ensembles représentés par des listes non triées d'éléments deux-à-deux
distincts. Rédiger une fonction \verb!subset! qui prend en argument un ensemble et qui
renvoie l'ensemble de ses parties. Par exemple \verb!subset([1,2,3])! doit
renvoyer \verb![[], [3], [2], [2, 3], [1], [1, 3], [1, 2], [1, 2, 3]]!.

\exercice{nom={Somme dans un arbre}}
Écrire une fonction \verb!somme(t)! qui prend en paramètre une séquence imbriquée, de
profondeur et de structure quelconque, dont tous les composants élémentaires sont des nombres,
et qui calcule la somme de tous ces éléments. Par exemple
\verb!somme([[[1, 2], [3, 4, 5]], 6, [7, 8], 9])! devra renvoyer 45. Pour écrire cette
fonction on pourra utiliser la fonction \verb!isinstance(t, list)! qui permet de savoir si
$t$ est une liste.

\exercice{nom={Dénombrabilité de $\N^2$}}
On démontre que l'ensemble $\N\times\N$ est dénombrable en numérotant chaque couple $(x,y)\in\N^2$ suivant le procédé suggéré par la figure ci-dessous.
\begin{center}
\includegraphics[width=0.3\textwidth]{../../Commun/Images/python-exos-rec-1.pdf}
\end{center}
\begin{questions}
\question Rédiger une fonction récursive qui renvoie le numéro du point de coordonnées $(x,y)\in\N^2$.
\question Rédiger la fonction réciproque, là encore, de façon récursive.
\end{questions}

\magsubsection{Fonction récursive impérative}

\exercice{nom={Triangles}}
\begin{questions}
\question Écrire une fonction récursive \verb!triangle(n)! affichant un triangle de la manière
  suivante.
\begin{pythoncode}
In [1]: triangle(5)
(*@\textcolor{purple}{*}@*)
(*@\textcolor{purple}{**}@*)
(*@\textcolor{purple}{***}@*)
(*@\textcolor{purple}{****}@*)
(*@\textcolor{purple}{*****}@*)
\end{pythoncode}
\question Écrire une fonction récursive \verb!triangle_inverse(n)! affichant un triangle de la manière
  suivante.
\begin{pythoncode}
In [2]: triangle_inverse(5)
(*@\textcolor{purple}{*****}@*)
(*@\textcolor{purple}{****}@*)
(*@\textcolor{purple}{***}@*)
(*@\textcolor{purple}{**}@*)
(*@\textcolor{purple}{*}@*)
\end{pythoncode}
\question Écrire une fonction récursive \verb!sablier(n)! affichant un demi-sablier de la manière
  suivante.
\begin{pythoncode}
In [3]: sablier(4)
(*@\textcolor{purple}{****}@*)
(*@\textcolor{purple}{***}@*)
(*@\textcolor{purple}{**}@*)
(*@\textcolor{purple}{*}@*)
(*@\textcolor{purple}{**}@*)
(*@\textcolor{purple}{***}@*)
(*@\textcolor{purple}{****}@*)
\end{pythoncode}
\end{questions}

\exercice{nom={Cercles}}
On suppose disposer d'une fonction \verb!circle([x, y], r)! qui trace à l'écran un cercle de
centre $(x,y)$ et de rayon $r$.
\begin{questions}
\question Définir une fonction récursive permettant de tracer le dessin ci-dessous où le
  cercle le plus gros est de rayon 1 de centre de coordonnées $(0,0)$ et chaque cercle est
	de rayon deux fois plus petit que celui de la génération précédente.
\begin{center}
\includegraphics[width=0.2\textwidth]{../../Commun/Images/python-exos-rec-3.pdf}
\end{center}
\question Même question avec ce dessin.
\begin{center}
\includegraphics[width=0.2\textwidth]{../../Commun/Images/python-exos-rec-4.pdf}
\end{center}
\end{questions}

\exercice{nom={Variante sur les tours de \textsc{Hanoï}}}
Résoudre le problème des tours de Hanoï en s’imposant une contrainte supplémentaire~: tout mouvement entre les tiges 1 et 3 est interdit 

\exercice{nom={Problème des $n$ reines}}
Le problème des $n$ reines consiste à placer $n$ reines sur un échiquier de sorte que
deux reines quelconques ne puissent pas s'attaquer, c'est-à-dire qu'il ne faut pas que
deux reines partagent une même ligne, une même colonne ou une même diagonale. De telles
positions seront qualifiées dans cet exercice de \emph{valides}. Il est évident que
dans chaque colonne doit se trouver une et une seule reine. Ainsi, il est possible de
représenter ce problème par un tableau de $n$ cases $q=[q_0,\ldots,q_{n-1}]$ dans lequel
$q_j$ désigne la ligne dans laquelle est placée la reine de la colonne $j$. Par exemple,
le dessin ci-dessous représente la solution $[3, 6, 2, 7, 1, 4, 0, 5]$.
\begin{center}
\includegraphics[width=0.3\textwidth]{../../Commun/Images/python-exos-rec-8.pdf}
\end{center}
On appellera solution partielle de rang $j$ un tableau $q$ de longueur $n$ dont les $j$
premières cases sont remplies avec des positions valides pour les reines, les
$n-j$ autres cases restant à remplir.\\

Écrire une fonction \verb!reine(q, j)! qui prend pour arguments un entier $j$ et une solution
partielle $q$ et qui réalise les opérations suivantes.
\begin{itemize}
\item Si $j=n$, cette fonction se contente d'afficher le tableau $q$. Dans ce cas, le problème
  est résolu.
\item Si $j<n$, cette fonction recherche parmi les $n$ valeurs possibles pour $q_j$ celles qui
  correspondent à des positions valides et pour chacune d'elles poursuit la recherche au
	rang $j+1$.
\end{itemize}
% \exercice{nom={Trier 3 éléments}}
% Écrire une fonction \verb!tri(a, b, c)! qui prend en argument trois nombre réels $a$, $b$ et
% $c$ et qui renvoie le triplet formé de cet 3 éléments, triés par ordre croissant.

% \exercice{nom={Parité}}
% Créer une fonction \texttt{parite} qui prend pour arguments deux entiers \texttt{x} et \texttt{y}, et qui renvoie  le booléen disant si \texttt{x} et \texttt{y}  ont même parité ou non.

	
% \exercice{nom={Triangle rectangle}}
% Créer une fonction \texttt{est\_rectangle} prenant pour argument les trois longueurs d'un triangle, et qui renvoie le booléen disant si le triangle est rectangle ou non.

	
	
% \exercice{nom={Suite récurrence}}
% Écrire une fonction \texttt{suite\_cos} qui prend pour argument un entier naturel \texttt{n} et qui affiche toutes les valeurs de $\cos(i)$ pour $i$ variant de $0$ à \texttt{n-1}.
	
% \exercice{nom={Initiales}}
% Créer une fonction \texttt{initiales} qui prend pour arguments deux chaînes de caractères \texttt{prenom} et \texttt{nom}, et qui renvoie les initiales (on suppose que les noms et prénoms ne sont pas composés...).
	
% \exercice{nom={Age et prénom}}
% Créer une fonction \texttt{age\_et\_prenom} qui prend pour argument un entier (l'âge) et une chaîne de caractères (le prénom), et qui renvoie une phrase redonnant l'âge et le nombre de lettres dans le prénom.\\ Par exemple, \texttt{age\_et\_prenom(17,'Bob')} donnera \texttt{Tu as 17 ans et ton prénom a 3 lettres.}
	
% \exercice{nom={Age}}
% Créer une fonction \texttt{age} qui prend pour argument une date de naissance (sous la forme d'une chaîne de caractère '26/11/1987'), et qui renvoie l'âge de la personne à la date d'aujourd'hui.	
% \begin{sol}
% Il y a certainement plus malin...
% \begin{pythoncode}
% def age(birth,today):
% 		jourauj,moisauj,anneeauj=int(today[0:2]),int(today[3:5]),int(today[6:10])
% 		jourbirth,moisbirth,anneebirth=int(birth[0:2]),int(birth[3:5]),int(birth[6:10])
% 		if moisauj>moisbirth:
% 				annee=anneeauj-anneebirth
% 				if jourauj>=jourbirth:
% 						mois=moisauj-moisbirth
% 				else:
% 						mois=moisauj-moisbirth-1
% 		elif moisauj==moisbirth:
% 				if jourauj>=jourbirth:
% 						annee=anneeauj-anneebirth
% 						mois=0
% 				else: 
% 						annee=anneeauj-anneebirth-1
% 						mois=11
% 		else:
% 				annee=anneeauj-anneebirth-1
% 				if jourauj>=jourbirth:
% 						mois=(moisauj-moisbirth)%12
% 				else:
% 						mois=(moisauj-moisbirth)%12-1
% 		return ("Il a "+str(annee)+" ans et "+str(mois)+" mois.")
% \end{pythoncode}
% \end{sol}

% \exercice{nom={Fonctions mystère}}
% On suppose que la variable \verb_n_ est de type \verb_int_. Donner, en fonction de $n$, la valeur et le type de l’objet renvoyé par chacune des fonctions suivantes. On ne cherchera pas à simplifier les calculs de sommes et produits mais plutôt à donner une expression utilisant des $\sum$ et des $\prod$.
% \begin{questions}
% \question
% \begin{pythoncode}
% def a(n):
% 		res = 1
% 		for i in range(1, n + 1):
% 				res = res * (i / n)
% 		return res
% \end{pythoncode}
% \question
% \begin{pythoncode}   
% def b(n):
% 		som = 0
% 		prod = 1
% 		for i in range(1, n):
% 				prod = prod * i
% 				som = som + prod
% 		return som
% \end{pythoncode}
% \question
% \begin{pythoncode}   
% def c(n):
% 		res = 0
% 		for i in range(n, 0, -1):
% 				res = res + i
% 		return res
% \end{pythoncode}
% \question
% \begin{pythoncode}    
% def d(n):
% 		q = n
% 		k = 0
% 		while q > 0:
% 				q = q // 10
% 				k = k + 1
% 		return k
% \end{pythoncode}
% \question
% \begin{pythoncode}   
% def e(n):
% 		res = 0
% 		for i in range(1, n + 1):
% 				for j in range(1, n + 1):
% 						if i > j:
% 								res = res + j
% 						else:
% 								res = res + i
% 		return res
% \end{pythoncode}
% \question
% \begin{pythoncode}    
% def f(n):
% 		u, v = 0, 2
% 		for i in range(n):
% 				u = u + v
% 				v = v + 1
% 		return u
% \end{pythoncode}
% \end{questions}


% \exercice{nom={Fonctions à compléter}}
% Complétez les programmes suivants afin qu'ils répondent aux specifications données.
% \begin{questions}
% \question
% \begin{pythoncode}
% ___

% def a(n, u0):
% 		u = ___
% 		for i in range(0, ___):
% 				u = ___
% 		return u
% \end{pythoncode}
% Cette fonction doit renvoyer le terme d'indice $n$ de la suite $(u_n)$ définie par son premier terme $u_0$ et la relation de récurrence
% \[\forall n\in\N\qsep u_{n+1}=\sqrt{u_n^2+3}.\]
% \question
% \begin{pythoncode}   
% def b(n):
% 		u = ___
% 		v = ___
% 		for i in range(___):
% 				u, v = ___
% 		return u
% \end{pythoncode}
% Cette fonction doit renvoyer le terme d'indice $n$ de la suite $(F_n)$ définie par $F_{0}=0$, $F_{1}=1$ et
% \[\forall n\in\N\qsep F_{n+2}=F_{n+1}+F_{n}.\]
% \question
% \begin{pythoncode}		
% def c(n):
% 		res = 1
% 		for i in range(1, ___):
% 				res = ___
% 				for j in range(0, ___):
% 						res= ___
% 		return res
% \end{pythoncode}
% Cette fonction doit renvoyer
% \[\sum_{i=0}^{n} \left(3^{i} +\sum_{j=1}^{n} 2i\right).\]
% \question
% \begin{pythoncode}
% def d(n):
% 		res = 0
% 		for k in range(___):
% 				p = 0
% 				for i in range(___):
% 						p = ___
% 				res = ___
% 		return res 
% \end{pythoncode}
% Cette fonction doit renvoyer
% \[\sum_{k=1}^{n} \sum_{i=1}^{k} \left(2^{i}-3ik \right).\]
% \question
% \begin{pythoncode}
% def e(n, a, b):
% 		u, v = ___
% 		i = 0
% 		while i < ___:
% 				u, v, i = ___
% 		return u, v
% \end{pythoncode}
% Cette fonction doit renvoyer le couple $(u_{n}, v_{n})$ lorsque les suites $u$ et $v$ sont définies par $u_{1}=a$, $v_{1}=b$ et 
% \[\forall n \in \N\qsep u_{n+1}=\dfrac{1}{u_{n}v_{n}} \et v_{n+1}=\dfrac{ u_{n}+v_{n}}{2}.\]
% \end{questions}

\magsubsection{Fonctions mutuellement récursives}

%END_BOOK

\end{document}


\chapter{Liste}
\setcounter{numeroexercicecours}{1}
\input{cours-structure_sequentielle}
\section{Exercices}
\setcounter{numeroexercice}{1}
\input{exos-liste}

\chapter{Représentation des données}
\setcounter{numeroexercicecours}{1}
\input{cours-representation_donnees}
\section{Exercices}
\setcounter{numeroexercice}{1}
\documentclass{magnoliaold}

\magtex{tex_driver={pdftex}}
\magfiche{document_nom={Représentation des données},
          auteur_nom={François Fayard},
          auteur_mail={francois.fayard@auxlazaristeslasalle.fr}}
\magexos{exos_matiere={maths},
         exos_niveau={mpsi},
         exos_chapitre_numero={1},
         exos_theme={Représentation des données}}
\magmisenpage{}
\maglieudiff{}
\magprocess

\begin{document}

%BEGIN_BOOK

\magsection{Les entiers}
\magsubsection{Décomposition en base $b$}

\exercice{nom={Calculs en base 2}}
Réaliser les opérations suivantes en base 2, sans passer par la base 10, à l'aide des
algorithmes appris à l'école primaire.
\[\underline{101010}_2+\underline{11000}_2\qsep \underline{110101}_2-\underline{11001}_2\qsep \underline{11101}_2 \times \underline{1011}_2\qsep
  \underline{1100101}_2 / \underline{1011}_2.\]


\exercice{nom={Somme et produit en base $b$}}
Dans cet exercice, un entier $d\in\N$ est représenté par sa décomposition en base
$b\geq 2$, c'est-à-dire par une liste d'entiers $d_k\in\interefo{0}{b}$ pour $0\leq k<n$,
telle que
\[d=\sum_{k=0}^{n-1} d_k b^k.\]
Les chiffres de poids faible sont situés en début de liste alors que les chiffres
de poids fort sont quant à eux en fin de liste.  Le but de cet exercice est
d'implémenter l'addition et la multiplication en base $b$, comme on l'a appris à l'école
primaire.
\begin{questions}
% \question Montrer que lors de l'addition de deux nombres $d$ et $e$ en base $b$,
%   à chaque étape, la retenue est soit égale à 0, soit égale à 1.
\question Écrire une fonction \verb!chiffre(d: list[int], k: int) -> int! renvoyant
  le chiffre $d_k$ du nombre $d$. Si $k$ est plus grand que la longueur de
  la liste \verb!d!, cette fonction devra renvoyer 0.
\question Écrire une fonction
  \verb!addition(d: list[int], e: list[int], b: int) -> list[int]! réalisant l'addition
  de deux nombres $d$ et $e$ en base $b$.
\question
\begin{questions}
% \question Montrer que lors de la multiplication d'un nombre $d\in\N$ en base $b$ par
%   un chiffre $c\in\interefo{0}{b}$, la retenue est toujours dans $\interefo{0}{b}$. 
\question Écrire une fonction
  \verb!multiplication_chiffre(d: list[int], c: int, i: int, b: int) -> list[int]!
  réalisant la multiplication en base $b$ du nombre $d\in\N$ par $c b^i$, où
  $c\in\interefo{0}{b}$ et $i\in\N$.
\question En déduire la fonction \verb!multiplication(d: list[int], e: list[int], b: int) -> list[int]!
  réalisant la multiplication de deux nombres $d$ et $e$ en base $b$. 
\end{questions}
\end{questions}

\exercice{nom={Incrément binaire}}
Écrire une fonction \verb!increment(d: list[int]) -> NoneType! prenant en entrée la
décomposition binaire
\[d=\sum_{k=0}^{n-1} d_k 2^k\]
du nombre $d$ et la transformant en celle de $d+1$.


\exercice{nom={Espace binaire}}
On appelle espace binaire d'un entier naturel $n$ toute séquence consécutive de 0 délimités
par deux 1 dans la décomposition en base 2 de $n$. Par exemple, le nombre 529 possède deux
espaces binaires de longueurs respectives 3 et 4 car $529=\underline{1000010001}_2$. En revanche, 32 ne
possède pas d'espace binaire puisque $32=\underline{100000}_2$.\\

Écrire une fonction \verb!espace_binaire(n: int) -> int! qui prend pour argument un entier naturel
et renvoie la longueur du plus grand espace binaire présent dans $n$ s'il existe, et la valeur
0 sinon.


\exercice{nom={Factorion}}
On appelle factorion, tout entier naturel qui est égal à la somme des factorielles de
ses chiffres. Par exemple, 145 est un factorion en écriture décimale, car
\[1! + 4! + 5! = 1 + 24 + 120 = 125.\]
\begin{questions}
\question Écrire une fonction \verb!factorielle(n : int) -> int!, calculant la
  factorielle d'un entier $n$.
\question Écrire une fonction \verb!factorion(m: int, b: int) -> bool!, déterminant
  si $m$ est un factorion en base $b$.
\question En déduire une fonction \verb!liste_factorions(b: int, p:int) -> list[int]!,
  renvoyant l'ensemble des factorions inférieurs ou égaux à $p$. 
\question
\begin{questions}
\question Montrer que si $m\in\N$ est un factorion de $n$ chiffres en base $b$, alors
  \[b^{n-1} \leq m \leq n(b-1)!.\]
\emph{En particulier, si
  \[u_n \defeq \frac{n(b-1)!}{b^{n-1}}<1,\]
  alors il n'existe aucun factorion de $n$ chiffres.}
\question Montrer que la suite $(u_n)$ est décroissante et tend vers 0,
  puis écrire une fonction
\begin{center}
  \verb!les_factorions(b: int) -> list[int]!
\end{center}
  renvoyant l'ensemble des factorions en base $b$.
\end{questions}
\end{questions}

\exercice{nom={Toblerone}}
Après un changement de l'équipe dirigeante, l'entreprise Mondelez
International, qui fabrique les barres Toblerone, décide de rationaliser
sa production pour maximiser ses revenus. En effet, leur chaine de production
fabrique des barres de $n$ «~carreaux~» qui sont ensuite coupées en barres plus
petites avant d'être vendues. Mais une récente étude de marché a déterminé le
prix auquel on pouvait vendre des barres de longueur $k$ (pour
$0 \leq k \leq n$), et ce prix s'avère ne pas avoir de relation simple avec $k$.
Le problème est donc de décider comment découper la barre initiale de $n$
carreaux en des barres plus petites pour maximiser le prix de vente total.\\

Dans tout le problème, on considèrera que l'on dispose d'un tableau
\verb!p!, indicé de $0$ à $n$, tel que \verb!p[k]! est le prix de
vente d'une barre de longueur $k$. Bien entendu, le prix d'un morceau de taille 0
est 0. Remarquez que si \verb!p[n]! est suffisamment grand, la solution optimale peut
très bien être de ne pas découper la barre.

\begin{center}
\includegraphics[width=0.5\textwidth]{../../commun/images/python-exos-toblerone}
\end{center}

\noindent On donne un exemple de tableau $p$ pour $n = 10$.
\begin{center}
  \begin{tabular}{r *{11}{c}}
    \toprule
    $k$   & 0 & 1 & 2 & 3 & 4 & 5 & 6 & 7 & 8 & 9 & 10\\
    \midrule
    ${p_k}$ & 0 & 1 & 5 & 8 & 9 & 10& 17& 17 & 20& 24& 26 \\
    \bottomrule
  \end{tabular}
\end{center}

Pour résoudre ce problème, on commence par établir une correspondance entre
les découpes d'un Toblerone composé de $n$ carreaux et les tableaux $d$
de $n-1$ booléens~: on coupe après le carreau d'indice $k$ si et seulement si
$d_k$ est vrai.

\begin{center}
\includegraphics{../../commun/images/info-cours-algo-decoupe}\\
La découpe $1, 3, 2, 1, 3$ et le tableau de booléens correspondant.
\end{center}

\begin{questions}
\question Écrire une fonction \verb!decomposition(d: int, n: int) -> list[bool]!
  prenant en entrée un entier $n\in\N$ ainsi qu'un entier $d\in\interefo{0}{2^n}$,
  et renvoyant une liste de booléens $d_k$ de longueur $n$ telle que
  \[d=\sum_{k=0}^{n-1} d_k 2^k\]
  où le booléen \verb!True! représente le bit 1 et le booléen \verb!False! représente
  le bit 0.
\question Écrire une fonction \verb!prix_decoupe(d: list[bool], p: list[int]) -> int!
  prenant en entrée un tableau $d$ de longueur $n-1$ représentant une découpe d'une
  barre de longueur $n$, ainsi qu'un tableau $p$ de longueur $n+1$ représentant la
  liste des prix des différentes longueurs et renvoyant le prix de revente de la découpe
  $d$.
\question En déduire une fonction \verb!meilleur_prix(p: list[int]) -> tuple[list[bool], int]! renvoyant une meilleure découpe d'une barre de longueur $n$ ainsi que son prix
de revente correspondant.
\question Donner le prix ainsi qu'une découpe optimale associée pour l'exemple de tableau
  $p$ donné plus haut. 
\end{questions}

\emph{Nous verrons dans l'année des algorithmes dit de \og programmation dynamique \fg
permettant de résoudre ce type de problème plus efficacement.}
\magsubsection{Représentation mémoire des entiers non signés}


\magsubsection{Représentation mémoire des entiers signés}
\exercice{nom={Complément à 2}}
Dans une représentation en complément à 2 sur 8 bits, quels sont les entiers relatifs
représentés par $01101101$ et $10010010$~?

\exercice{nom={Machine 16~bits}}
On considère une architecture 16~bits. On considère les opérations
suivantes entre les entiers signés. Donner leur résultat mathématique
(en les considérant comme entiers naturels) puis leur résultat sur
16 bits signés.
\begin{questions}
\question $10\times 10$
\question $32767+1$
\question $256 \times (-256)$
\question $32767 - (-32768)$
\end{questions}

\magsection{Les nombres flottants}
\magsubsection{Représentation mémoire des flottants}

\exercice{nom={Le type float16}}
Dans le type \verb!float16! utilisé par Numpy, les nombres flottants sont représentés
sur 16 bits~: 1 bit pour le signe, 5 bits pour l'exposant, 10 bits pour la mantisse.
\begin{questions}
\question Donner la représentation machine dans de type de 1, de $-2$, puis du plus petit
  nombre strictement supérieur à 1, ainsi que se valeur.
\question Donner les représentations machine et la valeur des plus petits et des plus grands
  nombres normalisés.
\question Déterminer quel nombre est représenté par $0|01110|1001001000$.
\end{questions}

\magsubsection{Problèmes liés à l'arithmétique des nombres flottants}

\exercice{nom={Hamster jovial}}
  À votre grand bonheur, vous avez reçu pour Noël une balance d'excellente
  qualité : elle offre trois chiffres décimaux de précision, et ce autant
  pour des masses de l'ordre du gramme que de l'ordre de la tonne. Vous
  décidez d'utiliser cette balance pour mesurer la masse $m_h$ de votre
  hamster $h$. On suppose pour simplifier que $m_h$ est de l'ordre de
  100 grammes (un gros hamster, d'après Wikipedia).
  \begin{enumerate}
    \item Si $h$ accepte de monter docilement sur la balance et d'y rester
    le temps qu'elle fasse sa mesure, avec quelle précision obtiendrez-vous
    $m_h$ ?
    \item $h$, qui est d'une intelligence assez rare pour un rongeur,
    vous soupçonne, à tort ou à raison, de vouloir utiliser cette pesée pour
    justifier une mise au régime. Il descend donc immédiatement de la balance
    à chaque fois que vous l'y posez, et ce avant que la mesure n'ait été
    faite. Vous décidez alors de le peser indirectement : vous vous pesez une
    première fois avec $h$ dans la main, puis une deuxième fois sans $h$,
    et vous faites la différence. Avec quelle précision obtenez-vous $m_h$ ?
  \end{enumerate}

\exercice{nom={Ordre de sommation}}
On définit la suite $(u_n)$ par
\[\forall n\in\Ns\qsep u_n\defeq \sum_{k=2}^n \frac{1}{k(k-1)}.\]
\begin{questions}
\question En remarquant que
  \[\forall k\geq 2\qsep \frac{1}{k(k-1)}=\frac{1}{k-1} - \frac{1}{k},\]
  calculer explicitement $u_n$.
\question Afin de calculer $u_n$ à l'aide d'une somme, on écrit~:
\begin{pythoncodeline}
def sum_1(n):
    s = 0.0
    for k in range(2, n + 1):
        s = s + 1 / (k * (k - 1))
    return s
\end{pythoncodeline}
  Écrire le programme \verb!sum_2! calculant la même somme, mais sommant les $1/(k(k-1))$
  non pas avec $k$ allant de 2 à $n$ de manière croissante,d mais allant de $n$ à 2 de
  manière décroissante.
\question Comparer le résultat des deux fonctions précédentes pour $n = 10\ 000\ 000$.
  Quelle est la fonction la plus précise~? Comment expliquez-vous ce phénomène~?
\end{questions}


\magsection{Caractères et chaines de caractères}
\magsubsection{Codes Ascii et Unicode}
\magsubsection{Lecture et écriture dans un fichier}



% \begin{exoC}{}{}
%   On considère deux flottants $a$ et $b$ connus chacun avec $k$
%   chiffres significatifs (en base 10). On suppose que
%   $\abs{\frac{a - b}{a}} \simeq 10^{-d}$.
%   \begin{enumerate}
%     \item Que peut-on dire des $d$ chiffres les plus significatifs de $a$
%     et de $b$ ?
%     \item Combien y a-t-il de chiffres significatifs dans $a - b$ ?
%   \end{enumerate}
% \end{exoC}

% \begin{exemple}{}{}
%   \begin{multicols}{2}
%     On a :
%     \[
%       \frac{1-\cos(x)}{x^2} =
%       \frac 12 \left( \frac{ \sin \left( \frac x2\right)}{\frac x2}\right)^2
%       \Tend{x}{0} \frac 12
%     \]
%     et même plus précisément
%     \[
%       \abs{\frac{1 - \cos x}{x^2} - \frac 12} \Equiv{x}{0}
%       \frac{x^2}{24} \simeq \nombre{0,0417}\cdot x^2.
%     \]
%     Mais les deux expressions se comportent très différemment lors d'un calcul
%     en virgule flottante :
%     \begin{center}
%       \begin{tabular}{ccc}
%         \toprule
%         $x$ & $\abs{\frac{1-\cos(x)}{x^2} - \frac 12}$
%         &  $\abs{\frac 12 \left( \frac{ \sin \left( \frac x2\right)}{\frac x2}\right)^2
%           - \frac 12}$ \\
%         \otoprule
%         \py!1e-01! & \py!4.17e-04! & \py!4.17e-04! \\
%         \midrule
%         \py!1e-02! & \py!4.17e-06! & \py!4.17e-06! \\
%         \midrule
%         \py!1e-03! & \py!4.17e-08! & \py!4.17e-08! \\
%         \midrule
%         \py!1e-04! & \py!3.04e-09! & \py!4.17e-10! \\
%         \midrule
%         \py!1e-05! & \py!4.14e-08! & \py!4.17e-12! \\
%         \midrule
%         \py!1e-06! & \py!4.45e-05! & \py!4.17e-14! \\
%         \midrule
%         \py!1e-07! & \py!4.00e-04! & \py!4.44e-16! \\
%         \midrule
%         \py!1e-08! & \py!5.00e-01! & \py!0.00e+00! \\
%         \bottomrule
%       \end{tabular}
%     \end{center}
%   \end{multicols}
% \end{exemple}

% \subsubsection{Calcul de sommes}


%END_BOOK

\end{document}


% \chapter{Récursivité}
% \setcounter{numeroexercicecours}{1}
% \input{cours-recursivite}
% \section{Exercices}
% \setcounter{numeroexercice}{1}
% \input{exos-recursivite}

\chapter{Complexité}
\setcounter{numeroexercicecours}{1}
\input{cours-complexite}
\section{Exercices}
\setcounter{numeroexercice}{1}
\documentclass{magnolia}

\magtex{tex_driver={pdftex}}
\magfiche{document_nom={Complexité},
          auteur_nom={François Fayard},
          auteur_mail={francois.fayard@auxlazaristeslasalle.fr}}
\magexos{exos_matiere={maths},
         exos_niveau={mpsi},
         exos_chapitre_numero={1},
         exos_theme={Complexité}}
\magmisenpage{}
\maglieudiff{}
\magprocess

\begin{document}

%BEGIN_BOOK

\magsection{Complexité}
\magsubsection{Notation mathématique}
\magsubsection{Type de ressource}
\magsubsection{Complexité dans le pire des cas}
\magsubsection{Complexité en moyenne}
\magsubsection{Complexité temporelle et temps de calcul}
\magsection{Calcul de complexité temporelle}
\magsubsection{Algorithme itératif}

\exercice{nom={Le mur}}
Vous êtes face à un mur qui s'étend à l'infini dans les deux directions. Il y a une porte dans ce mur, mais vous ne connaissez ni la distance, ni la direction dans laquelle elle se trouve. Par ailleurs, l'obscurité vous empêche de voir la porte à moins d'être juste devant elle.
Décrire un algorithme vous permettant de trouver cette porte en un temps linéaire vis-à-vis de la distance qui vous sépare de celle-ci.

\exercice{nom={Problème proposé par le CCC (Comité Contre les Chats)}}
Le problème est de déterminer à partir de quel étage d'un immeuble sauter du balcon est
fatal à un chat. Vous êtes dans un immeuble à $n$ étages (numérotés de 1 à $n$) et vous disposez de $k$ chats. Il n'y a qu'une opération possible pour tester si la hauteur d'un étage est fatale~: faire sauter un chat du balcon. S'il survit, vous pouvez le réutiliser ensuite, sinon vous ne pouvez plus. Vous devez proposer un algorithme pour trouver la hauteur à partir de laquelle un saut est fatal en faisant le minimum de lancers.
\begin{questions}
\question Si $k\geq \ents{\log n}$, proposer un algorithme en ${\rm O}(\log n)$ sauts.
\question Si $k<\ent{\log n}$, proposer un algorithme en
  \[{\rm O}\p{k+\frac{n}{2^{k-1}}}\]
  sauts.
\question Si $k=2$, proposer un algorithme en ${\rm O}(\sqrt{n})$ sauts.
\end{questions}

\exercice{nom={Poulidor forever}}
Expliquer comment trouver le deuxième plus grand élément d'un tableau $[a_0,\ldots,a_{n-1}]$
en effectuant au plus $n+\ent{\log n}-2$ comparaisons. Vous pouvez procéder par analogie avec
un tournoi à élimination directe en remarquant que le deuxième joueur le plus fort fait
nécessairement partie des adversaires malheureux du vainqueur.


\exercice{nom={Cherche un entier comme somme de deux entiers}}
\begin{questions}
  \question Écrire une fonction
  \verb!cherche_somme(t: list[int], s: int) -> tuple[int, int]! ayant
  la spécification suivante :
  \begin{itemize}
    \item Si la fonction renvoie \verb!(i, j)!, alors on a
    $0 \leq i \leq j < |t|$ et $t_i + t_j = s$.
    \item Si la fonction renvoie \verb!None!, alors il n'existe pas de
    couple $(i, j)$ vérifiant $0 \leq i \leq j < |t|$ et
    $t_i + t_j = s$.
  \end{itemize}
  Quelle est sa complexité~?
\begin{pythoncode}
In [1]: cherche_somme([15, 1, 3, 5, 6, 7, 10, 1, 8], 11)
Out[1]: (6, 7)

In [2]: cherche_somme([15, 1, 3, 5, 6, 7, 10, 1, 8], 14)
Out[2]: (5, 5)

In [3]: cherche_somme([15, 1, 3, 5, 6, 7, 10, 1, 8], 19)
Out[3]: None
\end{pythoncode}
  \question On suppose maintenant que le tableau \verb!t! est trié par ordre
  croissant. Écrire une fonction
  \begin{center}\verb!cherche_somme_croissant(t: list[int], s: int) -> tuple[int, int]!\end{center} ayant la
  même spécification que \verb!cherche_somme! mais de complexité linéaire en
  la taille du tableau.
\begin{pythoncode}
In [4]: cherche_somme_croissant([1, 1, 3, 5, 6, 7, 7, 10, 12, 15], 13)
Out[4]: (0, 8)

In [5]: cherche_somme_croissant([1, 1, 3, 5, 6, 7, 7, 10, 12, 15], 17)
Out[5]: (3, 8)
\end{pythoncode}
\question Déterminer un algorithme permettant d'implémenter \verb!cherche_somme! avec
une complexité quasi-linéaire.
\end{questions}


\exercice{nom={Exemples}}
Pour chacune des fonctions suivantes, évaluez la complexité temporelle en fonction de $n$.
\begin{questions}
\question $\ $
\begin{pythoncode}
def f1(n):
    x = 0
    for i in range(n):
        for j in range(n):
            x = x + 1
    return x
\end{pythoncode}
\question $\ $
\begin{pythoncode}
def f2(n):
    x = 0
    for i in range(n):
        for j in range(i):
            x = x + 1
    return x
\end{pythoncode}
\question $\ $
\begin{pythoncode}
def f3(n):
    x = 0
    for i in range(n):
        j = 0
        while j * j < i:
            x = x + 1
            j = j + 1
    return x
\end{pythoncode}
\question $\ $
\begin{pythoncode}
def f4(n):
    x = 0
    i = n
    while i > 1:
        x = x + 1
        i = i // 2
    return x
\end{pythoncode}
\question $\ $
\begin{pythoncode}
def f5(n):
    x = 0
    i = n
    while i > 1:
        for j in range(n):
            x = x + 1
        i = i // 2
    return x
\end{pythoncode}
\question $\ $
\begin{pythoncode}
def f6(n):
    x = 0
    i = n
    while i > 1:
        for j in range(i):
            x = x + 1
        i = i // 2
    return x
\end{pythoncode}
\end{questions}

\exercice{nom={Doublons dans un tableau}}
On désire obtenir un algorithme qui détermine si un tableau présente des doublons en son sein.
\begin{questions}
\question Rédiger un algorithme naïf qui résout le problème. Quelle est sa complexité~?
\question Rédiger maintenant un second algorithme en supposant cette fois le tableau trié.
  Quelle est sa complexité~? A-t-on intérêt à trier le tableau pour résoudre ce problème~?
\end{questions}

\exercice{nom={Équilibre d'un tableau}}
On se donne un tableau non vide $t$ de $n$ entiers relatifs et on cherche la valeur minimale
de
\[\Delta_k=(t_0+\cdots+t_k)-(t_{k+1}+\cdots+t_{n-1})\]
lorsqu'on fait varier $k$ dans $\intere{0}{n-2}$.
\begin{questions}
\question Rédiger une fonction \verb!delta(t, k)! qui calcule la quantité $\Delta_k$ et en
  déduire ne fonction \verb!equilibre(t)! qui résout le problème posé. Évaluer la
	complexité temporelle de cette dernière fonction.
\question Écrire une fonction \verb!equilibreLineaire(t)! qui résout ce problème en temps
  linéaire.
\end{questions}

\exercice{nom={Trouvez l'intrus}}
Un tableau contient un nombre impair d'entier positifs. Chacun de ces entiers est présent un
nombre pair de fois, à l'exception d'un seul.
\begin{questions}
\question Rédiger une fonction \verb!impair(t)! qui prend pour argument un tel tableau et
  renvoie cet unique entier présent un nombre impair de fois.
\question Analysez la complexité de cet algorithme.
\end{questions}

\exercice{nom={Insertion dans une pile}}
\begin{questions}
\question
À l'aide d'une pile auxiliaire, rédiger une fonction \verb!insere(x, p)! qui prend pour
argument un entier $x$ et une pile $p$ formée d'entiers triés par ordre croissant et qui
insère $x$ au sein de $p$ en préservant l'ordre relatif des éléments.
\begin{center}
\includegraphics[width=0.3\textwidth]{../../Commun/Images/python-exos-tableau-2.pdf}
\end{center}
\question En déduire une fonction \verb!tri(p)! qui prend pour argument une pile d'entiers
  et qui renvoie une nouvelle pile contenant les mêmes éléments triés par ordre croissant.
\question Quelle est la complexité temporelle de cette fonction~?
\end{questions}



\magsubsection{Algorithme récursif}

\exercice{nom={Somme des éléments d'un tableau}}
Écrire une fonction \verb!somme_tableau(t: list[int]) -> int! de manière
récursive en utilisant le fait que la somme des éléments d'un tableau vide est
nulle et que si \verb!t! est un tableau de taille $n\geq 1$, la somme de ses
éléments est la somme de son premier élément et des éléments restants.
Calculer la complexité de cette fonction en utilisant le fait que la création d'un
d'une slice de taille $n$ nécessite $n$ opérations élémentaires.
Comparer la performance de cette fonction avec une version itérative de cet algorithme.

\exercice{nom={Coefficients binomiaux}}
\begin{questions}
\question
Écrire une fonction récrusive \verb!binome(k: int, n: int) -> int!
calculant le coefficient $\binom{n}{k}$ en utilisant le fait que
\[\forall n\in\N\qsep \binom{n}{0}=\binom{n}{n}=1\]
et
\[\forall k,n\in\N\qsep \binom{n+1}{k+1}=\binom{n}{k}+\binom{n}{k+1}.\]
Que pensez-vous de la performance de cette fonction~?
\question Afin d'améliorer les performances de la fonction précédente, on se donne
	une valeur $N\in\N$ et on crée un tableau bidimensionnel 
\begin{pythoncode}
t = [[-1 for n in range(N + 1)] for k in range(N + 1)]
\end{pythoncode}
	Écrire une fonction \verb!binome_memoisation(k: int, n: int) -> int! qui,
	lorsqu'il est appelé avec $n\leq N$ et $k\leq N$, 
	renvoie $\binom{n}{k}$ s'il est déjà stocké dans le tableau \verb!t!
	à la place \verb!t[n][k]! et qui le calcule de manière récursive dans
	le cas où ce n'est pas déjà le cas (dans ce cas, on a \verb!t[n][k] = 1!).
	On veillera, dans ce dernier cas, à stocker le résultat du calcul dans
	le tableau \verb!t! avant de renvoyer ce résultat.
\end{questions}

\exercice{nom={Autour de l'exponentiation rapide}}
On souhaite écrire une fonction récursive qui calcule $a^n$.
\begin{questions}
\question Écrire une telle fonction qui exploite la relation
  \[\forall n\in\N\qsep a^n=a^{\ent{\frac{n}{2}}} a^{\ents{\frac{n}{2}}}.\]
\question Évaluer le nombre de multiplications à effectuer et comparer cet algorithme avec
  l'algorithme d'exponentiation rapide. 
\end{questions}


\exercice{nom={Minimum local}}
On suppose donné un tableau $t$ de longueur $n$ contenant au moins 3 éléments et possédant
la propriété suivante~: $t_0\geq t_1$ et $t_{n-2}\leq t_{n-1}$. Pour tout $k\in\intere{1}{n-2}$, on dit que le tableau possède un minimum local en $k$ lorsque $t_k\leq t_{k-1}$ et $t_k\leq t_{k+1}$.
\begin{questions}
\question Justifier l'existence d'un minimum local dans le tableau $t$.
\question Écrire une fonction qui détermine un minimum local en cout linéaire.
\question Écrire une fonction récursive qui détermine un minimum local en cout logarithmique.
\end{questions}

\exercice{nom={Rotation}}
Les processeurs graphiques possèdent en général une fonction de bas niveau appelée
\emph{blit} (ou transfert de bloc) qui copie rapidement un bloc rectangulaire d'une image
d'un endroit à un autre. L'objectif de cet exiercice est de faire tourner une image carrée
de $n\times n$ pixels de $90^{\circ}$ dans le sens direct en adoptant une stratégie récursive.
\begin{itemize}
\item On découpe l'image en 4 blocs de taille $(n/2)\times(n/2)$.
\item On déplace chacun de ses blocs à sa position finale à l'aide de 5 blits.
\item On fait tourner récursivement chacun de ces blocs.
\end{itemize}
\begin{center}
\includegraphics[width=0.6\textwidth]{../../Commun/Images/python-exos-rec-2.pdf}
\end{center}
On supposera dans tout l'exercice que $n$ est une puissance de 2.
\begin{questions}
\question Exprimer en fonction de $n$ le nombre de fois que la fonction blit est utilisée.
\question Quel est le cout total de cet algorithme lorsque le cout d'un blit d'un bloc
  $k\times k$ est en $\Theta(k^2)$.
\question Et lorsque ce cout est en $\Theta(k)$.
\end{questions}

\magsection{Calcul de complexité spatiale}
\magsubsection{Algorithme itératif}

\exercice{nom={Grenouille}}
Une petite grenouille se trouve face à une rivière. Initialement située sur une des deux rives
(position 0), elle veut se rendre sur la rive opposée (position $x+1$) mais ne peut réaliser
que des sauts d'une unité. Heureusement pour elle, des feuilles tombent sur la surface de la
rivière et peuvent lui permettre de sauter de feuille en feuille.\\
On se donne un tableau $t$ composée de $n$ entiers représentant les feuilles qui tombent~:
$t_k$ représente la position où la feuille tombe à l'instant $k$. L'objectif est de trouver
le moment le plus précoce où la grenouille pourra passer d'une rive à l'autre,
c'est-à-dire la date à laquelle toutes les positions de 1 à $x$ seront couvertes par une
feuille.
\begin{questions}
\question Rédiger une fonction \verb!grenouille(x, t)! qui résout ce problème. Par exemple,
  pour $x=5$ et $t=[1, 3, 1, 4, 5, 3, 2, 4]$, cette fonction devra renvoyer l'entier 6..
\question Évaluer la complexité temporelle et spatiale de cette fonction.
\end{questions}


\magsubsection{Algorithme récursif}

\exercice{nom={Complexité de l'exponentiation rapide}}
  On considère la fonction suivante :
\begin{pythoncode}
def expo(a, n):
    if n == 0:
        return 1
    elif if n % 2 == 0:
        return expo(a * a, n // 2)
    else:
        return a * expo(a, n - 1)
\end{pythoncode}
  \begin{enumerate}
    \item On note $M(n)$ le nombre de multiplications effectuées
          lors de l'appel \verb!expo(a, n)! (pour $n \geq 0$).
          Exprimer $M(2n + 1)$ et $M(2n)$ en fonction de $M(n)$.
    \item En déduire que $M(n) \leq 1 + 2 \log_{2} n$, pour $n \geq 1$.
    \item Quelle est la complexité de la fonction \verb!expo! ?
    \item Expérimentalement, en mesurant le temps d'exécution de \verb!expo(2, n)! et
    de \verb!expo(1.0007, n)!, on obtient
          les courbes suivantes :
          % \newpage
          \begin{multicols}{2}
            \begin{figure}[H]
              \centering
              \includegraphics[width = 8cm]{../../Commun/Images/info-exos-complexite-expo-entier}
              \caption{Graphique semilog pour l'exponentiation rapide d'un entier.}
            \end{figure}
            \begin{figure}[H]
              \centering
              \includegraphics[width = 8cm]{../../Commun/Images/info-exos-complexite-expo-flottant}
              \caption{Graphique semilog pour l'exponentiation rapide d'un flottant.}
            \end{figure}
          \end{multicols}
          Comment expliquer ce phénomène ?
  \end{enumerate}
%   \tcblower
%   \begin{enumerate}
%     \item La lecture du code fournit immédiatement, pour $n \geq 1$ :
%           \[
%           \begin{cases}
%             M(2n) = 1 + M(n) \\
%             M(2n + 1) = 1 + M(2n) = 2 + M(n)
%           \end{cases}
%           \]
%     \item On procède par récurrence forte sur $n \geq 1$.
%           \begin{description}
%             \item[Initialisation :] pour $n = 1$, on a d'après le code
%                   $M(1) = 1$, et $1 + 2\log_{2} 1 = 1$, la propriété
%                   est initialisée.
%             \item[Hérédité :] soit $n > 1$, supposons la propriété vérifiée
%                   pour $1 \leq k < n$.
%                   \begin{itemize}
%                     \item Si $n$ est pair, $n = 2k$ avec $1 \leq k < n$,
%                           alors :
%                           \begin{align*}
%                             M(n) & = M(2k) \\
%                                  & = 1 + M(k) \\
%                                  & \leq 2 + 2\log_{2} k & \text{d'après } H_{k} \\
%                                  & = 2(1 + \log_{2} k) \\
%                                  & = 2\log_{2} (n) & n = 2k \\
%                                  & < 1 + 2\log_{2} n
%                           \end{align*}
%                     \item Si $n$ est impair, $n = 2k + 1$ avec $1 \leq k < n$,
%                           alors :
%                           \begin{align*}
%                             M(n) & = M(2k + 1) \\
%                                  & = 2 + M(k) \\
%                                  & \leq 3 + 2\log_{2} k & \text{d'après } H_{k}
%                                  & = 1 + 2(1 + \log_{2} k) \\
%                                  & = 1 + 2\log_{2} (2k) \\
%                                  & \leq 1 + 2\log_{2} (2k + 1) \\
%                                  & = 1 + 2\log_{2} n
%                           \end{align*}
%                   \end{itemize}
%                   On a donc bien $M(n) \leq 1 + 2\log_{2} n$ pour $n \geq 1$.
%           \end{description}
%     \item La complexité de la fonction est proportionnelle au nombre de multiplications,
%           c'est donc un $\O(\log n)$. \\
%           \emph{Ce grand-$\O$ est en fait un grand-$\Theta$ : on pourrait facilement
%           montrer que $M(n) \geq 1 + \log_{2} n$.}
%     \item La majoration du nombre de multiplications reste bien sûr valable, c'est
%           donc la proportionnalité de la complexité à ce nombre de multiplications
%           qui doit être fausse dans le cas où $a$ est entier. Effectivement,
%           les entiers en Python ne sont pas des entiers machine, mais peuvent croître
%           arbitrairement : par exemple, \\verb!expo(3, 10000)! a plus de \nombre{4000}
%           chiffres en base 10. Les multiplications deviennent donc de plus en
%           plus coûteuses, et la complexité n'est plus logarithmique. \\
%           Ce problème ne se pose pas pour les flottants, et on obtient alors bien
%           une droite dans le graphique semilog (ce qui correspond à une complexité
%           de la forme $A\log n$).
%   \end{enumerate}
% \end{exo}

\exercice{nom={Fibonacci}}
Dans l'algorithme d'exponentiation rapide, rien n'impose qu'il
s'agisse d'un entier élevé à une certaine puissance. Dès lors qu'on
dispose d'une unité et d'une opération associative, on peut appliquer
cet algorithme pour calculer $x^n$ où $n$ est un entier. Cela
s'applique en particulier au calcul matriciel.
\begin{questions}
\question Écrire une fonction Python qui réalise la multiplication
  de deux matrices à coefficients entiers, supposées carrées de
  même dimension $m\times m$. Quelle est la complexité de cette
  fonction~?
\question En déduire une fonction qui calcule $M^n$ pour une matrice
  $M$ et un entier $n$. Donner la complexité de ce calcul en fonction
  de $m$ et $n$.
\question On définit la suite de Fibonacci par
  \[F_0\defeq 0,\qquad F_1\defeq   1,\quad\et\quad
    \forall n\in\N\qsep F_{n+2}\defeq F_{n+1}+F_n.\]
  Montrer que
  \[\forall n\in\Ns\qsep
    \begin{pmatrix}
      1 & 1\\
      1 & 0
    \end{pmatrix}^n =
    \begin{pmatrix}
      F_{n+1} & F_n\\
      F_n & F_{n-1}
    \end{pmatrix}\]
    et en déduire une fonction qui calcule $F_n$ avec seulement
    ${\rm O}(\log n)$ opérations arithmétiques.
\end{questions}

\exercice{nom={Memoization}}
  On veut calculer les termes de la suite définie par
  \[
    u_0 \defeq 1 \et \forall n \in \Ns\qsep \ u_{n} \defeq \frac{u_{n-1}}{1} + \frac{u_{n-2}}{2}
    + \dots + \frac{u_0}{n}.
  \]
  Pour cela, on peut utiliser la fonction suivante :
\begin{pythoncode}
def suite(n):
    """suite(n: int) -> float"""
    if n == 0:
        return 1.0
    else:
        s = 0.0
        for k in range(1, n + 1):
            s = s + suite(n - k) / k
        return s
\end{pythoncode}
  \begin{enumerate}
    \item On note $D(n)$ le nombre de divisions qui seront
          effectuées au total lors de l'appel \verb!suite(n)!
          (en comptant les appels récursifs).
          Donner une relation de récurrence reliant $D(n)$
          et les $D(k)$ pour $0 \leq k < n$.
    \item En déduire la valeur de $D(n)$, puis la complexité
          de la fonction \verb!suite!.
    \item Écrire une fonction (non récursive) effectuant le calcul
          en un temps plus raisonnable, au prix de l'utilisation d'un
          stockage auxiliaire. On déterminera les complexités temporelle
          et spatiale de cette nouvelle fonction.
  \end{enumerate}
    % Nous verrons plus tard que la première version de \verb!suite! a en fait
    % une complexité spatiale en $\Theta(n)$, ce qui signifie que la
    % deuxième version est strictement meilleure (même complexité spatiale,
    % complexité temporelle incomparablement meilleure).









% \exercice{nom={Retour sur la recherche dichotomique}}
% Il y a bien d'autres façons que celle vue en cours de programmer la méthode de recherche par dichotomie dans une liste triée.  Voici deux tentatives, très ressemblantes. Ces fonctions répondent-elles à la question ? 
	
% \begin{pythoncode}
% def dichot_v2(x,l):
%     g = 0
%     d = len(l) - 1
%     while g < d:
%         m = (g + d) // 2
%         if x > l[m]:
%             g = m + 1
%         else:
%             d = m
%     return l[g] == x
% \end{pythoncode}

% \begin{pythoncode}
% def dichot_v3(x,l):
%     g = 0
%     d = len(A) - 1
%     while g < d:
%         m = (g + d) // 2
%         if x < l[m]:
%             d = m - 1
%         else:
%             g = m
%      return l[g] == x   
% \end{pythoncode}
	
% 	{\it Indications} : Pour la preuve de validité, on pourra considérer la propriété ($\mathcal{H}$) suivante: "L'élément {\tt x} est présent dans la liste {\tt l} si et seulement si il est présent dans la sous-liste {\tt [l[i] for i in range(g,d)]}" et montrer que c'est un invariant de boucle. Pour la preuve de terminaison, on pourra considérer la quantité $d-g$. Si ça ne marche pas, on pourra tester les fonctions sur des listes de longueur 1 ou 2.

% \exercice{nom={La complexité du tri par sélection et du tri à bulle}}
% Reprendre les algorithmes de tri par sélection et de tri à bulle  du chapitre précédent (cf exercices 10 et 11) et montrer qu'ils ont une complexité quadratique.

% \exercice{nom={Écriture des chiffres d'un entier en base 2}}
% On rappelle (cf exercice 1 du chapitre précédent) que la fonction Python suivante renvoie la liste des chiffres de l'écriture de l'entier $n$ en base 2 en commençant par les chiffres de poids faibles :
% \begin{pythoncode}
% def vers_base2(n):               
%     x, res = n, [ ]                           
%     while x != 0:                        
%         x, r = x // 2, x % 2          
%         res.append(r)                      
%     return res 
% \end{pythoncode}
% Prouver la validité et la terminaison de cette fonction et en déterminer la complexité.

% {\it Indications} : Pour la preuve de validité, on pourra considérer la propriété ($\mathcal{H}$) suivante :

% " si on note $k$ la longueur de la liste {\tt res} alors $x\cdot 2^k\:+\:\sum_{i=0}^{k-1}${\tt res[i]}$\cdot 2^i\:=\:n$ "
% et montrer que c'est un invariant de boucle.
	
% \exercice{nom={Longueur de l'écriture d'un entier en base 2}}
% On considère le script suivant :
% \begin{pythoncode}
% def Longueur(n):               
%     u, p = n, 0                          
%     while u != 0:                        
%         ...                 
%         ...                    
%     return p
% \end{pythoncode}
% Compléter ce script afin que son appel renvoie le nombre de chiffres de l'écriture de l'entier $n$ en base 2. Prouver la validité et la terminaison de cette fonction. 

% {\it Indication} : On pourra montrer que la propriété $({\mathcal H})\:\:[\, 2^pu\leqslant n < 2^p(u+1)\:]$ est un invariant de boucle.
	

% \exercice{nom={L'exponentiation rapide}}
% Dans cet exercice, on se donne un entier $n\in\N$ et un réel $x$ et on cherche à calculer $x^n$ de manière efficace.
% 	\begin{enumerate}
% 	\item {\it Un algorithme naïf et pas bien méchant}
	
% 	On considère la fonction suivante :
% \begin{pythoncode}
% def expo_naif(x,n):
%     res = 1
%     for i in range(n):
%         res = res * x
%     return res
% \end{pythoncode}
% 	Combien cette fonction utilise-t-elle de multiplications pour calculer $x^n$ ? 
	
% 	Proposer un invariant de boucle et prouver la validité de cet algorithme
% 	\item {\it L'exponentiation rapide}
	
% 	On note $n\:=\:\underline{a_{p-1} \dots a_1 a_0}_2$ l'écriture de l'entier $n$ en base 2 de sorte que $n\:=\:\sum_{k=0}^{p-1}a_k 2^k$.
% 	\begin{enumerate}
% 	\item Expliquer comment calculer $x^{2^k}$ sans utiliser plus de $k$ multiplications.
% 	\item Pour $i\in[\![1,p-1]\!]$, exprimer $x^{\underline{a_{p-1} \dots a_{p-i-1}}_2}$ en fonction de $a_{p-i-1}$ et $x^{\underline{a_{p-1} \dots a_{p-i}}_2}$
% 	\item En déduire une fonction Python {\tt expo\_rapide(x,n)} dont l'appel renvoie $x^n$ et qui utilise les coefficients de l'écriture de $n$ en base 2. On pourra utiliser la fonction {\tt Vers\_base2}.
% 	\item Montrer que la complexité de cette fonction en termes du nombre de multiplications est logarithmique en $n$.
% 	\item Que fait la fonction suivante ? \footnote{Si vous séchez, vous pouvez demander à Perceval.}
% \begin{pythoncode}
% def sloubi(x, n):
%     y, m, z = x, n, 1
%     while m > 0:
%         m, r = m // 2, m % 2
%         if r == 1:
%             z = z * y
%         y = y * y
%     return z
% \end{pythoncode}
% 	Prouver sa correction (validité et terminaison).
	
% 	{\it Indication} : on pourra montrer que la propriété $({\mathcal H}) \:\:[\,x^n\:=\:z\cdot y^m\:]$ est un invariant de boucle. 
% 	\end{enumerate}
	
% 	\end{enumerate}

% \exercice{nom={À la recherche de nombres premiers : une méthode naïve}}
% On admettra les résultats suivants :
% \[
% \sum_{k=1}^n \frac{1}{k}\:=\: {\mathcal O}(\ln n) \quad \text{et} \quad \sum_{k=1}^n k\ln k\:=\: {\mathcal O}(n^2\ln n)\:\:.
% \]
% \begin{enumerate}
% \item En adaptant l'algorithme de la division euclidienne par différences successives, écrire une fonction Python {\tt reste(a,b)} qui prend en argument deux entiers $a\in\N$ et $b\in\N^*$ et renvoie le reste dans la division euclidienne de $a$ par $b$. Évaluer la complexité de cette fonction.
% \item Écrire une fonction Python {\tt nb\_div(n)} qui prend en argument un entier $n$ et dont l'appel renvoie le nombre de diviseurs de $n$. 
% Prouver la validité de cette fonction avec un invariant de boucle bien choisi. Montrer que la complexité de cette fonction est en $\mathcal{O}(n\ln n)$.
% \item
% \begin{enumerate}
% 		\item On suppose que la variable {\tt k} contient un entier naturel non nul.
		
% Compléter l'instruction suivante, pour qu'elle devienne un booléen de valeur {\tt True} si {\tt k} est un  nombre premier et un booléen de 
% valeur {\tt False} sinon: 
% 		\begin{center}
% 					{\tt nb\_div(k) == ...}
% 		\end{center}
% 		\item En déduire une fonction Python {\tt premiers\_naive(n)} prenant en argument un entier naturel non nul $n$ et dont l'appel affiche successivement tous les nombres premiers inférieurs ou égaux à $n$. 
% 		\item Évaluer la complexité de la fonction  {\tt premiers\_naive}.
% \end{enumerate}
% \end{enumerate}

% \exercice{nom={À la recherche des nombres premiers avec Ératosthène}}
% 	On reprend la méthode du crible d'Eratosthène vue dans les exercice 2 du chapitre précédent et on la détaille un peu pour en évaluer la complexité.
	
% 	Là, on aura besoin de savoir que :
% 	\[
% 	\sum_{k=1, \: k\,\text{premier}}^n \frac{1}{k}\:=\: {\mathcal O}(\ln \ln n) \:\:.
% 	\]
% 	\begin{enumerate}
% 	\item Écrire une fonction Python {\tt recherche(L,n,deb)} qui prend en argument une liste {\tt L} de $n$ cases et un indice de case {\tt deb} valide et renvoie le plus petit indice $k$ tel que $k \geqslant deb$  et $L[k]\neq 0$ et renvoie $n+1$ s'il n'existe pas de tel indice.
% 	\item  
% 		\begin{enumerate}
% 		\item  On suppose que $j \in \N$ et que $j^2 \leqslant n$. Écrire une fonction {\tt barre(L,n,j)} qui, lorsque {\tt L} est une liste 
% 		de $n$ cases, affecte à toutes les cases dont l'indice est supérieur ou égal à $j^2$ et 
% 		est divisible par $j$ la valeur $0$ et laisse inchangées les autres cases. 
% 		\item  Vérifier que cette fonction utilise $\lfloor \frac{n}{j} \rfloor -j+1$ affectations. 
% 		\end{enumerate}
% 	\item En déduire une fonction {\tt barre\_pas\_premier(n)} qui lorsque $n$ est un entier supérieur ou égal à $2$ renvoie la liste $P$ de $n$ cases, définies par $P[i]=i$ si $i$ est un nombre premier et $P[i]=0$ sinon. 
% 	\item En déduire une fonction {\tt premiers\_mieux(n)} qui affiche successivement tous les nombres premiers inférieurs ou égaux  $n$. 
% 	\item Donner une majoration du nombre d'opérations élémentaires (affectation ou test logique) utilisés lors de l'exécution de {\tt premiers\_mieux}(n).
% 	\end{enumerate}


%END_BOOK

\end{document}


\chapter{Correction}
\setcounter{numeroexercicecours}{1}
\input{cours-correction}
\section{Exercices}
\setcounter{numeroexercice}{1}
\documentclass{magnoliaold}

\magtex{tex_driver={pdftex}}
\magfiche{document_nom={Complexité},
          auteur_nom={François Fayard},
          auteur_mail={francois.fayard@auxlazaristeslasalle.fr}}
\magexos{exos_matiere={maths},
         exos_niveau={mpsi},
         exos_chapitre_numero={1},
         exos_theme={Complexité}}
\magmisenpage{}
\maglieudiff{}
\magprocess

\begin{document}

%BEGIN_BOOK

\magsection{Correction}
\magsubsection{Spécification d'une fonction}
\magsubsection{Correction partielle, correction totale}
\magsection{Algorithme itératif}
\magsubsection{Terminaison}
\magsubsection{Correction}

\exercice{nom={Fonction mystère}}
En déterminant un invariant, déterminer le rôle de la fonction suivante.
\begin{pythoncode}
def f(n):
    i = 0
    s = 0
    while s < n:
        s = s + 2 * i + 1
        i = i + 1
    return i
\end{pythoncode}

% \exercice{nom={Recherche dichotomique}}
% Voici une variante de la recherche dichotomique.
% \begin{pythoncodeline}
% def dichotomique(x, a):
%     lo = 0
%     hi = len(a)
%     while lo + 1 < hi:
%         mid = (lo + hi) / 2
%         if x < a[mid]:
%             hi = mid
%         else:
%             lo = mid
%     if a[lo] == x:
%         return lo
%     else:
%         return -1
% \end{pythoncodeline}

\magsubsection{Exemples fondamentaux}
\magsection{Algorithme récursif}
\magsubsection{Principe général}

\exercice{nom={Fonction mystère}}
On considère la fonction récursive suivante.
\begin{pythoncode}
def f(n):
    if n > 100:
        return n - 10
    return f(f(n + 11))
\end{pythoncode}
\begin{questions}
\question Prouver sa terminaison lorsque $n\in\N$. 
\question Déterminer ce qu'elle calcule.
\end{questions}

\exercice{nom={Fonction de Hofstadter}}
On considère la fonction $g$ de Hofstadter définie sur $\N$ de la manière suivante. 
\begin{pythoncode}
def g(n):
    if n == 0:
        return 0
    return n - g(g(n - 1))
\end{pythoncode}
\begin{questions}
\question Prouver sa terminaison lorsque $n\in\N$. 
\question Si vous avez l'inspiration, prouvez que
  \[\forall n\in\N\qsep g(n)=\ent{\frac{n+1}{\phi}}\]
	où $\phi\defeq\frac{1+\sqrt{5}}{2}$ est le nombre d'or.
\end{questions}


% \exercice{nom={À la recherche de nombres premiers : une méthode naïve}}
% On admettra les résultats suivants :
% \[
% \sum_{k=1}^n \frac{1}{k}\:=\: {\mathcal O}(\ln n) \quad \text{et} \quad \sum_{k=1}^n k\ln k\:=\: {\mathcal O}(n^2\ln n)\:\:.
% \]
% \begin{enumerate}
% \item En adaptant l'algorithme de la division euclidienne par différences successives, écrire une fonction Python {\tt reste(a,b)} qui prend en argument deux entiers $a\in\N$ et $b\in\N^*$ et renvoie le reste dans la division euclidienne de $a$ par $b$. Évaluer la complexité de cette fonction.
% \item Écrire une fonction Python {\tt nb\_div(n)} qui prend en argument un entier $n$ et dont l'appel renvoie le nombre de diviseurs de $n$. 
% Prouver la validité de cette fonction avec un invariant de boucle bien choisi. Montrer que la complexité de cette fonction est en $\mathcal{O}(n\ln n)$.
% \item
% \begin{enumerate}
% 		\item On suppose que la variable {\tt k} contient un entier naturel non nul.
		
% Compléter l'instruction suivante, pour qu'elle devienne un booléen de valeur {\tt True} si {\tt k} est un  nombre premier et un booléen de 
% valeur {\tt False} sinon: 
% 		\begin{center}
% 					{\tt nb\_div(k) == ...}
% 		\end{center}
% 		\item En déduire une fonction Python {\tt premiers\_naive(n)} prenant en argument un entier naturel non nul $n$ et dont l'appel affiche successivement tous les nombres premiers inférieurs ou égaux à $n$. 
% 		\item Évaluer la complexité de la fonction  {\tt premiers\_naive}.
% \end{enumerate}
% \end{enumerate}

% \exercice{nom={À la recherche des nombres premiers avec Ératosthène}}
% 	On reprend la méthode du crible d'Eratosthène vue dans les exercice 2 du chapitre précédent et on la détaille un peu pour en évaluer la complexité.
	
% 	Là, on aura besoin de savoir que :
% 	\[
% 	\sum_{k=1, \: k\,\text{premier}}^n \frac{1}{k}\:=\: {\mathcal O}(\ln \ln n) \:\:.
% 	\]
% 	\begin{enumerate}
% 	\item Écrire une fonction Python {\tt recherche(L,n,deb)} qui prend en argument une liste {\tt L} de $n$ cases et un indice de case {\tt deb} valide et renvoie le plus petit indice $k$ tel que $k \geqslant deb$  et $L[k]\neq 0$ et renvoie $n+1$ s'il n'existe pas de tel indice.
% 	\item  
% 		\begin{enumerate}
% 		\item  On suppose que $j \in \N$ et que $j^2 \leqslant n$. Écrire une fonction {\tt barre(L,n,j)} qui, lorsque {\tt L} est une liste 
% 		de $n$ cases, affecte à toutes les cases dont l'indice est supérieur ou égal à $j^2$ et 
% 		est divisible par $j$ la valeur $0$ et laisse inchangées les autres cases. 
% 		\item  Vérifier que cette fonction utilise $\lfloor \frac{n}{j} \rfloor -j+1$ affectations. 
% 		\end{enumerate}
% 	\item En déduire une fonction {\tt barre\_pas\_premier(n)} qui lorsque $n$ est un entier supérieur ou égal à $2$ renvoie la liste $P$ de $n$ cases, définies par $P[i]=i$ si $i$ est un nombre premier et $P[i]=0$ sinon. 
% 	\item En déduire une fonction {\tt premiers\_mieux(n)} qui affiche successivement tous les nombres premiers inférieurs ou égaux  $n$. 
% 	\item Donner une majoration du nombre d'opérations élémentaires (affectation ou test logique) utilisés lors de l'exécution de {\tt premiers\_mieux}(n).
% 	\end{enumerate}


%END_BOOK

\end{document}



\chapter{Graphe}
\setcounter{numeroexercicecours}{1}
\input{cours-graphe}

\part{TPs}

\chapter{Introduction, logo}
\documentclass{magnoliaold}

\magtex{tex_driver={pdftex},
        tex_packages={siunitx}}
\magfiche{document_nom={Cours sur les nombres complexes},
          auteur_nom={François Fayard},
          auteur_mail={fayard.prof@gmail.com}}
\magcours{cours_matiere={maths},
          cours_niveau={mpsi},
          cours_chapitre_numero={1},
          cours_chapitre={TP Python}}
\magmisenpage{}
\maglieudiff{}
\magprocess

\begin{document}

% https://pythonsandbox.com/turtle

%BEGIN_BOOK
\section{Découverte de Python}


\subsection{Les entiers}

\begin{questions}
\question Utiliser Python pour calculer $2022 + 2$, $2^{10}$, $3^2-2^3$.
\question Prédire avant de vérifier avec Python les résultats des opérations suivantes.
  \begin{center}
  \verb_10 // 2_, $\qquad$\verb_11 // 3_, $\qquad$\verb_-5 // 2_, $\qquad$\verb_11_ \% \verb_3_, $\qquad$\verb_-5_ \% \verb_2_
  \end{center}
\question Calculer $2^{\p{3^2}}$, puis $\p{2^3}^2$.
\question Afficher les nombres de Mersenne premiers suivants.
  \[2^3-1,\quad 2^7-1,\quad 2^{31}-1,\quad 2^{127}-1,\quad 2^{8191}-1,\quad 2^{131071}-1.\]
	% ,\quad 2^{524287}-1, \quad 2^{2147483647}-1\]
\end{questions}
%

\subsection{Les flottants}

\begin{questions}
\question Prédire puis vérifier le résultat des calculs suivants.
  \begin{center}
  \verb_0.1 + 0.2_, $\qquad$\verb_1.0 / 3.0_, $\qquad$\verb_0.5**3_.
  \end{center}
  Expliquer le premier résultat.
\question Calculer
  \begin{center}
  \verb_2 * 3.14159_, $\qquad$\verb_1 / 3_.
  \end{center}
\question En utilisant le fait que la vitesse de la lumière est de $c \approx 3.0\times 10^{8}\ {\rm m}.{\rm s}^{-1}$ et que la distance $d$ entre le soleil et la terre est de $d \approx 1.5\times 10^{11}\ {\rm m}$, déterminer le temps qu'il s'écoule avant que la lumière émise par le soleil arrive à la terre.
\question La distance moyenne entre 2 molécules d'eau liquide est de $3.4\times 10^{-10}\ {\rm m}$. Sachant qu'une mole contient de l'ordre de $6.0\times 10^{23}$ particules, donner un ordre de grandeur du nombre de moles dans un litre d'eau liquide.
\question À l'aide des fonctions du module \verb!math!, calculer
  \[\frac{1+\sqrt{5}}{2}, \quad \cos\p{\frac{\pi}{3}}, \quad \cos(\pi), \quad \sin(\pi), \quad\ln(2), \quad \log_{10}\p{1+\frac{1}{3}},\quad \e^{\pi\sqrt{163}}.\]
  Expliquer pourquoi l'un de ces résultats est surprenant.
\question Calculer puis expliquer le résultat des calculs suivants.
  \[(1 + 10^{-15}) - 1, \qquad (1 + 10^{-16}) - 1, \qquad (0.1 + 0.2) - 0.3, \qquad 0.1 + (0.2 - 0.3)\]
  On retiendra que les erreurs d'arrondis font que l'addition sur les nombres flottants n'est pas associative.
\question Prévoir, puis vérifier le type de
  \begin{center}
  \verb_5 // 2_, \quad \verb_5 / 2_, \quad \verb_(1 + math.sqrt(5)) / 2_, \quad \verb_4 / 2_
  \end{center}
\enonce Python permet de manipuler directement les nombres complexes. Le langage adopte les notations utilisées en physique~: le $\ii$ mathématique est noté $\jj$. Les nombres complexes en Python fonctionnent de la même manière que les nombres flottants. Le langage stocke la partie réelle et sa partie imaginaire comme des nombres flottants. Le nombre $2+\ii$ est entré sous la forme \verb_2+1j_.
\question Calculer $(2+\ii)(3+\ii)$.
\end{questions}

\subsection{Variables}

\begin{questions}
\question Quelle valeur contient la variable \verb_a_ après avoir écrit \verb_a = 1_ et exécuté 9 fois l'instruction \verb_a = 2 * a_~?
\question On considère les instructions suivantes
\begin{pythoncode}
In [1]: a = 1
In [2]: b = a
In [3]: c = b
In [4]: b = b + 1
\end{pythoncode}
Prévoir les valeurs de \verb_a_, \verb_b_ et \verb_c_ après l'exécution de ces instructions. Vérifier ensuite votre prédiction avec l'interpréteur Python.
\question À l'aide d'une troisième variable \verb_c_, échanger le contenu des variables \verb_a_ et \verb_b_.
\end{questions}

% \subsection{La méthode de Héron}

% La méthode de Héron est une méthode historique pour obtenir une valeur
% approchée de la racine carrée d'un nombre $a>0$. Pour cela, on définit la suite $(u_n)$ par
% \[u_0\defeq a, \quad\et\quad \forall n\in\N\qsep u_{n+1}\defeq\frac{u_n + \frac{a}{u_n}}{2}.\]
% Déterminer la plus petite valeur de $n$ pour laquelle la précision sur les nombres
% flottants ne permet plus de distinguer $u_{n+1}$ de $u_n$ lors du calcul de $\sqrt{2}$.

\section{Un peu de Logo}

Ce qu'on appelle aujourd'hui les graphismes \og tortue \fg trouvent leur origine dans le langage Logo, inventé dans
les années 1960 dans un but pédagogique. Ce langage comportait un certain nombre d'instructions qui permettaient de commander
un robot pouvant tracer des lignes sur une feuille de papier. La forme de ce robot, qui évoquait vaguement une tortue, a donné
son nom à cette manière de produire des dessins. Le module turtle de Python implémente une tortue virtuelle qui reproduit les
déplacements de son ancêtre dans une fenêtre de l'écran de votre ordinateur. C'est par son intermédiaire que nous allons
découvrir quelques concepts de base de la programmation. Dans le shell, tapez l'instruction suivante~:

\begin{pythoncode}
In [1]: import turtle as lg
In [2]: lg.reset()
\end{pythoncode}
\noindent
Une nouvelle fenêtre doit apparaitre à l'écran. Elle est peut-être cachée par Pyzo, donc redimensionnez votre espace de
travail afin de pouvoir visualiser en même temps l'éditeur, le shell et cette nouvelle fenêtre. Celle-ci représente
la feuille sur laquelle la tortue, représentée par défaut par la pointe d'une flèche, va réaliser ses dessins. Initialement
la tortue est située au point de coordonnées $(0, 0)$ situé au centre de la feuille et est orientée vers l'est, qui
correspond à un angle de $\ang{0}$. On rappelle les fonctions essentielles de la tortue~:

\begin{itemize}
\item \verb!lg.forward(c)! permet à la tortue d'avancer d'une distance de $c$.
\item \verb!lg.left(a)! permet à la tortue de tourner vers la gauche d'un angle de $a$ degrés.
\item \verb!lg.right(a)! permet à la tortue de tourner vers la droite d'un angle de $a$ degrés.
\item \verb!lg.reset()! permet de remettre à zéro le contenu de la fenêtre graphique.
\end{itemize}

\begin{questions}
\question Remettre à zéro le contenu de la fenêtre graphique, puis faire dessiner par la tortue un carré de côté 100 en tournant vers la gauche.
\enonce On souhaite désormais reproduire le dessin de la figure suivante.
\begin{center}
  \includegraphics[width=0.4\textwidth]{../../commun/images/python-tp-logo-1}\\
  \textsc{Figure 1}
  \end{center}
Il pourrait être utile de disposer d'une fonction dessinant un carré de taille $c$. Puisqu'il n'existe pas de
telle fonction, nous allons la définir. Afin de définir une fonction, il faut lui donner un nom, préciser la
liste de ses paramètres et enfin décrire les différentes instructions à réaliser. La syntaxe générale est
la suivante.
\begin{pythoncode}
def (*@\textcolor{purple}{nom\_fonction}@*)((*@\textcolor{purple}{arg1, ..., argn}@*)):
    (*@\textcolor{purple}{bloc....................}@*)
    (*@\textcolor{purple}{..........d'instructions}@*)
\end{pythoncode}
Attention de bien respecter la syntaxe de l'instruction \verb!def!. La ligne contenant cette instruction
doit obligatoirement se terminer par un double point, et le bloc d'instructions qui suit doit être indenté,
c'est-à-dire décalé vers la droite de 4 espaces. Heureusement, si vous n'avez pas fait d'erreur de syntaxe,
l'éditeur de code se chargera pour vous d'indenter automatiquement votre code. Par exemple, pour
définir la fonction qui nous intéresse et que nous allons nommer \verb!carre! et qui dessine un carré de côté
$c$ en tournant vers la gauche, on commencera par écrire
\begin{pythoncode}
def carre(c):
    (*@\textcolor{purple}{bloc....................}@*)
    (*@\textcolor{purple}{..........d'instructions}@*)
\end{pythoncode}
\question Achever la définition de la fonction \verb!carre(c)! en prenant soin à ce que la tortue regarde dans
  la même direction qu'au début une fois le tracé effectué. Puis, à l'aide de celle-ci, reproduire le dessin de la
  figure 1, où les carrés ont respectivement pour côté 50, 100, 150 et 200. On écrira pour cela une fonction \verb!figure1()!.
\enonce Nous allons maintenant chercher à reproduire le dessin de la figure suivante.
\begin{center}
  \includegraphics[width=0.3\textwidth]{../../commun/images/python-tp-logo-2}\\
  \textsc{Figure 2}
  \end{center}
  À l'aide de la fonction \verb!carre!, le script n'est pas difficile à réaliser, mais peut s'avérer
  fastidieux à écrire. Pour en faciliter l'écriture, nous allons introduire la notion de boucle. Si
  $a$ et $b$ sont deux entiers, le script
\begin{pythoncode}
for k in range(a, b):
    (*@\textcolor{purple}{bloc....................}@*)
    (*@\textcolor{purple}{..........d'instructions}@*)
\end{pythoncode}
va exécuter le bloc d'instruction en utilisant successivement $a, a+1, \ldots, b - 1$ pour valeur de $k$.
\question À l'aide d'une boucle, écrire une fonction
  \verb!figure2(c)! qui reproduit le dessin de la figure 2 où les carrés ont
  pour côtés respectifs $c, 2c, 3c, \ldots, 10c$.
\enonce Effacer l'écran, puis augmenter la vitesse de la tortue jusqu'à sa vitesse maximale, car le dessin
  qui va suivre va être un peu long à tracer. Pour cela, on utilisera la fonction \verb!lg.speed(s)! avec
  pour valeur de $s$ un entier compris entre 1 (lent) et 10 (rapide); la valeur 0 permet aussi d'obtenir la vitesse la plus rapide. Attention, en
  plus d'effacer l'écran, la commande \verb!lg.reset()! rétablit la vitesse d'origine.
\question Écrire une fonction \verb!tourne(n)! qui, à l'aide d'une boucle, fait avancer la tortue de 1, puis de $2, 3, 4, \ldots, n$ pas en
  tournant d'un angle de \ang{91} entre chaque déplacement. Une fois
  la vitesse réglée au maximum, on pourra faire un essai avec
  \verb!tourne(500)!.
\question Tracer un triangle équilatéral, un carré, un pentagone ou un hexagone régulier ne sont
  pas des tâches fondamentalement différentes. Définir une fonction \verb!polygone(n, c)! qui
  trace un polygone régulier à $n$ côtés, chacun de longueur $c$, puis reproduire à l'aide de cette
  fonction la figure 3.
\begin{center}
\includegraphics[width=0.3\textwidth]{../../commun/images/python-tp-logo-3}\\
\textsc{Figure 3}
\end{center}
\enonce Effacer de nouveau l'écran à l'aide de la fonction \verb!lg.reset()!, puis réaliser la commande
  \verb!polygone(100, 5)!. La courbe obtenue vous étonne-t-elle~? Elle peut aussi être obtenue à l'aide
  d'une fonction prédéfinie du module \verb!turtle! qui se nomme \verb!circle!, mais dont les
  paramètres sont différents. Pour en connaitre les spécifications, vous pouvez chercher les mots-clés
  \verb!turtle python! avec votre moteur de recherche internet et vous diriger vers le site \verb!docs.python.org!.
  Vous constaterez que cette fonction possède trois arguments, dont deux optionnels. Le second, \verb!extent!,
  lorsqu'il est présent, permet de tracer un arc de cercle plutôt qu'un cercle complet. Le troisième,
  \verb!steps!, permet de tracer des polygones.
\question Écrire une fonction \verb!figure4()! qui, à l'aide de la seule fonction \verb!circle!, permet de reproduire le dessin de la figure 4. Une commande
  permet de tracer le cercle et c'est une utilisation astucieuse de la
  fonction \verb!circle! qui permet de tracer l'étoile.
  \emph{Si vous ne trouvez pas rapidement, passez à la question suivante.}
\begin{center}
\includegraphics[width=0.3\textwidth]{../../commun/images/python-tp-logo-4}\\
\textsc{Figure 4}
\end{center} 
\question Écrire enfin une fonction \verb!figure5()! permettant de reproduire
  le dessin de la figure 5. Ce dernier est constitué de 72 demi-cercles
  régulièrement espacés. Avant de tracer cette figure, il pourra être utile d'accélérer la vitesse de la tortue
  pour éviter une attente trop longue.
\begin{center}
\includegraphics[width=0.3\textwidth]{../../commun/images/python-tp-logo-5}\\
\textsc{Figure 5}
\end{center}
\enonce Le triangle de Sierpi\'nski est une fractale qui s'obtient à partir d'un triangle plein par une infinité de
  répétitions consistant à diviser par 2 la taille du triangle puis à les accoler en trois exemplaires par leurs
  sommets pour former un nouveau triangle. On trouvera figure 6 la représentation graphique des trois premières
  générations de cette transformation.
\begin{center}
\includegraphics[width=0.6\textwidth]{../../commun/images/python-tp-logo-6}\\
\textsc{Figure 6}
\end{center}
Pour colorier l'intérieur d'une courbe fermée, il faut faire précéder le début du tracé par la commande
\verb!lg.begin_fill()! et terminer celui-ci par \verb!lg.end_fill()!. Par exemple, le script suivant
\begin{pythoncode}
lg.begin_fill()
lg.circle(100)
lg.end_fill()
\end{pythoncode}
crée un disque plein.
\question Définir une fonction \verb!sierp1(c)! qui dessine la première étape de la construction du triangle
  de Sierpi\'nski, c'est-à-dire un triangle plein de côté $c$, en prenant soin à ce que la tortue regarde dans
  la même direction qu'au début une fois le tracé effectué.
\question À l'aide de la fonction précédente, définir une fonction \verb!sierp2(c)! qui dessine la deuxième
  étape de la construction du triangle de Sierpi\'nski.
\question À l'aide de la fonction précédente, définir une fonction \verb!sierp3(c)! qui dessine la troisième
  étape de la construction du triangle de Sierpi\'nski.
\enonce Les questions précédentes ont du vous convaincre qu'à l'exception de la première, la $n^e$ génération
  du triangle de Sierpi\'nski s'exprime toujours de la même manière en fonction de la précédente. Pour exploiter
  cette remarque, nous allons utiliser une particularité des fonctions Python~: elles ont la possibilité de
  faire appel à elle-mêmes dans leur définition. Autrement dit, si nous choisissons de passer en paramètre l'ordre
  $n$ de la génération que nous voulons tracer en définissant une fonction \verb!sierpinski(n, c1)!, il est possible,
  au sein de cette définition, de faire appel à la fonction \verb!sierpinski(n - 1, c2)!. Mais pour cela, il faut
  distinguer le cas $n=1$ qui se traite différemment. Pour effectuer cette distinction, on utilise l'instruction
  conditionnelle \verb!if! dont la syntaxe est la suivante~:
\begin{pythoncode}
if test:
    (*@\textcolor{purple}{bloc....................}@*)
    (*@\textcolor{purple}{........d'instructions 1}@*)
else:
    (*@\textcolor{purple}{bloc....................}@*)
    (*@\textcolor{purple}{........d'instructions 2}@*)
\end{pythoncode}
Notez bien l'indentation qui permet de délimiter chacun des deux blocs d'instructions. Le fonctionnement de
cette instruction est le suivant~: si le résultat du test est positif, le premier bloc d'instruction est
réalisé. Dans le cas contraire, c'est le second. Notez que l'instruction \verb!else! est optionnelle si aucune
instruction de doit être réalisée dans le cas d'un test négatif. 
\question Écrire la fonction \verb!sierpinski(n, c)! qui dessine la génération d'ordre $n$ du triangle de
  Sierpi\'nski avec une longueur de côté égale à $c$. Utiliser ensuite cette fonction pour effectuer le tracé
  avec $n=5$. N'oubliez pas d'accélérer au maximum la vitesse de la tortue car sinon, le tracé sera très long.
\enonce D'autres structures fractales peuvent être dessinées par la tortue, à commencer par l'arbre binaire
  représenté figure 7. Pour construire l'arbre d'ordre $n$ et de hauteur $h$, on procède de la manière
  suivante~:
  \begin{itemize}
  \item On trace le tronc de hauteur $h/3$ et d'épaisseur $n$.
  \item Puis on trace ses deux branches, qui sont des arbres d'ordre $n-1$ et de hauteur $2h/3$, inclinés
    d'un angle de \ang{30}.
  \end{itemize}
\begin{center}
\includegraphics[width=0.6\textwidth]{../../commun/images/python-tp-logo-7}\\
\textsc{Figure 7} -- L'arbre d'ordre $n=4$ et sa règle de construction
\end{center}
\question Définir une fonction \verb!bin_tree(n, h)! qui dessine l'arbre binaire d'ordre $n$ et de hauteur $h$,
  puis dessiner l'arbre d'ordre 8.
\end{questions}

% \subsection{Noeud papillon}

% Tracez en \textsc{logo} le noeud papillon suivant. La hauteur du noeud papillon est de
% 100 et sa largeur est de 200. On utilisera la fonction \verb_atan(theta: float) -> float_ du module
% \verb_math_ qui calcule la mesure $\theta\in\intero{-\pi/2}{\pi/2}$ en radian de
% l'unique angle dont la tangente vaut $x$.
% \begin{center}
% \includegraphics[width=0.25\textwidth]{../../Commun/Images/python-exos-val-1.pdf}
% \end{center}


% \subsection{Suites récurrentes}

% Étant donnés $\alpha,\beta\in\RP$, on définit les suites $(a_n)$ et $(b_n)$ par
% \[a_0\defeq\alpha\qsep b_0\defeq\beta\qsep\et\quad \cro{
% 	\forall n\in\N\qsep a_{n+1}\defeq\frac{a_n+b_n}{2}\quad\et\quad b_{n+1}\defeq\sqrt{a_n b_n}}.\]
% \begin{questions}
% \question On suppose que les variables \verb!a! et \verb!b! contiennent respectivement
%   les valeurs $a_n$ et $b_n$. Écrire un code Python tel qu'après
% 	l'execution de ce code, les variables \verb!a! et \verb!b! contiennent respectivement
%   les valeurs $a_{n+1}$ et $b_{n+1}$.
% \question On suppose que $\alpha=1.0$ et que $\beta=2.0$. Quelle est la plus petite valeur
%   de $n$ pour laquelle la précision du type \verb!float! ne permet plus de discerner $a_n$
% 	de $b_n$~?
% \end{questions}
% \begin{sol}
% On trouve $n=4$.
% \end{sol}


% \section{Fonction, flot d'execution}

% \subsection{Boucle for}

% \begin{questions}
% \question  Écrire une fonction \verb+somme_cube(n: int) -> int+ calculant la somme des cubes de entiers
%   allant de 0 à $n$.
% \question Écrire une fonction \verb+somme_3_5(n: int) -> int+ calculant la somme de tous les
%   entiers inférieurs ou égaux à $n$ qui sont multiples de 3 ou de 5.
% \question Écrire une fonction \verb_factorielle(n: int) -> int_ calculant $n!$.
% \question
%   \begin{questions}
% 	\question Écrire une fonction \verb_exponentielle(x: float, n: int) -> float_ calculant
%     \[\sum_{k=0}^n \frac{x^k}{k!}.\]
% 		Comparer \verb_exponentielle(x, n)_ et \verb_exp(x: float) -> float_ du module \verb_math_
% 		pour quelques valeurs de $x$ et de $n$.
% 	\question Essayer d'écrire cette fonction de manière efficace sans utiliser, ni la
% 	  fonction puissance, ni la fonction factorielle. 
% 	\end{questions}
% \question On définit la suite de \textsc{Fibonacci} par
%   \[F_0\defeq 0, \quad F_1\defeq 1, \et \forall n\in\N\qsep F_{n+2}\defeq F_{n+1}+F_n.\]
%   Écrire une fonction \verb_fibo(n: int) -> int_ calculant le $n$-ième terme de la suite de
% 	\textsc{Fibonacci}.

% \question Une tortue ivre avance de $n$ pas. Avant chaque pas, elle tourne d'un angle
%   choisi de manière aléaoire uniformément entre 0 et 360. Écrire une fonction
% 	\verb+tortue_ivre(pas: int, n: int) -> NoneType+ simulant une telle marche, le paramètre \verb_pas_
% 	désignant la longueur d'un pas de la tortue. On utilisera pour cela,
% 	après avoir importé le module \verb_random_, l'expression \verb_random.randrange(360)_
% 	qui renvoie un entier $\theta$ tel que $0\leq \theta< 360$.
% \begin{center}
% \includegraphics[width=0.1\textwidth]{../../Commun/Images/python-tps-valeur_flot-1.pdf}
% \end{center}
% % \question \textsc{John Von Neumann} a proposé le générateur (uniforme sur $\interf{0}{1}$) de nombres aléatoires suivant. On se donne une graine $x\in\interf{0}{1}$ et on définit la suite $(u_n)$ par
% %   \[u_0\defeq x \et \forall n\in\N\qsep u_{n+1}\defeq 4 u_n(1-u_n).\]
% % On définit ensuite la suite $(r_n)$ par
% % \[\forall n\in\N\qsep r_n\defeq\frac{\arccos(1-2u_n)}{\pi}.\]
% % Afin de vérifier la qualité de ce générateur $(r_n)$, écrire une fonction \verb_moyenne(x, n)_ calculant la moyenne des $n$ premiers nombres de cette suite ainsi que la fonction \verb_variance(x, n)_ calculant la variance de ces nombres.
% \question Écrire une fonction \verb+polygone(r: int, n: int) -> NoneType+ dessinant un polygone régulier 
%   à $n$ côtés inscrit dans un cercle de rayon $r$.
% \end{questions}



% \subsection{Constante d'\textsc{Euler}}

% On note pour tout $n \in \Ns$
% \[S_n\defeq \sum_{k=1}^{n} \frac{1}{k},\qquad
% 	u_n\defeq S_n-\ln(n) \et v_n\defeq u_n-\frac{1}{n}.\] 
% On admet que $(u_n)$ et $(v_n)$ tendent vers la même limite $\gamma$ appelée constante
% d'\textsc{Euler} et que l'on a
% \[\forall n \in\Ns \qsep v_n \leq \gamma \leq u_n.\]
% Écrire une fonction \verb_euler(eps: float) -> float_ qui calcule une approximation à $\epsilon$ près de $\gamma$.


% \subsection{Suite de \textsc{Conway}}

% Les premiers termes de la suite de \textsc{Conway} sont $1, 11, 21, 1211, 111221,\ldots$
% chaque terme étant obtenu en lisant à haute voix le terme précédent. C'est pourquoi
% \textsc{Conway} avait baptisé cette suite \emph{look and say}. Par exemple, le terme
% 1211 se lit \og un 1, un 2, deux 1\fg donc le terme suivant est 111221.
% \begin{questions}
% \question Écrire une fonction \verb!look_and_say(s: string)! prenant en paramètre une chaîne de
%   caractères représentant un entier et renvoyant la chaîne de caractère représentant
% 	l'entier suivant dans la suite de \textsc{Conway}.
% \question À l'aide de cette fonction, afficher les 20 premiers termes de la suite de
%   \textsc{Conway}.
% \question Il a été démontré que si on note $u_n$ le nombre de chiffres du $n$-ième
%   nombre de \textsc{Conway}, le rapport $u_{n+1}/u_n$ admet une limite finie $l$.
% 	Donnez une valeur approchée de $l$.
% \question Une autre propriété de cette limite est qu'elle ne dépend pas de la
%   valeur initiale (excepté 22). Le vérifier expérimentalement.
% \question Démontrer que dans a suite de \textsc{Conway}, ne peuvent apparaître que les chiffres 1, 2 et 3.
% \end{questions}

%END_BOOK
\end{document}

\chapter{Altimètre, Syracuse}
\documentclass{magnolia}

\magtex{tex_driver={pdftex}}
\magfiche{document_nom={Flot d'exécution},
          auteur_nom={François Fayard},
          auteur_mail={francois.fayard@auxlazaristeslasalle.fr}}
\magcours{cours_matiere={maths},
          cours_niveau={mpsi},
          cours_chapitre_numero={1},
          cours_chapitre={Flot d'exécution}}
\magmisenpage{}
\maglieudiff{}
\magprocess

\begin{document}

%BEGIN_BOOK

\section{Altimètre}
Un altimètre est un instrument de mesure permettant de déterminer la distance verticale entre un
point et une hauteur de référence. Lors d'une randonnée en montagne, Max étalonne son altimètre
à son point de départ, puis mesure à chaque heure l'altitude relative atteinte. À la fin de sa
randonnée, il obtient une liste d'entiers naturels qu'il range dans une liste Python 
\verb!alt = [0, 300, 500, 600, 1000, 800, 900, 500, 600, 200, 0]!. 

\begin{center}
\includegraphics[width=0.65\textwidth]{../../commun/images/python-tp-altimetre}\\
\textsc{Figure 1} -- La randonnée de Max a duré 10 heures, il a atteint une
altitude relative\\ de 300~m au bout d'une heure, de 500~m au bout de deux heures, etc.
\end{center}

Max souhaite maintenant calculer certaines valeurs relatives à son parcours.

\begin{questions}
\question Comment obtenir la durée de la randonnée~? 
\question Définir en Python une fonction \verb!altmax(alt: list[int]) -> int!, qui prend en argument
  une liste \verb!alt! et qui renvoie l'altitude relative maximale atteinte lors de sa randonnée. Par
  exemple, dans le cas de l'illustration numérique donnée ci-dessus, la fonction devra renvoyer la
  valeur 1000.
\enonce Max cherche maintenant à savoir à quel moment son ascension a été la plus rapide en
  calculant le dénivelé maximal réalisé en une heure. Dans toute la suite de l'exercice,
  le dénivelé dont nous parlons est le dénivelé algébrique, qui est positif lorsque Max
  monte et négatif lorsque Max descend.  Il cherche donc la plus grande différence
  entre deux emplacements consécutifs de la liste. Par exemple, pour la liste donnée en illustration,
  le dénivelé maximal est égal à 400 et a été réalisé entre la troisième et la quatrième heure.
\question Définir en Python une fonction baptisée \verb!deniv_max(alt: list[int]) -> int! prenant
  en argument une liste \verb!alt! et renvoyant le dénivelé maximal réalisé en une heure durant
  sa randonnée. 
\question Écrire une fonction \verb!heure_deniv_max(alt: list[int]) -> int! renvoyant l'heure
  à laquelle débute la réalisation de ce dénivelé. Pour la liste donnée en exemple, cette fonction
  devra donc renvoyer la valeur 3.
\question Définir une fonction \verb!deniv_positif_total(alt: list[int]) -> int! renvoyant la somme des
  dénivelés positifs réalisés durant cette randonnée. Pour l'exemple donné en illustration, cette
  fonction devra renvoyer la valeur 1200.
\enonce Enfin, on appelle \emph{sommet} tout point dont l'altitude relative est strictement supérieure à
  l'altitude qu'elle précède et à l'altitude qui lui succède dans la liste \verb!alt!. Dans
  l'exemple qui nous sert à illustrer cet exercice, la randonnée de Max présente trois
  sommets d'altitudes 1000, 900 et 600 atteints à la quatrième, sixième et huitième heure de
  marche.
\question Écrire une fonction \verb!sommets(alt: list[int]) -> NoneType! affichant les
  différentes heures et altitudes des sommets de la randonnée.
\end{questions}

\section{La conjecture de Syracuse}

On doit cette conjecture au mathématicien allemand Lothar Collatz qui, en 1937, proposa à la
communauté mathématique le problème suivant~:\\

\parbox[c]{0.92\linewidth}{\emph{On part d'un nombre entier strictement positif. S'il est pair,
on le divise par 2. S'il est impair, on le multiplie par 3 et on ajoute 1. On réitère ensuite
cette opération.}}\\

Par exemple, à partir de 14 on construit la suite de nombres~: 14, 7, 22, 11, 34, 17, 52, 26,
13, 40, 20, 10, 5, 16, 8, 4, 2, 1, 4, 2, 1, etc. Après que le nombre 1 ait été atteint, la suite
des valeurs 4, 2, 1 se répète indéfiniment en un cycle de longueur 3.

\begin{center}
\includegraphics[width=0.45\textwidth]{../../commun/images/python-tp-collatz}\\
\textsc{Figure 2} -- Le graphe de la suite de Collatz pour $c=14$
\end{center}

La conjecture de Syracuse est l'hypothèse mathématique selon laquelle n'importe quel entier
de départ conduit à la valeur 1 au bout d'un certain temps. Nous allons expérimenter cette
conjecture en programmant l'évolution de la suite $(u_n)$ définie par les relations
\[u_0\defeq c\qquad\text{et}\qquad \forall n\in\N\qsep
  u_{n+1}=\begin{cases}\frac{u_n}{2} & \text{si $u_n$ est pair,}\\ 3u_n+1 & \text{si $u_n$ est impair.}\end{cases}\]

\subsection{Temps de vol et altitude maximale}

\begin{questions}
\question Écrire une fonction \verb!f(u: int) -> int! prenant en entrée un entier $u$ et renvoyant
  $u/2$ si $u$ est pair et $3u+1$ si $u$ est impair.
\enonce Le \emph{temps de vol} d'un entier $c$ est le plus petit entier $n$ (en admettant qu'il existe) pour lequel
  $u_n=1$. Par exemple, le temps de vol pour $c=14$ est égal à 17.
\question Définir une fonction \verb!temps_de_vol(c: int) -> int! prenant un paramètre entier $c$ et renvoyant
  le plus petit entier $n$ pour lequel $u_n=1$.
\enonce De manière tout aussi imagée, on appelle \emph{altitude maximale} de $c$ la valeur maximale de la suite $(u_n)$.
  Par exemple, l'altitude maximale de $c=14$ est égale à 52.
\question Écrire une fonction \verb!altitude(c: int) -> int! qui calcule cette fois-ci l'altitude maximale pour un
  entier $c$ donné en paramètre.
\end{questions}

\subsection{Vérification expérimentale de la conjecture}

On désire désormais vérifier la validité de la conjecture pour toute valeur $c\leq 1\ 000\ 000$. Une première solution
consisterait à calculer le temps de vol pour toutes ces valeurs, mais ce calcul est long et il y a mieux à faire en
observant que si la conjecture a déjà été vérifiée pour toute valeur $c'<c$, il suffit qu'il existe un rang $n$ pour
lequel $u_n<c$ pour être certain que la conjecture sera aussi vérifiée au rang $c$. On appelle donc
\emph{temps d'arrêt} le premier entier $n$ (en admettant qu'il existe) pour lequel $u_n<c$.

\begin{questions}
\question Écrire une fonction \verb!temps_d_arret(c: int) -> int! prenant un paramètre entier $c$ et renvoyant le temps
  d'arrêt de la suite de Syracuse correspondante.
\enonce Nous souhaitons maintenant mesurer le temps nécessaire pour vérifier la conjecture jusqu'à un paramètre entier
  $m$. Pour cela, nous allons utiliser la fonction \verb!time! du module \verb!time! du même nom, sans argument, qui
  renvoie le temps en secondes depuis une date de référence (qui dépend du système).
\question À l'aide de cette fonction, écrire une fonction \verb!verification(m: int) -> float! qui prend en argument
  un entier $m$ et renvoie le temps nécessaire pour vérifier que toutes les valeurs de $c\in\intere{2}{m}$ ont bien
  un temps d'arrêt fini. Quelle durée d'exécution obtient-on pour $m=1\ 000\ 000$~?
\question Quel est le temps d'arrêt d'un entier pair~? d'un entier de la forme $c=4n+1$~? En déduire qu'on peut restreindre
  la recherche aux entiers de la forme $4n+3$ et modifier en conséquence la fonction précédente. Combien de temps
  gagne-t-on par rapport à la version précédente pour $m=1\ 000\ 000$~? Vérifier ensuite la conjecture pour
  $n=10\ 000\ 000$.
\end{questions}

\subsection{Records}

\begin{questions}
\question Déterminer l'altitude maximale que l'on peut atteindre lorsque $c\in\intere{1}{1\ 000\ 000}$, ainsi que la
  valeur minimale de $c$ permettant d'obtenir cette altitude.
\question Déterminer le temps d'arrêt maximal lorsque $c\in\intere{1}{1\ 000\ 000}$ ainsi que la valeur de $c$
  correspondante.
\enonce On appelle \emph{vol en altitude de durée record} un vol dont tous les temps d'arrêt de rangs inférieurs
  sont plus courts. Par exemple, le vol réalisé pour $c=7$ est un vol en altitude de durée record (égale à 11)
  car tous les vols débutant par $c=1, 2, 3, 4, 5, 6$ ont des temps d'arrêt de durées inférieures à 11.
\question Déterminez tous les vols en altitude de durée record pour $c\leq 1\ 000\ 000$.
\end{questions}

\subsection{Affichage du vol}

Pour obtenir des graphes analogues à celui de la figure 2, on utilise la fonction \verb!plot! qui appartient à
un module appelé \verb!matplotlib.pyplot! et dédié au tracé des graphes. Vous allez donc commencer par importer
celui-ci à l'aide de la commande

\begin{pythoncode}
import matplotlib.pyplot as plt
\end{pythoncode}

Désormais, toutes les fonctions de ce module vous sont accessibles à condition de les préfixer par \verb!plt!.
Nous n'aurons besoin aujourd'hui que de deux fonctions \verb!plt.plot! et \verb!plt.show!. Sous sa forme la plus
simple, la fonction \verb!plt.plot! n'exige qu'une liste en paramètre \verb!plt.plot([a0, a1, ..., an])! et
crée un graphe constitué d'une ligne brisée reliant les points de coordonnées $(k,a_k)$ pour $k\in\intere{0}{n}$.
En Python, une liste est encadrée par des crochets et ses éléments sont séparés par des virgules. Nous étudierons
les listes plus tard dans le cours, mais pour l'instant nous n'aurons que besoin du fait que \verb![]! représente la liste
vide et si \verb!lst! est une liste, on ajoute un élément à la fin de celle-ci à l'aide de la commande
\verb!lst.append(x)!. Une fois votre graphe créé par la fonction \verb!plt.plot!, il reste à le faire
apparaitre dans une fenêtre annexe à l'aide de l'instruction \verb!plt.show()!.

\begin{questions}
\question Définir une fonction \verb!graphique(c: int) -> NoneType! qui prend un entier $c$ en paramètre et
  qui construit le graphe de la suite $(u_n)$ durant son temps de vol.
\end{questions}

\subsection{Encore une optimisation}

Nous avons vu plus haut qu'il suffit de restreindre l'étude de la conjecture aux entiers de la forme $4n+3$,
soit à $25\%$ des entiers. On peut chercher à affiner cette démarche en s'intéressant aux entiers de la forme
$8n+k$ mais ceci ne conduit pas à une amélioration des performances puisqu'on ne peut que restreindre l'étude
aux entiers de la forme $8n+3$ et $8n+7$. En revanche, il est possible de restreindre l'étude aux entiers
de la forme $16n+7$, $16n+11$ et $16n+15$, soit $18.75\%$ des entiers.

\begin{questions}
\question Si on écrit les entiers sous la forme $65536n+k$ ($65536=2^{16}$), à combien de valeurs de $k$
peut-on restreindre l'étude~?
\end{questions}
%END_BOOK
\end{document}

\chapter{Cryptage de César}
\documentclass{magnolia}

\magtex{tex_driver={pdftex}}
\magfiche{document_nom={TP Cryptographie},
          auteur_nom={Victor Lambert},
          auteur_mail={victor.lambert@ens-cachan.org}}
\magcours{cours_matiere={informatique},
          cours_niveau={mpsi},
          cours_chapitre_numero={2},
          cours_chapitre={TP Python}}
\magmisenpage{}
\maglieudiff{}
\magprocess


\begin{document}

%BEGIN_BOOK
La cryptographie a pour but de transformer des messages afin qu'ils ne puissent être lus
que par des personnes de confiance. Ces dernières sont les seules à connaitre le
mécanisme permettant d'effectuer la transformation inverse afin de retrouver le message
d'origine.\\

Au premier siècle de notre ère est apparu un chiffrement par substitution, connu sous le
nom de code de César, car l'empereur en a été l'un des plus assidus utilisateurs.
Le chiffrement de César consiste à assigner à chaque lettre de l'alphabet une
autre lettre, résultant du décalage de l'alphabet d'un certain nombre de lettres. Par exemple, avec le décalage suivant
\begin{center}
\begin{tabular}{|*{26}{c|}}
\hline
a&b&c&d&e&f&g&h&i&j&k&l&m&n&o&p&q&r&s&t&u&v&w&x&y&z\\
\hline
f&g&h&i&j&k&l&m&n&o&p&q&r&s&t&u&v&w&x&y&z&a&b&c&d&e\\
\hline
\end{tabular}
\end{center}
le texte \verb_"vous suivez le cours de python"_ devient
\verb_"atzx xznaje qj htzwx ij udymts"_. Pour décoder le message, il suffit de connaitre
la clé, c'est-à-dire la lettre qui correspond à la lettre \verb_"a"_. Dans l'exemple
ci-dessus, la clé est donc \verb_"f"_.\\

Le but de ce TP est d'écrire un programme permettant de coder un message par cette
méthode, puis un autre permettant de décoder ce message. On verra aussi comment attaquer
cette méthode de cryptage, c'est-à-dire comment un pirate peut intercepter le message
et retrouver la clé du cryptage afin de le décoder. 

\section{Codage et décodage par la méthode de César}

\begin{questions}
\question Écrire une fonction \verb_ordre(c: str) -> int_ qui à une lettre de l'alphabet
  latin associe sa position dans l'alphabet. Par exemple, à la lettre \verb_"a"_,
	la fonction \verb_ordre_ associera l'entier 0 et à la lettre \verb_"z"_ elle associera
	l'entier 25. On utilisera pour cela la fonction \verb_ord_ de Python.
\question Écrire une fonction réciproque \verb_lettre(n: int) -> str_  qui à l'entier
  $n\in\intere{0}{25}$ associe la lettre associée. Par exemple \verb_lettre(2)_ doit renvoyer
	\verb_"c"_.
\question Écrire une fonction \verb!est_lettre_alphabet(c: str) -> bool! qui renvoie
  \verb_True_ si \verb_c_ est une lettre de l'alphabet latin et \verb_False_ sinon.
	Par exemple \verb!est_lettre_alphabet("a")! va renvoyer \verb_True_ et
	\verb!est_lettre_alphabet(".")! va renvoyer \verb_False_.
\question Écrire une fonction \verb_code(m: str, c: str) -> str_ qui au message \verb_m_
  et à la clé \verb_c_ associe le message obtenu en utilisant le codage de
	César. Par exemple, l'appel \verb_code("lazos rocks", "c")_ doit renvoyer
	\verb_"ncbqu tqemu"_.
\question Écrire la fonction \verb_decode(s: str, c: str) -> str_ qui effectue
  la transformation inverse. L'appel à la fonction
	\verb_decode("ncbqu tqemu", "c")_ doit renvoyer \verb_"lazos rocks"_.
\end{questions}

\section{Déterminer la clé}

On intercepte un message codé par le chiffrement de César dont on ignore la clé.
On veut déterminer une méthode automatique qui nous donnera la clé et le message original.
Pour cela, nous allons exploiter l'idée que, dans une langue donnée, la fréquence
d'apparition de chacune des lettres de l'alphabet n'est pas la même.\\

Au début du fichier \texttt{crypto.py}, il y a une liste \verb_g_ des fréquences
d'apparition des lettres de l'alphabet en français. Par exemple, la lettre \verb_"a"_
apparaît, dans un texte comportant  $100$ caractères alphabétiques, en moyenne $8.4$
fois. Pour casser le codage de César, nous allons comparer cette liste \verb_g_
à la liste \verb_f_ des fréquences obtenues dans le texte encrypté.

\begin{questions}
\question Écrire une fonction \verb_frequence(m: str) -> list[float]_, qui a pour argument
  un message \verb_m_, et qui renvoie une liste de $26$ éléments contenant la fréquence 
	d'apparition de chacune des lettres de l'alphabet dans le message \verb_m_. Pour cela
	on commencera par initialiser une liste de 26 zéros à l'aide de l'instruction
	\verb_f = 26 * [0]_.
\enonce Pour trouver la clé, on va faire \og tourner \fg le tableau \verb_f_ des fréquences
  d'apparition de notre message pour le faire coïncider le mieux possible avec le tableau
	\verb_g_ des fréquences des lettres dans la langue française. 
\question Écrire une fonction \verb_distance(f: list[float], g: list[float], i: int) -> float_
  qui a pour argument deux listes de fréquences \verb_f_ et \verb_g_ ainsi qu'un décalage
	$i\in\intere{0}{25}$ et qui renvoie la distance
  \[d_i\defeq \sum_{j=0}^{25}\abs{g_j-f_{i+j\ ({\rm mod}\ 26)}}\]
\question Écrire une fonction \verb!indice_minimum_distance(m: str, g: list[float]) -> int!
  qui a pour argument un message \verb_m_ et qui renvoie l'entier  \verb_i0_ tel que
	\[d_{i_0}=\min\ensim{d_i}{i\in\intere{0}{25}}.\]
\question Écrire une fonction \verb!decrypte(m: str, g: list[float]) -> str! qui a pour
  argument un message \verb_m_ et qui renvoie le message en clair.
\question Tester la fonction précédente sur le message codé
  \verb_message_ présent dans le fichier distribué.
	%dans \texttt{messages.py}.
\end{questions}

%\section{Amélioration}
%
%Nous allons faire en sorte que la méthode de codage fonctionne sur des textes écrits avec des majuscules et des accents.
%\smallskip
%
%\emph{Remarques}: 
%\begin{itemize}
%\item[\textbullet] On supposera, dans la suite, que les majuscules ne sont pas accentuées.
%\item[\textbullet] On n'utilisera pas de fonction prédéfinie en python pour répondre aux question suivantes.
%\end{itemize}
%\smallskip
%
%\begin{enumerate}
%\item \'Ecrire une fonction \texttt{enleve\_majuscules} d'argument une chaîne \texttt{texte} qui renvoie la même chaîne en transformant les majuscules en minuscules.
%
%\item \'Ecrire une fonction \texttt{enleve\_accents} d'arguments une chaîne \texttt{texte}  qui renvoie la même chaîne en enlevant les accents et en remplaçant les ç par des c.
%
%\item Avec les fonctions suivantes, transformez \texttt{message4} pour ne plus avoir ni majuscules, ni accents. Proposez un codage du message obtenu avec la clé de votre choix puis effectuer un décodage automatique.
%\end{enumerate}


\section{Chiffre de Vigenère}

L'idée de Vigenère est d'utiliser un chiffre de César, mais où le
décalage utilisé change de lettre en lettre. Pour cela, on utilise une table composée de
26 alphabets, écrits dans l'ordre, mais décalés de ligne en ligne d'un caractère. On
écrit encore en haut un alphabet complet, pour la clé, et à gauche, verticalement, un
dernier alphabet, pour le texte à coder. 

\begin{center}\includegraphics[width=7cm]{../../Commun/Images/python-tps-vigenere.png}\end{center}

Pour coder un message, par exemple \verb_"cryptographie de vigenere"_, on choisit une clé
qui sera un mot de longueur arbitraire ; prenons \texttt{mathweb}. On écrit ensuite cette
clé sous le message à coder. Dans cette partie, on supposera que le message à coder ne comporte que
des lettres de l'alphabet et pas de point, de virgule, ni d'espace. On répète la clé
aussi souvent que nécessaire pour que sous chaque lettre du message à coder, on trouve
une lettre de la clé.
\begin{center}
\begin{tabular}{|*{25}{c|}}
\hline
c&r&y&p&t&o&g&r&a&p&h&i&e&d&e&v&i&g&e&n&e&r&e\\
\hline
m&a&t&h&w&e&b&m&a&t&h&w&e&b&m&a&t&h&w&e&b&m&a\\
\hline
\end{tabular}
\end{center}

\noindent
Pour coder, on regarde dans le tableau l'intersection de la ligne de la lettre à coder avec
la colonne de la lettre de la clé. Dans notre exemple, on commence par coder la lettre
\verb_"c"_ . La clé est donnée par la lettre  \verb_m_. On regarde dans le tableau 
l'intersection de la \og ligne \fg \verb_c_ et de la \og colonne\fg \verb_m_. Ainsi ce
\verb_"c"_ sera codé par \verb_"o"_. Ensuite, on code la lettre  \verb_"r"_, dont la clé est
\verb_a_. La lecture du tableau donne la lettre  \verb_"r"_ (ligne \verb_r_ et colonne
\verb_a_). Ainsi de suite. Notre message sera codé par \verb_"orrwpshdaioei eq vbnarfde"_. 

\begin{center}\includegraphics[width=7cm]{../../Commun/Images/python-tps-vigenere2.png}\end{center}

L'intérêt par rapport au codage de César est qu'une même lettre sera codée par
plusieurs lettres différentes ; par exemple ici \verb_"e"_ est codé par \verb_"i"_,
\verb_"q"_, \verb_"a"_, \verb_"f"_ et \verb_"e"_.

\begin{questions}
\question Écrire une fonction \verb!code_vigenere(m: str, cle: str) -> str! d'arguments
un message en clair \verb_m_ et une clé \verb_cle_. Cette fonction renverra le message
codé selon le code de Vigenère avec la clé \verb_cle_.
\question Écrire une fonction \verb!decode_vigenere(m: str, cle:str) -> str! d'arguments
un message codé \verb_m_ selon Vigenère avec la clé \verb_cle_ . Cette
fonction renverra le message décodé.
\question Écrire une fonction \verb!decrypte_vigenere(m: str, n: int, g: list[float]) -> str! d'arguments un message codé \verb_m_ selon Vigenère, la longueur de la clé
\verb_n_ et le tableau \verb_g_ des fréquences des lettres dans la langue française. 
\question Décoder \verb!message_vigenere! qui a été codé avec la méthode de
  Vigenère, sachant que la clé possède $11$ lettres.
\end{questions}

%END_BOOK
\end{document}


% \chapter{Levenshtein, Hamming}
% \documentclass{magnolia}

\magtex{tex_driver={pdftex},
        tex_packages={siunitx}}
\magfiche{document_nom={Levenshtein, Hamming},
          auteur_nom={François Fayard},
          auteur_mail={francois.fayard@auxlazaristeslasalle.fr}}
\magcours{cours_matiere={maths},
          cours_niveau={mpsi},
          cours_chapitre_numero={1},
          cours_chapitre={Levenshtein, Hamming}}
\magmisenpage{}
\maglieudiff{}
\magprocess

\begin{document}
%BEGIN_BOOK

\section{Distance de Levenshtein}

Un séquence d'\textsc{ADN} est une succession de nucléotides pouvant être de 4 types
distincts~: adénine (A), cytosine (C), guanine (G) et thymine (T). Dans ce problème, une séquence d'\textsc{ADN}
sera donc représentée par une chaine de caractères composée uniquement des lettres A, C, G et T;
on notera $\mathcal{S}$ l'ensemble de ces séquences.  Étant donnée
une séquence $u\in\mathcal{S}$, on appelle \emph{opération élémentaire d'édition} l'une des opérations suivantes~:
\begin{itemize}
\item La suppression d'une lettre.
\item L'insertion d'une lettre.
\item La substitution d'une lettre par une autre lettre.
\end{itemize}
Ces opérations élémentaires permettent par exemple de transformer la séquence $u={\rm ATGC}$ en $v={\rm AAC}$ à
l'aide de 2 opérations élémentaires.
\[{\rm ATGC}
  \xrightarrow[\text{Suppression du T}]{}  {\rm AGC}
  \xrightarrow[\text{Substitution du G par A}]{} {\rm AAC}.\]
Étant données deux séquences $u$ et $v$, il est bien entendu toujours possible de transformer $u$ et $v$
à l'aide d'une succession d'opérations élémentaires d'édition. On appelle \emph{distance de Levenshtein} entre
$u$ et $v$ et on note ${\rm d}(u, v)$, le nombre minimal d'opérations élémentaires permettant de passer
de $u$ à $v$.

\begin{questions}
\question Montrer que ${\rm d}$ est une distance, c'est-à-dire que
  \begin{eqnarray*}
  \forall u,v\in\mathcal{S},& & {\rm d}(u, v) = 0 \ssi u=v\\
  \forall u,v\in\mathcal{S},& & {\rm d}(u, v) = {\rm d}(v, u)\\
  \forall u,v,w\in\mathcal{S},& & {\rm d}(u, w) \leq {\rm d}(u, v) + {\rm d}(v, w).
  \end{eqnarray*}
\question Soit $u$ et $v$ des séquences de longueurs respectives $n$ et $m$. Montrer que
  \[\abs{m-n}\leq {\rm d}(u, v)\leq m + n.\]
\enonce Pour tout $i\in\intere{0}{n}$, on note $u[0:i]$ la sous-séquence de $u$ formée des $i$ premières
  lettres de $u$. On définit de même $v[0:j]$ pour tout $j\in\intere{0}{m}$ et on pose ${\rm D}_{i,j}\defeq
  {\rm d}(u[0:i], v[0:j])$. Le but de ce problème est de calculer efficacement
  ${\rm d}(u,v)={\rm D}_{n, m}$.
\question Combien valent respectivement ${\rm D}_{i, 0}$ et ${\rm D}_{0, j}$~?
\question Pour $i\in\intere{1}{n}$ et $j\in\intere{1}{m}$, on note $\alpha_{i,j}$ le nombre égal à 2 si
  $u_i\neq v_j$ et à 0 dans le cas contraire ($u_i$ et $u_j$ représentant respectivement la lettre d'indice
  $i$ de $u$ et la lettre d'indice $j$ de $v$). Montrer que
  \[{\rm D}_{i, j}=\min({\rm D}_{i-1, j-1}+\alpha_{i,j}, {\rm D}_{i-1, j}+1, {\rm D}_{i, j-1}+1).\]
\question En déduire une fonction \verb!dist_dynamique(u: str, v: str) -> list[list[int]]! prenant
  en entrée deux séquences $u$ et $v$ et renvoyant la matrice des $({\rm D}_{i,j})_{0\leq i\leq n, 0\leq j\leq m}$.
  On commencera par remplir la première colonne et la première ligne de la matrice grâce à la question 3, puis
  on remplira la matrice ligne par ligne grâce à la relation démontrée en question 4.
\question En déduire une fonction \verb!levenshtein(u: str, v: str) -> int! calculant la distance de Levenshtein
  entre $u$ et $v$.
\question La fonction précédente nécessite de stocker $(n+1)(m+1)$ entiers en mémoire. Améliorer votre
 programme pour qu'il soit nécessaire de stocker seulement $\min(n, m)+1$ entiers dans un tableau.
\end{questions}


\section{Nombres de Hamming}
Les nombres de Hamming sont les entiers naturels non nuls dont les seuls facteurs premiers éventuels
sont 2, 3, et 5~:
\[1, 2, 3, 4, 5, 6, 8, 9, 10, 12, 15, 16, 18, 20, 24, 25, 27, 30, \ldots\]
Le but de cet exercice est de les générer de manière croissante. Évidemment, on peut parcourir un à un
tous les entiers en testant à chaque fois si ceux-ci sont des entiers de Hamming, mais cette démarche
montre très vite des limites~: songez que le $1\ 999^{\rm e}$ entier de Hamming est égal à $8\ 100\ 000\ 000$
et le $2\ 000^{\rm e}$ à $8\ 153\ 726\ 976$; il faudrait tester plus de 53 millions de nombres avant
d'augmenter notre liste d'un élément. On adopte donc la démarche suivante~: on utilise trois files
$f_2$, $f_3$, $f_5$ contenant initialement le nombre 1, et on suit la démarche suivante~:
\begin{itemize}
\item On détermine le plus petit des trois têtes de file, que l'on note $k$ et que l'on ajoute à notre liste.
\item On retire cet élément des files où il se trouve.
\item On insère en queues des files $f_2$, $f_3$ et $f_5$ les entiers $2k$, $3k$ et $5k$.
\end{itemize}
Vous l'avez compris~: cette démarche utilise le fait que tout nombre de Hamming différent de 1 est le produit
par 2, 3 ou 5 d'un nombre plus petit. On utilisera pour cela un type \verb!queue[int]! et ses fonctions associées
\begin{pythoncode}
queue_new() -> queue[int]
queue_is_empty(q: queue[int]) -> bool
queue_push(q: queue[int], x: int) -> NoneType
queue_pop(q: queue[int]) -> int
queue_peek(q: queue[int]) -> int
\end{pythoncode}

\begin{questions}
\question Rédiger une fonction \verb!hamming(n: int) -> list[int]! générant les $n$ premiers nombres de Hamming.
\question L'inconvénient de cette démarche est que le même nombre peut se retrouver dans plusieurs des 3 files.
  Modifier votre fonction pour que ce ne soit plus le cas.
\end{questions}

%END_BOOK
\end{document}

\chapter{Sudoku}
\documentclass{magnolia}

\magtex{tex_driver={pdftex}}
\magfiche{document_nom={Fonctions et procédures},
          auteur_nom={François Fayard},
          auteur_mail={fayard.prof@gmail.com}}
\magcours{cours_matiere={maths},
          cours_niveau={mpsi},
          cours_chapitre_numero={5},
          cours_chapitre={TP Python~: Sudoku}}
\magmisenpage{misenpage_presentation={tikzvelvia},
              misenpage_format={a4},
              misenpage_nbcolonnes={1},
              misenpage_preuve={non},
              misenpage_sol={non}}
\maglieudiff{lieu_lycee={Aux Lazaristes},
             lieu_classe={Sup},
             lieu_annee={2020--2021}}
             
             
\usepackage{tikz}
\usepackage{tkz-tab}


\magprocess

\begin{document}


%BEGIN_BOOK

\tikzset{math3d/.style={x={(-0.353cm,-0.353cm)},z={(0cm,1cm)},y={(1cm,0cm)}}}

\tikzset{
xmin/.store in=\xmin, xmin/.default=-3, xmin=-3,
xmax/.store in=\xmax, xmax/.default=3, xmin=3,
ymin/.store in=\ymin, ymin/.default=-3, ymin=-3,
ymax/.store in=\ymax, ymax/.default=-3, ymax=-3,
}

\usetikzlibrary{matrix}

\newcommand{\grille}{\draw[help lines] (\xmin,\ymin) grid (\xmax,\ymax);}

\newcommand{\axes}{%
\draw[->] (\xmin,0)--(\xmax,0);
\draw[->] (0,\ymin)--(0,\ymax);
\draw[->][very thick] (0,0)--(1,0);
%\draw (1 , 0) node[below] {$1$};
%\draw (0.5 , 0) node[below] {$\vec{i}$};
\draw (0.5 , 0) node[below] {$\vec{\imath}$};
\draw[->] [very thick](0,0)--(0,1);
%\draw (0,0.5) node[left] {$\vec{j}$};
\draw (0 , 0.5) node[left] {$\vec{\jmath}$};
%\draw (0,1) node[left] {$1$};
%\draw (0,0) node[below left]{$O$};
}

\newcommand{\axesbis}{%
\draw[->] (\xmin,0)--(\xmax,0)node[below]{$x$};
\draw[->] (0,\ymin)--(0,\ymax)node[left]{$y$};
}
\newcommand{\aaa}{\hphantom{\hspace{0.5cm}}\vphantom{9}}

\section{Règles du Sudoku}

Le sudoku est un jeu de logique de la famille des carrés latins. L'origine de son nom vient des deux mots japonais \og su \fg{} qui signifie chiffre et \og doku \fg{} qui signifie seul.\\

On dispose d'une grille de $9\times 9=81$ cases divisées en $3\times 3=9$ régions. Initialement, certaines cases de la grille sont préremplies par des chiffres compris entre
$1$ et $9$. Le but du jeu est de remplir les cases restantes avec des chiffres compris
entre $1$ et $9$ en respectant les contraintes suivantes:
\begin{itemize}
\item Chaque chiffre doit apparaître une seule fois dans chaque ligne.
\item Chaque chiffre doit apparaître une seule fois dans chaque colonne.
\item Chaque chiffre doit apparaître une seule fois dans chaque région.
\end{itemize}
% \bigskip
% La grille que l'on va compléter est la suivante~:

% \begin{center}
% \begin{tikzpicture}[xmin=-12,xmax=12,ymin=-12,ymax=12,scale=0.5]
% %\tkzInit[xmin=\xmin, xmax=\xmax, ymin=\ymin, ymax=\ymax]
% %\tkzGrid
% %\tkzAxeXY
% \matrix(M)[matrix of math nodes, nodes = {minimum size = 0.8cm}]at(0,0)
% {\aaa&8&\aaa&\aaa&4&1&\aaa&\aaa&9\\
% 1&6&\aaa&\aaa&\aaa&5&\aaa&\aaa&\aaa\\
% 4&\aaa&\aaa&6&\aaa&\aaa&\aaa&\aaa&\aaa\\
% \aaa&1&8&5&\aaa&\aaa&6&\aaa&\aaa\\
% \aaa&\aaa&2&\aaa&\aaa&\aaa&7&\aaa&\aaa\\
% \aaa&\aaa&5&\aaa&\aaa&9&3&8&\aaa\\
% \aaa&\aaa&\aaa&\aaa&\aaa&7&\aaa&\aaa&3\\
% \aaa&\aaa&\aaa&9&\aaa&\aaa&\aaa&7&8\\
% 9&\aaa&\aaa&1&8&\aaa&\aaa&2&\aaa\\
% };
% \foreach \j in {1,2,...,9}
% {\draw(M-1-\j.north west)--(M-9-\j.south west);}
% \foreach \i in {1,2,...,9}
% {\draw(M-\i-1.south west)--(M-\i-9.south east);}
% \draw(M-1-1.north west)--(M-1-9.north east);
% \draw[very thick](M-1-1.north west)--(M-9-1.south west);
% \draw[very thick](M-1-1.north west)--(M-1-9.north east);
% \foreach \j in{3,6,9}
% {\draw[very thick](M-1-\j.north east)--(M-9-\j.south east);}
% \foreach \i in{3,6,9}
% {\draw[very thick](M-\i-1.south west)--(M-\i-9.south east);}
% %\draw[->,>=latex](-9.5,9.4)--(-9.5,-9.4)node[below]{$i$};
% %\draw(-10,8)node{$0$};
% %\draw(-10,6)node{$1$};
% %\draw(-10,4)node{$2$};
% %\draw[->,>=latex](-11,9.4)--(-11,-9.4)node[below]{$a$};
% %\draw(-12,6)node{\Large{$0$}};
% %\draw(-12,0)node{\Large{$1$}};
% %\draw(-12,-6)node{\Large{$2$}};
% %\draw[->,>=latex](-9.4,9.6)--(9,9.6)node[right]{$j$};
% %\draw(-8,10)node{$0$};
% %\draw(-6,10)node{$1$};
% %\draw(-4,10)node{$2$};
% %\draw[->,>=latex](-9.4,11)--(9.4,11)node[right]{$b$};
% %\draw(-6,12)node{\Large{$0$}};
% %\draw(0,12)node{\Large{$1$}};
% %\draw(6,12)node{\Large{$2$}};
% \end{tikzpicture}
% \end{center}

% \section{Modélisation}
\bigskip

La grille de sudoku est modélisée par une matrice carrée de 9 par 9,
à l'aide d'une liste de listes. Pour repérer une case dans la grille, on utilise ses coordonnées $(i,j)$ comme dans le schéma ci-dessous où $0\leq i,j < 9$. Une région est, quant à elle, définie par ses coordonnées $(a,b)$ où $0\leq a,b<3$.

\begin{center}
\begin{tikzpicture}[xmin=-12,xmax=12,ymin=-12,ymax=12,scale=0.4]
%\tkzInit[xmin=\xmin, xmax=\xmax, ymin=\ymin, ymax=\ymax]
%\tkzGrid
%\tkzAxeXY
\matrix(M)[matrix of math nodes, nodes = {minimum size = 0.8cm}]at(0,0)
{\aaa&8&\aaa&\aaa&4&1&\aaa&\aaa&9\\
1&6&\aaa&\aaa&\aaa&5&\aaa&\aaa&\aaa\\
4&\aaa&\aaa&6&\aaa&\aaa&\aaa&\aaa&\aaa\\
\aaa&1&8&5&\aaa&\aaa&6&\aaa&\aaa\\
\aaa&\aaa&2&\aaa&\aaa&\aaa&7&\aaa&\aaa\\
\aaa&\aaa&5&\aaa&\aaa&9&3&8&\aaa\\
\aaa&\aaa&\aaa&\aaa&\aaa&7&\aaa&\aaa&3\\
\aaa&\aaa&\aaa&9&\aaa&\aaa&\aaa&7&8\\
9&\aaa&\aaa&1&8&\aaa&\aaa&2&\aaa\\
};
\foreach \j in {1,2,...,9}
{\draw(M-1-\j.north west)--(M-9-\j.south west);}
\foreach \i in {1,2,...,9}
{\draw(M-\i-1.south west)--(M-\i-9.south east);}
\draw(M-1-1.north west)--(M-1-9.north east);
\draw[very thick](M-1-1.north west)--(M-9-1.south west);
\draw[very thick](M-1-1.north west)--(M-1-9.north east);
\foreach \j in{3,6,9}
{\draw[very thick](M-1-\j.north east)--(M-9-\j.south east);}
\foreach \i in{3,6,9}
{\draw[very thick](M-\i-1.south west)--(M-\i-9.south east);}
\draw[->,>=latex](-9.5,9.4)--(-9.5,-9.4)node[below]{$i$};
\draw(-10,8)node{$0$};
\draw(-10,6)node{$1$};
\draw(-10,4)node{$2$};
\draw[->,>=latex](-11,9.4)--(-11,-9.4)node[below]{$a$};
\draw(-12,6)node{\Large{$0$}};
\draw(-12,0)node{\Large{$1$}};
\draw(-12,-6)node{\Large{$2$}};
\draw[->,>=latex](-9.4,9.6)--(9,9.6)node[right]{$j$};
\draw(-8,10)node{$0$};
\draw(-6,10)node{$1$};
\draw(-4,10)node{$2$};
\draw[->,>=latex](-9.4,11)--(9.4,11)node[right]{$b$};
\draw(-6,12)node{\Large{$0$}};
\draw(0,12)node{\Large{$1$}};
\draw(6,12)node{\Large{$2$}};
\end{tikzpicture}
\end{center}

\begin{enumerate}
\item Récupérez le fichier \texttt{grille.py}. Dans ce fichier, vous trouverez la matrice \texttt{g}, nous donnant un exemple de grille à traiter. Les cases vides sont symbolisées par des $0$. Copiez cette matrice dans votre script.
\item Quelle doit être la valeur de \texttt{g[3][2]}? Vérifiez sur votre machine.
\item Écrire une fonction \verb!cases_vides(g: list[list[int]]) -> list[tuple[int, int]]!  prenant pour argument une grille \verb!g! et renvoyant la liste des coordonnées $(i,j)$
des cases vides. Dans cet algorithme, le parcours de la grille se fera ligne par ligne.
\item Écrire une fonction \verb!compat_ligne(g: list[list[int]], i: int, c: int) -> bool!
   prenant pour argument une grille $g$, une ligne $i$ et un chiffre
   $c$ compris entre $1$ et $9$, et renvoyant \verb!True! s'il est possible
   d'ajouter le chiffre $c$ sur une des cases vides de la ligne $i$, tout en satisfaisant
   les contraintes de ligne du plateau. Autrement dit, cette fonction doit renvoyer
   \verb!True! si $c$ est différent de toutes les autres chiffres présents sur la ligne
   $i$, et \verb!False! sinon.
\item Écrire une fonction analogue
  \verb!compat_colonne(g: list[list[int]], j: int, c: int) -> bool! pour les colonnes,
  ainsi qu'une fonction
  \verb!compat_region(g: list[list[int]], a: int, b: int, c: int) -> bool! pour les
  régions.
\item En utilisant la division entière en Python, écrire une fonction
  \verb!region(i: int, j: int) -> tuple[int, int]! prenant pour arguments les
  coordonnées $(i,j)$ d'une case et renvoyant les coordonnées $(a,b)$ de la région
  dans laquelle elle se trouve.
\item Déduire des questions précédentes une fonction
  \begin{center}
    \verb!compat(g: list[list[int]], i: int, j: int, c: int) -> bool!
  \end{center}
  prenant pour
  arguments une grille $g$, les coordonnées $(i,j)$ d'une case vide
  de la grille et un chiffre $c$ entre 1 et 9 et renvoyant \verb!True! s'il est
  possible d'ajouter le chiffre $c$ sur la case vide  de coordonnées $(i,j)$ tout
  en respectant les contraintes du Sudoku portant sur les lignes, les colonnes et
  les régions.
\end{enumerate}


\section{Le backtracking}

Il existe de nombreuses méthodes de résolution du Sudoku. Les méthodes simulant le raisonnement humain sont efficaces mais difficiles à programmer. La méthode du \emph{retour arrière systématique} ou \emph{backtracking} est préférable~: elle teste
toutes les possibilités de remplissage. Étant donnée la puissance des ordinateurs actuels, une grille solution est généralement  donnée en quelques millisecondes. Détaillons cette
méthode pour le jeu du Sudoku.\\

On commence par déterminer la liste \verb!cv! des cases initialement vides de la grille.
Ensuite, on teste
la possibilitée de placer un 1 dans la première case vide. Si cette valeur ne rentre
pas en conflit avec les chiffres déjà présents, on passe à la case vide suivante.
Sinon, on teste si 2 convient, puis 3, etc. Si aucun chiffre entre 1 et 9 ne convient
pour remplir une case, nous sommes face à une incompatibilité et il est alors
nécessaire de revenir en arrière dans la liste de nos cases vides. On tente alors de
la remplir avec une nouvelle valeur, plus grande que celle tentée précédemment.\\

Concrètement, supposons qu'on a déjà rempli les $k$ premières cases vides de la liste
\verb!cv!. La prochaine case vide à traiter dont les coordonnées sont données par
\verb!cv[k]! contient une valeur $c$.
\begin{itemize}
\item Si $c=0$, c'est que nous venons juste de tester une nouvelle valeur pour la
  case vide d'indice $k-1$.
\item Si $c\in\intere{1}{9}$, c'est que nous venons de réaliser que cette
  valeur a conduit à une incompatibilité en tentant de remplir les cases vides
  suivantes. 
\end{itemize}
\medskip

Dans les deux cas, nous allons tenter de remplir cette case avec un chiffre
strictement supérieur à $c$~: on commence par $c+1$, puis $c+2$, jusqu'à 9.
\begin{itemize}
\item Si on trouve un chiffre compatible avec le reste de la grille, on place
  ce chiffre dans la grille, puis on passe à la case vide d'indice $k+1$.
\item Si aucun chiffre n'est compatible, c'est que nous sommes face à une incompatibilité
  et qu'il faut revenir en arrière pour tenter un nouveau chiffre. On place donc
  le chiffre 0 dans la case pour signifier qu'elle est de nouveau vide et on passe à la case
  initialement vide d'indice $k-1$.
\end{itemize}
\medskip

Si on note $n$ la longueur de la liste \verb!cv!, l'algorithme peut se terminer de deux
manières distinctes~:
\begin{itemize}
\item Si $k=n$, c'est que la grille a été entièrement remplie et donc qu'une solution a été trouvée.
\item Si $k=-1$, c'est qu'aucune solution n'a été trouvée. L'algorithme étant exhaustif,
  cela prouve qu'il n'est pas possible de compléter la grille.
\end{itemize}


\begin{enumerate}
\setcounter{enumi}{8}
\item Écrire une fonction
  \begin{center}
  \verb!prochain(g: list[list[int]], cv: list[tuple[int, int]], k: int) -> int!
  \end{center}
  qui prend une telle grille où les $k$ premières cases vides de la grille $g$ ont
  été remplies, où \verb!cv! contient la liste des coordonnées des cases
  initialement vides et qui traite comme l'on vient de voir la case
  de coordonnées \verb!cv[k]!. Cette fonction devra renvoyer l'index de la prochaine
  case à traiter, qui sera soit $k+1$ si on a trouvé un chiffre compatible, soit
  $k-1$ si une incompatibilité a été trouvée.
\item Écrire la fonction \verb!solution(g: list[list[int]]) -> bool!
  prenant en entrée une grille de Sudoku $g$ et renvoyant \verb!True! s'il est possible
  de la résoudre (et en transformant au passage la grille $g$ en une solution)
  et renvoyant \verb!False! dans le cas contraire.
\item A l'aide de la macro \verb!%timeit! de IPython, déterminer le temps de
  nécessaire à la résolution de la grille proposée en exemple.
\end{enumerate}


% \section{Partie facultative}

% \begin{enumerate}
% \item \'Ecrire une fonction \texttt{grille} d'argument une matrice \texttt{M} qui renvoie une chaîne de caractères représentant la grille de sudoku associée à \texttt{M}.

% Par exemple, en écrivant \texttt{print(grille(M))} dans le script avec la matrice \texttt{M} initiale (non remplie), on voit s'afficher après exécution:
% \begin{center}
% \includegraphics[width=7cm]{../../Commun/Images/python-tps-sudoku.png}
% \end{center}
% \item \'Ecrire une fonction \texttt{est\_un\_sudoku} d'arguments une matrice \texttt{M} (représentant une grille de sudoku remplie). Cette fonction renverra un booléen permettant de savoir si \texttt{M} remplit bien les conditions pour être un sudoku.

% \emph{Indication}: On pourra utiliser des fonctions intermédiaires.

% \end{enumerate}


%END_BOOK
 
%\end{enumerate}
\end{document}

\chapter{Récursif, ligne d'horizon}
\documentclass{magnolia}

\magtex{tex_driver={pdftex},
        tex_packages={siunitx}}
\magfiche{document_nom={Récursivité, horizon},
          auteur_nom={François Fayard},
          auteur_mail={francois.fayard@auxlazaristeslasalle.fr}}
\magcours{cours_matiere={maths},
          cours_niveau={mpsi},
          cours_chapitre_numero={1},
          cours_chapitre={Ligne d'horizon}}
\magmisenpage{}
\maglieudiff{}
\magprocess

\begin{document}
%BEGIN_BOOK


\section{Ligne d'horizon}

Le problème de la ligne d'horizon est le suivant~: étant donnés un certain nombre de rectangles posés sur
l'axe des abscisses, quelle sera la ligne d'horizon visible. Les rectangles seront représentés par des
tuples \verb!(a, b, h)! où $a$ est l'abscisse du coin inférieur gauche, $b$ est l'abscisse du coin
inférieur droit et $h$ la hauteur du bâtiment. La liste des bâtiments nous sera donc donnée par une
liste de type \verb!list[tuple[int, int, int]]!. Par exemple, la figure 1, représente les bâtiments donnés
par la liste \verb!bat = [(0, 5, 3), (1, 3, 2), (2, 4, 4), (6, 7, 2)]!.
\begin{center}
\includegraphics[width=0.5\textwidth]{../../Commun/Images/python-tp-horizon-1}\\
\textsc{Figure 1}
\end{center}
La ligne d'horizon est quant à elle donnée par la liste des points à chaque changement d'ordonnée~: on donne
l'abscisse et l'ordonnée à droite du changement. Dans notre exemple, la ligne d'horizon visualisée sur
la figure 2 sera représentée par \verb!hor = [(0, 3), (2, 4), (4, 3), (5, 0), (6, 2), (7, 0)]!.
\begin{center}
\includegraphics[width=0.5\textwidth]{../../Commun/Images/python-tp-horizon-2}\\
\textsc{Figure 2}
\end{center}
\begin{questions}
\question Si \verb!hor! est une ligne d'horizon, que peut-on dire de \verb!hor[-1][1]!~?
\end{questions}

\subsection{Un cas simple}

Dans cette partie, on suppose que les abscisses des bords des rectangles sont des entiers compris entre
0 et $n-1$.
\begin{questions}
\question Écrire une fonction \verb!hauteurs(bat: list[tuple[int, int, int]], n: int) -> list[int]! qui prend
  en argument une liste de rectangles et leur abscisse maximale, et renvoie la liste \verb!h! des hauteurs de la
  ligne d'horizon~: pour tout $i\in\intere{0}{n-1}$, \verb!h[i]! est la hauteur de la ligne d'horizon
  juste à droite du point d'abscisse $i$. Avec l'exemple ci-dessus, cette fonction appelée avec $n=8$ nous
  renverrait la liste \verb![3, 3, 4, 4, 3, 0, 2, 0]!.
\question En utilisant ce qui précède, écrire une fonction
\begin{center}
  \verb!horizon_entier(bat: list[tuple[int, int, int]], int: n) -> list[tuple[int, int]]!
\end{center}
qui renvoie la liste des points de la ligne d'horizon.
\question Exprimer la complexité de ce calcul en fonction de paramètres de votre choix.
\end{questions}

\subsection{Le cas général}

On ne fait cette fois-ci plus aucune hypothèse sur les abscisses des rectangles. Nous supposerons par contre
pour simplifier que les abscisses des bords des rectangles sont toujours distinctes.
\begin{questions}
\question Écrire une fonction \verb!horizon_rectangle(a: int, b: int, h: int) -> list[tuple[int, int]]! qui prend
  en argument les paramètres d'un rectangle et renvoie la liste des deux points qui décrivent sa ligne
  d'horizon.
\question Écrire une fonction \verb!fusion(ligne1: tuple[int, int, int], ligne2: tuple[int, int, int])! qui prend
  en argument deux lignes d'horizon et renvoie la ligne associée à l'union des deux lignes, comme sur la figure
  3. On souhaite une complexité linéaire en le nombre de points dans les deux lignes.
\begin{center}
\includegraphics[width=0.8\textwidth]{../../Commun/Images/python-tp-horizon-3}\\
\textsc{Figure 3} -- Fusion de deux lignes d'horizon.
\end{center}
\question Écrire une fonction \verb!ligne_horizon1(bat: list[tuple[int, int, int]]) -> list[tuple[int, int]]!
  qui renvoie la ligne d'horizon associée aux bâtiments \verb!bat!, en partant d'une ligne d'horizon
  plate et en la fusionnant avec les lignes d'horizon des différents bâtiments. Quelle est sa complexité~?
\question En se basant sur la méthode \og diviser pour régner \fg, écrire une fonction
\begin{center}
  \verb!ligne_horizon2(bat: list[tuple[int, int, int]]) -> list[tuple[int, int]]!
\end{center}
  ayant les mêmes spécifications. Déterminer sa complexité.

\end{questions}


\section{Fonction $G$ de Hofstadter}
On définit la fonction $G$ de Hofstadter par
\[G(0)\defeq 0, \qquad\et\qquad \forall n\in\Ns\qsep G(n)\defeq n - G(G(n-1)).\]
\begin{questions}
\question Écrire une fonction récursive \verb!G(n: int) -> int! calculant $G(n)$.
\question Chronométrer le temps que votre ordinateur prend pour calculer $G(195)$.
\question Montrer par récurrence sur $n$ que quel que soit $n\in\N$, le calcul de $G(n)$ termine
  et $G(n)\leq n$.
\enonce Un problème posé par l'implémentation récursive de \verb!G! et qu'on se retrouve à calculer
  plusieurs fois de nombreux termes $G(k)$ lors du calcul de $G(n)$.
\question Écrire une fonction \verb!G_memoization(n: int) -> int!
  qui renvoie $G(n)$ mais qui évite de calculer plusieurs fois les différents $G(k)$. Cette fonction pourra
  utiliser un tableau d'entiers \verb!memo! de taille $n$ et remplir petit à petit les cases \verb!memo[k]!
  par $G(k)$.  Quel est le temps qui est désormais nécessaire pour calculer $G(195)$~?
\end{questions}
%END_BOOK
\end{document}

\chapter{Coupe de somme minimale}
\documentclass{magnolia}

\magtex{tex_driver={pdftex},
        tex_packages={siunitx}}
\magfiche{document_nom={Coupe de somme minimale},
          auteur_nom={François Fayard},
          auteur_mail={francois.fayard@auxlazaristeslasalle.fr}}
\magcours{cours_matiere={maths},
          cours_niveau={mpsi},
          cours_chapitre_numero={1},
          cours_chapitre={Coupe de somme minimale}}
\magmisenpage{}
\maglieudiff{}
\magprocess

\begin{document}
%BEGIN_BOOK



\section{Coupe de somme minimale}

Il n'est pas possible de déterminer à l'avance le temps que prendra un ordinateur pour exécuter un algorithme~:
cette caractéristique dépend de trop nombreux paramètres, tant matériels que logiciels. En revanche, il est
souvent possible d'évaluer l'\emph{ordre de grandeur} du temps d'exécution en fonction des paramètres de
l'algorithme. Durant cette séance de travaux pratiques, nous allons écrire plusieurs algorithmes résolvant le
même problème~: le premier aura un temps d'exécution en $\Theta(n^3)$,le second en $\Theta(n^2)$ et le troisième
en $\Theta(n)$ et nous constaterons la différence considérable qui peut exister concernant le temps d'exécution
de chacune de ces trois fonctions sur des données de grande taille.\\

Pour mesurer le temps d'exécution, nous allons commencer par importer une fonction nommée \verb!time! qui
appartient à un module lui aussi nommé \verb!time!. Votre code devra donc commencer par la ligne suivante~:
\begin{pythoncode}
from time import time
\end{pythoncode}
Une fois cette commande interprétée, vous disposerez d'une fonction \verb!time()! vous donnant la durée exprimée
en secondes depuis une date de référence qui dépend de votre système. Pour mesurer la durée d'exécution d'une
portion de code, il vous suffira d'encadrer celle-ci de la façon suivante~:
\begin{pythoncode}
debut = time()
(*@\textcolor{purple}{bloc....................}@*)
(*@\textcolor{purple}{..........d'instructions}@*)
fin = time()
duree = fin - debut
\end{pythoncode}

Dans ce problème, on considère des listes d'entiers relatifs $a=[a_0,\ldots,a_{n-1}]$, et on appelle \emph{coupe}
de $a$ toute suite non vide d'éléments consécutifs de cette liste. Ainsi, une coupe est une liste de la forme
$[a_i,\ldots,a_{j-1}]$ avec $0\leq i<j\leq n$ qu'on notera désormais $a[i:j]$. À toute coupe $a[i:j]$, on associe
la somme
\[{\rm s}(i,j)\defeq\sum_{k=i}^{j-1} a_k\]
des éléments qui la composent. Le but de ce problème est de déterminer un algorithme efficace pour déterminer
la valeur minimale des sommes des coupes de $a$. À titre d'exemple, la somme minimale des coupes du tableau
$a\defeq[4, -4, 1, -1, -9, 8, -3, 8, -5, 5]$ est égale à $-13$, valeur atteinte pour la coupe $a[1:5]$.

\subsection{Un générateur pseudo aléatoire}
On considère la suite $(u_n)$ définie par la donnée de $u_0=42$ et la relation de récurrence
\[\forall n\in\N\qsep u_{n+1}\defeq \p{163811 u_n \ {\rm mod}\  655211} - 327607.\]
Pour expérimenter les différentes fonctions que nous allons écrire, nous allons avoir besoin de trois listes
Python de longueurs respectives $1\ 000$, $10\ 000$ et $100\ 000$ qu'on nommera \verb!lst1!, \verb!lst2!
et \verb !lst3!.
\begin{questions}
\question Rédiger un script générant chacune de ces trois listes, avec pour contenu
  \[\verb!lst1!=\left[u_i\,:\,0\leq i< 1\ 000\right], \qquad
    \verb!lst2!=\left[u_i\,:\,1\ 000\leq i< 11\ 000\right], \qquad 
    \verb!lst3!=\left[u_i\,:\,11\ 000\leq i< 111\ 000\right].\]
\end{questions}

\subsection{Un algorithme naïf}

\begin{questions}
\question Définir une fonction \verb!somme(a: list[int], i: int, j: int) -> int! renvoyant la somme de
  la coupe $a[i:j]$.
\question En déduire une fonction \verb!coupe_min1(a: list[int]) -> int! prenant en paramètre une liste $a$
  et renvoyant la somme minimale d'une coupe de $a$.
\question Mesurer le temps d'exécution de la fonction \verb!coupe_min1! pour la liste \verb!list1!.
\question Montrer que si $n$ est la longueur de la liste, le nombre d'additions effectué par cet algorithme
  vérifie $c(n)=\Theta(n^3)$.
\question Si nous avions la mauvaise idée d'utiliser cette fonction pour la liste \verb!list2!, et en
  admettant que le temps d'exécution soit effectivement proportionnel à $n^3$, combien de temps peut-on
  prévoir d'attendre~? Et pour \verb!lst3!~? On répondra à ces questions en remplissant le tableau
  ci-dessous~:
  \begin{center}
  \includegraphics[width=0.7\textwidth]{../../Commun/Images/python-tp-coupe-1}
  \end{center} 
\end{questions}

\subsection{Un algorithme de coût quadratique}

\begin{questions}
\question Définir, sans utiliser la fonction \verb!somme!, une fonction \verb!mincoupe(a: list[int], i: int) -> int!
  prenant en paramètres\ une liste $a$ et un entier $i$ et calculant la valeur minimale de la somme d'une de coupe de
  $a$ dont le premier élément est $a_i$. En comptant toujours les additions effectuées, quelle est la complexité
  de cette fonction~?
\question En déduire une fonction \verb!coupe_min2(a: list[int]) -> int! dont la complexité est en $\Theta(n^2)$,
  prenant en paramètre une liste $a$ et renvoyant la somme minimale d'une coupe de $a$.
\question Mesurer le temps d'exécution de la fonction \verb!coupe_min2! pour les listes \verb!lst1! et \verb!lst2!.
  Les deux valeurs obtenues sont-elles compatibles avec une croissance quadratique~? Combien de temps peut-on
  prévoir d'attendre si nous utilisons cette fonction pour calculer la somme minimale d'une coupe de la liste
  \verb!lst3!~?
  \begin{center}
    \includegraphics[width=0.7\textwidth]{../../Commun/Images/python-tp-coupe-2}
    \end{center} 
\end{questions}

\subsection{Un algorithme de coût linéaire}

Étant donnée une liste $a$, on note $m_i$ la somme minimale d'une coupe quelconque de la liste $a[0:i]$ et $c_i$
la somme minimale d'une coupe de $a[0:i]$ se terminant par $a_{i-1}$.
\begin{questions}
\question Montrer que $c_{i+1}=\min(c_i+a_i, a_i)$ et $m_{i+1}=\min(m_i, c_{i+1})$ et en déduire une fonction
  \verb!coupe_min3! de temps d'exécution linéaire calculant la valeur minimale de la somme d'une coupe de $a$.
\question Mesurer le temps d'exécution de la fonction \verb!coupe_min3! pour les listes \verb!lst1!, \verb!lst2!
  et \verb!lst3!. Ces valeurs sont-elles compatibles avec une croissance linéaire~?
  \begin{center}
    \includegraphics[width=0.7\textwidth]{../../Commun/Images/python-tp-coupe-3}
    \end{center} 
\end{questions}

\subsection{Un algorithme de coût quasi-linéaire}

Un algorithme de type \emph{diviser pour régner} est un algorithme qui scinde le problème initial en plusieurs
problèmes de taille plus petite, par exemple en deux sous-problèmes de taille deux fois plus petite que le
problème initial.

\begin{questions}
\question Soit $k\defeq\ent{n/2}$. Démontrer que la coupe minimale de $a$ est
  \begin{itemize}
  \item Soit entièrement contenue dans $a[0:k]$.
  \item Soit entièrement contenue dans $a[k:n]$.
  \item Soit constituée de la concaténation d'une coupe $a[i_0:k]$ minimale parmi celles de la forme a$[i:k]$
    pour $0\leq i<k$ et d'une coupe $a[k:j_0]$ minimale parmi celles de la forme $a[k:j]$ pour $k<j\leq n$.
  \end{itemize}
  et en déduire une fonction \verb!coupe_min4! utilisant ce principe pour résoudre le problème de la coupe
  minimale.
\question Mesurer le temps d'exécution de cette fonction pour chacune des listes \verb!lst1!, \verb!lst2! et \verb!lst3!.
\begin{center}
  \includegraphics[width=0.7\textwidth]{../../Commun/Images/python-tp-coupe-4}
  \end{center}
\question Il est possible de montrer que la complexité de cet algorithme est en $\Theta(n \log n)$. Compte tenu des
  mesures de temps obtenues, comprenez-vous la raison pour laquelle un algorithme ayant une telle complexité est
  qualifié de \emph{quasi-linéaire}~?
\end{questions}
%END_BOOK
\end{document}

\part{Langage Python}
Cette annexe liste limitativement les éléments du langage Python (version 3 ou supérieure) dont la connaissance est exigible des étudiants. Aucun concept sous-jacent n'est exigible au titre de la présente annexe.\\

Aucune connaissance sur un module particulier n'est exigible des étudiants.\\

Toute utilisation d'autres éléments du langage que ceux que liste cette annexe, ou d'une fonction d'un module, doit obligatoirement être accompagnée de la documentation utile, sans que puisse être attendue une quelconque maîtrise par les étudiants de ces éléments.

\subsubsection*{Traits généraux}
\begin{itemize}
\item Typage dynamique~: l'interpréteur détermine le type à la volée lors de l'exécution du code.
\item Principe d'indentation.
\item Portée lexicale~: lorsqu'une expression fait référence à une variable à l'intérieur d'une fonction, Python cherche la valeur définie à l'intérieur de la fonction et à défaut la valeur dans l'espace global du module.
\item Appel de fonction par valeur~: l'exécution de \verb|f(x)| évalue d'abord $x$ puis exécute $f$ avec la valeur calculée.
\end{itemize}

\subsubsection*{Types de base}
\begin{itemize}
\item Opérations sur les entiers (\verb|int|)~: \verb|+|, \verb|-|, \verb|*|, \verb|//|, \verb|**|, \verb|%| avec des opérandes positifs.
\item Opérations sur les flottants (\verb|float|)~: \verb|+|, \verb|-|, \verb|*|, \verb|/|, \verb|**|.
\item Opérations sur les booléens (\verb|bool|)~: \verb|not|, \verb|or|, \verb|and| (et leur caractère paresseux).
\item Comparaisons \verb|==|, \verb|!=|, \verb|<|, \verb|>|, \verb|<=|, \verb|>=|.
\end{itemize}

\subsubsection*{Types structurés}
\begin{itemize}
\item Structures indicées immuables (chaînes, tuples)~: \verb|len|, accès par indice positif valide, concaténation \verb|+|, répétition \verb|*|, tranche.
\item Listes~: création par compréhension \verb|[e for x in s]|, par \verb|[e] * n|, par \verb|append| successifs~; \verb|len|, accès par indice positif valide~; concaténation \verb|+|, extraction de tranche, copie (y compris son caractère superficiel)~; \verb|pop| en dernière position.
\item Dictionnaires~: création \verb|{c_1 : v_1, ..., c_n : v_n}|, accès, insertion, présence d'une clé \verb|k in d|, \verb|len|, \verb|copy|.
\end{itemize}

\subsubsection*{Structures de contrôle}
\begin{itemize}
\item Instruction d'affectation avec \verb|=|. Dépaquetage de tuples.
\item Instruction conditionnelle~: \verb|if|, \verb|elif|, \verb|else|.
\item Boucle \verb|while| (sans \verb|else|). \verb|break|, \verb|return| dans un corps de boucle.
\item Boucle \verb|for| (sans \verb|else|) et itération sur \verb|range(a, b)|, une chaîne, un tuple, une liste, un dictionnaire au travers des méthodes \verb|keys| et \verb|items|.
\item Définition d'une fonction \verb|def f(p_1, ..., p_n)|, \verb|return|.
\end{itemize}

\subsubsection*{Divers}
\begin{itemize}
\item Introduction d'un commentaire avec \verb|#|.
\item Utilisation simple de \verb|print|, sans paramètre facultatif.
\item Importation de modules avec \verb|import module|, \verb|import module as alias|, \verb|from module import f, g, ...|
\item Manipulation de fichiers texte (la documentation utile de ces fonctions doit être rappelée~; tout problème relatif aux encodages est éludé)~: \verb|open|, \verb|read|, \verb|readline|, \verb|readlines|, \verb|split|, \verb|write|, \verb|close|.
\item Assertion~: \verb|assert| (sans message d'erreur).
\end{itemize}


\end{document}
