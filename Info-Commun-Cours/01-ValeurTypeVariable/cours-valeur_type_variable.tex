\documentclass{magnoliaold}

\magtex{tex_driver={pdftex},
        tex_packages={xypic}}
\magfiche{document_nom={Cours Python sur les valeurs et types},
          auteur_nom={François Fayard},
          auteur_mail={fayard.prof@gmail.com}}
\magcours{cours_matiere={python},
          cours_niveau={mpsi},
          cours_chapitre_numero={1},
          cours_chapitre={Valeur, type et variable}}
\magmisenpage{}
\maglieudiff{}
\magprocess

\begin{document}
%BEGIN_BOOK
\hfill\includegraphics[width=0.7\textwidth]{../../Commun/Images/maths-cours-tech-calvin.png}\\

\magtoc

\section{Valeur, type}

Le langage Python manipule des valeurs de différents \emph{types}.
Nous rencontrerons d'abord les types numériques~: les \emph{entiers} ainsi que les
\emph{nombres flottants} que l'on utilise pour représenter les réels. Nous verrons
ensuite les \emph{chaines de caractères}, les \emph{booléens} et les \emph{tuples}.

\subsection{Nombre entier}

Nous utiliserons Python le plus souvent dans ce qu'on
appelle le shell ou la boucle interactive. Ce mode est aussi appelé
\og {\sc Repl} \fg pour~: Read, Evaluate, Print, Loop. Autrement dit, lorsqu'on entre
une expression, Python la lit, l'évalue, affiche le résultat, et est
prêt pour l'interaction suivante.\\

On écrit les entiers de manière naturelle. D'une manière générale,
on obtient le type d'une valeur grâce à la fonction \verb_type_. 

\begin{pythoncode}
In [1]: 42
Out[1]: 42

In [2]: type(42)
Out[2]: int
\end{pythoncode}

\noindent Le type des entiers est donc \verb_int_. 
Les opérateurs usuels d'addition \og\verb_+_\fg, de soustraction \og\verb_-_\fg, de
multiplication \og\verb+*+\fg et d'exponentiation \og\verb_**_\fg sont disponibles pour
créer des \emph{expressions} qui sont \emph{évaluées} par l'interpréteur. Ces opérateurs
possèdent différents niveaux de priorité. L'exponentiation est prioritaire sur 
la multiplication. L'addition et la soustraction ont la priorité la plus basse.

\begin{pythoncode}
In [3]: 2 + 3 * 5
Out[3]: 17

In [4]: -1 + 2 ** 8
Out[4]: 255
\end{pythoncode}

\noindent
On utilise les parenthèses pour grouper différentes sous-expressions lorsque les calculs
que l'on souhaite effectuer diffèrent de ceux fixés par les règles de priorité.

\begin{pythoncode}
In [5]: (2 + 3) * 5
Out[5]: 25 
\end{pythoncode}

\noindent
Les différentes règles de priorité sont parfois subtiles. Il est donc souhaitable, pour des raisons de lisibilité,
d'ajouter des parenthèses dès lors que l'évaluation de notre expression repose sur leur
connaissance fine. N'oubliez jamais qu'un programme est écrit pour être lu non seulement par
un ordinateur, mais aussi par des humains, qu'ils soient programmeurs ou correcteurs de
concours. Par exemple, on n'écrira pas \verb_2 ** 2 ** 3_ mais plutôt l'une
des deux expressions suivantes~:

\begin{pythoncode}
In [6]: (2 ** 2) ** 3
Out[6]: 64

In [7]: 2 ** (2 ** 3)
Out[7]: 256
\end{pythoncode}

\noindent 
Ces opérateurs sont des opérateurs \emph{binaires}~: ils nécessitent deux arguments.
Le \textsc{Pep8}, qui fixe les règles de bon usage
en Python, recommande de mettre un espace de part et d'autre de tels opérateurs.
Cependant, lorsqu'on construit des expressions mélangeant des opérateurs ayant
différents niveaux de priorité, il est parfois plus lisible d'omettre cet espace autour des
opérateurs ayant la priorité la plus forte. Par exemple, on écrira \verb!2**10 - 1!.\\

Le symbole \og\verb_-_\fg, utilisé comme opérateur
binaire de soustraction, est aussi utilisé pour la négation. Dans ce cas, c'est
un opérateur \emph{unaire} ne nécessitant qu'une opérande.

\begin{pythoncode}
In [8]: -2 * 3
Out[8]: -6
\end{pythoncode}
\noindent
Contrairement à ce qui se passe dans la plupart des autres langages comme le \textsc{C} ou
OCaml, Python peut représenter des nombres aussi grands que l'on souhaite.
Prenons l'exemple du $n$-ième nombre de Mersenne définit par
$M_n\defeq 2^n-1$. Il est courant de chercher des nombres premiers parmi ces entiers. Le
dixième nombre de Mersenne qui est premier est $M_{89}$ et son calcul ne pose aucun
problème à Python.

\begin{pythoncode}
In [9]: 2**89 - 1
Out[9]: 618970019642690137449562111
\end{pythoncode}
\vspace{1ex}

Python offre deux types de division. Commençons par la division entière. Rappelons le théorème de la division euclienne sur $\Z$~:

\begin{proposition}
Soit $a\in\Z$ et $b\in\Ns$. Alors il existe un unique couple $\p{q,r}\in\Z^2$
tel que
\[a=qb+r \et 0\leq r<b.\]
$q$ est appelé \emph{quotient} de la division euclidienne de $a$ par $b$, et $r$ son
\emph{reste}.
\end{proposition}

\noindent
Par exemple $7 = 2\times 3 + 1$, donc 2 est le quotient de la division
euclidienne de 7 par 3 et son reste est $1$. De même $-7 = (-3)\times 3 + 2$, donc $-3$ est
le quotient de la division euclidienne de $-7$ par 3 et son reste est 2. En Python,
on obtient le quotient de la division euclidienne de $a$ par $b$ avec
\verb_a // b_ et son reste avec \verb_a % b_.

\begin{pythoncode}
In [10]: -7 // 3
Out[10]: -3

In [11]: -7 % 3
Out[11]: 2
\end{pythoncode}

\noindent La division par 0 est une erreur et lève ce
qu'on appelle une \emph{exception}. Même s'il est possible de rattraper les exceptions,
nous ne le ferons pas dans ce cours et une division par 0 aura pour effet de produire
l'erreur suivante~:

\begin{francois}
\begin{pythoncode}
In [12]: 1 // 0
(*@\textcolor{violet}{ZeroDivisionError: integer division or modulo by zero}@*)
\end{pythoncode}
\end{francois}
\begin{victor}
\begin{pythoncode}
>>> 1 / 0
Traceback (most recent call last):
  File "<stdin>", line 1, in <module>
ZeroDivisionError: division by zero
\end{pythoncode}
\end{victor}

% \noindent
% Notons que si tous les langages sont d'accord pour définir la division entière de la même
% manière lorsque $a\in\N$ et $b\in\Ns$, ce n'est pas le cas si $a<0$. Par exemple, en
% \textsc{C} et en Ocaml, ce n'est pas la division euclidienne qui est effectuée
% lorsque l'on fait une division entière de $-7$ par 3. En effet, ces langages renvoient $-2$.
% Ne parlons pas de la division par des entiers strictement négatifs qui est bien
% définie en Python, mais que l'on ne cherchera pas à utiliser.\\
\vspace{1ex}

Python offre aussi une division plus classique, notée \verb_/_. Elle produit
une valeur d'un type différent~: celui des nombres flottants.

\begin{pythoncode}
In [13]: 3 / 2
Out[13]: 1.5

In [14]: type(1.5)
Out[14]: float
\end{pythoncode}

\subsection{Nombre flottant}

Les nombres flottants sont utilisés pour représenter les nombres réels. Comme tous
les langages de programmation, Python utilise le \og\verb_._\fg comme séparateur décimal.
Pour calculer une valeur approchée de la circonférence d'un cercle de diamètre 2, on entre
donc~:

\begin{pythoncode}
In [1]: 3.14 * 2.0
Out[1]: 6.28
\end{pythoncode}

\noindent Commençons par remarquer que les opérateurs \verb!+!, \verb!-!, \verb!*!, \verb!/! 
et \verb!**! sont
disponibles pour les nombres flottants. On peut par ailleurs mélanger flottants et
entiers dans les calculs. Comme pour les entiers, la division par \verb_0.0_ lève une
exception.\\

Pour simplifier l'écriture de grands et de petits nombres, on utilise la notation
scientifique. Ainsi, l'âge de l'univers étant estimé à 13.8 milliards d'années et la vitesse
de la lumière étant de l'ordre de $3.0\times 10^8$ mètres par seconde,
le calcul suivant nous montre qu'il est impossible
d'observer des endroits de l'univers à une distance supérieure à $1.31\times 10^{26}$
mètres de la terre~:

\begin{pythoncode}
In [2]: 13.8e9 * 365 * 24 * 60 * 60 * 3e8
Out[2]: 1.3055904e+26
\end{pythoncode}

% \noindent
% Remarquons qu'on se place ici dans un modèle où il n'y aurait pas eu de \og big bang \fg, et
% donc une expansion de l'univers, mais dans un modèle simpliste où il serait apparu
% de manière uniforme dans l'espace à un même instant.
Pour le calcul de la circonférence du cercle de diamètre 2, on a approché plus haut $\pi$ par $3.14$.
On pourrait bien sûr utiliser une approximation plus précise comme $3.14159$,

\begin{pythoncode}
In [3]: 2 * 3.14159
Out[3]: 6.28318
\end{pythoncode}
 
\noindent mais la précision disponible avec Python n'est pas illimitée. En première approximation,
on peut considérer que Python ne peut travailler qu'avec des nombres flottants ayant
une précision de 16 chiffres significatifs~: tout excès de précision est ignoré.

\begin{pythoncode}
In [4]: 1234567890.12345678 - 1234567890.1234567
Out[4]: 0.0
\end{pythoncode}

\noindent Le premier nombre possède 18 chiffres
significatifs alors que le second en possède 17. Avant même d'effectuer la soustraction, les deux
nombres sont arrondis au même nombre~: le résultat final est donc nul. La situation est
en fait plus complexe que cela, car tout comme les entiers, les flottants ne sont pas
représentés en interne en base 10 mais en base 2. Ne soyez donc pas surpris si, en faisant vos propres
essais, vous avez parfois l'impression que Python garde 16 chiffres significatifs,
parfois 17.\\

Pour les mêmes raisons, le résultat de chaque opération arithmétique est arrondi. Cela 
conduit à des résultats surprenants comme le calcul suivant qui n'est pas égal à
$10^{-16}$ comme on pourrait s'y attendre.

\begin{pythoncode}
In [5]: (1.0 + 1.0e-16) - 1.0
Out[5]: 0.0
\end{pythoncode}

\noindent En effet, $1.0 + 10^{-16}$ possède 17 chiffres significatifs. Il est arrondi
à $1.0$ avant d'effectuer la soustraction qui donne donc 0. Comme les
flottants sont stockés en base 2 et non en base 10, même les nombres décimaux les plus
simples ne sont pas représentables exactement. On peut donc avoir des résultats
surprenants~:

\begin{pythoncode}
In [6]: 0.1 + 0.2 - 0.3
Out[6]: 5.551115123125783e-17
\end{pythoncode}

\noindent Vous comprendrez pourquoi les logiciels de comptabilité ne travaillent pas en
interne avec des nombres flottants. Ils ont cependant de nombreuses qualités et sont utilisés en simulation numérique ainsi qu'en intelligence artificielle. On n'oubliera cependant jamais que des arrondis sont effectués à
chaque opération et nous verrons que les erreurs accumulées peuvent parfois devenir 
significatives et fausser complètement un résultat.\\

Lorsqu'on mélange des entiers et des flottants dans une expression,
une conversion préalable des entiers vers les flottants est réalisée automatiquement.
Cette conversion automatique est un choix raisonnable, car contrairement
aux nombres décimaux, les nombres entiers qui ne sont pas trop grands (disons, ceux qui s'écrivent
avec moins de 16 chiffres) sont représentables de manière exacte par des flottants.
La conversion des flottants vers les entiers est aussi possible. Cependant, comme elle fait perdre
de l'information, il faut la demander explicitement en utilisant la fonction \verb_int_.
Cette fonction arrondit un flottant au premier entier rencontré lorsqu'on se rapproche de 0.

\begin{pythoncode}
In [7]: int(2.718)
Out[7]: 2

In [8]: int(-2.718)
Out[8]: -2
\end{pythoncode}

\noindent De manière générale, chaque type numérique possède une fonction associée pour
forcer une conversion.\\

Les fonctions usuelles sont disponibles dans la bibliothèque \verb_math_. On y trouve
par exemple la fonction \verb_floor_ qui, pour chaque nombre $x$, renvoie sa partie entière dont on
rappelle la définition ci-dessous.

\begin{proposition}
Soit $x\in\R$. Il existe un unique $n\in\Z$ tel que
\[n\leq x<n+1.\]
Cet entier est appelé \emph{partie entière} de $x$ et est noté $\ent{x}$.
\end{proposition}

\noindent
Pour charger la bibliothèque \verb_math_, on utilise l'instruction suivante~:

\begin{pythoncode}
In [9]: import math

In [10]: math.floor(2.718)
Out[10]: 2

In [11]: math.floor(-2.718)
Out[11]: -3
\end{pythoncode}

\noindent De même, on définit la partie entière supérieure d'un réel.
\begin{proposition}
Soit $x\in\R$. Il existe un unique $n\in\Z$ tel que
\[n-1< x\leq n.\]
Cet entier est appelé \emph{partie entière supérieure} de $x$ et est noté $\ceil{x}$.
\end{proposition}

\noindent
La fonction \verb_ceil_ de la même bibliothèque permet d'y accéder.

\begin{pythoncode}
In [12]: math.ceil(2.718)
Out[12]: 3

In [13]: math.ceil(-2.718)
Out[13]: -2
\end{pythoncode}

On peut calculer la racine carrée d'un nombre avec la fonction \verb_sqrt_,
abréviation de \og square root\fg. Les autres fonctions usuelles sont aussi disponibles.

\begin{pythoncode}
In [14]: math.sqrt(2.0)
Out[14]: 1.4142135623730951

In [15]: math.exp(1.0)
Out[15]: 2.718281828459045
\end{pythoncode}

\noindent Le logarithme naturel $\ln$ (\verb_log_ en Python), le logarithme en base 10 (\verb_log10_) et
le logarithme en base 2 (\verb_log2_) sont aussi disponibles. Bien sûr, les fonctions
trigonométriques circulaires $\cos$, $\sin$ et $\tan$ sont présentes, tout comme la
constante $\pi$.

\begin{pythoncode}
In [16]: math.cos(math.pi / 17)
Out[16]: 0.9829730996839018
\end{pythoncode}

La principale devise de Python est \og batteries included\fg. Autrement dit,
de nombreuses bibliothèques (\og libraries\fg en anglais), sont disponibles. Nous
venons d'utiliser notre première bibliothèque~: le module \verb_math_. Il existe de
nombreuses manières de les rendre accessibles. La plus simple est d'écrire \verb_import_
suivi du nom de la bibliothèque. Les fonctions et constantes seront alors disponibles,
préfixées par le nom du module. Si vous souhaitez seulement en utiliser certaines sans
avoir à taper à chaque fois le nom du module, vous pouvez entrer la commande~:

\begin{pythoncode}
In [17]: from math import cos, pi

In [18]: cos(pi / 17)
Out[18]: 0.9829730996839018
\end{pythoncode}

\noindent Il est d'ailleurs possible d'importer tous les composants du module \verb!math! avec la commande

\begin{pythoncode}
In [19]: from math import *
\end{pythoncode}

\noindent
C'est cependant une opération dangereuse, car si on importe ainsi plusieurs modules, on ne sait rapidement plus
d'où viennent nos fonctions. En pratique, il est préférable d'utiliser \verb!import math!, quitte
à renommer le module en un nom plus court. On utilise pour cela la commande~:

\begin{pythoncode}
In [20]: import math as ma

In [21]: ma.cos(ma.pi / 3)
Out[21]: 0.5000000000000001
\end{pythoncode}

% \subsection{Nombres complexes}

% Les nombres complexes sont des cousins des nombres flottants. Ils sont formés de deux
% nombres flottants~: un représentant leur partie réelle et un autre représentant leur partie
% imaginaire. Le nombre imaginaire $i$ est représenté par la lettre \verb_j_ que l'on accole 
% à la partie imaginaire du nombre. Bien entendu, on peut mélanger entiers, flottants et
% nombres complexes. Tout se passe comme si tous les nombres étaient convertis en nombres
% complexes avant d'effectuer les opérations.

% \begin{pythoncode}
% >>> type(1.0 + 2.0j)
% <class 'complex'>
% >>> 1 / (1.0 + 2.0j)
% (0.2-0.4j) 
% \end{pythoncode}

% \noindent
% Le module, la partie réelle et la partie imaginaire sont aussi accessibles avec les
% fonctions \verb_abs_, \verb_real_ et \verb_imag_.

% \begin{pythoncode}
% >>> import numpy as np
% >>> np.abs(1.0 + 1.0j)
% 1.41421356237
% >>> np.real(2.0 + 3.0j)
% 2.0
% \end{pythoncode}


% Les types que nous venons de découvrir, \verb_int_, \verb_float_ et \verb_complex_ sont des
% types numériques. Cette suite de type a la particularité intéressante que l'on peut presque
% considérer que chaque type peut représenter de manière exacte toutes les valeurs du type
% qui le précède. En effet, les nombres entiers de taille raisonnable sont représentables
% de manière exacte par des nombres flottants et les nombres flottants sont tous
% représentables par des nombres complexes. Toute expression mélangeant ces types
% numériques sera donc évaluée en convertissant tous les types vers le type le plus large présent. C'est ce que l'on appelle la promotion automatique.


% Commen\c{c}ons d'abord
% par remarquer qu'il peut paraître surprenant que \verb_bool_ soit un type numérique.
% C'est cependant le cas si on identifie \verb_False_ à 0 et \verb_True_ à 1. C'est ce que
% fait Python pour des raisons historiques. Il vous est cependant demandé d'oublier au plus
% vite cette curiosité.\\



\subsection{Chaine de caractères}

Python nous permet de travailler avec du texte. Pour cela, on utilise des chaines de caractères que l'on
encadre en utilisant soit des \og\verb_"_\fg, soit des \og\verb_'_\fg.

\begin{pythoncode}
In [1]: "hello, world"
Out[1]: 'Hello, world'

In [2]: 'Ça dépend, ça dépasse.'
Out[2]: 'Ça dépend, ça dépasse.'
\end{pythoncode}

\noindent Python permet d'utiliser des accents dans les chaines de caractères.
Vous pouvez même utiliser des caractères espagnols, allemands, russes, hébreux, arabes
ou chinois. Bref, tous les caractères \textsc{Unicode} sont supportés.
De nombreux caractères \og spéciaux \fg existent. Par exemple, le retour à la ligne est un
\og caractère \fg que l'on obtient en écrivant \og\verb_\n_\fg. De même, la tabulation est
un caractère que l'on obtient en écrivant \og\verb_\t_\fg. La plupart des éditeurs de texte
affichent la tabulation en la remplaçant par 2 ou 4 espaces,
mais il est important d'être conscient que dans les fichiers textes, c'est un caractère à part entière. Comme il
est difficile de le distinguer d'une succession d'espaces, on convient
en général de ne pas l'utiliser~: beaucoup d'éditeurs de texte insèrent des espaces
plutôt qu'un caractère de tabulation lorsque l'on utilise la touche \og \textsc{Tab} \fg de notre clavier. Enfin, si
vous souhaitez utiliser des guillemets dans une chaine de caractères et que votre
choix du délimiteur vous en empêche, vous pouvez l'insérer avec \og\verb_\"_\fg ou
\og\verb_\'_\fg. Par exemple~:

\begin{pythoncode}
In [3]: "What do you mean \"ew\"? I don't like Spam!"
Out[3]: (*@\textcolor{red}{'What do you mean "ew"? I don't like Spam!'}@*)
\end{pythoncode}
Il est possible d'obtenir la longueur d'une chaine de caractères en utilisant la
fonction \verb_len_.

\begin{pythoncode}
In [4]: len("hello, world")
Out[4]: 12
\end{pythoncode}

Les chaines de caractères ont leur propre type~: le type \verb_str_.

\begin{pythoncode}
In [5]: type("Il suffit pas d'y dire, y faut aussi y faire.")
Out[5]: str
\end{pythoncode}

\noindent
On ne confondra pas les chaines de caractères et les entiers. En particulier, si on
essaie d'ajouter un entier à une chaine de caractères en écrivant \og\verb_2 + "2"_\fg,
on obtient une erreur de type. Cependant, on peut utiliser le symbole \verb_+_ entre
deux chaines de caractères, ce qui a pour effet de les concaténer~:

\begin{pythoncode}
In [6]: "Tic" + "Tac"
Out[6]: 'TicTac'
\end{pythoncode}

\noindent
De la même manière, il est possible de multiplier une chaine de caractères par un entier.

\begin{pythoncode}
In [7]: "G" + 10 * "o" + "al"
Out[7]: 'Gooooooooooal'
\end{pythoncode}

On peut convertir une chaine de caractères en un entier ou un nombre flottant.
C'est une conversion de type et il suffit
pour cela d'appliquer la fonction portant le nom du type désiré à notre chaine.


\begin{pythoncode}
In [8]: int("2")
Out[8]: 2

In [9]: float("13.1")
Out[9]: 13.1

In [10]: int("2") * float("13.1")
Out[10]: 26.2   
\end{pythoncode}

\noindent Si la chaine de caractères ne peut être interprétée comme une valeur du type demandée, une exception sera
levée. La conversion inverse est possible avec la fonction \verb_str_~:

\begin{pythoncode}
In [11]: "Fahrenheit " + str(45) + str(1)
Out[11]: 'Fahrenheit 451'
\end{pythoncode}

Le standard \textsc{Ascii} associe une valeur entre 0 et 127 à chacun des caractères les plus courants.
Le tableau ci-dessous donne ces valeurs, les cases grisées représentant des caractères non imprimables.
Par exemple, le caractère A est associé à la valeur 65.

\begin{center}
\begin{tabular}{|c||c|c|c|c|c|c|c|c|c|c|}
\hline
&0&1&2&3&4&5&6&7&8&9\\
\hline
\hline
0&\cellcolor[gray]{0.9}&\cellcolor[gray]{0.9}&\cellcolor[gray]{0.9}&\cellcolor[gray]{0.9}&\cellcolor[gray]{0.9}&\cellcolor[gray]{0.9}&\cellcolor[gray]{0.9}&\cellcolor[gray]{0.9}&\cellcolor[gray]{0.9}&\cellcolor[gray]{0.9}\\
\hline
10&\cellcolor[gray]{0.9}&\cellcolor[gray]{0.9}&\cellcolor[gray]{0.9}&\cellcolor[gray]{0.9}&\cellcolor[gray]{0.9}&\cellcolor[gray]{0.9}&\cellcolor[gray]{0.9}&\cellcolor[gray]{0.9}&\cellcolor[gray]{0.9}&\cellcolor[gray]{0.9}\\
\hline
20&\cellcolor[gray]{0.9}&\cellcolor[gray]{0.9}&\cellcolor[gray]{0.9}&\cellcolor[gray]{0.9}&\cellcolor[gray]{0.9}&\cellcolor[gray]{0.9}&\cellcolor[gray]{0.9}&\cellcolor[gray]{0.9}&\cellcolor[gray]{0.9}&\cellcolor[gray]{0.9}\\
\hline
30&\cellcolor[gray]{0.9}&\cellcolor[gray]{0.9}&&!&"&\#&\$&\%&\&&'\\
\hline
40&(&)&*&+&,&-&.&/&0&1\\
\hline
50&2&3&4&5&6&7&8&9&:&;\\
\hline
60&<&=&>&?&@&A&B&C&D&E\\
\hline
70&F&G&H&I&J&K&L&M&N&O\\
\hline
80&P&Q&R&S&T&U&V&W&X&Y\\
\hline
90&Z&[&\textbackslash&]&\char`\^&\_&\`\ &a&b&c\\
\hline
100&d&e&f&g&h&i&j&k&l&m\\
\hline
110&n&o&p&q&r&s&t&u&v&w\\
\hline
120&x&y&z&\{&|&\}&\char`\~&\cellcolor[gray]{0.9}&&\\
\hline
\end{tabular}
\end{center}

\noindent
Pour obtenir cet entier, on
utilise la fonction \verb_ord_.

\begin{pythoncode}
In [12]: ord('A')
Out[12]: 65
\end{pythoncode}

\noindent
On peut obtenir un caractère à partir de son code \textsc{Ascii} grâce à la fonction
\verb!chr!.

% \begin{pythoncode}
% In [13]: "I " + chr(9829) + " les Lazos."
% Out[13]: 'I les Lazos.'
% \end{pythoncode}
\begin{lstlisting}[language=python, escapeinside=||]
In [13]: chr(65)
Out[13]: 'A'
\end{lstlisting}
% R code
% @
% |\includegraphics[width=0.5\textwidth]{example-image}|






% \vspace{2ex}
% Il est possible d'accèder au $k$-ième caractère d'une chaine en utilisant la syntaxe
% suivante.

% \begin{pythoncode}
% In [12]: s = "hello, world"

% In [13]: s[0]
% Out[13]: 'h'
% \end{pythoncode}

% \noindent
% Notons que comme la plupart des langages, Python indece les chaines à partir de 0
% et non de 1. Par contre, contrairement à ce qui se passe en \textsc{C} et en Ocaml,
% Python ne possède pas de type \og caractère \fg. L'instruction \verb_s[k]_
% renvoie donc une chaine de caractère de longueur 1. Bien entendu, si \verb_s_ est une
% chaine de longueur \verb_n_, on n'utilisera que des indices \verb_k_ tels que
% $0\leq k< n$. Si $k\geq n$, Python générera une exception et donc une erreur.

% \begin{francois}
% \begin{pythoncode}
% In [14]: s = "hello, world"

% In [15]: len(s)
% Out[15]: 12

% In [16]: s[12]
% (*@\textcolor{violet}{IndexError: string index out of range}@*)
% \end{pythoncode}
% \end{francois}
% \begin{victor}
% \begin{pythoncode}
% >>> s = "hello, world"
% >>> len(s)
% 12
% >>> s[12]
% Traceback (most recent call last):
%   File "<stdin>", line 1, in <module>
% IndexError: string index out of range
% \end{pythoncode}
% \end{victor}

% \noindent
% Python accepte des indices négatifs $-n\leq k\leq -1$ pour désigner les caractères
% d'une chaine en partant de la fin. Cependant, le programme de classes préparatoires
% proscrit une telle utilisation.\\

% % Les chaines de caractère sont immuables. Il est impossible de les modifier et le code suivant
% % renvoie une erreur.
% % \begin{francois}
% % \begin{pythoncode}
% % >>> s = "hello, world"
% % >>> s[0] = "a"
% % (*@\textcolor{violet}{Traceback (most recent call last):}@*)
% % (*@\textcolor{violet}{\ \ File "<stdin>", line 1, in <module>}@*)
% % (*@\textcolor{violet}{TypeError: 'str' object does not support item assignment}@*)
% % \end{pythoncode}
% % \end{francois}
% % \begin{victor}
% % \begin{pythoncode}
% % >>> s = "hello, world"
% % >>> s[0] = "a"
% % Traceback (most recent call last):
% %   File "<stdin>", line 1, in <module>
% % TypeError: 'str' object does not support item assignment
% % \end{pythoncode}
% % \end{victor}

% Il est possible d'extraite une sous-chaine à l'aide de \emph{slices}. Dans son utilisation
% la plus élémentaire, \verb_s[i:j]_ renvoie la chaine composée des caractères d'indices
% $k$ tels  que $i\leq k< j$. Il est possible d'omettre $i$. Dans ce cas, la valeur de 0
% est utilisée. Si on omet $j$, c'est la longueur de la chaine qui est utilisée.

% \begin{pythoncode}
% In [17]: s = "hello, world"

% In [18]: s[1:5]
% Out[18]: 'ello'

% In [19]: s[:5]
% Out[19]: 'hello'

% In [20]: s[7:]
% Out[20]: 'world'
% \end{pythoncode}

% \noindent
% Il est aussi possible d'utiliser une forme plus avancée \verb_s[i:j:p]_
% où $p$ est le pas. Si $i\leq j$ et $p>0$, la chaine obtenue est formée des caractères
% d'indices $i + k p$ pour $i \leq i + k p < j$.

% \begin{pythoncode}
% In [21]: s = "hello, world"

% In [22]: s[1::3]
% Out[22]: 'eowl'
% \end{pythoncode}


\subsection{Booléen}

Python possède un type booléen qui n'a que deux valeurs distinctes~:
\verb_True_ et \verb_False_. 

\begin{pythoncode}
In [1]: True
Out[1]: True

In [2]: type(True)
Out[2]: bool
\end{pythoncode}

\noindent
Les opérateurs logiques usuels \og et \fg, \og ou \fg et \og non \fg sont disponibles. On
remarquera que le \og ou \fg est bien le ou inclusif, comme en mathématiques.

\begin{pythoncode}
In [3]: True and True
Out[3]: True

In [4]: True and False
Out[4]: False

In [5]: True or True
Out[5]: True

In [6]: not True
Out[6]: False
\end{pythoncode}

Pour savoir si deux valeurs sont égales, on utilise le symbole \og\verb_==_\fg.

\begin{pythoncode}
In [7]: 1 + 1 == 2
Out[7]: True
\end{pythoncode}

\noindent La valeur renvoyée par un test d'égalité est un \emph{booléen}. Attention
à ne jamais utiliser de test d'égalité entre deux flottants, car du fait de l'arithmétique
de ces derniers, certains résultats sont surprenants~!

\begin{pythoncode}
In [8]: 0.1 + 0.2 == 0.3
Out[8]: False
\end{pythoncode}

En général, deux valeurs de types différents ne sont pas égales.
\begin{pythoncode}
In [9]: 2 == "2"
Out[9]: False
\end{pythoncode}
\noindent
Les opérateurs \verb_!=_, \verb_<_, \verb_>_, \verb_<=_, \verb_>=_ sont aussi disponibles.

\begin{center}
\begin{tabular}{cc}
\hline
\verb_x == y_& x est égal à y\\
\verb_x != y_ & x est différent de y\\
\verb_x < y_ & x est strictement inférieur à y\\
\verb_x > y_ & x est strictement supérieur à y\\
\verb_x <= y_ & x est inférieur ou égal à y\\
\verb_x >= y_ & x est supérieur ou égal à y\\
\hline
\end{tabular}
\end{center}

\begin{pythoncode}
In [10]: 3 <= 3.14 and 3.14 <= 4
Out[10]: True
\end{pythoncode}

% \noindent
% Python comprend les expressions du type  \og\verb_3 <= x <= 4_\fg, mais
% nous éviterons ces constructions dans nos programmes.\\


Pour la comparaison des chaines de caractères, l'ordre utilisé est l'ordre
lexicographique, chaque caractère étant ordonné dans l'ordre de la table
\textsc{Ascii}/\textsc{Unicode}.

\begin{pythoncode}
In [11]: "OL" > "OM"
Out[11]: False
\end{pythoncode}

\noindent
Faites attention à l'ordre de ces caractères. Les minuscules sont bien évidemment
dans l'ordre alphabétique, tout comme les majuscules, mais la lettre \og Z\fg est avant
la lettre \og a\fg et donc de manière surprenante \verb_"Zorro" < "algèbre"_.
\vspace{2ex}
\begin{exoUnique}
\exo En utilisant le tableau \textsc{Ascii}, classer ces chaines de caractère dans l'ordre
  lexicographique~: \verb_"9"_, \verb_"34"_, \verb_"Maison"_, \verb_"la"_ et \verb_"laisser"_.
\end{exoUnique}

\subsection{Tuple}

Afin de grouper plusieurs valeurs, Python propose un type
appelé \verb_tuple_.

\begin{pythoncode}
In [1]: (1.0, 2.0)
Out[1]: (1.0, 2.0)

In [2]: "Teddy", "Riner", 1989
Out[2]: ("Teddy", "Riner", 1989)

In [3]: type((2.0, 1.0))
Out[3]: tuple
\end{pythoncode}

\noindent
Les parenthèses regroupant ces valeurs sont optionnelles.
Il arrive qu'on utilise des tuples pour grouper des valeurs n'ayant pas
le même type, comme dans notre second exemple.

\section{Programmation impérative}
\subsection{Variable}

Les valeurs déjà calculées peuvent être gardées en mémoire afin de les utiliser plus tard.
Pour cela, on utilise des variables. Les types que nous avons vus jusqu'ici
seront plus tard décrits comme \emph{immuables} et lorsqu'on travaille avec de tels types, une
variable peut être conceptualisée par une boite portant un \emph{nom} et contenant une
\emph{valeur}. Afin de stocker une valeur dans une boite, on utilise le symbole
d'\emph{affectation} \og\verb_=_\fg.

\begin{pythoncode}
In [1]: a = 6   
\end{pythoncode}

\noindent À gauche du symbole d'affectation, on place le nom de la boite qui doit
être utilisée. À droite, on doit trouver une expression qui sera évaluée en une valeur. Cette
valeur sera alors stockée dans la boite.\\

On accède ensuite à la valeur mémorisée en utilisant le nom de la variable. Lors de
l'évaluation de chaque expression, les noms de variables sont remplacés par les valeurs
qu'elles contiennent.

\begin{pythoncode}
In [2]: a * (a + 1)
Out[2]: 42 
\end{pythoncode}



\begin{exoUnique}
\exo Dans cet exercice on s'interdit d'utiliser l'exponentiation \og\verb_**_\fg.
\begin{questions}
\question Montrer que l'on peut calculer $a^8$ avec 3 multiplications.
\question Donner une manière de calculer $a^7$ avec 4 multiplications.
\end{questions}
\end{exoUnique}
\bigskip

Une fois qu'une variable est définie, il est possible de la redéfinir en utilisant une nouvelle valeur.

\begin{pythoncode}
In [3]: a = 7

In [4]: a = a + 1

In [5]: a
Out[5]: 8
\end{pythoncode}

\noindent Pour l'entrée \verb!a = a + 1!, le membre de droite est d'abord évalué pour produire la valeur 8.
Cette valeur est ensuite stockée dans la variable \verb_a_. L'ancienne valeur est \og
écrasée\fg et il n'est plus possible d'y accéder. Ce type d'instruction nous rappelle que le
symbole d'affectation est dissymétrique, contrairement au symbole d'égalité utilisé en
mathématiques. En particulier, l'instruction \og\verb_a + 1 = a_\fg n'a aucun sens et sera
signalée par Python comme une erreur. Remarquons que c'est bien une valeur qui est
stockée dans une variable. En particulier, si l'on définit
\og\verb_b = a_\fg et que l'on change ensuite la valeur de \verb_a_, celle de
\verb_b_ reste inchangée.
\vspace{2ex}
\begin{exoUnique}
\exo La méthode de Héron est une méthode historique pour obtenir une valeur
  approchée de la racine carrée d'un nombre $a>0$. Pour cela, on définit
	la suite $(u_n)$ par
	\[u_0\defeq a, \quad\et\quad \forall n\in\N\qsep u_{n+1}\defeq\frac{u_n + \frac{a}{u_n}}{2}.\]
	Déterminer la plus petite valeur de $n$ pour laquelle la précision sur les nombres
	flottants ne permet plus de distinguer $u_n$ de $u_{n+1}$ lors du calcul de $\sqrt{2}$.
\end{exoUnique}
\vspace{2ex}

Python est un langage de programmation à \emph{typage dynamique}~: une même
variable peut à un moment donné stocker un entier et plus tard une chaine de caractères.
Le type d'une variable, c'est-à-dire le type de la valeur stockée par cette variable est
donc autorisé à changer lors de l'exécution d'un programme. Cette manière de programmer
rend cependant les programmes plus difficiles à lire et nous éviterons de le faire.\\

Pour les noms de variables, nous nous limiterons aux noms composés de lettres minuscules
(\verb_a-z_) et majuscules (\verb_A-Z)_, ainsi qu'au caractère \og tiret du bas\fg ou \og
underscore\fg (\verb-_-) disponible sur la touche 8 des claviers français. L'utilisation de chiffres (\verb_0-9_) à
la fin d'un nom est autorisée. On évitera
d'utiliser les accents dans les noms de variables. 
Choisir
judicieusement le nom de ses variables est un art qu'il est important de cultiver. Les noms
de variables courts ont l'avantage d'être rapides à taper et à lire. On les utilisera donc
pour stocker des valeurs que nous utiliserons souvent. Les noms de variables longs ont
l'avantage d'être plus descriptifs. On les utilisera donc pour faire référence à des valeurs
que nous utiliserons plus rarement. Pour des noms de variables composés de plusieurs mots, on utilise
souvent un underscore comme dans \verb-nb_eleves- ou une lettre majuscule comme dans
\verb_nbEleves_.

\subsection{État du système}

Contrairement aux expressions dont la finalité est de produire une valeur, une affectation
a pour effet de changer l'état des variables. On dit qu'elle agit par \emph{effet de bord}.
Pour représenter l'état du système, nous utiliserons la notation suivante
\verb_{a: 2, b: 7}_. Elle signale que la variable \verb_a_ contient la valeur 2 tandis que
\verb_b_ contient la valeur 7. La programmation \emph{impérative} consiste à écrire une
succession d'instructions pour changer l'état de la machine. Le langage machine, qui est utilisé par les processeurs, fonctionne de cette manière. C'est une des raisons pour
lesquelles ce style est central dans de nombreux langages de programmation. C'est le cas pour
Python, et c'est un style que nous adopterons souvent dans ce cours. Afin de visualiser l'état
dans lequel se trouve la machine, on le décrira sur
une ligne de commentaire. Ces lignes commencent par le caractère \verb_#_ et sont ignorées
par Python.\\

Si par exemple les variables \verb_a_ et \verb_b_ contiennent respectivement 2 et 7, les
instructions suivantes modifient l'état de la machine comme suit~:

\begin{pythoncode}
# etat {a: 2, b: 7}
In [1]: a = b
In [2]: b = a
# etat {a: 7, b: 7} 
\end{pythoncode}

\noindent En particulier, ces deux instructions n'ont pas eu pour effet d'échanger le contenu
des variables \verb_a_ et \verb_b_. La première instruction a eu pour effet d'écraser la
valeur contenue dans \verb_a_ qui est alors définitivement perdue. Si on possède un verre
d'eau et un verre de vin, le meilleur moyen pour échanger le contenu de ces verres est
d'utiliser un troisième verre. Pour échanger deux variables, on peut donc utiliser la séquence
d'instructions suivante~:

\begin{pythoncode}
# etat {a: 2, b: 7}
In [1]: c = a
In [2]: a = b
In [3]: b = c
# etat {a: 7, b: 2, c: 2}
\end{pythoncode}

%%{r = 1 et n >= 0}

%%invariant: {r * n! = a!}
%%while n > 0:
%%  r = r * n
%%  n = n - 1
%%{r = n!}

Notons que Python permet d'utiliser les \verb_tuple_ pour effectuer des affectations
\og simultanées \fg. Par exemple après l'instruction
\begin{pythoncode}
In [4]: a, b = 7, 2
\end{pythoncode}
\verb!a! contient 7 et \verb!b! contient 2.
Puisque l'expression de droite est évaluée avant l'affectation, cette construction est
très utile pour échanger le contenu de deux variables.
\begin{pythoncode}
In [5]: a, b
Out[5]: (7, 2)

In [6]: a, b = b, a

In [7]: a, b
Out[7]: (2, 7)
\end{pythoncode}
En pratique, on réservera ces affectations simultanées aux cas où plusieurs affectations
les unes à la suite des autres ne permettent pas d'obtenir un résultat similaire.\\

Notons qu'il est possible de supprimer une variable avec l'instruction \verb_del_, mais
nous nous contenterons d'ignorer les variables dont nous n'avons plus l'utilité.\\

On notera parfois $\mathcal{E}_0, \mathcal{E}_1, \dots$ l'état du système à
différentes étapes de l'exécution de notre code. On utilisera aussi la convention
suivante~: si \verb_a_ est une variable, $a_k$ sera la valeur contenue par cette dernière
quand le système est dans l'état $\mathcal{E}_k$. Par exemple

\begin{pythoncode}
# etat0 {a: a0, b: b0}
In [8]: a = a + b
# etat1 {a: a0 + b0, b: b0} 
\end{pythoncode}

\noindent signifie qu'après notre affectation $a_1=a_0+b_0$ et $b_1=b_0$.

% \begin{exoUnique}
% \exo Montrer qu'il est possible d'échanger deux variables contenant des entiers avec des
%   opérations arithmétiques sans utiliser de troisième variable.
% \end{exoUnique}

\subsection{Entrée, sortie}

Le langage Python permet d'interagir avec l'utilisateur en demandant d'entrer des valeurs avec lesquelles il va travailler puis en affichant les résultats de son calcul.\\

Pour afficher une valeur, on utilise la fonction \verb_print_. On l'utilise pour afficher des chaines de caractères, mais aussi des entiers ou des nombres flottants.


\begin{pythoncode}
In [1]: print("hello, world")
(*@\textcolor{purple}{hello, world}@*)

In [2]: print(2**10)
(*@\textcolor{purple}{1024}@*)
\end{pythoncode}


\noindent La fonction \verb_print_ travaille par effet de bord. Elle a pour effet de
changer l'état du système, à savoir ce qui est affiché par la \emph{console} de l'ordinateur.
Il est essentiel de bien faire la différence entre l'expression \verb_2**10_ qui s'évalue en
1024 et l'appel \verb_print(2**10)_ qui affiche 1024 sur la console. Cet appel 
renvoie \verb_None_, l'unique valeur du type \verb_NoneType_ qui est renvoyée par les 
fonctions travaillant par effet de bord. On ne voit pas cette valeur sur une ligne \verb!Out! car le shell a pour habitude de ne jamais l'afficher. La différence peut paraitre subtile lorsque l'on 
travaille avec Python en mode interactif mais elle existe bien~:
\begin{pythoncode}
In [3]: 2**10
Out[3]: 1024

In [4]: print(2**10)
(*@\textcolor{purple}{1024}@*)
\end{pythoncode}

La fonction \verb_print_ peut être utilisée avec plusieurs valeurs, séparées par des 
virgules~: elles sont affichées sur une même ligne, les unes à la suite des autres.

\begin{pythoncode}
In [5]: nb_eleves = 43

In [6]: print("Il y a", nb_eleves, "éleves dans la classe.")
(*@\textcolor{purple}{Il y a 43 élèves dans la classe.}@*)
\end{pythoncode}

\noindent
Par défaut, un retour à la ligne est automatiquement ajouté après chaque appel à
\verb_print_. Pour éviter cela, on peut utiliser l'option \verb_end_ et remplacer le
caractère \og\verb_\n_\fg, par la chaine de votre choix. Le plus courant est d'utiliser
une chaine vide.

\begin{francois}
\begin{pythoncode}
In [7]: print("hello, ", end="")
   ...: print("world")
(*@\textcolor{purple}{hello, world}@*)
\end{pythoncode}
\end{francois}
\begin{victor}
\begin{pythoncode}
>>> print("hello, ", end="")
... print("world")
hello, world
\end{pythoncode}
\end{victor}

\noindent
Enfin, lorsque vous voulez afficher plusieurs valeurs sur une même ligne en les séparant par
des virgules, Python va ajouter un espace entre chaque valeur. Cela peut être utile, mais 
dans les cas où vous ne le souhaitez pas, vous pouvez construire les chaines de 
caractères à la main.

\begin{francois}
\begin{pythoncode}
In [8]: n = 3

In [9]: print("Sup" + str(n) + " rocks!")
(*@\textcolor{purple}{Sup3 rocks!}@*)
\end{pythoncode}
\end{francois}
\begin{victor}
\begin{pythoncode}
>>> n = 1
>>> print("Sup" + str(n) + " rocks!")
Sup1 rocks!
\end{pythoncode}
\end{victor}

%\noindent Pour plus de précisions sur l'affichage, on peut utiliser la méthode \verb_format_ qui s'emploie sur les chaines de caractères. Son usage élémentaire est le suivant~:
%
%\begin{pythoncode}
%>>> a = 2
%>>> b = 3
%>>> print("{} * {} = {}".format(a, b, a * b))
% 2 * 3 = 6
%\end{pythoncode}
%
%\noindent Cette syntaxe a l'avantage de permettre un formatage précis des données. On peut souhaiter par exemple écrire le résultat de \verb_a * b_ sur deux colonnes et justifier le résultat à droite. Cela peut être utilise pour écrire des tables de multiplication.
%
%\begin{pythoncode}
%>>> print("{} * {} = {:>2})".format(2, 3, 6))
%    print("{} * {} = {:>2})".format(6, 7, 42))
% 2 * 3 =  6
% 6 * 7 = 42
%\end{pythoncode}
%
%\noindent De nombreux autres options sont disponibles pour formatter précisément vos sorties.\\

La fonction \verb_input_ permet quant à elle de demander des valeurs à l'utilisateur.
Quelle que soit la valeur attendue, c'est sous la forme d'une chaine de caractères que
Python la renvoie au programmeur. Il convient donc d'effectuer explicitement une
conversion lorsque l'on souhaite une valeur d'un autre type.

\begin{francois}
\begin{pythoncode}
In [10]: nom = input("Quel est votre nom ? ")
(*@\textcolor{purple}{Quel est votre nom ?}@*) (*@\textcolor{brown}{Teddy Riner}@*)

In [11]: entree = input(nom + ", quelle est votre année de naissance ? ")
(*@\textcolor{purple}{Teddy Riner, quelle est votre année de naissance ?}@*) (*@\textcolor{brown}{1989}@*)

In [12]: annee = int(entree)
In [13]: print("Vous aurez", 2024 - annee, "ans l'année des jeux de Paris.")
(*@\textcolor{purple}{Vous aurez 35 ans l'année des jeux de Paris.}@*)
\end{pythoncode}
\end{francois}
\begin{victor}
\begin{pythoncode}
>>> nom = input("Quel est votre nom ? ")
... entree = input(nom + ", quelle est votre année de naissance ? ")
... annee = int(entree)
... print("Vous aurez", 2024 - annee, "ans l'année des jeux de Paris.")
Quel est votre nom ? Teddy Riner
Teddy Riner, quelle est votre année de naissance ? 1989
Vous aurez 35 ans l'année des jeux de Paris.
\end{pythoncode}
\end{victor}

\noindent Bien qu'elle renvoie une valeur, on dit aussi que la fonction \verb_input_
travaille par effet de bord, car elle attend une entrée de l'utilisateur.
C'est la seule fois dans ce cours où vous verrez cette fonction. Nous sommes en
2023, les téléphones ont des interfaces tactiles et la reconnaissance vocale est
plutôt efficace. Tout cela pour vous dire qu'il vaut mieux laisser gérer l'interface utilisateur
par des personnes maitrisant ces technologies. Comme nous travaillerons en mode
interactif, \verb_print_ et \verb_input_ nous seront de toute façon le plus souvent inutiles et nous vous
demandons de les laisser de côté, sauf si on vous demande explicitement de les utiliser.

% \subsection{Logo}

% Le \textsc{Logo} est un langage de programmation permettant d'apprendre la programmation
% impérative. On l'utilisera donc dans ce cours pour quelques exercices. On dispose
% d'une tortue qui laisse derrière elle une trace dans le sable ce qui permet de
% faire des dessins. Le \textsc{Logo} est disponible en Python par
% l'intermédiaire du module \verb_turtle_. Afin de charger ce module et éviter d'avoir à
% écrire le mot complet \verb_turtle_, on peut utiliser un alias grâce à l'instruction suivante.

% \begin{pythoncode}
% In [1]: import turtle as lg
% \end{pythoncode}

% On dispose alors des commandes suivantes. La commande \verb_lg.forward(n)_ fait avancer
% la tortue d'une distance $n$. La commande \verb_lg.left(theta)_ fait tourner la tortue
% vers la gauche d'un angle de $\theta$ degrés. La commande \verb_lg.right(theta)_ fait
% tourner la tortue vers la droite d'un angle de $\theta$ degrés. Par exemple, les
% commandes suivantes permettent de dessiner un triangle équilatéral.

% \begin{pythoncode}
% In [2]: lg.forward(100)
% In [3]: lg.left(120)
% In [4]: lg.forward(100)
% In [5]: lg.left(120)
% In [6]: lg.forward(100)
% \end{pythoncode}

% \noindent
% La commande \verb_lg.reset()_ permet d'effacer les traces de la tortue et de la replacer
% à son point de départ.

% \begin{exoUnique}
% \exo Écrire une séquence d'instructions qui dessine un carré de côté 100.
% \end{exoUnique}


%  Même si le type d'une liste ne dépend pas des
% éléments qu'ils contient, on utilisera la notation \verb!list[a]! pour désigner une liste
% d'éléments de type \verb!a! dans les docstrings.


%END_BOOK
\end{document}