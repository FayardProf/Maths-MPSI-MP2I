\documentclass{magnolia}

\magtex{tex_driver={pdftex}}
\magfiche{document_nom={Corps, polynomes},
          auteur_nom={Emmanuel Roblet, François Fayard},
          auteur_mail={fayard.prof@gmail.com}}
\magexos{exos_matiere={maths},
         exos_niveau={mpsi},
         exos_chapitre_numero={12},
         exos_theme={Corps, polynomes}}
\magmisenpage{}
\maglieudiff{}
\magprocess


\begin{document}
%BEGIN_BOOK

\magsection{Anneau, corps}

\magsubsection{Anneau}

\exercice{nom={Anneau de Boole}}
Soit $E$ un ensemble. Pour tout $A,B\in\mathcal{P}(E)$, on définit $A\Delta B$ par
\[A\Delta B\defeq(A\setminus B)\cup(B\setminus A).\]
Pour tout $A\in\mathcal{P}(E)$, on définit la fonction caractéristique de $A$ comme
l'application $\mathds{1}_A:A\to\mathbb{F}_2$ définie par
\[\forall x\in E \qsep \mathds{1}_A(x)\defeq
\begin{cases}
\bar{1} & \text{si $x\in A$}\\
\bar{0} & \text{sinon.}
\end{cases}\]
\begin{questions}
\question Montrer que pour tout $A,B\in\mathcal{P}(E)$,
  $\mathds{1}_{A\Delta B}=\mathds{1}_A + \mathds{1}_B$ et
  $\mathds{1}_{A\cap B}=\mathds{1}_A \mathds{1}_B$.
\question En déduire que $(\mathcal{P}(E),\Delta,\cap)$ est un anneau commutatif.
\question Montrer que cet anneau est intègre si et seulement si $E$ possède un unique
  élément.
\end{questions}

\exercice{nom={Fonction définie sur un anneau}}
Soit $(A,+,\times)$ un anneau et $f$ une application de $A$ dans $\RP$ telle
que
\begin{itemize}
\item $\forall x\in A \qsep f(x)=0 \ssi x=0$.
\item $\forall x,y\in A \qsep f(xy)=f(x)f(y)$.
\item $\forall x,y\in A \qsep f(x+y)\leq \max\p{f(x),f(y)}$.
\end{itemize}
Montrer que $\enstq{x\in A}{f(x)\leq 1}$ est un sous-anneau de~$A$.

\magsubsection{Corps}

\exercice{nom={Exemple de corps}}
On définit sur $\R$ deux lois $\oplus$ et $\otimes$ par
\begin{eqnarray*}
\forall x,y\in\R\qsep x\oplus y&\defeq&x+y-1\>,\\
\forall x,y\in\R\qsep x\otimes y&\defeq&x+y-xy\>.
\end{eqnarray*}
Montrer que $(\R,\oplus,\otimes)$ est un corps.

\exercice{nom={Morphisme de corps}}
Montrer qu'un morphisme de corps est toujours injectif. Trouver un morphisme
d'anneaux non injectif.

\begin{sol}
Soient $K_1, K_2$ des corps et $f$ un morphisme de $K_1$ dans $K_2$. Comme $f$ est un morphisme de corps, $f(1)=1$. Supposons $f(a)=f(b)$. On a alors $f(a-b)=0$. Si jamais on avait $a \neq b$, on aurait $a-b$ inversible (notons $c$ son inverse) et par suite $1=f(1)=f(c(a-b))=f(c)f(a-b)=f(c)\times 0=0$, ce qui n'est pas possible (dans un corps, $0$ et $1$ sont distincts). On a donc nécessairement $a=b$. Cela signifie bien que $f$ est injective.
\end{sol}

\exercice{nom={Extension quadratique}}
Soit $\alpha\in\N$ tel que $\sqrt{\alpha}\not\in\Q$. On pose 
\[\Q(\sqrt{\alpha})\defeq\ensim{a+b\sqrt{\alpha}}{(a,b)\in\Q^2}\]
\begin{questions}
\question Soit $x\in\Q\p{\sqrt{\alpha}}$. Montrer qu'il existe un unique
  couple $\p{a,b}\in\Q^2$ tel que $x=a+b\sqrt{\alpha}$.
\question Montrer que $\Q(\sqrt{\alpha})$ est un sous-corps de $(\R,+,\times)$.
% \enonce Dans la suite, on suppose que $\alpha=2$ et on note $\K$ le corps
%   $\Q(\sqrt{2}) $.
\question Pour $x=a+b\sqrt{\alpha}\in\Q(\sqrt{\alpha})$, on pose
  $\overline{x}=a-b\sqrt{\alpha}$; on l'appelle le conjugué de $x$. Montrer que
  l'application  $x\mapsto \overline{x}$ est bien définie et est un
  automorphisme du corps $\Q(\sqrt{\alpha})$.
\question Montrer que l'automorphisme construit à la question précédente est
  le seul automorphisme non trivial de $\Q(\sqrt{\alpha})$.
% \question Pour $x\in\K$, on pose $N(x)=x\overline{x}$. Pour tous  $x$ et $y$
%   dans~$\K$, montrer que 
%   \begin{itemize}
%   \item $N(x)=0$ si et seulement si $x=0$ ;
%   \item $N(xy)=N(x)N(y)$.
%   \end{itemize}
%   Que dire de l'application $N$ de $(\Ks,\times)$ dans $(\Qs,\times)$~?
%   Calculer $N(1/x)$ lorsque $x$ est un élément non nul de~$\K$.
% \question Soit $A=\enstq{a+b\sqrt{2}}{(a,b)\in\Z^2}$. Montrer que $A$ est un
%   sous-anneau intègre du corps $\K$. Montrer que $x\in A$ admet un inverse dans
%   $A$ si et seulement si $|N(x)|=1$ ; que vaut alors l'inverse de $x$~?
%    Donner un exemple non trivial d'élément de~$A$ inversible dans $A$.
\end{questions}

% \exercice{nom={Théorème de Wilson}}
% Soit $p$ un nombre premier.
% \begin{questions}
% \question Montrer que l'application $\phi$ de $(\Z/p\Z)^*$ dans $(\Z/p\Z)^*$
%   qui à $x$ associe $1/x$ est une bijection. Quels sont les points fixes de
%   cette bijection~?
% \question En déduire que $(p-1)!\equiv -1\ [p]$.
% \end{questions}

% \begin{sol}
% \begin{questions}
% \question C'est une bijection d'inverse elle-même. Elle a pour point fixe $1$ et $-1$.
% \question On a un nombre pair d'éléments dans ce produit. En regroupant chaque élément avec son inverse, il reste $1$ et $-1$ dont le produit fait bien $-1$.
% \end{questions}
% \end{sol}

\magsection{Espace vectoriel, algèbre}
\magsubsection{Espace vectoriel}
\magsubsection{Algèbre}


\magsection{L'algèbre $\polyK$}

\magsubsection{Définition}

\exercice{nom={Produit}}
On note $\mathcal{A}$ l'ensemble des polynômes $P\in\polyR$ tels qu'il existe
$a_0,\ldots,a_n\in\RP$ tels que
\[P=\sum_{k=0}^n (-1)^k a_k X^k.\]
Montrer que $\mathcal{A}$ est stable par produit.


\magsubsection{Substitution}
\magsubsection{Degré d'un polynôme}

\exercice{nom={Équations sur $\polyR$}}
Déterminer l'ensemble des $P\in\polyR$ tels que
\[P(X^2)=(X^2+1)P(X).\]
On commencera par effectuer une analyse et un cherchera des informations sur
le degré de $P$.

% On trouve P = lambda(X^2 - 1)

\exercice{nom={Division euclidienne}}
Soit $n\in\N$. Calculer le reste de la division euclidienne
\begin{questions}
\question de $X^n(X+1)^2$ par $(X-1)(X-2)$.
\question de $X^n$ par $(X-1)^2(X+1)$.
% \question de $X^{2n}$ par $(X^2+1)^n$.
\question de $(X+1)^{2n+1} - X^{2n+1}$ par $X^2+X+1$.
\end{questions}

\magsubsection{Racines, fonctions polynôme}

\exercice{nom={Exercice}}
\begin{questions}
\question Déterminer tous les polynômes $P\in\polyR$ tels que
  \[\forall n\in\N\qsep P(n)=n^2.\]
\question Déterminer tous les polynômes $P\in\polyR$ tels que
  \[\forall n\in\N\qsep P(n)=n^2+(-1)^n.\]
\end{questions}


\exercice{nom={Polynômes de \nom{Tchebychev}}}
On définit la suite de polynômes $\p{T_n}$ par
\[T_0\defeq 1, \quad T_1\defeq X \et \forall n\in\N \qsep T_{n+2}\defeq 2XT_{n+1}-T_n.\]
\begin{questions}
\question Calculer les polynômes $T_n$ pour $n\in\intere{0}{5}$.
\question Calculer le degré de $T_n$ et son coefficient dominant.
\question Montrer que pour tout $n\in\N$, $T_n$ est l'unique polynôme
  tel que
  \[\forall x\in\R \qsep T_n\p{\cos x}=\cos\p{nx}.\]
\question En déduire les racines de $T_n$.
\question En dérivant deux fois la relation obtenue dans la question 3, montrer que
  \[\forall n\in\N\qsep (X^2-1)T_n''+XT_n'-n^2T_n=0.\]
\question Déterminer une expression explicite de $T_n$ en exploitant la formule de
  Moivre.
\end{questions}
\begin{sol}
$\quad$
\begin{questions}
\question On trouve $P_0=1$, $P_1=X$, $P_2=2X^2-1$, $P_3=4X^3-3X$,
  $P_4=8X^4-8X^2+1$, $P_5=16X^5-20X^3+5X$.
\question Faire une récurrence double sur $n$.
\question Ce sont les
  \[\cos\p{\frac{\pi}{2n}\p{1+2k}}\]
  pour $k\in\intere{0}{n-1}$. Ce sont des racines et il faut démontrer par
  récurrence sur $n$ que le degré de $P_n$ est $n$.
\end{questions}
\end{sol}

\exercice{nom={Exercice}}
On note $\mathcal{A}$ l'ensemble des fonctions $f\in\mathcal{F}(\R,\R)$ telles qu'il
existe $\lambda_0,\ldots,\lambda_n\in\R$ tels que
\[\forall x\in\R\qsep f(x)=\sum_{k=0}^n \lambda_k \e^{kx}.\]
\begin{questions}
\question Montrer que $\mathcal{A}$ est une sous-algèbre de $\mathcal{F}(\R,\R)$.
\question Montrer que $\mathcal{A}$ est intègre.
\question Déterminer $U_{\mathcal{A}}$. 
\end{questions}


\exercice{nom={Polynômes de Lagrange}}
Soit $n\geq 2$, $x_1,\ldots,x_n\in\K$ deux à deux distincts et $L_1,\ldots,L_n$ les
polynôme de Lagrange associés. Simplifier les sommes
\[\sum_{i=1}^n L_i \quad\text{et}\quad \sum_{i=1}^n x_i L_i.\]

\exercice{nom={Polynômes de Lagrange}}
On note $L_0,\ldots,L_n$ les polynômes de Lagrange de $0,\ldots,n$.
\begin{questions}
\question Pour tout $k\in\intere{0}{n}$, exprimer le coefficient dominant de $L_k$ au
  moyen de factorielles.
\question Exprimer de deux manières différentes l'unique polynôme $P\in\polyR$ de degré
  inférieur ou égal à $n$ tel que
  \[\forall k\in\intere{0}{n}\qsep P(k)=k^n.\]
\question En déduire une simplification de
  \[\sum_{k=0}^n \binom{n}{k}(-1)^{n-k} k^n.\]
\end{questions}

% Cela donne n!

\exercice{nom={Exercice}}
Soit $P\in\polyC$.
\begin{questions}
\question Donner une condition nécessaire et suffisante sur les coefficients de $P$
  pour que \[\forall x\in\R\qsep P(x)\in\R.\]
\question Donner une condition nécessaire et suffisante sur les coefficients de $P$
  pour que \[\forall x\in\Q\qsep P(x)\in\Q.\]
  \emph{On pourra utiliser les polynômes de Lagrange.}
\end{questions}


\magsubsection{Polynôme dérivé}

\exercice{nom={Exercice}}
Soit $n\in\N$ et $k\in\intere{0}{n}$. On pose
\[S\defeq\sum_{i=k}^n \binom{i}{k} \quad\text{et}\quad
  P\defeq\sum_{i=k}^n (X+1)^i.\]
\begin{questions}
\question Exprimer $S$ en fonction de $P^{(k)}(0)$.
\question Simplifier $(X+1)P - P$, puis dériver $k+1$ fois la
  relation obtenue.
\question En déduire une expression simple de $S$.
\end{questions}

\exercice{nom={Équations sur $\polyR$}}
\begin{questions}
\question Déterminer les polynômes $P\in\polyR$ tels que
  \[P\p{2X}=P'(X)P''(X).\]
\question Déterminer les polynômes $P\in\polyR$ tels que
  \[X\p{X+1}P''+\p{X+2}P'-P=0.\]
\end{questions}
\begin{sol}
$\quad$
\begin{questions}
\question Regarder le degré du polynôme. On trouve qu'il est inférieur ou égal
  à 3. Finalement, les solutions sont $0$ et $\frac{4}{9}X^3$.
\question Regarder le coefficient dominant. Cela implique que le degré du
  polynôme est inférieur ou égal à 1. On trouve finalement les solutions
  suivantes~: les $\lambda(X+2)$ avec $\lambda\in\R$.
\end{questions}
\end{sol}



\exercice{nom={Coefficients binomiaux}}
On donne un entier $n\geq 1$.
\begin{questions}
\question Pour $a,b\in\R$, calculer la dérivée $n$-ième du polynôme
  $P=\p{X-a}^n\p{X-b}^n$.
\question En déduire une expression simplifiée de la somme
  \[S=\sum_{k=0}^n \binom{n}{k}^2\]
\end{questions}


\exercice{nom={Primitive}}
Soit $P\in\polyC$ un polynôme de degré inférieur ou égal à $n$. Montrer que
\[Q\defeq \sum_{k=0}^n \frac{(-1)^k P^{(k)}(X)}{(k+1)!}X^{k+1}\]
est l'unique primitive de $P$ qui s'annule en 0.

% \exercice{nom={Calcul de degré}} %%Tout-en-un (2ème édition), p.749
% Soit $P$ un polynôme de $\polyR$. Déterminer le degré du polynôme
% $$P(X+1)-P(X)$$
% en fonction du degré de $P$.













% \exercice{nom={Une base de $\polyK[n]$}}
% Soit $n\in\N$. On définit pour tout $k\in\intere{0}{n}$~:
% \[P_k=X^k\p{1-X}^{n-k}\]
% Montrer que $P_0,\ldots,P_n$ est une base de $\polyK[n]$.



% \exercice{nom={Polynômes de Hilbert}}
% Soit $n\in\N$. Pour tout $k\in\intere{0}{n}$, on définit le polynôme $P_k$ par~:
% \[P_k=\frac{1}{k!}\prod_{l=0}^{k-1} \p{X-l}\]
% \begin{questions}
% \question Montrer que la famille $P_0,\ldots,P_n$ est une base de $\polyR[n]$.
% \question On définit l'application $\Delta$ par~:
%   \[\dspappli{\Delta}{\polyR[n]}{\polyR[n]}{P}{P\p{X+1}-P(X)}\]
%   \begin{questions}
%   \question Montrer que $\Delta$ est linéaire.
%   \question Pour tout $k\in\intere{0}{n}$, calculer $\Delta\p{P_k}$.
%   \question En déduire $\ker \Delta$ et $\im \Delta$.
%   \end{questions}
% \question
%   \begin{questions}
%   \question Montrer que~:
%     \[\forall P\in\polyR[n] \quad P=\sum_{k=0}^n \p{\Delta^k P}\p{0}P_k\]
%   \question Montrer que~:
%     \[\forall k\in\intere{0}{n} \quad \forall l\in\Z \quad P_k(l)\in\Z\]
%   \question En déduire que si $P\in\polyR[n]$~:
%     \[\cro{\forall l\in\Z \quad P(l)\in\Z} \quad \ssi\]
%     \[\cro{\text{Les coordonnées de $P$ dans la base $P_0,\ldots,P_n$ sont
%       entières.}}\]
%   \end{questions}
% \end{questions}







% \exercice{nom={Bases de $\polyR[n]$}}
% \begin{questions}
% \question Soit $n\in\N$ et $P_0,\ldots,P_n$ $n+1$ polynômes tels que :
%   $$\forall k\in\intere{0}{n} \quad \deg P_k=k$$
%   Montrer que la famille $P_0,\ldots,P_n$ est une base de $\polyR[n]$.
% \question Soit $\alpha\in\R$. Montrer que la famille
%   $\p{(X-\alpha)^n}_{n\in\N}$ est une base de $\polyR$.
% \end{questions}


% \exercice{nom={Inversion d'une matrice}}
% Soit $n\in\N$. On considère l'application $\phi$ qui au polynôme $P$
% associe le polynôme $P(X+1)$.
% \begin{questions}
% \question Montrer que $\phi$ est un isomorphisme de $\polyC[n]$ et déterminer
%   $\phi^{-1}$.
% \question Expliciter la matrice de $\phi$ dans la base canonique de
%   $\polyC[n]$.
% \question Montrer que cette matrice est inversible et calculer son inverse.
% \end{questions}


%END_BOOK


\end{document}

