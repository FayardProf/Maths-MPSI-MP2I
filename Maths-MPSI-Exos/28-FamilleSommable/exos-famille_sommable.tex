\documentclass{magnolia}

\magtex{tex_driver={pdftex}}
\magfiche{document_nom={Exercices sur les développements limités},
          auteur_nom={François Fayard},
          auteur_mail={fayard.prof@gmail.com}}
\magexos{exos_matiere={maths},
         exos_niveau={mpsi},
         exos_chapitre_numero={22},
         exos_theme={Famille sommable}}
\magmisenpage{}
\maglieudiff{}
\magprocess

\begin{document}
%BEGIN_BOOK

\magsection{Famille sommable}
\magsubsection{Famille sommable de réels positifs}

\exercice{nom={Exercice}}
Soit $\alpha\in\R$. Calculer
\[\sum_{n\in\Ns} \sum_{k\geq n} \frac{1}{k^\alpha}, \qquad
  \sum_{m\in\Ns}\sum_{n\in\Ns} \frac{1}{(m+n)^{\alpha}}.\]

\magsubsection{Famille sommable d'éléments de $\K$}

\exercice{nom={Exercice}}
Pour tout $k\in\N$, on pose
\[L_k\defeq \prod_{i=0}^{k-1} (X-i).\]
\begin{questions}
\question Montrer que pour tout $P\in\polyC$, la famille
  \[\p{\frac{P(n)}{n!}}_{n\in\N}\]
  est sommable.
\question Calculer
  \[\sum_{n\in\N} \frac{L_k(n)}{n!}\]
  pour tout $k\in\N$.
\question En déduire
  \[\sum_{n\in\N} \frac{n^4}{n!}.\]
\end{questions}

\exercice{nom={Exercice}}
\begin{questions}
\question Prouver la sommabilité et calculer
\[\sum_{n\in\N} \sum_{k\geq n} \frac{1}{k!}, \qquad
  \sum_{n\in\Ns} \sum_{k\geq n} \frac{(-1)^k}{k^3}, \qquad
  \sum_{n\in\Ns} (-1)^n \sum_{k\geq n} \frac{1}{k^3}.\]
\question Soit $z\in\C$ tel que $\abs{z}<1$. Prouver la sommabilité et calculer
  \[\sum_{n\in\N} \frac{z^{2^n}}{1-z^{2^{n+1}}}.\]
\end{questions}

\exercice{nom={Exercice}}
À quelles conditions nécessaires et suffisantes sur $a,b,z\in\C$ les familles suivantes
sont-elles sommables~?
\[\p{\frac{z^p}{q!}}_{p,q\in\N}, \qquad
  \p{\frac{a^p b^q}{p!q!}}_{p,q\in\N}, \qquad
  \p{\frac{q^p z^p}{p!q!}}_{p,q\in\N}, \qquad
  \p{\binom{p+q}{p}z^{p+q}}_{p,q\in\N}.\]

\exercice{nom={Exercice}}
\begin{questions}
\question Soit $n\in\N$. Décomposer en éléments simples la fraction rationnelle
  \[F_n\defeq \frac{1}{X(X+1)\cdots(X+n)}.\]
\question En déduire que pour tout $z\in\C\setminus(-\N)$
  \[\sum_{n\in\N} \frac{1}{z(z+1)\cdots(z+n)}=\e\sum_{n\in\N} \frac{(-1)^n}{n!(z+n)}.\]
\end{questions}










%[$\star$]
%Quels sont les réels dont le développement décimal est périodique à partir d'un certain rang ?
%



%END_BOOK

\end{document}
