\documentclass{magnolia}

\magtex{tex_driver={pdftex}}
\magfiche{document_nom={Exercices sur le dénombrement},
          auteur_nom={François Fayard},
          auteur_mail={fayard.prof@gmail.com}}
\magexos{exos_matiere={maths},
         exos_niveau={mpsi},
         exos_chapitre_numero={10},
         exos_theme={Dénombrement}}
\magmisenpage{}
\maglieudiff{}
\magprocess

\begin{document}
%BEGIN_BOOK
\magsection{Cardinal}
\magsubsection{Équipotence}

\exercice{nom={Non dénombrabilité de $[0,1]$}}
Le but de cet exercice est de démontrer que $\interf{0}{1}$ n'est pas dénombrable,
c'est-à-dire qu'il n'existe pas de surjection de $\N$ dans $\interf{0}{1}$. On raisonne
par l'absurde et on suppose qu'il existe une suite $(u_n)$ d'éléments de $[0,1]$ telle que
\[\forall x\in[0,1]\qsep \exists n\in\N\qsep u_n=x.\]
\begin{questions}
\question Construire deux suites $(a_n)$ et $(b_n)$ de réels telles que
  \begin{itemize}
  \item $a_0= 0 \et b_0=1.$
  \item $\forall n\in\N \qsep a_n\leq a_{n+1}< b_{n+1}\leq b_n.$
  \item $\forall n\in\N \qsep b_{n+1}-a_{n+1}=\frac{1}{3}\p{b_n-a_n}.$
  \item $\forall n\in\N \qsep u_n \not\in \interf{a_{n+1}}{b_{n+1}}.$
  \end{itemize}
\question Montrer que les suites $(a_n)$ et $(b_n)$ convergent vers une limite commune 
  $l\in[0,1]$, puis que
  \[\forall n\in\N \qsep u_n\neq l.\]
  Conclure.
\end{questions}


\begin{sol}
Tout roule bien. En particulier, on fait de la trichotomie, car avec de la dichotomie, $u_n$ pourrait être le point milieu à l'étape $n$ de la construction. Pour la fin, fixons $n_0\in \N$, $u_{n_0}\notin \interf{a_{n_0}}{b_{n_0}}$ donc $u_{n_0}$ n'est pas dans l'intersection donc n'appartient pas à $\set{\ell}$. Ainsi, $u_{n_0}\neq \ell$ et ce quelque soit $n_0$.\\

Pour conclure, pour toute suite à valeurs dans $[0,1]$, c'est-à-dire toute fonction de $\N$ dans $[0,1]$, on a construit un élément jamais atteint. Donc elle n'est pas surjective.
\end{sol}

\magsubsection{Ensemble fini, cardinal}

\exercice{nom={Ensemble d'entiers}}
Étant donnés 51 entiers compris entre 1 et 100, montrer qu'il en existe toujours 2 consécutifs.

\exercice{nom={Théorème d'approximation de \nom{Dirichlet}}}
\begin{questions}
\question Soit $x\in\R$ et $n\in\Ns$. Pour tout $k\in\intere{0}{n}$, on pose
  $\delta_k\defeq kx-\ent{kx}$. En appliquant le principe des tiroirs aux
  réels $\delta_k$, montrer le théorème d'approximation de \nom{Dirichlet}
  \[\exists p\in\Z\qsep \exists q\in\intere{1}{n}\qsep \abs{x-\frac{p}{q}}<\frac{1}{nq}.\]
\question Soit $x\in\R\setminus\Q$.
  \begin{questions}
  \question Montrer qu'il existe une infinité de $(p,q)\in\Z\times\Ns$ tels
    que
    \[\abs{x-\frac{p}{q}}<\frac{1}{q^2}.\]
  \question Montrer qu'il existe une infinité de $p\in\Z$ pour lesquels
    \[\exists q\in\Ns\qsep \abs{x-\frac{p}{q}}<\frac{1}{q^2}.\]
  \end{questions}
\question On admet que $\pi$ est irrationnel. Dans ces conditions
  $\sin n\neq 0$ pour tout $n\in\Ns$ et on peut poser
  \[\forall n\in\Ns\qsep u_n\defeq\frac{1}{n\sin n}.\]
  On souhaite montrer que la suite $(u_n)$ n'admet pas de limite. On raisonne
  par l'absurde et on suppose que $(u_n)$ admet une limite
  $l\in\Rbar$.
  \begin{questions}
  \question Montrer que $l=0$.
  \question Obtenir une contradiction en appliquant les résultats de
    la question 2.b au réel $\pi$.
  \end{questions}

\end{questions}
% Soit $x\in\R$ et $q\geq 2$. Montrer qu'il existe $a\in\Z$ et
%   $b\in\intere{1}{q-1}$ tels que
%   \[\abs{bx-a}\leq \frac{1}{q}.\]
%   \begin{sol}
%   En effet, on considère 0, 1 et
%   les $\{bx\}$ pour $b\in\intere{1}{q-1}$ où $\{y\}$ désigne la distance de $y$
%   à sa partie entière. D'après la remarque précédente, il
%   existe deux éléments deux à deux distincts dont la distance est inférieure à
%   $1/q$ ce qui permet de conclure. (ENS Ulm/Lyon 1995) 
%   \end{sol}

\exercice{nom={7 nombres réels}}
Soit sept nombres réels $x_1,\ldots,x_7$. Montrer qu'il existe deux indices $i$
et $j$ distincts tels que 
\[0<\frac{x_i-x_j}{1+x_ix_j}<\frac{1}{\sqrt{3}}\>\cdot\]
\emph{Rappelez vous que la trigonométrie se cache même aux endroits où on ne l'attend pas}.
% {\it On pourra introduire, pour tout $k\in\intere{1}{7}$, $\theta_k$ tel que
% $\tan \theta_k=x_k$.}

\magsection{Dénombrement}
\magsubsection{Dénombrement élémentaire}


\exercice{nom={Couples dans le plan}}
Combien y a-t-il de couples $(i,j)$
\begin{questions}
\question dans $\intere{1}{n}^2$ pour lesquels $i+j=n$~?
\question dans $\intere{1}{n}^2$ pour lesquels $i<j$~?
\question dans $\intere{1}{n}\times\intere{1}{2n}$ pour lesquels $i<j$~?
\question dans $\intere{1}{n}^2$ pour lesquels $\abs{i-j}\leq 1$~?
\end{questions}

\exercice{nom={Autour du crible}}
Une certaine ville compte $17\ 500$ actifs dans sa population. À l'issue
d'un recensement, on a obtenu les informations suivantes sur ces $17\ 500$
actifs~:
\begin{itemize}
\item 4 actifs sur 7 sont des femmes et 6 d'entre elles sur 10 ont voté
  aux dernières municipales.
\item 3 actifs sur 5 ont voté aux dernières municipales et $40 \%$ de ces
  personnes sont au chômage.
\item Le chômage touche 1 actif sur 4 et $60 \%$ des demandeurs d'emploi
  sont des femmes.
\item $60 \%$ des femmes au chômage ont voté aux dernières municipales.
\end{itemize}
Combien d'hommes qui ne sont pas au chômage sont restés chez eux le jour des
élections municipales~?



\magsubsection{Arrangement, combinaison} 



\exercice{nom={Anagrammes}}
Dénombrer les anagrammes des mots suivants
\begin{center}
{\sc couvert, couteau, fourchette}.
\end{center}
\begin{sol}
Pour couvert, il y en a 7!. Pour couteau, il y en a 7!/2!. Pour fourchette, il
y en a 10!/(2! 2!).
\end{sol}

\exercice{nom={Le livreur}}
Un livreur doit distribuer des colis à cinq personnes A, B, C, D, E. Combien y
a-t-il de trajets possibles~? S'il souhaite livrer A avant B et C, combien
y a-t-il de trajets possibles~?
\begin{sol}
Il y a 5! trajets possibles. Si on livre $A$ avant $B$ et $C$, il y a
$2\times 4\times 5$ possibilités.
\end{sol}

\exercice{nom={Surjections d'un ensemble fini dans un autre}}
Quels que soient $n,p\in\N$, on note $S_{p,n}$ le nombre de surjections d'un
ensemble à $p$ éléments dans un ensemble à $n$ éléments.
\begin{questions}
\question On suppose que $p\leq n$. Que vaut $S_{p,n}$ ?
\question Calculer $S_{n+1,n}$ et $S_{p,2}$.
\question Montrer que
  \[\forall p,n \in \N \qsep n^p=\sum_{i=0}^n \binom{n}{i}S_{p,i}.\]
\end{questions}
\begin{sol}
$\quad$
\begin{questions}
\question Si $p<n$ il n'y a pas de surjection, donc $S_{p,n}=0$.
\question On a $S_{n+1,n}=\binom{n+1}{2}n!$. Si $p\leq 1$, alors $S_{p,n}=0$.
  Sinon, il faut choisir les antécédents d'un des élément. Il faut donc
  prendre une partie non vide de l'ensemble de départ différente de l'ensemble
  en entier. Il y a donc $2^p-2$ possibilités.
\question Considérer l'ensemble des fonctions d'un ensemble à $p$ éléments dans
  un ensemble à $n$ éléments. Faire ensuite une partition selon le cardinal de
  l'image de $f$.
\end{questions}
\end{sol}

\exercice{nom={Partitions d'un ensemble}}
Quels que soient $n,p\in\N$, on note $P_{n,p}$ le nombre de partitions d'un ensemble
de cardinal $np$ en $n$ parties à $p$ éléments. Montrer que
\[\forall n,p\in\N\qsep P_{n+1,p}=\frac{1}{n+1}\binom{(n+1)p}{p}P_{n,p}.\]
En déduire $P_{n,p}$.

\exercice{nom={Tours de Hanoï}}
Le jeu des tours de Hanoï se compose de trois tiges sur lesquelles on peut
empiler $n$ disques deux à deux distincts ($n\ge1$). Initialement, les $n$
disques sont empilés sur la première tige, par ordre décroissant de
taille, du bas vers le haut. Le but du jeu est de transporter la tour
complète sur une autre tige par une suite de mouvements consistant à
déplacer un disque à la fois, et en respectant les deux règles suivantes :
\begin{itemize}
\item on ne peut ôter d'une tige que le disque se trouvant au sommet de la
  pile ;
\item on ne peut empiler un disque sur une tige que si elle est vide ou bien si
  l'on pose le disque en question sur un autre plus grand.
\end{itemize}

\bigskip
\setlength{\unitlength}{2pt}
\begin{center}
\begin{picture}(130,15)
\put(0,0){\line(1,0){130}}
\put(5,2){\line(1,0){30}}
\put(7,4){\line(1,0){26}}
\put(9,6){\line(1,0){22}}
\put(11,8){\line(1,0){18}}
\put(13,10){\line(1,0){14}}
\put(5,0){\line(0,1){2}}
\put(35,0){\line(0,1){2}}
\put(7,2){\line(0,1){2}}
\put(33,2){\line(0,1){2}}
\put(9,4){\line(0,1){2}}
\put(31,4){\line(0,1){2}}
\put(11,6){\line(0,1){2}}
\put(29,6){\line(0,1){2}}
\put(13,8){\line(0,1){2}}
\put(27,8){\line(0,1){2}}
\put(19,10){\rule{2mm}{5mm}}
\put(59,0){\rule{2mm}{15mm}}
\put(99,0){\rule{2mm}{15mm}}
\end{picture}
\end{center}

\bigskip
\noindent
Notons $a_n$ le nombre minimal de mouvements nécessaires au transport de
la tour initiale de $n$ disques.
\begin{questions}
\question Montrer que $a_1=1$ et $a_2=3$. 
\question Établir une relation entre $a_{n+1}$ et $a_n$ pour tout $n\ge 1$.
\question Montrer que $a_n=2^n-1$ pour tout $n\ge 1$. 
\end{questions}





\exercice{nom={Exercice}}
Sur une étagère, on range les $n$ tomes d'une encyclopédie. Combien y a-t-il de manières de les ranger tout en étant sûr que le tome 1 et le tome 2 sont côte à côte et dans cet ordre ?


\exercice{nom={Exercice}}%http://images.math.cnrs.fr/Juin-2016-3e-defi.html
De combien de façons différentes peut-on ranger les nombres $1,2,\dots,n$ si l'on veut que le produit de deux nombres voisins soit toujours pair ?



\exercice{nom={Exercice}}%Monier-Haberer-Lardon Livre d'exos MPSI
Une urne contient 10 boules numérotées de $1$ à $10$. La boule 1 est jaune,
les boules 2 et 3 sont bleues, les boules 4,5,6 sont rouges et les boules 7,8,9,10
sont vertes. On tire dans l'urne successivement et avec remise 5 boules. Le résultat
est donc la liste ordonnée des cinq numéros des boules tirées. Déterminer le nombre de
résultats
\begin{enumerate}
\item en tout,
\item pour lesquels les cinq boules sont toutes de la même couleur,
\item pour lesquels les quatre couleurs apparaissent parmi les cinq boules,
\item pour lesquels la boule numéro 8 a été tirée et exactement deux des boules tirées sont rouges.
\end{enumerate}
\begin{sol}
\begin{questions}
\question Il y a $10^5$ tirages.
\question Il y a $1+2^5+3^5+4^5=1300$ tirages.
\question 4 couleurs différents. Si il y a deux jaunes, cela fait $2\times 3\times 4\times 5!/2=1440$.
  Si il y a deux bleues, cela fait $3\times 4\times 5!+2\times 3\times 4\times 5!/2=2880$.
  Si il y a deux rouges, 4320. Si il y a deux verts, 5760. Au total, cela donne $14400$.
\question Si on tire 3 fois 8 et 2 rouges, on a $(3 parmi 5)3^2=90$. Si on tire 2 fois 8 et 2 rouges,
  $(2 parmi 5)(2 parmi 3)3^2 6^1=1620$. Si on tire une fois 8 et 2 rouges $9720$.
  Cela fait au total 11430.\\
  Autre méthode~: On choisit où placer les rouges. Il y a 2 parmi 5 possibilités. Ensuite, on choisit
  les rouges, il y a $3^2$ possibilités. Il faut ensuite trouver les 3-listes avec au moins un 8. On peut
  utiliser le complémentaire. On trouve
  $(2 parmi 5) 3^2(7^3-6^3)$.
\end{questions}
\end{sol}




\exercice{nom={Exercice}}%Monier-Haberer-Lardon Livre d'exos MPSI
On dispose de trois urnes notées $A,B,C$ et de six boules. On répartit les six boules dans
les trois urnes (chaque urne peut contenir de $0$ à $6$ boules). Une répartition est une
liste ordonnée de trois nombres indiquant le nombre de boules contenues dans les urnes
$A,B,C$. Par exemple, la répartition $(2,4,0)$ indique que l'urne $A$ contient $2$ boules,
l'urne $B$ en contient $4$ et l'urne $C$ est vide.
Déterminer le nombre de répartitions
\begin{enumerate}
\item en tout,
\item telles que l'urne $A$ est vide,
\item telles que l'urne $A$ est la seule urne vide,
\item telles qu'une urne et une seulement est vide,
\item telles qu'aucune urne est vide,
\item telles qu'au moins une urne est vide.
\end{enumerate}
\begin{sol}
\begin{questions}
\question En utilisant la représentation $OOOIOOOI$, on trouve $\choose{8}{2}=28$ possibilités.
\question En utilisant la représentation $OOOIOOO$, on trouve $\choose{7}{1}=7$ possibilités.
\question On enlève les deux possibilités $B$ vide et $C$ vide. On a $7-2=5$ possibilités.
\question On a $3\times 5=15$ possibilités.
\question Il y a 10 possibilités.
\question Complémantaire du cas précédent. $28-10=18$.
\end{questions}
\end{sol}



\exercice{nom={Exercice}}
On considère un quadrillage de $n$ lignes et $m$ colonnes. On part de la case en haut à gauche pour arriver à la case en bas à droite. Les seuls mouvements possibles sont de se déplacer d'une case à droite ou d'une case en bas. 
Combien existe-t-il de chemins ?


%\exercice{nom={Exercice}}
%Sur un échiquier (donc 8 lignes et 8 colonnes), il ne reste plus que les 2 tours blanches et les deux noires. Combien de positions possibles pour que les tours ne se menacent pas entre-elles ?
%


\exercice{nom={Exercice}}%vincent
Un domino est un rectangle constitué de deux carrés, chacun comportant entre 0 et 6 points. 
% \begin{center}
% \includegraphics[width=5cm]{domino.jpg}
% \end{center}
\begin{enumerate}
\item Combien existe-t-il de dominos ?
\item Combien de paires peut-on former avec des dominos ayant un nombre en commun~?
\end{enumerate}
\begin{sol}
\begin{questions}
\question Si on ordonne les chiffres des dominos, il y en a $7\times 7=49$. Si on n'ordonne pas, il y en a $(49-7)/2+7=28$.
\question On choisit le chiffre en commun. Puis on choisit 2 chiffres parmi 7. On trouve 147.
\end{questions}
\end{sol}


\exercice{nom={Exercice}}
Soit $E$ un ensemble de cardinal $2n$. On appelle partition de $E$ en paires tout ensemble 
$\{ \{a_1,b_1 \}, \dots, \{ a_n, b_n \} \}$ avec pour tout $i \in \intere{1}{n}$, 
$a_i \neq b_i$ et $\p{\{a_1,b_1 \}, \dots, \{ a_n, b_n \}}$ est une partition de $E$.
Dénombrer les  partitions de $E$ en paires.


\exercice{nom={Exercice}}
Soit $E$ une partie de cardinal $n\in\Ns$. Dénombrer de deux manières différentes les couples
$(a,A)\in E\times\mathcal{P}(E)$ tels que $a\notin A$. En déduire
\[\forall n\in\Ns \qsep \sum_{k=1}^n k \binom{n}{k}=n 2^{n-1}.\]
\begin{sol}
On choisit d'abord $a\in A$. On a $n$ possibilités. On choisit ensuite une partie dans ce qui reste. On a $2^{n-1}$ possibilités.

\end{sol}

\exercice{nom={Compter les matrices}}
Combien existe-t-il de matrices de $\mat{q,p}{\R}$ dont les entrées sont
\og 0 \fg ou \og 1 \fg et
\begin{questions}
\question dont chaque ligne contient exactement un coefficient \og 1 \fg ~?
\question dont chaque ligne contient exactement deux coefficients \og 1 \fg ~?
\question dont chaque ligne et chaque colonne contiennent exactement un
  coefficient \og 1 \fg (on suppose ici $q=p$)~?
\end{questions}

\exercice{nom={Le crible}}
Soit $n\in\Ns$.
\begin{questions}
\question Montrer que
  \[\sum_{k=0}^n (-1)^{n-k}\binom{n}{k}k^n = n!.\]
  On pourra appliquer la formule du crible aux ensembles
  $\intere{1}{n}\setminus\ens{k}$ pour $k\in\intere{1}{n}$.
\question Simplifier de même
  \[\sum_{k=0}^n (-1)^{n-k}\binom{n}{k}k^p\]
  pour tout $p\in\intere{0}{n}$.
\end{questions}

%\exercice{nom={Exercice}}
%On dit qu'une partie $A$ de $\intere{1}{n}$ vérifie la propriété $(\star)$ ssi pour tout $a \in A$, $a+1$ et $a+2 \notin A$. On note $u_n$ le nombre de parties de $\intere{1}{n}$ vérifiant la propriété $(\star)$.
%
%Déterminer une relation de récurrence vérifiée par la suite $(u_n)$. En déduire la forme d'un équivalent de $u_n$.
%

\exercice{nom={Exercice}}
Pour tout $n\in\N$, on note $u_n$ le nombre d'applications $f$ de $\intere{1}{n}$ dans
$\intere{1}{n}$ telles que $f\circ f=\id$. Déterminer une relation de récurrence vérifiée
par la suite $(u_n)$.


%END_BOOK

\end{document}
















