\documentclass{magnolia}

\magtex{tex_driver={pdftex},
        tex_packages={xypic}}
\magfiche{document_nom={Exercices sur les espaces vectoriels},
          auteur_nom={François Fayard},
          auteur_mail={fayard.prof@gmail.com}}
\magexos{exos_matiere={maths},
         exos_niveau={mpsi},
         exos_chapitre_numero={7},
         exos_theme={Espaces vectoriels}}
\magmisenpage{}
\maglieudiff{}
\magprocess

\begin{document}
%BEGIN_BOOK
\magsection{Espace vectoriel, application linéaire}
\magsubsection{Définition, propriétés élémentaires}
\magsubsection{Sous-espace vectoriel}

\exercice{nom={Exemples d'espaces vectoriels}}
\begin{questions}
\question Les ensembles $E$ suivants sont-ils des sous-espaces vectoriels de l'espace vectoriel
	des suites réelles~? Si oui, le prouver.
\begin{questions}
\question L'ensemble des suites réelles ayant une limite finie lorsque $n$ tend vers
$+\infty$.
\question L'ensemble des suites réelles bornées, c'est-à-dire l'ensemble des suites
	réelles $(u_n)$ telles qu'il existe $M\geq 0$ tel que
	\[\forall n\in\N\qsep \abs{u_n}\leq M.\]
\end{questions}
\question Les ensembles $E$ suivants sont-ils des sous-espaces vectoriels de l'espace vectoriel $\mathcal{F}(\R,\R)$~?  Si oui, le prouver.
\begin{questions}
\question L'ensemble des fonctions 1-périodiques.
\question L'ensemble des fonctions croissantes.
\question L'ensemble des fonctions qui sont la somme d'une fonction croissante et d'une fonction
  décroissante.
\question L'ensemble des solutions de l'équation différentielle
	\[\forall t\in\R\qsep y'(t)+\e^{t \sin(t)}y(t)=0.\]
\end{questions}
\end{questions}

% \begin{questions}
% \question Montrer que $\vect (u,v)=\vect (u,w)$ si et seulement si
% \[\exists \alpha,\beta,\gamma \in \K \qsep \alpha u+\beta v+\gamma w=0 \quad
%   \text{et} \quad \beta\gamma\not=0.\]
% \question Soit $F$ un sous-espace vectoriel de $E$. Montrer que $F+\K v=F+\K w$
%   si et seulement si
%   \[\exists u \in F \qsep \exists \alpha,\beta \in \K \quad u+\alpha v+\beta
%     w=0 \quad \text{et} \quad \alpha\beta\not=0.\]
% \end{questions}

\exercice{nom={Combinaison linéaire}}
\begin{questions}
\question Dans $\R^3$, donner une condition nécessaire et suffisante sur $a\in\R$ pour
  que le vecteur $(1,-a,1)$ soit combinaison linéaire de $(1,1,1)$ et $(a,0,2)$.
\question Dans $\mathcal{F}(\R,\R)$, $x\mapsto \cos^2(x)$ est-elle combinaison linéaire
  de $x\mapsto 1$ et $x\mapsto \cos(2x)$~?
\question Dans $\mathcal{F}(\R,\R)$, $x\mapsto \sin(2x)$ est-elle combinaison linéaire
de $x\mapsto \cos(x)$ et $x\mapsto \sin(x)$~? 
\end{questions}

\exercice{nom={Fonctions trigonométriques}}
On pose $E\defeq\mathcal{F}(\R,\R)$. Pour tout $n\in\N$, on définit les fonctions
$f_n$ et $g_n$ par
\[\forall x\in\R\qsep f_n(x)\defeq\cos(nx) \quad\et\quad g_n(x)\defeq\cos^n(x).\]
\begin{questions}
\question Montrer que pour tout $n\in\N$, $g_n\in\vect\p{f_0,\ldots,f_n}$ et
  $f_n\in\vect\p{g_0,\ldots,g_n}$.
\question En déduire que pour tout $n\in\N$,
  $\vect\p{f_0,f_1,\ldots,f_n}=\vect\p{g_0,g_1,\ldots,g_n}$.
\end{questions} 

\exercice{nom={Espace vectoriel engendré}}
Soit $u$, $v$ et $w$ trois vecteurs d'un \Kev $E$. Montrer que $\vect (u,v)=\vect (u,w)$
si et seulement si
\[\exists \alpha,\beta,\gamma \in \K \qsep \alpha u+\beta v+\gamma w=0 \quad
  \text{et} \quad \beta\gamma\not=0.\]

\exercice{nom={Union de sous-espaces vectoriels}}
Soit $E$ un \Kev.
\begin{questions}
\question Soit $F$ et $G$ deux sous-espaces vectoriels de $E$. Montrer que $F\cup G$ est un sous-espace
  vectoriel de $E$ si et seulement si $F\subset G$ ou $G\subset F$.
\question Soit $(F_i)_{i\in I}$ une famille de sous-espaces vectoriels de $E$ pour laquelle
  \[\forall i,j\in I\qsep \exists k\in I\qsep F_i \cup F_j \subset F_k.\]
  Montrer que $\cup_{i\in I} F_i$ est un sous-espace vectoriel de $E$.
\end{questions}

\magsubsection{Application linéaire}

\exercice{nom={Caractérisation des homothéties}}

Soit $f\in\mathcal{L}(E)$. Le but de cet exercice est de montrer que $f$ est une homothétie
si et seulement si, quel que soit $x\in E$, $x$ et $f(x)$ sont colinéaires.
\begin{questions}
\question Montrer que si $f$ est une homothétie, quel que soit $x\in E$, $x$ et $f(x)$ sont colinéaires.
\question Réciproquement, on suppose que quel que soit $x\in E$, $x$ et $f(x)$ sont colinéaires.
  \begin{questions}
  \question Montrer que pour tout $x\in E\setminus\ens{0}$, il existe un unique $\lambda_x\in\K$ tel
    que $f(x)=\lambda_x x$.
  \question Montrer que si $x$ et $y\in E\setminus\ens{0}$ sont colinéaires, alors $\lambda_x=\lambda_y$.
  \question Montrer que si $x$ et $y\in E\setminus\ens{0}$ ne sont pas colinéaires, alors
    $\lambda_x=\lambda_y$.
  \question Conclure.
  \end{questions}
\end{questions}







% \exercice{nom={$E$ n'est pas réunion de sous-espaces vectoriels
%   stricts}}
% Soit $E$ un \Kev et $F_1,\ldots,F_n$ $n$ sous-espaces vectoriels de $E$
% strictement inclus dans $E$. Le but de cet exercice est de montrer que si $\K$
% est infini, alors :
% \[\cup_{k\in\intere{1}{n}} F_k \not= E\]
% \begin{questions}
% \question Traiter le cas $n=2$ pour un corps $\K$ fini ou infini.
% \question Pour le cas général $\K$ est supposé infini. On suppose que
%   $F_n\not\subset F_1 \cup \ldots \cup F_{n-1}$ et on choisit un élément $x$
%   de $F_n$ n'appartenant pas à $F_1 \cup \ldots \cup F_{n-1}$, ainsi qu'un
%   élément $y$ de $E$ n'appartenant pas à $F_n$.
%   \begin{questions}
%   \question Montrer que :
%     \[\forall \lambda \in \K \quad \lambda x + y \not\in F_n\]
%   \question Montrer que pour tout entier $i\in\intere{1}{n-1}$, il existe au
%     plus un $\lambda \in \K$ tel que $\lambda x + y \in F_i$.
%   \question Conclure
%   \end{questions}
% \end{questions}

\magsection{L'algèbre $\Endo{E}$}

\magsubsection{$\lin{E}{F}$}

\exercice{nom={Calcul dans $\Endo{E}$}}
\begin{questions}
\question Soit $f\in\Endo{E}$ tel que $f^3=f^2+f+\id$. Montrer que $f$ est un
  automorphisme.
\question Soit $E$ un \Kev et $f$ un endomorphisme de $E$. On suppose que $f$ est
  nilpotent, c'est-à-dire qu'il existe $n\in\N$ tel que :
  \[f^n=0\]
  Montrer que $\id_E+f$ est un automorphisme et calculer son inverse.
\end{questions}

% \exercice{nom={Commutant d'un projecteur}}
% Soit $E$ un \Kev et $f_0$ un endomorphisme de $E$. On note~:
% $$\mathcal{C}(f_0)=\left\{f \in \Endo{E} : f\circ f_0=f_0\circ f \right\}$$
% \begin{questions}
% \question Montrer que $\mathcal{C}(f)$ est une sous algèbre de $\Endo{E}$.
% \question On suppose que $f_0$ est un projecteur. Montrer que $f$ commute avec
%   $f_0$ si et seulement si $f$ laisse stable le noyau et l'image de $f_0$.
% \end{questions}

\magsubsection{Le groupe linéaire}


% \exercice{nom={Formes linéaires}}
% Soit $\phi_1$ et $\phi_2$ deux formes linéaires non nulles sur un \Kev $E$.
% Montrer que~:
% \[\ker \phi_1=\ker \phi_2 \quad\ssi\quad \cro{\exists \lambda\in\Ks \quad
%   \phi_2=\lambda \phi_1}\]

\exercice{nom={Automorphisme de $\R^3$}}
Soit
\[\dspappli{f}{\R^3}{\R^3}{(x,y,z)}{(x+z,-2x+y,x+3z)}\]
Montrer que $f\in\gl{}{\R^3}$.


\magsection{Somme, somme directe, projecteur, hyperplan}
\magsubsection{Somme, somme directe}

\exercice{nom={Exercice}}
Soit $v$ et $w$ deux vecteurs d'un \Kev $E$ et $F$ un sous-espace vectoriel de $E$.
Montrer que $F+\K v=F+\K w$ si et seulement si
\[\exists u \in F \qsep \exists \alpha,\beta \in \K \qsep u=\alpha v+\beta
  w \quad \text{et} \quad \alpha\beta\not=0.\]

\exercice{nom={Exercice}}
Soit $E$ un espace vectoriel et $F,G$ et $H$ trois sous-espaces vectoriels de $E$.
\begin{questions}
\question  Montrer que $(F\cap G)+(F\cap H)\subset F\cap (G+H)$.
  Vérifiez sur un dessin qu'il est possible que cette inclusion soit stricte.
\question Établir que l'on a $(F\cap G)+(F\cap H)=F\cap [G+(F\cap H)]$.
\end{questions}

\exercice{nom={Exercice}}
$E$, $F$   et  $G$  sont  trois \Kevs, $f\in\lin{E}{F}$  et  $g\in\lin{F}{G}$.
Montrer que
\[\ker\p{g \circ f} = \ker f \quad\ssi\quad
  \ker g \cap \im f  = \ens{0}.\]
\[\im\p{g \circ f} = \im g \quad\ssi\quad
  \ker g+\im f = F.\]


\exercice{nom={Fonctions paires et impaires}}
Dans $E=\mathcal{F}(\R,\R)$, on pose
\[\mathcal{P}\defeq\enstq{f\in E}{\forall x\in\R\qsep f(-x)=f(x)} \quad\et\quad
  \mathcal{I}\defeq\enstq{f\in E}{\forall x\in\R\qsep f(-x)=-f(x)}.\]
Montrer que $E=\mathcal{P}\oplus\mathcal{I}$.

\exercice{nom={Somme directe}}
Soit $E$ le \Rev des fonctions réelles de classe $\classec{1}$ sur $\R$. On
définit
\[A\defeq\enstq{f\in E}{\exists a,b\in\R \qsep \forall x\in\R \qsep f(x)=ax+b}\]
\[B\defeq\enstq{f\in E}{f\p{0}=0 \et f'\p{0}=0}\]
\begin{questions}
\question Montrer que $A$ et $B$ sont des sous-espaces vectoriels de $E$.
\question Montrer que $E=A\oplus B$.
\end{questions}


\exercice{nom={Rendre directe une somme}}
Soit $F$ et $G$ deux sous-espaces vectoriels d'un \Kev $E$ tels que $F+G=E$. On
note $F'$ un supplémentaire de $F\cap G$ dans $F$. Montrer que
\[E=F'\oplus G.\]

% \exercice{nom={Supplémentaire}}
% Soit $E$ le \Rev des fonctions continues de $\R$ dans $\R$. On considère :
% \[F=\left\{ f\in E : \int_0^1 f(t)\, dt=0 \right\}\]
% Montrer que $F$ est un sous-espace vectoriel de $E$ puis en donner deux
% supplémentaires.

\magsubsection{Projecteur}

\exercice{nom={Somme de deux projecteurs}}
Soit $E$ un \Kev et $p,q\in\Endo{E}$ deux projecteurs.
\begin{questions}
\question Montrer que $p+q$ est un projecteur si et seulement si
  $p\circ q=q\circ p=0$.
\question On suppose que $p+q$ est un projecteur. Montrer que
  \[\ker(p+q)=\ker p\cap\ker q \et \im(p+q)=\im p\oplus\im q.\]
\end{questions}

\begin{sol}
$p\circ q=p\circ p\circ q=p\circ(-q\circ p)=-p\circ q\circ p=q\circ p\circ p=q\circ p$.
\end{sol}


\exercice{nom={Réduction  d'une application linéaire}}
Soit $E$ un \Kev et $f\in\Endo{E}$ tel que
\[f^2-5f+6\id=0.\]
\begin{questions}
\question Montrer que $(f-2\id)\circ(f-3\id)=0$.
\question En déduire que $E=\ker (f-2\id)\oplus\ker (f-3\id)$.
\end{questions}

\exercice{nom={Projecteur}}
Soit $E$ un \Kev.
\begin{questions}
\question Soit $f\in\Endo{E}$ et $g$ un projecteur de $E$. Montrer que
  \[\ker(f\circ g)=\ker g \oplus (\ker f\cap\im g).\]
\question Soit $f$ un projecteur de $E$ et $g\in\Endo{E}$. Montrer que
  \[\im(f\circ g)=\im f \cap (\ker f + \im g).\]
\question Soit $f$ et $g$ deux projecteurs de $E$. Montrer que $f\circ g$ est
  un projecteur si et seulement si
  \[\im f \cap (\ker f + \im g)\subset \im g \oplus (\ker f \cap \ker g).\]
\end{questions}


\magsubsection{Symétrie}

\exercice{nom={Centre de $\Endo{E}$}}
Soit $E$ un \Kev. Le but de cet exercice est de montrer que les endomorphismes qui commutent avec tous les
autres sont les homothéties.
\begin{questions}
\question Montrer que si $f\in\Endo{E}$ est une homothétie, alors elle commute avec tous les
  endomorphismes de $E$.
\question Réciproquement, soit $f\in\Endo{E}$ un endomorphisme commutant avec tous les endomorphismes
  de $E$.
  \begin{questions}
  \question Soit $s$ une symétrie de $E$. Montrer que $\ker(s-\id)$ et $\ker(s+\id)$ sont stables par $f$.
  \question En admettant le fait que toute droite vectorielle admet un supplémentaire, montrer que
    quel que soit $x\in E$, $x$ et $f(x)$ sont colinéaires.
  \question Conclure.
  \end{questions}
\end{questions}

\magsubsection{Hyperplan}

\exercice{nom={Hyperplan}}
Soit $E$ un \Kev et $H_1, H_2$ deux hyperplans de $E$ tels que $H_1\subset H_2$. Montrer que $H_1=H_2$.

% \exercice{nom={Fonctions paires et impaires}}
% Soit $E$ le \Cev des fonctions de $\R$ dans $\C$. On note $\mathcal{I}$
% (respectivement $\mathcal{P}$) l'ensemble des fonctions impaires
% (respectivement paires) de $E$.
% \begin{questions}
% \question Montrer que $\mathcal{I}$ et $\mathcal{P}$ sont des sous-espaces
%   vectoriels de $E$.
% \question Montrer que $E=\mathcal{I}\oplus\mathcal{P}$.
% \end{questions}







% \exercice{nom={Définition d'une application linéaire}}
% Soit $E$ et $F$ deux \Kevs et $A$, $B$ deux sous-espaces vectoriels
% supplémentaires de $E$. On se donne $f_A\in\lin{A}{F}$ et $f_B\in\lin{B}{F}$.
% Montrer qu'il existe une unique application linéaire $f\in\lin{E}{F}$ telle
% que~:
% \[\cro{\forall a\in A \quad f(a)=f_A(a)} \et
%   \cro{\forall b\in B \quad f(b)=f_B(b)}\]







%END_BOOK
\end{document}

% \exercice{nom={Factorisation d'une application linéaire}}
% Soit $E$ et $F$ deux $\K$-espaces vectoriels, $\phi \in\Endo{E}$ et
%   $\psi \in\Endo{F}$. Soit $f$ une application linéaire de $E$ dans $F$.
% \begin{questions}
% \question Montrer qu'il existe $g \in \lin{E}{F}$ tel que $f=g\circ \phi$ si
%   et seulement si $\ker \phi \subset \ker f$. 
% \question Montrer qu'il existe $g \in \lin{E}{F}$ tel que $f=\psi\circ g$ si
%   et seulement si $\im f \subset \im \psi$. 
% \end{questions}

