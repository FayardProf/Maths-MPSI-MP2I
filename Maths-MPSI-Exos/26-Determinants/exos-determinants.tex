\documentclass{magnolia}

\magtex{tex_driver={pdftex}}
\magfiche{document_nom={Exercices sur les déterminants},
          auteur_nom={François Fayard},
          auteur_mail={fayard.prof@gmail.com}}
\magexos{exos_matiere={maths},
         exos_niveau={mpsi},
         exos_chapitre_numero={25},
         exos_theme={Déterminants}}
\magmisenpage{}
\maglieudiff{}
\magprocess

\begin{document}
%BEGIN_BOOK


\magsection{Déterminant}
\magsubsection{Forme $n$-linéaire alternée}


\exercice{nom={Exercice \textsc{ENS}}}
Soit $E$ un \Kev de dimension $d\in\N$. Pour tout $n\in\N$, calculer la dimension
de l'espace vectoriel $\mathcal{A}_n(E)$ des formes $n$-linéaires alternées sur $E$.

\magsubsection{Déterminant d'une famille de $n$ vecteurs}

\exercice{nom={Exercice}}
Soit $E$ un \Kev de dimension $n$, $f\in\mathcal{L}(E)$ et $\mathcal{B}$ une base
de $E$. Montrer que pour tous $x_1,\ldots,x_n\in E$
\[\sum_{k=1}^n \det{}_{\mathcal{B}}(x_1,\ldots,x_{k-1},f(x_k),x_{k+1},\ldots,x_n)=\tr(f)
  \det{}_{\mathcal{B}}(x_1,\ldots,x_n).\]

\magsubsection{Déterminant d'un endomorphisme}

\exercice{nom={Déterminant de la transposition}}
Soit $\phi$ l'application de $\mat{n}{\K}$ dans lui-même qui à la matrice $M$
associe sa transposée. Calculer le déterminant de $\phi$.
\begin{sol}
$\quad$
\begin{questions}
\question La transposition est une symétrie. La dimension de l'espace des
  matrices antisymétriques est $n\p{n-1}/2$. Donc le déterminant de la
  transposition est
  \[\p{-1}^{\frac{n\p{n-1}}{2}}\]
\end{questions}
\end{sol}

\magsubsection{Déterminant d'une matrice carrée}
\magsection{Calcul de déterminant}
\magsubsection{Méthode du pivot}

\exercice{nom={Calcul de déterminant}}
Soit $a_1,\ldots,a_n\in\R$. Calculer le déterminant de la matrice $A\in\mat{n}{\R}$
donnée par les coefficients
\[\forall i,j\in\intere{1}{n}\qsep a_{i,j}\defeq\sin(a_i+a_j).\]
% \[\left|
%   \begin{array}{cccc}
%   \sin(a_1+a_1) & \sin(a_1+a_2) & \dots & \sin(a_1+a_n)\\
%   \sin(a_2+a_1) &               &       & \vdots\\
%   \vdots        &               &       & \vdots\\
%   \sin(a_n+a_1) & \sin(a_n+a_2) & \dots & \sin(a_n+a_n)\\
%   \end{array}
%   \right|.\]
\begin{sol}
Si on note $A=\p{\sin a_1,\ldots,\sin a_n}$ et
$B=\p{\cos a_1,\ldots,\cos a_n}$, alors $C_j=\p{\cos a_j}A+\p{\sin a_j}B$.
Donc le rang de la matrice est inférieur ou égal à 2.
\begin{itemize}
\item Si $n\geq 3$, le déterminant est nul.
\item Si $n=1$, le déterminant est $\sin\p{2a_1}$.
\item Si $n=2$, le déterminant est $-\sin^2\p{a_1-a_2}$.
\end{itemize}
\end{sol}

\exercice{nom={Calculs de déterminants}}
Soit $a,b,c\in\K$. Calculer et factoriser les déterminants suivants
\[\begin{vmatrix}
  1 & 1 & 1\\
  a+b & c+a & b+c\\
  ab & ca & bc
  \end{vmatrix}\qquad
  \begin{vmatrix}
  a+b & b+c & c+a\\
  a^2+b^2 & b^2+c^2 & c^2+a^2\\
  a^3+b^3 & b^3+c^3 & c^3+a^3
  \end{vmatrix}\]
\[\begin{vmatrix}
  (b+c)^2 & b^2 & c^2\\
  a^2 & (c+a)^2 & c^2\\
  a^2 & b^2 & (a+b)^2
  \end{vmatrix}\]
\begin{sol}
$\quad$
\begin{questions}
\question On trouve $\p{c-b}\p{c-a}\p{b-a}$.
\question On trouve $2abc\p{a+b+c}^3$. On fait apparaître $a+b+c$ par
  l'opération $L_1\gets L_1-L_3$.
\end{questions}
\end{sol}

\exercice{nom={Calculs de déterminants}}
Soit $a,b,c,d\in\K$. Calculer et factoriser les déterminants suivants
\[\begin{vmatrix}
  1 & 1 & 1 & 1\\
  a & b & c & d\\
  a^2 & b^2 & c^2 & d^2\\
  bcd & acd & abd & abc
  \end{vmatrix}\qquad
  \begin{vmatrix}
  1 & a & b & ac\\
  1 & b & c & bd\\
  1 & c & d & ac\\
  1 & d & a & bd
  \end{vmatrix}
  \qquad
  \begin{vmatrix}
  0 & 1 & 1 & 1\\
  1 & 0 & a^2 & b^2\\
  1 & a^2 & 0 & c^2\\
  1 & b^2 & c^2 & 0
  \end{vmatrix}
  \]

\[\begin{vmatrix}
  1+a & b & a & b\\
  b & 1+a & b & a\\
  a & b & 1+a & b\\
  b & a & b & 1+a
  \end{vmatrix}\qquad
  \begin{vmatrix}
  a & c & c & b\\
  c & a & b & c\\
  c & b & a & c\\
  b & c & c & a
  \end{vmatrix}\]

\exercice{nom={Calcul de déterminant}}
Soit $a_1,\ldots,a_n\in\R$. Calculer
\[\left|
  \begin{array}{cccc}
  1        & 1 & \cdots & 1\\
  \cos(a_1) & \cos(a_2) & \cdots & \cos(a_n)\\
  \vdots   &   \vdots   &         & \vdots\\
  \cos ((n-1)a_1) & \cos((n-1)a_2) & \cdots & \cos((n-1)a_n)
  \end{array}
  \right|\]
\begin{sol}
Comme $\cos\p{k a_j}=P_k\p{\cos a_j}$ et que le coefficient dominant
de $P_k$ est $2^{k-1}$, le déterminant est égal à 
\begin{eqnarray*}
& & 2^{\frac{\p{n-2}\p{n-1}}{2}}V\p{\cos a_1,\ldots,\cos a_n}\\
&=& 2^{\frac{\p{n-2}\p{n-1}}{2}}\prod_{1\leq i<j\leq n} \p{\cos a_j-\cos a_i}\\
&=& \p{-1}^{\frac{n\p{n-1}}{2}}2^{\p{n-1}^2}\prod_{1\leq i<j\leq n}
    \sin\p{\frac{a_j+a_i}{2}}\sin\p{\frac{a_j-a_i}{2}}
\end{eqnarray*}
\end{sol}

\exercice{nom={Matrice circulante}}
\begin{questions}
\question Soit $a,b,c\in\C$. On pose
  \[M\defeq\begin{pmatrix}
    a&b&c\\
    c&a&b\\
    b&c&a
  \end{pmatrix} \quad\et\quad J\defeq\begin{pmatrix}
  1 & 1 & 1\\
  1 & \jj & \jj^2\\
  1 & \jj^2 & \jj
  \end{pmatrix}\]
  \begin{questions}
  \question Montrer que $\det J\neq 0$.
  \question Calculer $MJ$ et en déduire $\det M$.
  \end{questions}
\question Mêmes questions avec la \emph{matrice circulante}
  \[M\defeq\begin{pmatrix}
    a_0 & a_1 & \cdots & a_{n-1}\\
    a_{n-1} & a_0 & \cdots & a_{n-2}\\
    \vdots & \vdots & \ddots & \vdots\\
    a_1 & a_2 & \cdots & a_0
  \end{pmatrix} \quad\text{et}\quad 
  J\defeq \p{\omega^{(i-1)(j-1)}}_{1\leq i,j\leq n}\]
  pour $a_0,\ldots,a_{n-1}\in\C$ en posant $\omega\defeq \e^{\ii\frac{2\pi}{n}}$.
\end{questions}

\exercice{nom={Calculs de déterminants}}
Soit $a_1,\ldots,a_n,b_1,\ldots,b_n\in\K$. Calculer le déterminant
\[\left|
\begin{array}{ccccc}
a_1+b_1 & a_1 & \cdots & \cdots & a_1\\
a_2 & a_2+b_2 & a_2 & \cdots & a_2\\
a_3 & \ddots & \ddots & \ddots & \vdots\\
\vdots & & \ddots & \ddots & a_{n-1}\\
a_n & \cdots & \cdots & a_n & a_n+b_n
\end{array}
\right|\]

\begin{sol}
On trouve
  \[\prod_{k=1}^n b_k+\sum_{k=1}^n
    \cro{a_k \prod_{\substack{i=1\\i\neq k}}^n b_i}\]
  Pour cela, il y a plusieurs méthodes~:
  \begin{itemize}
  \item Le plus naturelle (et la plus longue). On enlève la dernière colonne
    à toutes les autres. On développe ensuite par rapport à la dernière ligne.
  \item Méthode plus fine. On utilise la linéarité du déterminant par rapport
    à la dernière colonne pour faire apparaître $\p{a_1,\ldots,a_n}$ dans la
    dernière colonne. Dans cette matrice, on enlève la dernière colonne à
    toutes les autres. Dans l'autre dont la dernière colonne est
    $\p{0,\ldots,0,b_n}$, on développe par rapport à cette dernière colonne.
  \item Méthode encore plus fine. On pose $u=\p{a_1,\ldots,a_n}$ et
    $\mathcal{B}=e_1,\ldots,e_n$ la base canonique de $\K^n$. Alors
    \begin{eqnarray*}
    \Delta=\det{}_{\mathcal{B}}\p{u+b_1 e_1,\ldots,u+b_n e_n}
    \end{eqnarray*}
    On utilise ensuite la linéarité du déterminant par rapport à chacune des
    variables.
  \end{itemize}
\end{sol}

\exercice{nom={Calcul de déterminant}}
Soit $n\in\Ns$ et $p\in\intere{1}{n}$. On définit la matrice $A_{n,p}$
de $\mat{p+1}{\R}$ par
\[A_{n,p}\defeq
  \begin{pmatrix}
  1 & \binom{n}{1} & \binom{n}{2} & \cdots & \binom{n}{p}\\ 
  1 & \binom{n+1}{1} & \binom{n+1}{2} & \cdots & \binom{n+1}{p}\\ 
  \vdots & \vdots &  &  & \vdots\\ 
  1 & \binom{n+p}{1} & \binom{n+p}{2} & \cdots & \binom{n+p}{p}
  \end{pmatrix}\]
Calculer $\det A_{n,p}$.


\exercice{nom={Opérations par blocs}}
\begin{questions}
\question Soit $A,B,C,D\in\mat{n}{\K}$. Quel produit matriciel permet de transformer
  \[\begin{pmatrix}A & B\\ C&D\end{pmatrix} \quad\text{en}\quad
    \begin{pmatrix}A+2C & B+2D\\ C&D\end{pmatrix}\]
\question Montrer que pour tous $A,B\in\mat{n}{\K}$
  \[\begin{vmatrix}I_n & B\\A & I_n\end{vmatrix} = \det(I_n - AB)=\det(I_n-BA).\]
\end{questions}


\magsubsection{Développement par rapport à une colonne}

\exercice{nom={Matrice tridiagonale}}
Soit $x\in\R$. Calculer le déterminant
\[\left|
\begin{array}{ccccc}
1+x^2 & x &  & & (0)\\
x & 1+x^2 & x & & \\
  & \ddots & \ddots & \ddots & \\
  &   & x & 1+x^2 & x\\
(0)  & &  & x & 1+x^2
\end{array}
\right|\]
\begin{sol}
On trouve la relation de récurrence $u_{n+2}=\p{1+x^2}u_{n+1}-x^2 u_n$.\begin{itemize}
\item Si $x^2=1$, on a $u_n=n+1$.
\item Sinon, $u_n=\frac{x^{2n+2}-1}{x^2-1}$.
\end{itemize}
\end{sol}



\exercice{nom={Calcul de rang}}
Résoudre le système linéaire dont la matrice $A$ est définie par
\[\forall i,j\in\intere{1}{n}\qsep \quad a_{i,j}\defeq
  \begin{cases}
  1 & \text{si} \abs{i-j}\leq 1,\\
  0 & \text{sinon.}
  \end{cases}\]
Quel est son rang ?



\magsubsection{Comatrice}


\exercice{nom={Comatrice du produit}}
Soit $\K$ un corps et $A,B\in\mat{n}{\K}$.
\begin{questions}
\question Montrer que les matrices $A-X I_n$ et $B-XI_n$ sont inversibles dans
  $\mat{n}{\fracK}$.
\question Montrer que $\Com(AB)=\Com(A)\Com(B)$.
\end{questions}










% \exercice{nom={Le déterminant de Vandermonde}}
% Soit $n\in\Ns$ et $x_0,\ldots,x_n$ $n+1$ scalaires. On définit le
% déterminant de Vandermonde par :
% \[V(x_0,x_1,\ldots,x_n)=
%   \begin{vmatrix}
%   1 & x_0 & x_0^2 & \dots & x_0^n\\
%   1 & x_1 & x_1^2 & \dots & x_1^n\\
%   \vdots & \vdots & \vdots & & \vdots\\
%   1 & x_n & x_n^2 & \dots & x_n^n
%   \end{vmatrix}\]
% Le but de cet exercice est de montrer par récurrence sur $n$ que :
% $$V(x_0,\ldots,x_n)=\prod_{0\leq i<j \leq n} (x_j-x_i)$$
% de deux manières distinctes.\\
% \begin{questions}
% \question Établir ce résultat en effectuant un pivot de Gauss.
% \question Établir ce résultat en étudiant le polynôme
%  \[V(x_0,\ldots,x_n,X)\]
% \end{questions}



% \exercice{nom={$\gl{n}{\Z}$}}
% Soit $M\in\mat{n}{\Z}$. Donner une condition nécessaire et
% suffisante pour que $M$ soit inversible et que
% $M^{-1}\in\mat{n}{\Z}$.
% \begin{sol}
% $\quad$
% \begin{questions}
% \question Il faut et il suffit que $\det A=\pm 1$.
% \end{questions}  
% \end{sol}

% \magsection{Déterminants en géométrie affine}
% \exercice{nom={Coordonnées barycentriques}}
% Soit $A,B,C$ trois points du plan non alignés.
% \begin{questions}
% \question On se donne $M_1,M_2,M_3$ trois points du plan dont les coordonnées
%   barycentriques dans le système $(A,B,C)$ sont respectivement
%   $(\alpha_1,\beta_1,\gamma_1)$, $(\alpha_2,\beta_2,\gamma_2)$ et
%   $(\alpha_3,\beta_3,\gamma_3)$. Montrer que $M_1,M_2$ et $M_3$ sont alignés si
%   et seulement si :
%   $$
%   \left|
%   \begin{array}{ccc}
%   \alpha_1 & \alpha_2 & \alpha_3\\
%   \beta_1 & \beta_2 & \beta_3\\
%   \gamma_1 & \gamma_2 & \gamma_3\\
%   \end{array}
%   \right|=0
%   $$
% \question Soit $A'$, $B'$, $C'$ trois points situés respectivement sur les
%   droites $\p{BC}$, $\p{CA}$ et $\p{AB}$ et distincts de $A,B,C$. Montrer que
%   $A'$, $B'$, $C'$ sont alignés si et seulement si~:
%   \[\frac{\overline{A'B}}{\overline{A'C}} \cdot
%     \frac{\overline{B'C}}{\overline{B'A}} \cdot
%     \frac{\overline{C'A}}{\overline{C'B}}=1\]
% \end{questions}


% \begin{sol}
% $\quad$
% \begin{questions}
% \question On peut se placer dans le repère $(A,\ve{AB},\ve{AC})$. Les points
%   $M_1,M_2,M_3$ sont alignés si et seulement si il existe $a,b,c\in\R$ tels
%   que $\p{a,b}\neq\p{0,0}$ et $M_1,M_2,M_3$ sont sur la droite d'équation
%   $ax+by+c=0$.
% \question Si on note $\alpha=\overline{A'B}/\overline{A'C}$, alors $A'$ est
%   barycentre de $B$ et $C$ avec les coefficients $1$ et $-\alpha$. Il suffit
%   ensuite d'appliquer la question précédente.
% \end{questions}
% \end{sol}


% \exercice{nom={Droites concourantes}}
% Dans le plan rapporté à un repère affine, on se donne $3$ droites
% $\mathcal{D}_1,\mathcal{D}_2$ et $\mathcal{D}_3$ d'équations respectives~:
% $$(\mathcal{D}_i) \quad \alpha_i x+\beta_i y+\gamma_i=0$$
% On suppose qu'au moins deux des droites ne sont pas parrallèles. Montrer
% que les trois droites sont concourantes si et seulement si :

% $$
% \left|
% \begin{array}{ccc}
% \alpha_1 & \alpha_2 & \alpha_3\\
% \beta_1 & \beta_2 & \beta_3\\
% \gamma_1 & \gamma_2 & \gamma_3\\
% \end{array}
% \right|=0
% $$




%END_BOOK

\end{document}


