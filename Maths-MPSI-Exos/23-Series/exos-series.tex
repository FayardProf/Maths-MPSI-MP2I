\documentclass{magnolia}

\magtex{tex_driver={pdftex}}
\magfiche{document_nom={Exercices sur les développements limités},
          auteur_nom={François Fayard},
          auteur_mail={fayard.prof@gmail.com}}
\magexos{exos_matiere={maths},
         exos_niveau={mpsi},
         exos_chapitre_numero={22},
         exos_theme={Séries}}
\magmisenpage{}
\maglieudiff{}
\magprocess

\begin{document}
%BEGIN_BOOK
\magsection{Série}
\magsubsection{Série}
\exercice{nom={Exercice}}
Établir la convergence et calculer
\[\sum_{n=0}^{+\infty}\frac{2^n}{(n+2)n!},\qquad
  \sum_{n=0}^{+\infty}\frac{2^n(5n+1)}{3^n n!}.\]
\begin{sol}
On trouve $(1/4)(1+\e^2)$, $(13/3)\e^{2/3}$.
\end{sol}

\exercice{nom={Exercice}}
Soit $(u_n)$ une suite positive décroissante tendant vers $0$. Montrer que
si $\sum u_n$ converge, alors
\[u_n = \petito{n}{+\infty}{\frac{1}{n}}.\]
On pourra considérer la suite de terme général $s_{2n} - s_n$. La réciproque est-elle vraie~?

\magsubsection{Série à termes positifs}

\exercice{nom={Exercice}}
Soit $(u_n)$ une suite réelle positive, décroissante.
\begin{questions}
\question Montrer que $\sum u_n$ converge si et seulement si $\sum 2^n u_{2^n}$ converge.
\question Soit $\alpha,\beta\in\R$. Déterminer la nature de
  \[\sum \frac{1}{n^{\alpha} \ln^\beta n}.\]
\end{questions}


\magsubsection{Série absolument convergente}

\exercice{nom={Exercice}}
Déterminer la nature des séries suivantes.
\[\sum \frac{\ln(n)}{n^2},\qquad \sum \frac{n!}{n^n},\qquad \sum \cro{\e - \p{1+\frac 1n}^n},\]
\[\sum \frac{1}{\binom{2n}{n}}, \qquad \sum \frac{1}{\ln(n)^{\ln(n)}}.\]

\exercice{nom={Exercice}}
Montrer que la suite $(u_n)$ définie par
\[\forall n\in\N\qsep u_n \defeq \prod_{k=1}^n \p{1+ \frac{1}{k^2}}\]
converge vers un réel non nul.

\exercice{nom={Exercice}}
Donner la nature de la série de terme général
\[\frac{n^{2 \lambda}}{\lambda^n + \ln n}\]
suivant $\lambda\in\R$.

\exercice{nom={Exercice}}
Soit $a,b\in\R$. Donner la nature de
\[\sum \p{ \sqrt[3]{n^3+n} - \sqrt[2]{n^2+an+b}}.\]

\exercice{nom={Exercice}}
Pour tout $r \in \N$, on pose
\[S_r(x) \defeq \sum_{n=r}^{+\infty} \binom{n}{r} x^{n-r}.\]
\begin{questions}
\question Pour quelles valeurs de $x\in\R$ la somme $S_r(x)$ est-elle définie~?
\question Calculer $(1-x)S_{r+1}(x)$.
\question En déduire la valeur de $S_r(x)$. 
\end{questions}

\exercice{nom={Exercice}}
Montrer qu'il existe $l\in \R$ tel que
\[\sum_{k=1}^n k^{\frac{1}{k}} = n + \frac{1}{2} \ln^2 n+l +\petito{n}{+\infty}{1}.\]

\magsubsection{Série semi-convergente}

\exercice{nom={Exercice}}
On définit les suites $(u_n)$ et $(v_n)$ par
\[\forall n\geq 2\qsep u_n \defeq \frac{(-1)^n}{\sqrt{n}} \et
  v_n \defeq \frac{(-1)^n}{\sqrt{n}+(-1)^n}.\]
\begin{questions}
\question Montrer que \[u_n \equi{n}{+\infty} v_n.\]
\question Montrer que $\sum u_n$ converge et que $\sum v_n$ diverge.
\end{questions}

\exercice{nom={Exercice}}
Déterminer les suites réelles $(v_n)$ telles que pour toute suite réelle positive $(u_n)$
\[\sum u_n \text{ converge } \quad\implique\quad \sum u_n v_n \text{ converge }.\]


%END_BOOK

\end{document}
