\documentclass{magnolia}

\magtex{tex_driver={pdftex}}
\magfiche{document_nom={Exercices sur les matrices et applications linéaires},
          auteur_nom={François Fayard},
          auteur_mail={fayard.prof@gmail.com}}
\magexos{exos_matiere={maths},
         exos_niveau={mpsi},
         exos_chapitre_numero={0},
         exos_theme={Matrices et applications linéaires}}
\magmisenpage{misenpage_presentation={tikzvelvia},
              misenpage_format={a4},
              misenpage_nbcolonnes={1},
              misenpage_sol={non}}
\maglieudiff{lieu_lycee={Aux Lazaristes},
             lieu_classe={MPSI 1},
             lieu_annee={2019--2020}}
\magprocess

\begin{document}
%BEGIN_BOOK
\magsection{Matrices, vecteur et application linéaire}
\magsubsection{Matrice d'une famille de vecteurs}
\magsubsection{Matrice d'une application linéaire}
\exercice{nom={Base, noyau et image}}
\begin{questions}
\question On considère l'endomorphisme $f$ de $\R^3$ dont la matrice dans la
  base canonique de $\R^3$ est
  \[A\defeq\begin{pmatrix}
      1 & 1 & 1\\
     -1 & 2 & -2\\
      0 & 3 & -1
      \end{pmatrix}\]
  Donner une base de $\ker f$ et $\im f$.
\question Soit $f$ l'application linéaire de $\R^4$ dans $\R^3$ canoniquement
  associée à la matrice
  \[A\defeq\begin{pmatrix}
      -11 & 7 & 0 & 3\\
       0 & 1 & 11 & 2\\
       1 & 0 & 7 & 1
      \end{pmatrix}\]
  Déterminer le rang de $f$, ainsi qu'une base de son noyau et de son image.
  Donner une équation de l'image.
\end{questions}
\begin{sol}
$\quad$
\begin{questions}
\question Le noyau est engendré par $(-4,1,3)$ donc le rang de $f$ est de 2.
  Donc $f(e_1),f(e_2)$ est une base de $\im f$ car c'est une famille libre.
\question La famille $(-7,-11,1,0),(-1,-2,0,1)$ est une base de $\ker f$.
  Donc le rang de $f$ est de 2. La famille $f(e_1),f(e_2)$ est libre, donc
  c'est une base de $\im f$. L'équation $\alpha-7\beta+11\gamma=0$.  
\end{questions}
\end{sol}

\magsubsection{Matrice de passage, changement de base}

\exercice{nom={Calcul de matrices}}
\begin{questions}
\question On considère l'endomorphisme $u$ de $\R_3[X]$ défini par
  \[\forall P\in\R_3[X] \qsep u(P)\defeq P'+P.\]
  Écrire la matrice de $u$ dans la base $(1,X,X^2,X^3)$.
\question Dans $\R^3$, déterminer la matrice de passage de la base $b_1$ à la
  base $b_2$ avec
  \[b_1\defeq((1,2,1), (2,3,3), (3,7,1))\]
  \[b_2\defeq((3,1,4), (5,3,2), (1,-1,7))\]
\end{questions}
\begin{sol}
\begin{questions}
\question On trouve
  \[
  \begin{pmatrix}
  1 & 1 & 0 & 0\\
  0 & 1 & 2 & 0\\
  0 & 0 & 1 & 3\\
  0 & 0 & 0 & 1  
  \end{pmatrix}\]
\question On a $P(b_1,b_2)=P(b,b_1)^{-1}P(b,b_2)$. On trouve
  \[P(b,b_1)^{-1}=
  \begin{pmatrix}
  -18 & 7 & 5\\
  5 & -2 & -1\\
  3 & -1 & -1
  \end{pmatrix}
  \et P(b_1,b_2)=
  \begin{pmatrix}
  -27 & -59 & 10\\
  9 & 17 & 0\\
  4 & 10 & -3 
  \end{pmatrix}\]
\end{questions}
\end{sol}

\magsubsection{Caractérisation des matrices inversibles}

\exercice{nom={Calcul dans l'anneau $\mat{n}{\K}$}}
Soit $A$ et $B$ deux matrices de $\mat{n}{\K}$ telles que $AB=A+B$. Montrer
que $A$ et $B$ commutent.

\exercice{nom={Matrices à diagonale dominante}}
Soit $A$ une matrice de $\mat{n}{\C}$ à coefficients diagonaux
  dominants, c'est-à-dire telle que
  \[\forall i\in\intere{1}{n} \qsep \abs{a_{i,i}} >
    \sum_{\substack{j=1\\ j\not=i}}^n \abs{a_{i,j}}.\]
  Montrer que $A$ est inversible.


\magsubsection{Rang d'une matrice}


\exercice{nom={Calcul de rang et d'inverse}}
Calculer les rangs des matrices suivantes et calculer leurs inverses quand il y
a lieu.
\[\begin{pmatrix}
   1 & 1 & 0 & 1\\
   3 & 2 & -1 & 3\\
   \lambda & 3 & -2 & 0\\
   -1 & 0 & -4 & 3
  \end{pmatrix}\qquad
  \begin{pmatrix}
  3 & 1 & 1\\
  1 & 1 & \lambda\\
  -4 & 4 & -4\\
  6 & 4 & 0
  \end{pmatrix}\qquad
  \begin{pmatrix}
  1 & 1 & -1 & 2\\
  \lambda & 1 & 1 & 1\\
  1 & -1 & 3 & -3\\
  4 & 2 & 0 & \lambda
  \end{pmatrix}\]
\begin{sol}
$\quad$
\begin{questions}
\question Le rang est 4 si $\lambda\neq -20$ et 3 sinon.
\question Le rang est 3 quel que soit $\lambda$.
\question Le rang est 4 si $\lambda\neq 3$ et 2 sinon.
\end{questions}
\end{sol}








% \exercice{nom={Centre de $\Endo{E}$}}
% Soit $E$ un \Kev de dimension finie. Le but de cet exercice est de montrer que
% le centre de $\Endo{E}$, c'est à dire l'ensemble des endomorphismes qui
% commutent avec tous les endomorphismes est l'ensemble des homothéties. On
% propose pour cela deux démonstrations : une géométrique et une autre
% matricielle.
% \begin{questions}
% \question 
%   \begin{questions}
%   \question Soit $u$ un endomorphisme de $E$ tel que~:
%     \[\forall x\in E \quad \exists \lambda\in\K \quad u(x)=\lambda x\]
%     Montrer qu'il existe $\lambda\in\K$ tel que $u=\lambda\id$.
%   \question Soit $u$ un endomorphisme de $E$ qui commute avec toutes les
%     applications linéaires.
%     \begin{questions}
%     \question Soit $x\in E$. En considérant une symétrie par rapport à la
%       droite vectorielle $\K x$, montrer qu'il existe $\lambda\in\K$ tel que
%       $u(x)=\lambda x$.
%     \question En déduire que $u$ est une homothétie, puis conclure.
%     \end{questions}
%   \end{questions}
% \question
%   \begin{questions}
%   \question Soit $A$ une matrice commutant avec toutes les matrices.
%     En considérant la base canonique de $\mat{n}{\K}$, montrer qu'il
%     existe $\lambda\in\K$ tel que $A=\lambda I_n$.
%   \question Conclure.
%   \end{questions}
% \end{questions}

\magsection{Matrices équivalentes, matrices semblables}
\magsubsection{Matrices équivalentes}

\exercice{nom={Rang}}
Soit $A,B\in\mat{q,p}{\K}$. Montrer que
\[\rg B \leq \rg A \quad\ssi\quad \cro{\exists Q\in\gl{q}{\K} \qsep
  \exists P\in\mat{p}{\K} \qsep B=QAP}.\]





\exercice{nom={Étude d'un système affine}}
Soit $a,b,c$ trois réels deux à deux distincts.
\begin{questions}
\question Montrer que le système
  \[\syslin{x&+ay&+a^2z&=&0\hfill\cr
            x&+by&+b^2z&=&0\hfill\cr
            x&+cy&+c^2z&=&0\hfill}\]
  est de Cramer.
\question Résoudre le système
  \[\syslin{
  x&+ay&+a^2z&=&a^4\cr
  x&+by&+b^2z&=&b^4\cr
  x&+cy&+c^2z&=&c^4}\]
\end{questions}

\magsubsection{Matrices semblables}



\exercice{nom={Réduction d'une matrice}}
Soit $A$ la matrice
\[A\defeq
  \begin{pmatrix}
  0  & 2  & -1\\
  3 & -2  & 0\\
  -2  & 2  & 1  
  \end{pmatrix}\]
\begin{questions}
\question Déterminer l'ensemble des valeurs propres de $A$.
\question Déterminer une matrice diagonale $D$ semblable à $A$. En déduire un
  polynôme annulateur non nul de $A$.
\question Expliciter les suites $u$, $v$ et $w$ définies par la donnée de
  $u_0$, $v_0$, $w_0$ et la relation de récurrence
  \[\forall n\in\N\qsep \syslin{u_{n+1}&\defeq&    & 2v_n&-w_n&\cr
            v_{n+1}&\defeq&3u_n&-2v_n&\cr
            w_{n+1}&\defeq&-2u_n&+2v_n&+w_n}\]
\end{questions}
\begin{sol}
$\quad$  
\begin{questions}
\question Échanger la troisième et la première colonne, puis faire un pivot de
  Gauss. Remarquer que la dernière ligne est factorisable par $\p{\lambda-2}$.
  On trouve comme valeurs propres~: 2,1,$-4$.
  De plus, $e_1=(4,3,-2)$ est un vecteur propre associé à 2, $e_2=(1,1,1)$
  est un vecteur propre associé à 1 et $e_3=(-2,3,-2)$ est un vecteur
  propre associé à $-4$.
\question
\question
  \[A^n=\frac{1}{30}\left( \begin {array}{ccc} 10\, \left( -1 \right) ^{n}{4}^{n}+20\,{2}
  ^{n}&12+12\, \left( -1 \right) ^{1+n}{4}^{n}&18+2\, \left( -1 \right) 
  ^{n}{4}^{n}-20\,{2}^{n}\\\noalign{\medskip}15\, \left( -1 \right) ^{1+
  n}{4}^{n}+15\,{2}^{n}&12+18\, \left( -1 \right) ^{n}{4}^{n}&18+3\,
   \left( -1 \right) ^{1+n}{4}^{n}-15\,{2}^{n}\\\noalign{\medskip}10\,
   \left( -1 \right) ^{n}{4}^{n}-10\,{2}^{n}&12+12\, \left( -1 \right) ^
  {1+n}{4}^{n}&18+2\, \left( -1 \right) ^{n}{4}^{n}+10\,{2}^{n}
  \end {array} \right) \]
\end{questions}
\end{sol}

\exercice{nom={Matrices telles que $M^2=0$}}
Déterminer les matrices $M\in\mat{3}{\K}$ telles que $M^2=0$.

\exercice{nom={Réduction des matrices nilpotentes}}
Soit $A\in\mat{n}{\K}$ une matrice nilpotente d'indice de nilpotence $n$,
c'est-à-dire telle que
\[A^n=0 \quad \text{et} \quad A^{n-1}\not=0.\]
Montrer que $A$ est semblable aux matrices
\[\begin{pmatrix}
  0       & \dots  & \dots  & \dots  & 0\\
  1       & \ddots &        &        & \vdots\\
  0       & \ddots & \ddots &        & \vdots\\
  \vdots  & \ddots & \ddots & \ddots & \vdots\\
  0       & \dots  & 0      & 1      & 0
  \end{pmatrix}\et
  \begin{pmatrix}
  0       & 1      & 0      & \dots  & 0\\
  \vdots  & \ddots & \ddots & \ddots & \vdots\\
  \vdots  &        & \ddots & \ddots & 0\\
  \vdots  &        &        & \ddots & 1\\
  0       & \dots  & \dots  & \dots  & 0
  \end{pmatrix}\]

\exercice{nom={Valeurs propres et de $AB$ et $BA$}}
On appelle valeur propre d'une matrice $X\in\mat{n}{\K}$ tout réel
$\lambda$ tel que $X-\lambda I_n$ ne soit pas inversible. Montrer que si $A$
et $B$ sont deux matrices de $\mat{n}{\K}$, alors $AB$ et $BA$ ont les
mêmes valeurs propres non nulles.

%END_BOOK
\end{document}