\documentclass{magnolia}

\magtex{tex_driver={pdftex}}
\magfiche{document_nom={Compléments d'algèbre},
          auteur_nom={François Fayard},
          auteur_mail={fayard.prof@gmail.com}}
\magexos{exos_matiere={maths},
         exos_niveau={mpsi},
         exos_chapitre_numero={4},
         exos_theme={Compléments d'algèbre}}
\magmisenpage{}
\maglieudiff{}
\magprocess

\begin{document}

%BEGIN_BOOK


% \exercice{nom={Décomposition en éléments simples}}
% Décomposer en éléments simples sur $\R$ les fractions
% \[\frac{10X^3}{(X^2+1)(X^2-4)} \et
%   \frac{X^3-1}{(X-1)(X-2)(X-3)}.\]
% % \[\frac{X^2}{(X^2+X+1)^2}, \qquad \frac{(X^2+4)^2}{(X^2+1)(X^2-2)^2}.\]
% \begin{sol}
% $\quad$
% \begin{questions}
% \question 
%   \[\frac{10X^3}{(X^2+1)(X^2-4)}=
%     \frac{4}{X-2}+\frac{4}{X+2}+\frac{2X}{X^2+1}\]
% % \question 
% %   \[\frac{X^4+1}{X^4+X^2+1}=
% %     1-\frac{1}{2}\cdot\frac{X}{X^2-X+1}+\frac{1}{2}\cdot\frac{X}{X^2+X+1}\]
% \question 
%   \[\frac{X^3-1}{(X-1)(X-2)(X-3)}=
%     1-\frac{7}{X-2}+\frac{13}{X-3}\]
% % \question 
% %   \[\frac{X^2}{(X^2+X+1)^2}=
% %     \frac{1}{X^2+X+1}-\frac{X+1}{\p{X^2+X+1}^2}\]\
% % \question 
% %   \begin{eqnarray*}
% %   \frac{(X^2+4)^2}{(X^2+1)(X^2-2)^2}&=&
% %     \frac{1}{X^2+1}+\frac{3\sqrt{2}}{4}\cdot\frac{1}{X+\sqrt{2}}
% %     +\frac{3}{2}\cdot\frac{1}{\p{X+\sqrt{2}}^2}\\
% %    & &
% %     -\frac{3\sqrt{2}}{4}\cdot\frac{1}{X-\sqrt{2}}
% %     +\frac{3}{2}\cdot\frac{1}{\p{X-\sqrt{2}}^2}
% %   \end{eqnarray*}
% \end{questions}
% \end{sol}


\magsection{Somme et produit}
\magsubsection{Somme}

\exercice{nom={Sommes}}
Simplifier les sommes suivantes.
\[\textbf{a.}\ \sum_{k=0}^n k(k-1), \qquad \textbf{b.}\ \sum_{k=1}^n (2k-1), \qquad \textbf{c.}\ \sum_{k=1}^n (-1)^k\]
\[\textbf{d.}\ \sum_{k=0}^n (k+n), \qquad \textbf{e.}\ \sum_{k=1}^{n+1} \frac{2^k}{3^{2k-1}}, \qquad
\textbf{f.}\ \sum_{k=2}^n \ln\p{1-\frac{1}{k^2}}.\]

\exercice{nom={Décomposition en éléments simples}}
\begin{questions}
\question Montrer qu'il existe $a,b\in\R$ tels que
  \[\forall k\in\N\qsep \frac{1}{(3k+1)(3k+4)}=\frac{a}{3k+1}+\frac{b}{3k+4}.\]
\question En déduire
  \[\sum_{k=0}^n \frac{1}{(3k+1)(3k+4)}.\]
\end{questions}

\exercice{nom={Sommes}}
\begin{questions}
\question Montrer que pour tout $n\in\N$
  \[\sum_{k=0}^n k^3=\p{\frac{n(n+1)}{2}}^2.\]
\question Montrer que pour tout $n\in\N$
  \[\sum_{k=0}^n (-1)^k k^2=(-1)^n \frac{n(n+1)}{2}.\]
\end{questions}

\exercice{nom={Récurrence}}
Montrer par récurrence que pour tout $n\in\Ns$
\[\sum_{k=1}^{2n} \frac{(-1)^{k-1}}{k}=\sum_{k=1}^n \frac{1}{n+k}.\]

\exercice{nom={Somme lacunaire de coefficients binomiaux}}
En considérant $(1+1)^n$ et $(1-1)^n$ calculer, pour tout $n\in\Ns$, les sommes
\[A_n\defeq \sum_{\substack{0\leq k\leq n\\\text{$k$ pair}}} \binom{n}{k}
  \qquad\text{et}\qquad
  B_n\defeq \sum_{\substack{0\leq k\leq n\\\text{$k$ impair}}} \binom{n}{k}.\]

\exercice{nom={Coefficients binomiaux}}
Montrer par récurrence que pour tout $p\in\N$ et $n\geq p$
\[\sum_{k=p}^n \binom{k}{p}=\binom{n+1}{p+1}.\]

\exercice{nom={Coefficients binomiaux}}
Simplifier, pour tout $n\in\N$, la somme
\[\sum_{k=0}^n \frac{(-1)^k}{k+1}\binom{n}{k}.\]

\magsubsection{Produit}

\exercice{nom={Produits}}
Simplifier les produits suivants en les exprimant le plus possible à l'aide
de puissances et de factorielles.
\[\textbf{a.}\ \prod_{k=1}^n \sqrt{k(k+1)},\qquad
  \textbf{b.}\ \prod_{k=1}^n (-5)^{k^2-k},\qquad
  \textbf{c.}\ \prod_{k=1}^n \frac{4^k}{k^2}\]
\[\textbf{d.}\ \prod_{k=0}^n (2k+1),\qquad
  \textbf{e.}\ \prod_{k=1}^n (4k^2-1),\qquad
  \textbf{f.}\ \prod_{p=0}^{n-1} \sum_{k=0}^p 2^{p! k}.\]

\exercice{nom={Produit}}
Soit $z\in\C$. Pour tout $n\in\N$, on pose
\[P_n\defeq \prod_{k=0}^n \p{1+z^{2^k}}.\]
Calculer $(1-z)P_n$ et en déduire une expression simple de $P_n$.

\exercice{nom={Majoration}}
\begin{questions}
\question
\begin{questions}
\question Montrer que
  \[\forall k\geq 2\qsep 1+\frac{1}{k^2} \leq \frac{k^2}{(k-1)(k+1)}.\]
\question En déduire que
  \[\forall n\in\Ns\qsep \prod_{k=1}^n \p{1+\frac{1}{k^2}}\leq 4.\]
\end{questions}
\question Montrer que pour tout $n\in\Ns$
  \[\prod_{k=1}^n \p{1+\frac{1}{k^3}}\leq 3-\frac{1}{n}.\]
\end{questions}

\exercice{nom={Limite}}
\begin{questions}
\question Factoriser $k^3-1$ par $k-1$ et $k^3+1$ par $k+1$ pour tout $k\geq 2$.
\question En déduire, sans récurrence, que pour tout $n\geq 2$
  \[\prod_{k=2}^n \frac{k^3-1}{k^3+1}=\frac{2}{3}\cdot\frac{n^2+n+1}{n(n+1)}.\]
\question En déduire la limite de la suite de terme général
  \[u_n\defeq \prod_{k=2}^n \frac{k^3-1}{k^3+1}.\] 
\end{questions}


\magsubsection{Somme et produit doubles}

\exercice{nom={Sommes}}
Simplifier les sommes suivantes.
\[\textbf{a.}\ \sum_{1\leq i\leq j\leq n} i, \qquad
  \textbf{b.}\ \sum_{1\leq i < j\leq n} j, \qquad
  \textbf{c.}\ \sum_{1\leq i,j\leq n} (i+j)\]
\[\textbf{d.}\ \sum_{1\leq i,j\leq n} x^{i+j},\qquad
  \textbf{e.}\ \sum_{1\leq i\leq j\leq n} \frac{i}{j+1},\qquad
  \textbf{f.}\ \sum_{1\leq i\leq j\leq n} (j-i)\]
\[\textbf{g.}\ \sum_{1\leq i\leq j\leq n} \frac{i^2}{j},\qquad
  \textbf{h.}\ \sum_{1\leq i,j\leq n} (i+j)^2, \qquad
  \textbf{i.}\ \sum_{1\leq i,j\leq n} \max(i,j).\]
% \exercice{nom={Une somme de série}}
% Calculer la limite lorsque $n$ tend vers $+\infty$ de
% \[\sum_{k=2}^n \frac{3k^2-1}{(k-1)^2k^2(k+1)^2}.\]
% On pourra calculer $P'/P^2$ avec $P\defeq(X-1)X(X+1)$.
% \begin{sol}
% On pose $P\defeq (X-1)X(X+1)$. Alors, il existe $a,b,c\in\R$ tels que
% \[\frac{1}{P}=\frac{1}{(X-1)X(X+1)}=\frac{a}{X-1}+\frac{b}{X}+\frac{c}{X+1}.\]


% On trouve
% \[\frac{3X^2-1}{(X-1)^2X^2(X+1)^2}=\frac{1}{2}\cdot\frac{1}{(X-1)^2}
%   -\frac{1}{X^2}+\frac{1}{2}\cdot\frac{1}{(X+1)^2}\] en primitivant la DES de $-1/P$.
% (Astuce de Anis Marrakchi : Pour connaître le coefficient au dessus de
% $1/(X+1)$ et $-1/(X-1)$ on regroupe ces éléments simples, on multiplie par
% $X^2$ et on fait tendre $x$ vers $+\infty$ // Autre astuce de Anis Marrakchi :
% on peut remarquer que $F=P'/P^2$).
% On obtient donc une somme téléscopique. La somme de la série est donc
% $\frac{3}{8}$.
% \end{sol}


\exercice{nom={Avec des racines $n$-ièmes}}
Soit $n\geq 2$ et $z\in\C$. On pose $\omega\defeq \e^{\ii\frac{2\pi}{n}}$ et
\[S_n\defeq \sum_{k=0}^{n-1} \p{z+\omega^k}^n.\]
Calculer $S_n$.


\exercice{nom={Somme double}}
Soit $n\in\N$. Calculer
\[\sum_{0 \leq i,j \leq n} 2^{i-j} \et \sum_{\substack{(i,j)\in\intere{0}{n}^2\\0\leq i+j \leq n}} \binom{n}{i+j}.\]
\begin{sol}
\begin{questions}
\question
\[\sum_{i=0}^n \sum_{j=0}^n 2^{i-j}=2^{n+2}-4+2^{-n}.\]
\question
\[\sum_{i=0}^n \sum_{j=0}^n \binom{n}{i+j}=2^{n-1}(n+2).\]
\end{questions}
\end{sol}



\exercice{nom={Somme de {\sc Gauss}}}
Soit $n\in\N$ un entier impair. On pose
\[\omega\defeq\e^{\frac{2\ii\pi}{n}} \quad\text{et}\quad S\defeq \sum_{k=0}^{n-1} \omega^{k^2}.\]
\begin{questions}
\question Écrire $\abs{S}^2$ comme une somme double, puis montrer que
  \[\abs{S}^2=\sum_{k=0}^{n-1}\sum_{p=-k}^{n-k-1} \omega^{2pk+p^2}.\]
\question Soit $k\in\intere{0}{n-1}$.
  \begin{questions}
  \question Montrer que la fonction
  \[\dspappli{\phi}{\Z}{\C}{p}{\omega^{2pk+p^2}}\]
  est $n$-périodique.
  \question En déduire pour tout $k\in\intere{0}{n-1}$ une écriture simplifiée de
  \[\sum_{p=-k}^{n-k-1} \omega^{2pk+p^2}.\]
  \end{questions}
\question Simplifier
  \[\sum_{k=0}^{n-1} \omega^{2pk}\]
  pour tout $p\in\Z$.
\question En déduire que $\abs{S}=\sqrt{n}$.
\end{questions}

\begin{sol}
\begin{questions}
\question 
On calcule :
\begin{eqnarray*}
|S|^2&=& \conj{S}S\\
&=& \p{\sum_{k=0}^{n-1} \omega^{-k^2}}\p{\sum_{s=0}^{n-1} \omega^{s^2}}\\
&=& \sum_{k=0}^{n-1}\p{\omega^{-k^2}\sum_{s=0}^{n-1} \omega^{s^2}}\\
&=& \sum_{k=0}^{n-1}\p{\omega^{-k^2}\sum_{p=-k}^{n-k-1} \omega^{(p+k)^2}}\\
&=& \sum_{k=0}^{n-1}\sum_{p=-k}^{n-k-1} \omega^{2pk+p^2}
\end{eqnarray*}
\question Soit $k\in\intere{0}{n-1}$.
  \begin{questions}
  \question Facile car $\omega^n=1$.
  \question Ainsi, on a :
  \[\sum_{p=-k}^{n-k-1} \omega^{2pk+p^2}=\sum_{p=0}^{n-1} \omega^{2pk+p^2}.\]
  \end{questions}
\question On a pour tout $p\in\Z$ :
  \[\sum_{k=0}^{n-1} \omega^{2pk}=\sum_{k=0}^{n-1}\p{\omega^{2p}}^k.\]
  On a une somme géométrique, reste à savoir si $\omega^{2p}=1$ et c'est le cas si et seulement si $n$ divise $2p$.
  
  $\bullet$ 1er cas : Si $n$ ne divise pas $2p$, la somme géométrique fait $0$.
   $\bullet$ 2ème cas : Si $n$ divise $2p$, la somme fait $n$.
 
\question Déjà, faisons le bilan des deux premières questions. 


\[\abs{S}^2=\sum_{k=0}^{n-1}\sum_{p=-k}^{n-k-1} \omega^{2pk+p^2}=\sum_{k=0}^{n-1}\sum_{p=0}^{n-1} \omega^{2pk+p^2}=\sum_{p=0}^{n-1}\sum_{k=0}^{n-1}\omega^{2pk+p^2}=\sum_{p=0}^{n-1}\omega^{p^2}\sum_{k=0}^{n-1}\omega^{2pk}\]
Comme $n$ est impair, le seul $p$ entre $0$ et $n-1$ tel que $n$ divise $2p$ est $0$. Donc seul le terme en $p=0$ n'est pas nul et vaut $n$. D'où $|S|^2=n$ et on a le résultat souhaité.

\end{questions}
\end{sol}

\magsubsection{Fonction polynôme}

\exercice{nom={Polynôme à coefficients symétriques}}
\begin{questions}
\question Montrer que le changement de variable $u\defeq z+1/z$ simplifie l'équation
  \[z^4-5z^3+6z^2-5z+1=0\]
  en une équation du second degré en $u$.
\question En déduire l'ensemble de ses solutions sur $\C$.
\question Sur le même modèle, résoudre l'équation
  \[z^4+z^3-10z^2-z+1=0.\]
\end{questions}
\begin{sol}
$\quad$
\begin{questions}
\question On trouve $2\pm\sqrt{3}$.
\question Faire le changement de variable $u=x-1/x$. On trouve
  \[\frac{-\p{1+\sqrt{33}}\pm\sqrt{50+2\sqrt{33}}}{4}, \qquad
    \frac{-1+\sqrt{33}\pm\sqrt{50-2\sqrt{33}}}{4}\]
\end{questions}
\end{sol}

\exercice{nom={Simplification de racine}}
On pose
\[a\defeq \sqrt[5]{\frac{5\sqrt{5}+11}{2}}, \quad b\defeq \sqrt[5]{\frac{5\sqrt{5}-11}{2}} \et
  x\defeq a-b.\]
On souhaite montrer que $x=1$.
\begin{questions}
\question Calculer $ab$.
\question En déduire que $x$ est racine de $A(t)\defeq t^5+5t^3+5t-11$.
\question Montrer que $1$ est la seule racine positive de $A$ et conclure.
\end{questions}
\begin{sol}
Écrire $x=a-b$ et remarquer que $ab=1$. Développer $x^5$ et en déduire que
$x^5+5x^3+5x-11=0$. Factoriser ce polynôme par $\p{x-1}$~:
$\p{x-1}\p{x^4+x^3+6x^2+6x+11}$. Comme le second polynôme n'a pas de racine
positive, on en déduit que $x=1$.
\end{sol}

% \exercice{nom={Équations algèbriques}}
% \begin{questions}
% \question Résoudre les équations
%   \[x=\sqrt{2-x}, \qquad \sqrt{x+7}=\sqrt{x+2}+\sqrt{x-1}.\]
% \question Résoudre les équations suivantes sur $\R$
%   \[1+x+x^2+x^3+x^4+x^5=0, \qquad 1-x+x^2-x^3+x^4=0.\]
% \end{questions}
% \begin{sol}
% $\quad$
% \begin{questions}
% \question Seul 1 est racine pour la première. Seul 2 est racine pour la
%   seconde.
% \question Pour la première, seul 1 est solution. Pour la seconde, il n'y a pas
%   de solution.
% \end{questions}
% \end{sol}

\exercice{nom={Inéquation}}
Résoudre l'inéquation
\[4x+2\leq\sqrt{7x^3+15x^2+11x+3}.\]
On commencera bien entendu par donner son domaine de définition.

\begin{sol}
$7x^3+15x^2+11x+3=(x+1)(7x^2+8x+3)$ avec le terme du second degré toujours positif. Donc la racine est définie pour $x\geq -1$.
Si $x\leq -1/2$, $4x+2\leq 0$ donc l'inégalité est vérifiée pour $x\in [-1;-1/2]$.

Si $x\geq 1/2$, on peut élever les deux termes au carré tout en gardant l'équivalence et l'inégalité est équivalente à $(x-1)(7x^2+6x+1)\geq 0$. L'étude du trinôme + tableau de signe montre que $S=[-1;-1/2]\cup[-1/2;\dfrac{-3+\sqrt{2}}{7}]\cup [1;+\infty[.$
\end{sol}

\exercice{nom={Coefficients binomiaux}}
Écrire la somme
\[\sum_{k=0}^n (1+z)^k\]
de deux manières différentes. En déduire
\[\sum_{k=j}^n \binom{k}{j}.\]

\begin{sol}
D'une part,
$$\sum_{k=0}^n (1+X)^k=\sum_{k=0}^n \sum_{j=0}^k \binom{k}{j}X^j=\sum_{j=0}^n\p{\sum_{k=j}^n\binom{k}{j}}X^j.$$
D'autre part, $$\sum_{k=0}^n (1+X)^k=\frac{1-(1+X)^{n+1}}{1-(1+X)}=\frac{\sum_{k=0}^{n+1}\binom{n+1}{k}X^k -1}{X}=\sum_{k=1}^{n+1}\binom{n+1}{k}X^{k-1}=\sum_{j=0}^{n}\binom{n+1}{j+1}X^{j}.$$
Par identification des coefficients, on a pour tout $j\in \llbracket 0,n\rrbracket, \sum_{k=j}^n \binom{k}{j}=\binom{n+1}{j+1}$.
\end{sol}


\exercice{nom={Méthode de {\sc Cardan}}}
Soit $a,b,c\in\C$ et $Q(z)\defeq z^3+az^2+bz+c$. 
\begin{questions}
\question Pour quels $\alpha\in\C$ le polynôme $P(z)\defeq Q(z+\alpha)$ n'a-t-il pas de coefficient en $z^2$~?
\enonce On choisit un tel $\alpha$ et on définit $p,q\in\C$ tels que $P(z)=z^3-3pz+2q$.
\question 
\begin{questions}
\question Soit $u\in\Cs$. Montrer que
  \[u+\frac{p}{u}\]
  est racine de $P$ si et seulement si $w\defeq u^3$ est racine d'un trinôme $R$ que l'on déterminera.
\question On note $w_1$ et $w_2$ les racines complexes de $R$. Soit $u_1$ une racine cubique de $w_1$. Montrer qu'il existe une unique racine cubique $u_2$ de $w_2$ telle que $u_1u_2=p$.
\question Déterminer les racines de $Q$ en fonction de $\alpha$, $u_1$, $u_2$ et $\jj$.
\end{questions} 
\question Déterminer les racines des polynômes
  \[z^3+3z^2+6z+2,\qquad z^3-3z-1.\]
\end{questions}

\textit{C'est pour résoudre de telles équations, pour lesquelles $P$ admet trois racines réelles, mais $R$ n'en n'a pas, que {\sc Rafael Bombelli} (1526--1572) a introduit les nombres complexes.}

\begin{sol}
\begin{questions}
\question On trouve $\alpha=-a/3$.
\question 
\begin{questions}
\question On trouve que $w\defeq u^3$ est racine de $X^2+2qX+p^3$.
\question On a $w_1w_2=p^3$ donc $w_2=p^3/u_1^3$. Ainsi, les racines cubiques de $w_2$ sont $\dfrac{p}{u_1}$, $j\dfrac{p}{u_1}$ et $j^2\dfrac{p}{u_1}$. 
\question 
\begin{eqnarray*}
u\in\ens{u_1,ju_1,j^2u_1,u_2,ju_2,j^2u_2} &\Longleftrightarrow& u^3 \text{ est racine de } R \\
&\Longleftrightarrow& u^+\dfrac{p}{u} \text{ est racine de } P\\
&\Longleftrightarrow& u^+\dfrac{p}{u} \text{ est racine de } Q(X+\alpha)\\
&\Longleftrightarrow& u^+\dfrac{p}{u}+\alpha \text{ est racine de } Q
\end{eqnarray*}
Ainsi, les racines de $Q$ sont $$\mathcal{S}=\ens{u_1+u_2+\alpha; ju_1+j^2u_2+\alpha; j^2u_1+ju_2+\alpha}.$$
\end{questions} 
\question Déterminer les racines des polynômes
  \[X^3+3X^2+6X+2,\qquad X^3-3X-1.\]
\end{questions}
\end{sol}

\exercice{nom={Méthode de {\sc Ferrari}}}
Soit $a,b,c,d\in\C$ et $P(z)\defeq z^4+az^3+bz^2+cz+d$. 
\begin{questions}
\question Montrer qu'il existe un unique $\alpha\in\C$ tel que $Q(z)\defeq P(z+\alpha)$ ne possède pas de terme en $z^3$.
\enonce Pour la suite, $\alpha$ désignera cette valeur. Il existe donc $p,q,r\in\C$ tels que $Q(z)=z^4+p z^2+q z+ r$.
\question Soit $v\in\C$. Montrer que
  \[\forall z\in\C\qsep Q(z)=0 \quad\ssi\quad \p{z^2+\frac{v}{2}}^2=\p{v-p}z^2-qz+\p{\frac{v^2}{4}-r}.\]
  On pose alors $A(z)\defeq \p{v-p}z^2-qz+\p{v^2/4-r}$.
\question Montrer que $A$ admet une racine double si et seulement si $v$ est racine d'un polynôme $B$ de degré 3 que l'on déterminera.
\enonce Pour la suite, on suppose que $P(z)\defeq z^4-2z^3+3z^2+4zQ-10$.
\question Montrer que $B$ admet une racine évidente. En déduire les racines de $P$.
\end{questions}

\textit{On a ainsi prouvé, dans les deux exercices précédents, que l'on pouvait calculer les racines des équations de degré 3 et 4 à l'aide de racines $n$-ièmes de nombres complexes. Ces résultats étaient connus dès le 16e siècle. {\sc Niels Abel} (1802--1829), puis {\sc Évariste Galois} (1811--1832), ont démontré qu'il n'était pas possible de résoudre l'équation générale de degré 5 en utilisant des racines $n$-ièmes de nombres complexes.}

\exercice{nom={Principe du maximum pour les polynômes}}
Soit $P(z)\defeq\sum_{k=0}^n a_k z^k$ une fonction polynôme à coefficients complexes de degré inférieur ou égal à $n$. On se donne un réel positif $M$ tel que
$$\forall z\in\U \qsep \abs{P(z)}\leq M.$$
\begin{questions}
\question On pose $\omega\defeq\exp\p{\ii\frac{2\pi}{n+1}}$. Calculer
  $$\sum_{j=0}^n P\p{w^j}.$$
\question En déduire que $\abs{P(0)}\leq M$.
\end{questions}

\begin{sol}
\begin{questions}
\question 
  $$\sum_{j=0}^n P\p{w^j}=\sum_{j=0}^n\sum_{k=0}^na_k\omega^{jk}=\sum_{k=0}^na_k\sum_{j=0}^n\p{\omega^{k}}^j=(n+1)a_0.$$
\question Ainsi, $|(n+1)a_0|=\abs{\sum_{j=0}^n P\p{w^j}|}\leq (n+1)M$ par I.T permet de conclure.
\end{questions}
\end{sol}

\magsection{Trigonométrie}
\magsubsection{Égalité modulaire}
\magsubsection{Formules de trigonométrie}

% \exercice{nom={Calcul du cosinus et du sinus de $\frac{\pi}{12}$ et
%   $-\frac{\pi}{8}$}}
% \begin{questions}
% \question En utilisant les formules de trigonométrie, calculer
%   \[\sin^2\p{\frac{\pi}{12}}.\]
% \question En déduire le sinus, le cosinus et la tangente de l'angle de mesure
%   $\pi/12$ en utilisant des racines carrées.
% \question En s'inspirant de cette méthode, calculer le sinus, le cosinus et la
%   tangente de l'angle $-\pi/8$.
% \end{questions}
% \begin{sol}
% $\quad$
% \begin{questions}
% \question On a
%   \[\sin^2\p{\frac{\pi}{12}}=\frac{2-\sqrt{3}}{4}\]
% \question On trouve
%   \[\sin\p{\frac{\pi}{12}}=\frac{\sqrt{2-\sqrt{3}}}{2}, \quad \cos\p{\frac{\pi}{12}}=\frac{\sqrt{2+\sqrt{3}}}{2}, \quad
%     \tan\p{\frac{\pi}{12}}=2-\sqrt{3}\]
% \question On trouve
%   \[\sin\p{-\frac{\pi}{8}}=-\frac{\sqrt{2-\sqrt{2}}}{2}, \quad \cos\p{-\frac{\pi}{8}}=\frac{\sqrt{2+\sqrt{2}}}{2}, \quad
%     \tan\p{-\frac{\pi}{8}}=1-\sqrt{2}\]
% \end{questions}
% \end{sol}

% \exercice{nom={Sur les angles d'un triangle}}
% Soit $a,b,c$ trois réels tels que $a+b+c=\pi$. Montrer que
% \[\cos^2 a+\cos^2 b+\cos^2 c+2\cos a \cos b\cos c=1.\]

% \exercice{nom={Équations trigonométriques}}
% Résoudre les équations suivantes sur $\R$
% \[\cos\p{2x}+\cos(x)=-1\]
% \[\sqrt{3}\cos\p{5x}=\cos\p{2x}+\cos\p{12x} \qquad
%   \tan\p{2x}=3\tan{x}\]
% \[\cos^6 x+\sin^6 x=\frac{1}{4} \qquad
%   2\cos\p{\frac{x}{3}}-\sin\p{\frac{x}{2}}=2.\]
% \begin{sol}
% \quad
% \begin{questions}
% \question Exprimer $\cos\p{2x}$ en fonction de $\cos x$. On trouve
%   \[x\equiv\frac{\pi}{2}\ [\pi] \ou x\equiv\frac{2\pi}{3}\ [2\pi] \ou
%     x\equiv -\frac{2\pi}{3}\ [2\pi]\]
% \question Factoriser le second membre. L'équation se factorise alors par
%   $\cos\p{5x}$. On trouve
%   \[x\equiv\frac{\pi}{10}\ \cro{\frac{\pi}{5}} \ou
%     x\equiv\frac{\pi}{42}\ \cro{\frac{2\pi}{7}} \ou
%     x\equiv -\frac{\pi}{42}\ \cro{\frac{2\pi}{7}}\]
% \question Il faut $x\not\equiv\pi/4\ [\pi/2]$. On change tout en $\sin$, $\cos$
%   puis on factoriser par $\sin x$. On obtient $\sin x\p{1-2\cos\p{2x}}=0$.
%   Les solutions sont donc
%   \[x\equiv 0\ [\pi] \ou x\equiv\frac{\pi}{6}\ [\pi] \ou 
%     x\equiv -\frac{\pi}{6}\ [\pi]\]
% \question On pose $u=\cos^2 x$. On obtient un polynôme du second degré en $u$
%   qui a une unique racine $1/2$. On trouve
%   \[x\equiv\frac{\pi}{4}\ \cro{\frac{\pi}{2}}\]
% \question On pose $x=6u$. On factorise $2\p{\cos\p{2u}-1}-\sin\p{3u}$ par
%   $\sin u$ et on obtient $\sin u\p{4\sin^2 u-4\sin u-3}=0$.
%   Une racine est en dehors de $\interf{-1}{1}$. L'autre est $-1/2$. Finalement
%   \[x\equiv 0\ [6\pi] \ou x\equiv-\pi\ [12\pi] \ou x\equiv-5\pi\ [12\pi]\]
% \end{questions}
% \end{sol}

% \exercice{nom={Inéquations trigonométriques}}
% Résoudre les inéquations suivantes sur $\R$
% \[\cos x+\cos\p{x+\pi/3}\geq 0, \qquad 2\cos x+\sin x \leq 2.\]

% \exercice{nom={Calcul de somme}}
% \begin{questions}
% \question Calculer $\tan p-\tan q$.
% \question En déduire la valeur de
%   \[S_n=\sum_{k=1}^n \frac{1}{\cos\p{k\theta}\cos\p{\p{k+1}\theta}}\]
%   où $\theta\in\R$.
% \end{questions}
% \begin{sol}
% Il faut que $\theta\not\equiv\frac{\pi}{2k} \cro{\frac{\pi}{k}}$ pour tout
% $k\in\intere{1}{n+1}$. En suite on trouve
% \[S_n=\begin{cases}
%       \frac{\sin(n\theta)}{\cos((n+1)\theta)\cos(\theta)\sin \theta} &
%         \text{si $\theta\not\equiv 0\ [\pi]$}\\
%       n & \text{si $\theta=2k\pi$}\\
%       -n & \text{si $\theta=(2k+1)\pi$}
%       \end{cases}\]
% \end{sol}

\exercice{nom={Inégalité}}
Montrer que pour tout $x\in \R$ et $n \in \N$
\[|\sin(nx)| \leq n |\sin(x)|.\]
\begin{sol}
Récurrence et I.T
\end{sol}

\exercice{nom={Équations trigonométriques}}
Résoudre les équations suivantes d'inconnue $x$.
\[\textbf{a.}\ \cos(3x)=\sin(x), \qquad \textbf{b.}\ \cos x+\sin x=1+\tan x,\qquad \textbf{c.}\ \sin x+\sin(2x)=0\]
\[\textbf{d.}\ \tan(2x)=3\tan x,\qquad \textbf{e.}\ 2\sin x+\sin(3x)=0,\qquad \textbf{f.}\ 3\tan x=2\cos x\]
\[\textbf{g.}\ \cos x=\sqrt{3}\sin x, \qquad \textbf{h.}\ 2\cos(4x)+\sin x=\sqrt{3}\cos x.\]


\exercice{nom={Équations trigonométriques}}
\begin{questions}
\question Résoudre les équations \[\cos(x)-\cos(2x)=\sin(3x), \qquad \cos(5x)+2\cos (3x)+3\cos(x)=0\]
\question Résoudre l'équation
  \[\cos(x)+\sin(x)=\tan \left( \frac{x}{2} \right)\]
  en posant (soigneusement) $t\defeq\tan(x/2)$.
\end{questions}

\begin{sol}
\begin{questions}
\question On exprime les $\cos(nx)$ comme polynôme en $\cos(x)$. On obtient donc
  \begin{eqnarray*}
\forall x\in\R\qsep & & \cos(5x)+2\cos (3x)+3\cos(x)=0\\
&\ssi& 16\cos^5(x)-12\cos^3(x)+2\cos(x)=0\\
&\ssi& u(16 u^4-12u^2+2)=0 \quad\text{en posant $u=\cos(x)$}\\
&\ssi& u = 0 \ou u^2=\frac{1}{4} \ou u^2=\frac{1}{2}\\
&\ssi& u =0 \ou u =\pm\frac{1}{2} \ou u=\pm\frac{1}{\sqrt{2}}\\
&\ssi& \cos(x)=0 \ou \cos(x)=\pm\cos\frac{\pi}{3} \ou \cos(x)=\pm\cos\frac{\pi}{4}\\
&\ssi& x\equiv\frac{\pi}{2}\ [\pi]\ou\cos(x)\equiv\pm\frac{\pi}{3}\, [2\pi]\ou\cos(x)\equiv\pm\frac{2\pi}{3}\, [2\pi]\ou\\
&    & x\equiv\pm\frac{\pi}{4}\ [2\pi]\ou x\equiv\pm\frac{3\pi}{4}\, [2\pi].
  \end{eqnarray*}
Pour $\cos(x)-\cos(2x)=\sin(3x)$. On utilise $\cos(a)-\cos(b)$ pour la gauche et $\sin(2a)$ pour la droite. On obtient :
$$-2\sin(3x/2)\sin(-x/2)=2\sin(3x/2)\cos(3x/2).$$
On peut transformer $\sin(x/2)=\cos(\pi/2-x/2)$ puis les différents cas mènent à $x\equiv \pi/4 [\pi]$, $x\equiv -\pi/2 [2\pi]$ et $x\equiv 0 [2\pi/3]$.
\question Le domaine de définition de l'équation est $\mathcal{D}=\R\setminus\p{\pi+2\pi\Z}$. Soit $x\in\mathcal{D}$. On pose $t\defeq\tan(x/2)$. Alors
  \begin{eqnarray*}
\cos x+\sin x=\tan\frac{x}{2}
&\ssi& \frac{1-t^2}{1+t^2}+\frac{2t}{1+t^2}=t\\
&\ssi& 1-t^2+2t=t(1+t^2)\\
&\ssi& t^3+t^2-t-1=0\\
&\ssi& (t-1)(t+1)^2=0\\
&\ssi& t=1 \ou t=-1\\
&\ssi& \tan\frac{x}{2}=\tan\frac{\pi}{4} \ou \tan\frac{x}{2}=\tan\p{-\frac{\pi}{4}}\\
&\ssi& \frac{x}{2}\equiv\frac{\pi}{4}\ [\pi] \ou \frac{x}{2}\equiv-\frac{\pi}{4}\ [\pi]\\
&\ssi& x\equiv\frac{\pi}{2}\ [2\pi] \ou x\equiv-\frac{\pi}{2}\ [2\pi]\\
&\ssi& x\equiv\frac{\pi}{2}\ [\pi]
  \end{eqnarray*}
\end{questions}
\end{sol}

\exercice{nom={Calcul de somme}}
Pour tout $n\in\Ns$, calculer
\[\sum_{k=0}^{n-1} 3^k \sin^3 \p{\frac{\theta}{3^{k+1}}}.\]

\begin{sol}
On linéarise $\sin^3(x)=3/4\sin(x)-1/4\sin(3x)$ puis quand on écrit la somme elle devient télescopique. On obtient
\[\frac{1}{4}\cro{\sin(\theta)-3^n\sin\p{\frac{\theta}{3^n}}}.\]
\end{sol}

\exercice{nom={Mon capitaine}}
Pour tout $n\in\N$, calculer
\[\sum_{k=0}^n k\binom{n}{k}\cos (k \theta).\]


\magsection{Récurrence linéaire}
\magsubsection{Récurrence linéaire d'ordre 1}
\exercice{nom={Récurrences d'ordre 1}}
Déterminer une expression explicite des suites définies par
\begin{questions}
\question $u_0\defeq 0$ et $\forall n\in\N\qsep u_{n+1}\defeq 2 u_n+1$.
\question $u_0\defeq 1$ et $\forall n\in\N\qsep u_{n+1}\defeq 3 - \frac{u_n}{2}$.
\question $u_0\defeq 1$ et $\forall n\in\N\qsep u_{n+1}\defeq 2 u_n^2$.
\end{questions}


\magsubsection{Récurrence linéaire d'ordre 2}

\exercice{nom={Récurrences doubles}}
Déterminer les suites définies par
\[\textbf{a.}\ a_0\defeq 0,\quad a_1\defeq -1 \et \forall n\in\N\qsep a_{n + 2} \defeq 5a_{n + 1}-6a_n,\]
\[\textbf{b.}\ b_0\defeq 1,\quad b_1\defeq 1 \et \forall n\in\N\qsep b_{n + 2} \defeq -b_{n + 1}-b_n,\]
\[\textbf{c.}\ c_0\defeq 1,\quad c_1\defeq -1 \et \forall n\in\N\qsep c_{n + 2} \defeq 4c_{n + 1}-4c_n+n.\]

\begin{sol}
Pour le premier, on trouve $a_n=2^n-3^n$. Pour le second, on trouve $b_n=2\sin(\pi/6+n2\pi/3)$.
Pour la troisième, on cherche une solution particulière du type $v_n=n+\alpha$. On trouve $\alpha=2$. On résout l'équation homogène puis on ajoute les $2$ et ensuite avec les valeurs initiales on obtient :
$$c_n=(-1-n)2^n+n+2.$$
\end{sol}

\exercice{nom={Récurrence double}}
Déterminer une expression explicite de la suite $(u_n)$ définie par
\[u_0\defeq 1,\qquad
  u_1\defeq 2\qquad\text{et}\qquad
  \forall n\in\N\qsep u_{n+2}\defeq\frac{u_{n+1}^6}{u_n^5}.\]

\exercice{nom={Récurrence double avec second membre polynomial}}
On dira qu'une suite réelle $(u_n)$ est solution de $(E)$ lorsque
\[\forall n\in\N\qsep u_{n+2}=2u_{n+1}+8u_n+9n^2.\]
\begin{questions}
\question Montrer que $(E)$ possède une solution de la forme $(an^2+bn+c)$ où $a,b,c\in\R$.
\question En déduire une expression explicite de l'unique solution de $(E)$ telle que $u_0=0$ et $u_1=1$.
\end{questions}

\exercice{nom={Équation fonctionnelle}}
Déterminer les fonctions $\app{f}{\RP}{\RP}$ telles que
\[\forall x \geq 0\qsep f(f(x))=6x-f(x).\]

\begin{sol}
On procède par analyse-synthèse. On suppose qu'une telle solution existe.\\

Fixons $x_0\geq 0$ et définissons $(u_n)$ par $u_n=f^n(x_0)$. On démontre (en appliquant l'égalité en $f(x)$) $u_{n+2}+u_{n+1}-6u_n=0$. D'où : $$u_n=\lambda (-3)^n +\mu 2^n \quad \text{ avec en particulier } \lambda=\dfrac{2}{5}x_0-\dfrac{1}{5}f(x_0).$$

On peut écrire $u_n=\displaystyle (-3)^n\p{\lambda+\mu\p{\frac{-2}{3}}^n}$. On peut avec cette forme montrer que $u_n\geq 0 \forall n\in \N$ implique $\lambda=0$. Ainsi $f(x_0)=2x_0$ et cela pour $x_0$ quelconque dans $\RP$.

Synthèse : cela fonctionne.
\end{sol}


\magsection{Système linéaire}
\magsubsection{Système linéaire à $q$ équations et $p$ inconnues}


% \exercice{nom={Résolution de systèmes linéaires}}
% Résoudre les systèmes linéaires suivants~:
% \[\syslin{ x&+3y&=&3\hfill\cr
%           3x&-y &=&1\hfill} \qquad
%   \syslin{3x&-2y&=&1\hfill\cr
%            x&+y &=&2\hfill}\]
% \[\syslin{ 2x&-y &=&2\hfill\cr
%           -4x&+2y&=&0\hfill} \qquad
%   \syslin{ 2x&-y &=&2\hfill\cr
%           -4x&+2y&=&-4\hfill}\]

\exercice{nom={Résolution de systèmes linéaires}}
\begin{questions}
\question Résoudre les systèmes linéaires suivants
  \[\syslin{2x&+y &+z&=&1\hfill\cr
             x&-y &  &=&0\hfill\cr
            3x&+5y&+z&=&3,\hfill} \qquad
    \syslin{2x&-y &+2z&=&2\hfill\cr
             x&+y &+z &=&1\hfill\cr
             x&   &+z &=&1.\hfill}\]
\question Soit $a,b,c,d\in\R$. Résoudre le système linéaire
  \[\syslin{x&+y&+z&+t&=&a\hfill\cr
            x&+2y&+3z&+4t&=&b\hfill\cr
            x&+3y&+6z&+10t&=&c\hfill\cr
            x&+4y&+10z&+20t&=&d.\hfill}\]
\end{questions}
\begin{sol}
\begin{questions}
\question Pour le premier, on trouve $x=y=2/5$ et $z=-1/5$. Pour le deuxième, $y=0$ et $x+z=1$ donc $x=1-t_1$ et $t_1=z$.
\question Le système admet une unique solution
  \[\syslin{x&=&4a&-6b&+4c&-d\cr
            y&=&-6a&+14b&-11c&+3d\cr
            z&=&4a&-11b&+10c&-3d\cr
            t&=&-a&+3b&-3c&+d}\]
\end{questions}
\end{sol}

\exercice{nom={Équation fonctionnelle}}
Déterminer les fonctions $f:\RPs\to\R$ telles que
\[\forall x\in\RPs\qsep f(x)+3f\p{\frac{1}{x}}=x^2.\]

\exercice{nom={Sytème de type \nom{Vandermonde}}}
Soit $a,b,c$ trois réels deux à deux distincts. Résoudre le système
  \[\syslin{x&+ay&+a^2z&=&a^4\hfill\cr
            x&+by&+b^2z&=&b^4\hfill\cr
            x&+cy&+c^2z&=&c^4.}\]
\begin{sol}
Pour le deuxième système, on trouve 
$$z=a^2+b+c^2+ab+ac+bc$$
$$y=-(bc^2+b^2c+a^2b+ab^2+a^2c+ac^2)-2abc$$
$$x=a^2bc+ab^2c+abc^2.$$
\end{sol}



\exercice{nom={Résolution de systèmes linéaires}}
\begin{questions}
\question Soit $m\in\R$. Résoudre le système
  \[\syslin{mx&+ y&+ z&=&1\hfill\cr
             x&+my&+ z&=&m\hfill\cr
             x&+ y&+mz&=&m^2.\hfill}\]
\question Soit $m,a,b,c,\in\R$. On considère le système
  \[\syslin{-mx&+\p{m-1}y&+mz&=&a\hfill\cr
            \p{2m-1}x&+\p{m-1}y&-mz&=&b\hfill\cr
            -2x&-4y&+2mz&=&c.\hfill}\]
  Calculer le rang du système en fonction de $m$. Le résoudre dans ces différents cas.
\end{questions}
\begin{sol}
$\quad$
\begin{questions}
\question Si $m=1$, $x=1-t_1-t_2$, $y=t_1$, $z=t_2$. Si $m=-2$ le système n'a
  pas de solution. Sinon
  \[x=-\frac{m+1}{m+2} \qquad y=\frac{1}{m+2} \quad z=\frac{\p{m+1}^2}{m+2}\]
\question Le système n'admet pas une unique solution si et seulement si
  $m=0,\frac{1}{3},1$.
  \begin{itemize}
  \item Si $m=0$, alors
    \begin{itemize}
    \item Si $a+b-\frac{c}{2}\neq 0$, alors le système n'a pas de solution.
    \item Sinon $x=a-b$, $y=-a$ et $z=t$.
    \end{itemize}
  \item Si $m=\frac{1}{3}$, alors
    \begin{itemize}
    \item Si $4a+2b-c\neq 0$, alors le système n'admet pas de solution.
    \item Sinon, $x=-2t-\frac{3}{2}b-\frac{3}{2}a$, $y=t$,
      $z=\frac{3}{2}a-\frac{3}{2}b$.
    \end{itemize}
  \item Si $m=1$, alors
    \begin{itemize}
    \item Si $a+b\neq 0$, alors le système n'admet pas de solution.
    \item Sinon, $x=t-a$, $y=\frac{1}{2}a-\frac{1}{4}c$, $z=t$.
    \end{itemize}
  \end{itemize}
\end{questions}
\end{sol}

\exercice{nom={Calcul de rang}}
Soit $m\in\R$. Résoudre et déterminer le rang du système linéaire
\[\syslin{ x&+ y& +z& +t&=&3\hfill\cr
           x&+my& +z&-mt&=&m+2\hfill\cr
          mx& -y&-mz& -t&=&-1.\hfill}\]
\begin{sol}
En deux étapes (Pivot en haut à gauche puis échange des colonnes $t$ et $y$ et des lignes $2$ et $3$ puis $L_3\leftarrow L_3-L_2$) on se ramène à un système échelonné :
\[\syslin{ x&+ t& +z& +y&=&3\hfill\cr
           &  -(m+1)t&-2mz&-(m+1)y&=&-1-3m \hfill\cr
          & & 2mz& +2my&=&4m\hfill}\]
\begin{itemize}
\item[$\bullet$] Si $m=0$ :
\[\syslin{ x&+ t& +z& +y&=&3\hfill\cr
           &  t& &+y&=&1 \hfill}\]
           Le rang du système est $2$ et on peut prendre $z$ et $y$ en paramètre...
\item[$\bullet$] Si $m\neq 0$, le système devient 
\[\syslin{ x&+ t& +z& +y&=&3\hfill\cr
           &  -(m+1)t&-2mz&-(m+1)y&=&-1-3m \hfill\cr
          & & z& +y&=&2\hfill}\]
          \begin{itemize}
\item[$\bullet$] Si $m=-1$ : le système devient 
\[\syslin{ x&+ t& +z& +y&=&3\hfill\cr
           &  &z&&=&1 \hfill\cr
          & & z& +y&=&2\hfill}\]
          On a donc $y=1$, $z=1$ et $x+t=1$ donc c'est de rang $3$ et on passe $t$ en paramètre.
\item[$\bullet$] Si $m\neq -1$, le système est de rang $3$, on peut fixer $y$ comme paramètre et exprimer les autres en fonction. 
\end{itemize}
\end{itemize}
\end{sol}

\exercice{nom={Somme lacunaire de coefficients binomiaux}}
On définit $A,B$ et $C$ par
$$A\defeq\sum_{\substack{k=0\\k\egmodulo{0}{3}}}^{n} \binom{n}{k}, \qquad
  B\defeq\sum_{\substack{k=0\\k\egmodulo{1}{3}}}^{n} \binom{n}{k} \qquad\text{et}\qquad
  C\defeq\sum_{\substack{k=0\\k\egmodulo{2}{3}}}^{n} \binom{n}{k}.$$
\begin{questions}
\question Calculer $A+B+C$, $A+\jj B+\jj^2C$ et $A+\jj^2B+\jj C$.
\question En déduire $A$.
\question Sur le même modèle, étant donnés $b\in\Ns$ et $a\in\intere{0}{b-1}$, calculer
  $$A=\sum_{\substack{k=0\\k\egmodulo{a}{b}}}^{n} \binom{n}{k}.$$
\end{questions}
\begin{sol}
$\quad$
\begin{questions}
\question
\question On trouve
  \[A=\frac{2^n+2\cos\frac{n\pi}{3}}{3}\]
\question
  \[A=\frac{2^n+2^{\frac{n+2}{2}}\cos\frac{n\pi}{4}}{4}\]
\end{questions}
\end{sol}



\magsubsection{Interprétation géométrique lorsque $p=2$ ou $p=3$}
%END_BOOK

\end{document}
