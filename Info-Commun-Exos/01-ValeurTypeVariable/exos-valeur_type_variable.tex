\documentclass{magnolia}

\magtex{tex_driver={pdftex}}
\magfiche{document_nom={Valeur, type, variable},
					auteur_nom={François Fayard},
					auteur_mail={francois.fayard@auxlazaristeslasalle.fr}}
\magexos{exos_matiere={maths},
         exos_niveau={mpsi},
         exos_chapitre_numero={1},
         exos_theme={Valeur, type, variable}}
\magmisenpage{}
\maglieudiff{}
\magprocess

\begin{document}

%BEGIN_BOOK
\magsection{Valeur, type}

\magsubsection{Nombre entier}

\exercice{nom={Évaluer une expression}}
Déterminer la valeur et le type de chacune des expressions suivantes~:
\begin{pythoncode}
In [1]: 5 * 2 + 1 ** 2
In [2]: 5 * (2 + 1) ** 2
In [3]: -16 // 5
In [4]: 8 / 2
\end{pythoncode}

\exercice{nom={Les oeufs}}
On suppose que la variable \verb_n_ contient le nombre d'oeufs dont on dispose et on souhaite
calculer le nombre \verb_b_ de boites de 6 oeufs nécessaires à leur transport.
\begin{questions}
\question Pour quelles valeurs de $n$ l'expression \verb_n // 6_ donne-t-elle la bonne
  réponse~?
\question Trouver une expression donnant la bonne réponse.
\end{questions}
\begin{sol}
\begin{questions}
\question c'est la bonne réponse si $n$ est multiple de 6.
\question \verb_(n + 5) // 6_.
\end{questions}
\end{sol}

\magsubsection{Nombre flottant}

% \exercice{nom={Entiers et flottants}}
% Évaluer les expressions suivantes en déterminant celles qui renvoient des résultats de
% type \verb!int!.
% \begin{pythoncode}
% In [1]: 4 + 2
% In [2]: 1 - 6 * 7
% In [3]: 18 // 7
% In [4]: 42 / 6
% In [5]: 22 / (16 - 2 * 8)
% \end{pythoncode}

\exercice{nom={Évaluer une expression}}
Déterminer la valeur et le type de chacune des expressions suivantes, d'abord sans utiliser
Python, puis en l'utilisant.
\begin{pythoncode}
In [1]: 2 ** 3.0 + 4
In [2]: int(8.6) + 2
In [3]: float(2) ** 3
\end{pythoncode}

% \exercice{nom={Conversion}}
% \begin{questions}
% \question Les expressions suivantes renvoient-elles la même valeur~?
% \begin{pythoncode}
% In [1]: 8.5 / 2.1
% In [2]: int(8.5) / int(2.1)
% In [3]: int(8.5 / 2.1)
% \end{pythoncode}
% \question Que dire des expressions suivantes~?
% \begin{pythoncode}
% In [1]: float(8 * 2)
% In [2]: 8 * 2
% In [3]: 8. * 2.
% \end{pythoncode}
% \end{questions}

% \exercice{nom={Fonctions usuelles}}
% Comment calculer les nombres suivants~?
% \[\e^2\qsep \sqrt{13}\qsep \cos\frac{\pi}{5}\qsep \e^{\sqrt{5}},\]
% \[\ln 2\qsep \ln 10\qsep \log_{2} 10\qsep \tan\frac{\pi}{2}.\]

\magsubsection{Chaine de caractères}

\magsubsection{Booléen}

% \exercice{nom={Évaluer une expression}}
% Déterminer la valeur et le type de chacune des expressions suivantes, d'abord sans utiliser
% Python, puis en l'utilisant.
% \begin{pythoncode}
% In [1]: 2 == 1 + 1
% In [2]: 2 == 1 + 1 + 1
% In [3]: (2 == 1 + 1 + 1) and (2 == 1 + 1)
% In [4]: 2 == 1 + 1 + 1 or 2 == 1 + 1
% In [5]: 1 == 0 // 0
% In [6]: (not (0 == 0)) and (1 == 0 // 0)
% In [7]: 2 >= 3 or 2 ** 4 * 5 ** 2 // 20 == 20
% In [8]: True or 4 > 3 and 3 > 4
% In [9]: not False and False
% \end{pythoncode}

\exercice{nom={Année bissextile}}
Une année est bissextile dans les deux cas suivants.
\begin{itemize}
\item Si l'année est divisible par 4 et non divisible par 100.
\item Si l'année est divisible par 400.
\end{itemize}
On suppose que la variable $n$ contient l'année qui nous intéresse. Donner une expression
Python qui s'évalue en \verb_True_ si l'année est bissextile et en \verb_False_
sinon. 

\magsubsection{Tuple}

\magsection{Programmation impérative}

\magsubsection{Variable}

\magsubsection{État du système}

% \exercice{nom={Changement d'état}}
% \begin{questions}
% \question Quelle est la valeur affichée par l'interprète après la séquence d'instructions
%   suivante~?
% \begin{pythoncode}
% a = 3
% a = 4
% a = a + 2
% a
% \end{pythoncode}
% \question Quelle est la valeur affichée par l'interprète après la séquence d'instructions
%   suivante~?
% \begin{pythoncode}
% a = 2
% b = a * a
% b = a * b
% b = b * b
% b
% \end{pythoncode}
% \end{questions}

\exercice{nom={Suites d'affectations}}
% \begin{questions}
% \question Quelles sont les valeurs de \verb!a! et \verb!b! après les instructions suivantes~?
% \begin{pythoncode}
% a = 7
% b = a
% a = 9
% \end{pythoncode}
% \question Quelles sont les valeurs de \verb!x!, \verb!y!, \verb!z! après les instructions suivantes~?
% \begin{pythoncode}
% x = 23
% y = 18
% z = x 
% x = y 
% y = z 
% \end{pythoncode}
% \question
Quelles sont les valeurs de \verb!x!, \verb!y! après les instructions suivantes~?
\begin{pythoncode}
In [1]: x = 23
In [2]: y = 18
In [3]: x = x + y 
In [4]: y = x - y 
In [5]: x = x - y 
\end{pythoncode}
% \end{questions}

\exercice{nom={Quésako}}
On suppose que les variables \verb_a_ et \verb_b_ contiennent initialement les valeurs
$a_0$ et $b_0$. Quelles sont les valeurs contenues par \verb_a_ et \verb_b_ après les
instructions suivantes~?
\begin{pythoncode}
In [1]: a = a + b
In [2]: b = a - b
In [3]: a = a - b
\end{pythoncode}

% \exercice{nom={Échange}}
% S'il n'était pas possible d'effectuer plusieurs affectations simultanément, comment
% procéderiez-vous pour échanger le contenu de deux variables \verb_a_ et \verb_b_.

\magsubsection{Entrée, sortie}

\exercice{nom={Entrée, sortie}}
Donner l'état du shell après chacune des commandes suivantes.
\begin{pythoncode}
In [1]: 1 + 2
In [2]: print(1 + 2)
In [3]: print(print(1 + 2))
In [4]: print(1) + 2
In [5]: print(1) + print(2)
\end{pythoncode}


% \magsubsection{Logo}

% \exercice{nom={Noeud papillon}}

% Tracez en \textsc{logo} le noeud papillon suivant. La hauteur du noeud papillon est de
% 100 et sa largeur est de 200. On utilisera la fonction \verb_atan(x)_ du module
% \verb_math_ qui calcule la mesure $\theta\in\intero{-\pi/2}{\pi/2}$ en radian de
% l'unique angle dont la tangente vaut $x$.
% \begin{center}
% \includegraphics[width=0.3\textwidth]{../../Commun/Images/python-exos-val-1.pdf}
% \end{center}

%END_BOOK

\end{document}
