\documentclass{magnolia}

\magtex{tex_driver={pdftex},
        tex_packages={epigraph,xypic}}
\magfiche{document_nom={Cours sur les séries},
          auteur_nom={François Fayard},
          auteur_mail={fayard.prof@gmail.com}}
\magcours{cours_matiere={maths},
          cours_niveau={mpsi},
          cours_chapitre_numero={21},
          cours_chapitre={Famille sommable}}
\magmisenpage{misenpage_presentation={tikzvelvia},
          misenpage_format={a4},
          misenpage_nbcolonnes={1},
          misenpage_preuve={non},
          misenpage_sol={non}}
\magmisenpage{}
\maglieudiff{}
\magprocess

\begin{document}

%BEGIN_BOOK
\setlength\epigraphwidth{.6\textwidth}
\epigraph{\og Divergent series are the invention of the devil, and it is shameful to base on them any demonstration whatsoever. \fg}{--- {\sc Niels Abel (1802--1829)}}

\setlength\epigraphwidth{.4\textwidth}
\epigraph{$\dsp \sum_{k=1}^{+\infty} k=-\frac{1}{12}$}{--- {\sc Srinivasa Ramanujan (1887--1920)}}

\magtoc
\vspace{2ex}
Dans ce chapitre, $\K$ désignera l'un des corps $\R$ ou $\C$.

\section{Famille sommable}


Les séries nous ont permis de donner un sens, lorsque c'est possible, à
\[\sum_{n=0}^{+\infty} u_n.\]
Cependant, de nombreux problèmes arrivent lorsque l'on souhaite sommer des familles $(u_{i,j})_{(i,j)\in\N^2}$
indexées par deux entiers. Il serait naturel de définir une telle somme, lorsque les séries en jeu
sont convergentes, par
\[\sum_{i=0}^{+\infty} \sum_{j=0}^{+\infty} u_{i,j}.\]
Mais on trouve rapidement des exemples pour lesquels
\[\sum_{i=0}^{+\infty} \sum_{j=0}^{+\infty} u_{i,j} \quad\et\quad
  \sum_{j=0}^{+\infty} \sum_{i=0}^{+\infty} u_{i,j}\]
ont toutes deux un sens et des valeurs différentes. Par exemple, si on définit la famille $(u_{i,j})_{(i,j)\in\N^2}$
par
\[\forall i,j\in\N\qsep u_{i,j}\defeq\begin{cases}
  1 & \text{si $i=j$,}\\
  -1 & \text{si $i=j+1$,}\\
  0 & \text{sinon,}\end{cases}\]
les séries doubles définies plus haut ont toutes deux un sens, mais la première est égale à 1 tandis que la seconde
vaut 0.\\

Contrairement à ce qui se passe dans le cas des sommes finies, il arrive donc que la \og somme \fg des éléments
d'une famille infinie dépende de l'ordre de sommation. L'objet de la théorie des \emph{familles sommables} est d'avoir
un cadre dans lequel la somme de ces familles ne dépend pas de cet ordre.


\subsection{Famille sommable de réels positifs}

\begin{definition}
On pose $[0,+\infty]\defeq [0,+\infty[\cup\ens{+\infty}$.
\begin{itemize}
\item On étend la définition de $+$ sur $[0,+\infty]$ en posant
  \begin{eqnarray*}
  \forall x\in[0,+\infty],& &x + (+\infty) \defeq +\infty\\ 
                          & &(+\infty) + x \defeq +\infty
  \end{eqnarray*}
  On vérifie que $+$ est associative et commutative sur $[0,+\infty]$ et que 0 est élément
  neutre. 
\item On étend la définition de $\times$ en posant
  \begin{eqnarray*}
  \forall x\in]0,+\infty],& &x \times (+\infty) \defeq +\infty\\ 
                          & &(+\infty) \times x \defeq +\infty
  \end{eqnarray*}
  On pose enfin $0\times(+\infty)\defeq 0$ et $(+\infty)\times 0\defeq 0$. On vérifie que $\times$
  est associative et commutative sur $[0,+\infty]$ et que 1 est élément neutre. Enfin,
  $\times$ est distributive par rapport à $+$ sur $[0,+\infty]$.
\item On étend la définition de $\leq$ sur $[0,+\infty]$ en posant
  \[\forall x\in[0,+\infty]\qsep x\leq +\infty.\]
  Muni de $\leq$, $[0,+\infty]$ est un ensemble totalement ordonné.
\end{itemize}
\end{definition}

\begin{remarques}
\remarque Excepté $+\infty$, tous les éléments de $[0,+\infty]$ sont réguliers pour $+$.
  Excepté 0 et $+\infty$, tous les éléments de $[0,+\infty]$ sont réguliers pour $\times$.
\remarque La relation d'ordre $\leq$ reste compatible avec l'addition et la multiplication~:
  il est toujours possible d'ajouter et de multiplier entre elles des inégalités puisque
  ces dernières sont positives.
\end{remarques}

\begin{definition}
Soit $A$ une partie de $[0,+\infty]$. On dit que $A$ admet une \emph{borne supérieure dans
$[0,+\infty]$} lorsque l'ensemble des majorants de $A$ dans $[0,+\infty]$ admet un plus
petit élément. Si tel est le cas, on le note $\supb A$.
\end{definition}

\begin{proposition}
Toute partie de $[0,+\infty]$ admet une borne supérieure dans $[0,+\infty]$.
\end{proposition}

\begin{remarques}
\remarque Soit $A$ une partie de $[0,+\infty]$. Si $+\infty\in A$, alors $\supb A=+\infty$. Sinon,
  $A$ est une partie de $[0,+\infty[$ et
  \begin{itemize}
  \item Si $A$ est vide, alors $\supb A=0$.
  \item Si $A$ n'est pas majorée, alors $\supb A=+\infty$. 
  \item Sinon, $A$ est non vide majorée. Elle admet donc une borne supérieure $\sup A$
    dans $\R$ et $\supb A=\sup A$.
  \end{itemize}
\remarque Soit $A$ une partie de $[0,+\infty]$.
  \begin{itemize}
  \item Si $B$ est une partie de $A$, alors $\supb B\leq\supb A$.
  \item Si $x\in [0,+\infty]$, on définit $x+A$ par
    \[x+A\defeq \ensim{x+a}{a\in A}.\]
    Alors, si $A$ est non vide, $\supb(x+A)=x+\supb A$.
  \item Si $\lambda\in[0,+\infty]$, on définit $\lambda A$ par
    \[\lambda A\defeq \ensim{\lambda a}{a\in A}.\]
    Alors, $\supb(\lambda A)=\lambda\supb A$.
  \end{itemize}
% L'ensemble vide admet une borne supérieure dans $[0,+\infty]$ qui est 0. Si $A$
%   est une partie de $[0,+\infty[$ qui n'est pas majorée dans $[0,+\infty[$, alors elle
%   admet une borne supérieure dans $[0,+\infty]$ qui est $+\infty$.
% \remarque Si $A$ est une partie de $[0,+\infty[$ non vide majorée, elle admet une borne
%   supérieure $\alpha\defeq\sup A\in\R$ au sens usuel. Alors $\alpha$ est la borne supérieure de $A$
%   dans $[0,+\infty]$. 
\end{remarques}

\begin{definition}
Soit $(u_i)_{i\in I}$ une famille d'éléments de $[0,+\infty]$. On appelle \emph{somme des $u_i$}
pour $i\in I$ et on note $\sum_{i\in I} u_i$ la borne supérieure de
\[A\defeq\ensim{\sum_{i\in K} u_i}{\text{$K$ est une partie finie de $I$}}\]
dans $[0,+\infty]$.
\end{definition}

\begin{remarqueUnique}
\remarque Si $I$ est fini, la somme que l'on vient de définir n'est autre que la somme
  $\sum_{i\in I} u_i$ définie de manière classique.
% \remarque Soit $(u_i)_{i\in I}$ une famille de réels positifs.
%   \begin{itemize}
%   \item Alors $\sum_{i\in I} u_i=+\infty$ si et seulement si, quel que soit $m\in\RP$, il existe une partie
%     finie $J$ de $I$ tel que
%     \[\sum_{i\in J} u_i\geq m.\]
%   \item De plus, si $l\in\RP$, alors $\sum_{i\in I} u_i=l$ si et seulement si, quel que
%   soit la partie finie $J$ de $I$, on a
%   \[\sum_{i\in J} u_i\leq l\]
%   et quel que soit $\epsilon>0$, il existe une partie finie $J$ de $I$ telle que
%   \[\sum_{i\in J} u_i\geq l-\epsilon.\] 
%   \end{itemize}
\end{remarqueUnique}

\begin{exoUnique}
\exo Montrer que
  \[\sum_{(i,j)\in\Ns^2} \frac{1}{(i+j)^2}=+\infty.\]
\end{exoUnique}

\begin{definition}
On dit qu'une famille $(u_i)_{i\in I}$ d'éléments de $[0,+\infty]$ est \emph{sommable} lorsque
\[\sum_{i\in I} u_i < +\infty.\]
\end{definition}

\begin{remarqueUnique}
\remarque Si l'un des $x_i$ est égal à $+\infty$, alors
  \[\sum_{i\in I} u_i=+\infty.\]
  En particulier, tous les éléments d'une famille sommable sont réels.
\end{remarqueUnique}


\begin{proposition}
Soit $(u_n)_{n\in\N}$ une suite réelle positive. Alors, la famille $(u_n)_{n\in\N}$ est
sommable si et seulement si la série $\sum u_n$ est convergente. De plus, si tel est le
cas
\[\sum_{n\in\N} u_n=\sum_{n=0}^{+\infty} u_n.\]
\end{proposition}

\begin{remarqueUnique}
\remarque Si la série à termes positifs $\sum u_n$ diverge, alors
  \[\sum_{n\in\N} u_n=+\infty.\]
  C'est pourquoi, certains auteurs se permettent d'écrire
  $\sum_{n=0}^{+\infty} u_n=+\infty$.
\end{remarqueUnique}

\begin{proposition}
Soit $(u_i)_{i\in I}$ une famille d'éléments de $[0,+\infty]$ et $J$ une partie de
  $I$. Alors
  \[\sum_{i\in J} u_i \leq \sum_{i\in I} u_i.\]  
\end{proposition}

\begin{remarqueUnique}
\remarque En particulier, si $(u_i)_{i\in I}$ est sommable, alors $(u_i)_{i\in J}$ est sommable.
\end{remarqueUnique}



\begin{proposition}
Soit $(u_i)_{i\in I}$ et $(v_i)_{i\in I}$ deux familles d'éléments de $[0,+\infty]$ et $\lambda,\mu\in[0,+\infty]$.
Alors
\[\sum_{i\in I} \p{\lambda u_i+\mu v_i}=\lambda \sum_{i\in I} u_i + \mu \sum_{i\in I} v_i.\]
\end{proposition}

  

\begin{proposition}
Soit $(u_i)_{i\in I}$ et $(v_i)_{i\in I}$ deux familles d'éléments de $[0,+\infty]$
telles que
\[\forall i\in I\qsep u_i\leq v_i.\]
Alors
\[\sum_{i\in I} u_i \leq \sum_{i\in I} v_i.\]
\end{proposition}

\begin{remarqueUnique}
\remarque En particulier, si $(v_i)_{i\in I}$ est sommable, alors $(u_i)_{i\in I}$ est sommable.
\end{remarqueUnique}


\begin{proposition}
Soit $(u_i)_{i\in I}$ une famille d'éléments de $[0,+\infty]$ et $\sigma:J\to I$ une bijection.
Alors
\[\sum_{i\in I} u_i=\sum_{j\in J} u_{\sigma(j)}.\]
\end{proposition}


\begin{proposition}[nom={Théorème de sommation par paquets}]
Soit $(u_i)_{i\in I}$ une famille d'éléments de $[0,+\infty]$. Si $(I_j)_{j\in J}$ est
une partition de $I$, alors
\[\sum_{j\in J} \p{\sum_{i\in I_j} u_i}=\sum_{i\in I} u_i.\]
\end{proposition}

\begin{remarqueUnique}
\remarque En particulier, si $(u_i)_{i\in I}$ une famille d'éléments de $[0,+\infty]$ et
    $I_1,I_2\in\mathcal{P}(I)$ sont tels que $I=I_1 \sqcup I_2$, alors
    \[\sum_{i\in I} u_i = \sum_{i\in I_1} u_i + \sum_{i\in I_2} u_i.\]
\end{remarqueUnique}

\begin{exoUnique}
\exo Pour quelles valeurs de $\alpha\in\R$ la famille
  \[\p{\frac{pq}{(p+q)^{\alpha}}}_{(p,q)\in\Ns^2}\]
  est-elle sommable~?
\begin{sol}
Si et seulement si $\alpha>4$. 
\end{sol}
\end{exoUnique}

\begin{proposition}[nom={Théorème de \nom{Fubini}}]
Soit $(u_{i,j})_{(i,j)\in I\times J}$ une famille d'éléments de $[0,+\infty]$. Alors
\[\sum_{j\in J} \p{\sum_{i\in I} u_{i,j}}=
  \sum_{i\in I} \p{\sum_{j\in J} u_{i,j}}=
  \sum_{(i,j)\in I\times J} u_{i,j}.\]
\end{proposition}

% \begin{remarqueUnique}
%   \remarque Soit $(u_{i,j})_{(i,j)\in\N^2}$ une famille d'éléments de $[0,+\infty]$. Alors
%   \[
%   \sum_{(i,j)\in\N^2} u_{i,j} =
%     \sum_{i\in\N} \sum_{j\in\N} u_{i,j} =
%     \sum_{j\in\N} \sum_{i\in\N} u_{i,j}\]
%   % \remarque Nous verrons qu'il existe des familles $(u_{i,j})_{(i,j)\in\N^2}$ telles que les sommes
%   %   \[\sum_{i=0}^{+\infty} \sum_{j=0}^{+\infty} u_{i,j} \quad\et\quad 
%   %     \sum_{j=0}^{+\infty} \sum_{i=0}^{+\infty} u_{i,j}\]
%   %   soient bien définies et distinctes. Bien entendu, de telles familles ne sont pas sommables.
%   \end{remarqueUnique}

\begin{proposition}
Soit $(u_i)_{i\in I}$ et $(v_j)_{j\in J}$ deux familles d'éléments de $[0,+\infty]$.
Alors
\[\sum_{(i,j)\in I\times J} u_i v_j = \p{\sum_{i\in I} u_i} \p{\sum_{j\in J} v_j}.\]
\end{proposition}


\begin{exos}
\exo Soit $a,b\in\RPs$. Montrer que la famille
  \[\p{\e^{-(ap+bq)}}_{(p,q)\in\N^2}\]
  est sommable et calculer sa somme.
\exo On considère la fonction $\zeta$ de \nom{Riemann} définie par
  \[\zeta(x)\defeq\sum_{n\in\Ns} \frac{1}{n^x}.\]
  pour tout $x\in\R$ pour lequel la somme ci-dessus est finie.
  \begin{questions}
  \question Montrer que le domaine de définition de $\zeta$ est $]1,+\infty[$.
  \question Montrer que
    \[\sum_{n\geq 2} (\zeta(n)-1)=1.\] 
  \end{questions}
\end{exos}

% \subsection{Familles sommables de réels positifs}




% \begin{proposition}
% Soit $(u_i)_{i\in I}$ et $(v_i)_{i\in I}$ deux familles d'éléments de $\RP$ telles que
% \[\forall i\in I\qsep 0\leq u_i\leq v_i.\]
% Si $(v_i)_{i\in I}$ est sommable, alors $(u_i)_{i\in I}$ est sommable et
% \[\sum_{i\in I} u_i\leq\sum_{i\in I} v_i.\] 
% \end{proposition}


% \begin{proposition}
% Soit $(u_i)_{i\in I}$ et $(v_i)_{i\in I}$ deux familles sommables d'éléments de $\RP$ et
% $\lambda,\mu\in\RP$. Alors $(\lambda u_i+\mu v_i)_{i\in I}$ est sommable et
% \[\sum_{i\in I} \p{\lambda u_i+\mu v_i} = \lambda \sum_{i\in I} u_i + \mu \sum_{i\in I} v_i.\]
% \end{proposition}


% \begin{proposition}[nom={Théorème de \nom{Fubini}}]
% Soit $(u_{i,j})_{(i,j)\in I\times J}$ une famille d'éléments de $\RP$. Alors
% \[\sum_{(i,j)\in I\times J} u_{i,j}=
%   \sum_{j\in J} \p{\sum_{i\in I} u_{i,j}}=
%   \sum_{i\in I} \p{\sum_{j\in J} u_{i,j}}.\]
% \end{proposition}


\subsection{Famille sommable d'éléments de $\K$}

\begin{definition}
On dit qu'une famille $(u_i)_{i\in I}$ d'éléments de $\K$ est \emph{sommable} lorsque
la famille des réels positifs $(\abs{u_i})_{i\in I}$ est sommable.
\end{definition}

\begin{remarques}
\remarque L'ensemble des familles sommables indexées par $I$ est noté $\ell^1(I,\K)$
  ou $\ell^1(I)$.
  C'est un sous-espace vectoriel de $\K^I$.
\remarque Si $(u_i)_{i\in I}$ est une famille sommable et $J$ est une partie de $I$, alors $(u_i)_{i\in J}$ est sommable.
\end{remarques}

\begin{exos}
\exo Montrer que la famille
  \[\p{\frac{\sin(p+q)}{p^2 q^2}}_{(p,q)\in\Ns^2}\]
  est sommable.
\exo Montrer que la famille
  \[\p{z^{ij}}_{(i,j)\in\Ns^2}\]
  est sommable si et seulement si $\abs{z}<1$.
\end{exos}

\begin{definition}
Pour tout $x\in\R$ on définit respectivement la \emph{partie positive} $x^+$ 
et la \emph{partie négative} $x^-$ de $x$ par
\[x^+\defeq
  \begin{cases}
  x & \text{si $x\geq 0$}\\
  0 & \text{si $x<0$}
  \end{cases}
  \qquad \text{et} \qquad
  x^-\defeq
  \begin{cases}
  \abs{x} & \text{si $x\leq 0$}\\
  0 & \text{si $x>0$.}
  \end{cases}\]
\end{definition}

\begin{proposition}
Pour tout $x\in\R$
\[x=x^+ - x^-,\qquad \abs{x}=x^+ + x^-,\qquad
  0\leq x^+\leq \abs{x} \qquad\text{et}\qquad
  0\leq x^-\leq \abs{x}.\]
\end{proposition}

\begin{definition}
\begin{itemize}
\item Soit $(u_i)_{i\in I}$ une famille de réels sommable. Alors, les familles $(u_i^{+})_{i\in I}$ et
  $(u_i^{-})_{i\in I}$ sont sommables et on définit
  \[\sum_{i\in I} u_i\defeq \sum_{i\in I} u_i^{+} - \sum_{i\in I} u_i^{-}.\]
\item Soit $(u_i)_{i\in I}$ une famille de nombres complexes sommable. En décomposant $u_i = a_i + \ii b_i$ en sa
  partie réelle et sa partie imaginaire, les familles $(a_i)_{i\in I}$ et $(b_i)_{i\in I}$ sont sommables et on définit
  \[\sum_{i\in I} u_i\defeq \sum_{i\in I} a_i + \ii \sum_{i\in I} b_i.\]
\end{itemize}
\end{definition}


\begin{proposition}
Soit $(u_n)_{n\in\N}$ une suite d'éléments de $\K$. Alors, la famille $(u_n)_{n\in\N}$ est
sommable si et seulement si la série $\sum u_n$ est absolument convergente. De plus, si tel est le
cas
\[\sum_{n\in\N} u_n=\sum_{n=0}^{+\infty} u_n.\]
\end{proposition}
  

\begin{proposition}
Soit $(u_i)_{i\in I}$ une famille sommable d'éléments de $\K$ et $l\in\K$. Alors
\[\sum_{i\in I} u_i=l\]
si et seulement si, quel que soit $\epsilon>0$, il existe une partie finie $K$ de $I$ telle que
pour toute partie finie $L$ de $I$ telle que $K\subset L$, on a
\[\abs{\sum_{i\in L} u_i - l}\leq\epsilon.\]
\end{proposition}

\begin{remarqueUnique}
\remarque La définition \og historique \fg d'une famille sommable est la suivante~: on dit
  qu'une famille $(u_i)_{i\in I}$ d'éléments de $\K$ est sommable lorsqu'il existe $l\in\K$ tel que quel
  que soit $\epsilon>0$, il existe une partie finie $K$ de $I$ telle que pour toute partie finie $L$ de
  $I$ telle que $K\subset L$, on a
  \[\abs{\sum_{i\in L} u_i - l}\leq\epsilon.\]
  Si c'est le cas, $l$ est unique, et est appelé somme de la famille $(u_i)_{i\in I}$. La proposition
  précédente nous montre donc que si une famille est sommable pour le sens donné dans ce cours,
  alors elle est sommable pour le sens \og historique \fg. Réciproquement, on peut montrer que si une famille
  est sommable pour le sens \og historique \fg, elle est sommable pour le sens donné dans ce cours.
  Cela montre l'équivalence des deux approches.
\end{remarqueUnique}

\begin{proposition}
Soit $(u_i)_{i\in I}$ et $(v_i)_{i\in I}$ deux familles sommables d'éléments de $\K$ et
$\lambda,\mu\in\K$. Alors $(\lambda u_i+\mu v_i)_{i\in I}$ est sommable et
\[\sum_{i\in I} \p{\lambda u_i+\mu v_i} = \lambda \sum_{i\in I} u_i + \mu \sum_{i\in I} v_i.\]
\end{proposition}

\begin{remarqueUnique}
\remarque Attention, il est possible que $(\lambda u_i+\mu v_i)_{i\in I}$ soit sommable sans
  que $(u_i)_{i\in I}$ et $(v_i)_{i\in I}$ le soient. Dans ce cas, il est bien sur
  interdit d'écrire 
  \[\sum_{i\in I} \p{\lambda u_i+\mu v_i} = \lambda \sum_{i\in I} u_i + \mu \sum_{i\in I} v_i\]
  puisque l'expression à droite de l'égalité n'a aucun sens.
  % Par exemple, bien que la famille
  % \[\p{\frac{1}{n(n+1)}}_{n\in\Ns}\] soit sommable, on ne peut pas écrire
  % \[\sum_{n\in\Ns} \frac{1}{n(n+1)}=\sum_{n\in\Ns} \p{\frac{1}{n}-\frac{1}{n+1}}=
  %   \sum_{n\in\Ns} \frac{1}{n} - \sum_{n\in\Ns}\frac{1}{n+1}\]
  % puisque les familles $\p{\frac{1}{n}}_{n\in\Ns}$ et $\p{\frac{1}{n+1}}_{n\in\Ns}$ ne sont pas sommables.
\end{remarqueUnique}

\begin{proposition}
  Soit $(u_i)_{i\in I}$ et $(v_i)_{i\in I}$ deux familles réelles sommables telles que
  \[\forall i\in I\qsep u_i\leq v_i.\]
  Alors
  \[\sum_{i\in I} u_i \leq \sum_{i\in I} v_i.\]
  \end{proposition}

\begin{proposition}
  Soit $(u_i)_{i\in I}$ une famille sommable d'éléments de $\K$. Alors
  \[\abs{\sum_{i\in I} u_i}\leq\sum_{i\in I} \abs{u_i}.\]
  \end{proposition}

\begin{proposition}
Soit $(u_i)_{i\in I}$ une famille sommable d'éléments de $\K$ et $\sigma:J\to I$ une bijection.
Alors $(u_{\sigma(j)})_{j\in J}$ est sommable et
\[\sum_{i\in I} u_i=\sum_{j\in J} u_{\sigma(j)}.\]
\end{proposition}

\begin{remarques}
\remarque Soit $\sum u_n$ une série absolument convergente. Alors, quel que soit
  la bijection $\sigma:\N\to\N$, la série $\sum u_{\sigma(n)}$ est absolument convergente et
  \[\sum_{n=0}^{+\infty} u_n=\sum_{n=0}^{+\infty} u_{\sigma(n)}.\]
\remarque Cette propriété est fausse pour les séries semi-convergentes. En effet, le théorème de réarrangement
  de \nom{Riemann} montre que si $\sum u_n$ est une série réelle semi-convergente, alors
  quel que soit
  $l\in\R$, il existe une bijection $\sigma:\N\to\N$ telle que
  \[\sum_{n=0}^{+\infty} u_{\sigma(n)} = l.\]
\end{remarques}

\begin{proposition}[nom={Théorème de sommation par paquets}]
Soit $(u_i)_{i\in I}$ une famille sommable d'éléments de $\K$. Si $(I_j)_{j\in J}$ est
une partition de $I$, alors pour tout $j\in J$ la famille $(u_i)_{i\in I_j}$ est
sommable. De plus, la famille $(\sum_{i\in I_j} u_i)_{j\in J}$ est sommable et
\[\sum_{j\in J} \p{\sum_{i\in I_j} u_i}=\sum_{i\in I} u_i.\]
\end{proposition}

\begin{proposition}[nom={Théorème de \nom{Fubini}}]
Soit $(u_{i,j})_{(i,j)\in I\times J}$ une famille sommable d'éléments de $\K$. Alors,
pour tout $j\in J$ la famille $(u_{i,j})_{i\in I}$ est sommable et pour tout $i\in I$,
la famille  $(u_{i,j})_{j\in J}$ est sommable. De plus, les
familles $(\sum_{i\in I} u_{i,j})_{j\in J}$ et $(\sum_{j\in J} u_{i,j})_{i\in I}$ sont
sommables et
\[\sum_{j\in J} \p{\sum_{i\in I} u_{i,j}}=
  \sum_{i\in I} \p{\sum_{j\in J} u_{i,j}}=
  \sum_{(i,j)\in I\times J} u_{i,j}.\]
\end{proposition}


\begin{remarqueUnique}
\remarque Soit $(u_{i,j})_{(i,j)\in\N^2}$ une famille sommable d'éléments de $\K$. Alors
\begin{eqnarray*}
\sum_{(i,j)\in\N^2} u_{i,j} &=&
  \sum_{i\in\N} \sum_{j\in\N} u_{i,j} =
  \sum_{j\in\N} \sum_{i\in\N} u_{i,j}\\
  &=& \sum_{i=0}^{+\infty} \sum_{j=0}^{+\infty} u_{i,j} =
  \sum_{j=0}^{+\infty} \sum_{i=0}^{+\infty} u_{i,j}.
\end{eqnarray*}
% \remarque Nous verrons qu'il existe des familles $(u_{i,j})_{(i,j)\in\N^2}$ telles que les sommes
%   \[\sum_{i=0}^{+\infty} \sum_{j=0}^{+\infty} u_{i,j} \quad\et\quad 
%     \sum_{j=0}^{+\infty} \sum_{i=0}^{+\infty} u_{i,j}\]
%   soient bien définies et distinctes. Bien entendu, de telles familles ne sont pas sommables.
\end{remarqueUnique}

\begin{exoUnique}
  \exo Montrer que la famille
    \[\p{\frac{\e^{\frac{2\ii k\pi}{n}}}{2^n}}_{k\in\Ns, n> k}\]
    est sommable et calculer sa somme.
  % \exo On définit la famille $(u_{i,j})_{(i,j)\in\N^2}$ par
  %   \[\forall i,j\in\N\qsep u_{i,j}\defeq
  %     \begin{cases}
  %     1 & \text{si $i=j$,}\\
  %     -\frac{1}{2^{j-i}} & \text{si $i<j$,}\\
  %     0 & \text{sinon.}
  %     \end{cases}\]
  %   Calculer
  %   \[\sum_{i=0}^{+\infty} \sum_{j=0}^{+\infty} u_{i,j} \quad\et\quad
  %     \sum_{j=0}^{+\infty} \sum_{i=0}^{+\infty} u_{i,j}.\]
  \end{exoUnique}

\begin{proposition}
  Soit $(u_i)_{i\in I}$ et $(v_j)_{j\in J}$ deux familles sommables d'éléments de $\K$.
  Alors la famille $(u_i v_j)_{(i,j)\in I\times J}$ est sommable et
  \[\sum_{(i,j)\in I\times J} u_i v_j = \p{\sum_{i\in I} u_i} \p{\sum_{j\in J} v_j}.\]
  \end{proposition}



\begin{definition}
Soit $(u_n)_{n\in\N}$ et $(v_n)_{n\in\N}$ deux suites. On appelle
\emph{produit de \nom{Cauchy}} de ces suites la suite $(w_n)_{n\in\N}$
définie par
\[\forall n\in\N\qsep w_n\defeq\sum_{k=0}^n u_k v_{n-k}.\]
\end{definition}

\begin{proposition}
Soit $\sum u_n$ et $\sum v_n$ deux séries absolument convergentes. Alors, leur produit de
\nom{Cauchy} est absolument convergent et
\[\sum_{n=0}^{+\infty} \p{\sum_{k=0}^n u_k v_{n-k}}=
  \p{\sum_{n=0}^{+\infty} u_n}\p{\sum_{n=0}^{+\infty} v_n}.\]
\end{proposition}

\begin{exoUnique}
\remarque L'objet de cet exercice est de donner une définition \og moderne \fg de l'exponentielle.
  \begin{questions}
  \question Pour tout $z\in\C$, montrer que la série
    \[\sum \frac{z^n}{n!}\]
    est absolument convergente; on note $\exp(z)$ sa somme.
  \question Montrer que
  \[\forall a,b\in\C\qsep \exp(a+b)=\exp(a) \exp(b).\]
  \end{questions}


% \remarque Montrer que
%   \[\forall x\in\intero{-1}{1}\qsep \sum_{n=1}^{+\infty} n x^{n-1}=\frac{1}{(1-x)^2}.\]
\end{exoUnique}


%END_BOOK
\end{document}