\documentclass{magnoliaold}

\magtex{tex_driver={pdftex},
        tex_packages={epigraph,pgfplots,caption,float,xypic}}
\magfiche{document_nom={Cours sur les nombres complexes},
          auteur_nom={François Fayard},
          auteur_mail={fayard.prof@gmail.com}}
\magcours{cours_matiere={maths},
          cours_niveau={mpsi},
          cours_chapitre_numero={1},
          cours_chapitre={Nombres Complexes}}
\magmisenpage{misenpage_presentation={tikzvelvia},
          misenpage_format={a4},
          misenpage_nbcolonnes={1},
          misenpage_preuve={non},
          misenpage_sol={non}}
\maglieudiff{}
\magprocess

\begin{document}

%BEGIN_BOOK
\setlength\epigraphwidth{.6\textwidth}
\epigraph{\og La voie la plus courte et la meilleure entre deux vérités du domaine réel passe souvent par le domaine imaginaire. \fg}{--- \textsc{Jacques Hadamard (1865--1963)}}
\hidemesometimes{
\bigskip
\hfill\includegraphics[width=0.2\textwidth]{../../Commun/Images/maths-cours-imaginary_numbers.png}}

\magtoc

\section{Le corps des nombres complexes}
\subsection{Définition, conjugaison, module}
Le carré de tout nombre réel étant positif, l'équation
$$x^2=-1$$
n'admet aucune solution réelle. Nous admettrons qu'il existe un ensemble de
nombres $A$ ayant les propriétés suivantes.
\begin{itemize}
  \item $\R\subset A$.
  \item On peut additionner, soustraire et multiplier les éléments de $A$ en
    utilisant les règles usuelles de l'algèbre.
  \item L'équation $z^2=-1$ admet au moins une solution sur $A$.
\end{itemize}
On note $\ii$ une solution de cette équation.

\begin{definition}[utile=-3]
  On appelle corps des nombres complexes et on note $\C$
  l'ensemble des nombres $x+\ii y$ où $x$ et $y$ sont réels.
\end{definition}

\begin{remarques}
\remarque $\R$ est inclus dans $\C$. Autrement dit, tout nombre réel est un nombre
  complexe.
\remarque $\C$ est stable par les opérations d'addition, de soustraction et
  de multiplication.
\begin{preuve}
Soit $z$ et $z'$ deux nombres complexes. Il existe donc
$x,y,x',y'\in\R$ tels que $z=x+\ii y$ et $z'=x'+\ii y'$. Alors
\begin{eqnarray*}
z+z' &=& x+\ii y+x'+\ii y'\\
     &=& \underbrace{\p{x+x'}}_{\in\R}
         +\ii \underbrace{\p{y+y'}}_{\in\R}\in\C,\\
z-z' &=& \underbrace{\p{x-x'}}_{\in\R}
         +\ii \underbrace{\p{y-y'}}_{\in\R}\in\C,\\
z z' &=& \p{x+\ii y}\p{x'+\ii y'}\\
     &=& x x'+\ii x y'+\ii x' y+\underbrace{\ii^2}_{=-1} y y'\\
     &=& \underbrace{\p{x x'-y y'}}_{\in\R} +
         \ii\underbrace{\p{x y'+x' y}}_{\in\R} \in\C.
\end{eqnarray*}
\end{preuve}
\end{remarques}

\begin{definition}[utile=-3]
  Pour tout nombre complexe $z$, il existe un unique couple de réels $(x,y)$ 
  tel que $z=x+\ii y$. Les réels $x$ et $y$ sont respectivement appelés \emph{partie réelle} et \emph{partie imaginaire} de $z$. On note
  $$\Re(z)\defeq x \quad \text{et} \quad \Im(z)\defeq y$$
et on a donc $z=\Re(z)+\ii\Im(z)$.
\end{definition}
\begin{preuve}
  Soit $z$ un nombre complexe.
  \begin{itemize}
  \item \emph{Existence}~:
    Par définition de $\C$, il existe $x,y\in\R$ tels que $z=x+\ii y$.
  \item \emph{Unicité}~:
    Soit $x_1,y_1,x_2,y_2\in\R$ tels que $z=x_1+\ii y_1$ et $z=x_2+\ii y_2$. Alors
    \begin{equation*}\begin{split}
    & x_1+\ii y_1=x_2+\ii y_2\\
    \text{donc}\quad  & x_1-x_2=\ii\p{y_2-y_1}\\
    \text{donc}\quad  & \p{x_1-x_2}^2=\ii^2\p{y_2-y_1}^2\\   
    \text{donc}\quad  & \p{x_1-x_2}^2=-\p{y_2-y_1}^2\\   
    \text{donc}\quad  & \underbrace{\p{x_1-x_2}^2}_{\geq 0}+
                        \underbrace{\p{y_2-y_1}^2}_{\geq 0}=0
    \end{split}\end{equation*}
    La somme de deux réels positifs n'étant nulle que si ces deux réels sont
    nuls, on en déduit que $x_1=x_2$ et $y_1=y_2$.
  \end{itemize}
\end{preuve}

\begin{remarques}
\remarque On appelle \emph{forme cartésienne} de $z\in\C$, toute écriture de la forme $z=x+\ii y$ où $x,y\in\R$. La proposition précédente affirme que, quel que soit $z\in\C$, une telle écriture existe et est unique.
\remarque Si $x_1,y_1,x_2,y_2\in\R$ sont tels que
  \[x_1+\ii y_1=x_2+\ii y_2,\]
  alors $x_1=x_2$ et $y_1=y_2$. On dit souvent qu'on procède par \emph{identification}, mais cette terminologie est abusive. On utilise en fait l'unicité provenant de la proposition précédente.
\remarque Soit $\mathcal{R}=\p{O,\ve{e_1},\ve{e_2}}$ un repère orthonormé
  direct du plan. À tout nombre complexe $z=x+\ii y$, on associe le point $M$ dont les coordonnées dans le repère
  $\mathcal{R}$ sont $\p{x,y}$. On a donc
  \[\ve{OM}=x\ve{e_1}+y\ve{e_2}\]
  et on dit que $M$ a pour affixe $z$. Pout tout point $M$ du plan, il existe un unique $z\in\C$ tel que
  $M$ a pour affixe $z$; on dit alors que $z$ est \emph{l'affixe} du point $M$. On a ainsi identifié
  $\C$ avec l'ensemble des points du plan.
\remarque Un nombre complexe est réel si et seulement si sa partie imaginaire est nulle. On dit qu'un nombre complexe
  est \emph{imaginaire pur} lorsque sa partie réelle est nulle. L'ensemble des nombres imaginaires purs est
  donc
  \[\ii\R\defeq\ensim{\ii y}{y\in\R}.\]
\remarque De même qu'on ne peut pas écrire d'inégalités entre les points du plan, les inégalités entre nombres
  complexes n'ont aucun sens.
\end{remarques}

\begin{proposition}[utile=-3]
Un nombre complexe est nul si et seulement si sa partie réelle et sa partie imaginaire le sont.
\end{proposition}

\begin{preuve}
Soit $z$ un nombre complexe. Alors $z=\Re(z)+\ii \Im(z)$. 
\begin{itemize}
\item Si $\Re(z)=0$ et $\Im(z)=0$, alors $z=0$.
\item Réciproquement, si $z=0$, alors $\Re(z)+\ii \Im(z)=0+\ii 0$. Par unicité de la forme cartésienne, on en déduit que
  \[\Re(z)=0 \et \Im(z)=0.\]
\end{itemize}
\end{preuve}

\begin{proposition}[utile=-3]
Soit $z$ et $z'$ deux nombres complexes, $\lambda$ et $\mu$ deux réels. Alors
\begin{itemize}
\item $\Re(\lambda z+\mu z')=\lambda\Re(z)+\mu\Re(z'),$
\item $\Im(\lambda z+\mu z')=\lambda\Im(z)+\mu\Im(z').$
\end{itemize}
\end{proposition}

\begin{preuve}
Soit $z$ et $z'$ deux nombres complexes, $\lambda$ et $\mu$ deux réels.
Il existe donc $x,y,x',y'\in\R$ tels que $z=x+\ii y$ et $z'=x'+\ii y'$. Alors
\begin{eqnarray*}
\lambda z+\mu z'
&=& \lambda\p{x+\ii y}+\mu\p{x'+\ii y'}\\
&=& \underbrace{\p{\lambda x+\mu x'}}_{\in\R}+\ii
    \underbrace{\p{\lambda y+\mu y'}}_{\in\R}
\end{eqnarray*}
Puisque $x=\Re(z)$, $y=\Im(z)$, $x'=\Re(z')$, $y'=\Im(z')$, on en déduit que
\[\Re(\lambda z+\mu z')=\lambda\Re(z)+\mu\Re(z') \et
  \Im\p{\lambda z+\mu z'}=\lambda\Im(z)+\mu\Im(z').\]
\end{preuve}

\begin{remarqueUnique}
\remarque Attention, l'identité $\Re(z z')=\Re(z)\Re(z')$ est fausse.
  Par exemple $\Re(\ii\cdot \ii)=-1$ et $\Re(\ii)\Re(\ii)=0$.
\end{remarqueUnique}

\begin{definition}[utile=-3]
  Soit $z$ un nombre complexe. On appelle \emph{conjugué} de $z$ et on note
  $\conj{z}$ le nombre complexe
  $$\conj{z}\defeq x-\ii y$$
  où $x$ et $y$ sont respectivement la partie réelle et imaginaire de $z$.
\end{definition}

\begin{remarqueUnique}
\remarque Si $M$ est le point d'affixe $z\in\C$, le point $M'$ d'affixe $\conj{z}$ est le symétrique de $M$ par rapport à l'axe $(Ox).$
\end{remarqueUnique}

\begin{proposition}[utile=-3]
  Soit $z,z'\in\C$. Alors
 \[\conj{z+z'}=\conj{z}+\conj{z}', \quad \conj{z z'}=\conj{z}\ \conj{z}', \et \conj{\conj{z}}=z.\]
\end{proposition}
\begin{preuve} Soit $z,z'\in\C$. Il existe donc $x,y,x',y'\in\R$ tels que
    $z=x+\ii y$ et $z'=x'+\ii y'$
  \begin{itemize}
  \item On a $z+z'=\p{x+x'}+\ii \p{y+y'}$ donc
    \[\conj{z+z'}=\p{x+x'}-\ii \p{y+y'}.\] Or
    \begin{eqnarray*}
    \conj{z}+\conj{z}'
    &=& x-\ii y+x'-\ii y'\\
    &=& \p{x+x'}-\ii \p{y+y'}\\
    &=& \conj{z+z'}
    \end{eqnarray*}
  \item
    De même $zz'=\p{xx'-yy'}+\ii \p{xy'+x'y}$, donc
    \[\conj{zz'}=\p{xx'-yy'}-\ii \p{xy'+x'y}.\] Or
    \begin{eqnarray*}
    \conj{z}\ \conj{z}'
    &=& \p{x-\ii y}\p{x'-\ii y'}\\
    &=& xx'-\ii xy'-\ii x'y-yy'\\
    &=& \p{xx'-yy'}-\ii \p{xy'+x'y}\\
    &=& \conj{z z'}
    \end{eqnarray*}
  \item Enfin $\conj{z}=x-\ii y=x+\ii(-y)$, donc
    $\conj{\conj{z}}=x-\ii(-y)=x+\ii y=z$.
  \end{itemize}
\end{preuve}

\begin{definition}[utile=-3]
Soit $z$ un nombre complexe. On appelle \emph{module} de $z$ et on note $\abs{z}$ le nombre réel positif
\[\abs{z}\defeq\sqrt{x^2+y^2}\]
où $x$ et $y$ sont respectivement la partie réelle et imaginaire de $z$.
\end{definition}

\begin{remarques}
% \item On voit directement que $\overline{z\conj{z}}=\conj{z}\conj{\conj{z}}=\conj{z}z$.
%   En particulier $z\conj{z}\in\R$. Par contre cette méthode ne permet pas de
%   montrer que $z\conj{z}$ est positif.
\remarque Si $x\in\R$ est considéré comme un nombre complexe, son module est égal à sa valeur absolue.
\remarque Si $M$ est le point d'affixe $z$, le module de $z$ est la distance $OM$. Si $A$ et $B$ sont deux points d'affixes respectives $a$ et $b$, alors $AB=\abs{b-a}$.
\remarque Si $a\in\C$ et $r>0$
  \begin{itemize}
  \item $\enstq{z\in\C}{\abs{z-a}=r}$ est le \emph{cercle} de centre $a$ et de rayon $r$.
  \item $\enstq{z\in\C}{\abs{z-a}\leq r}$ est le \emph{disque fermé} de centre $a$ et de rayon $r$.
  \item $\enstq{z\in\C}{\abs{z-a}<r}$ est le \emph{disque ouvert} de centre $a$ et de rayon $r$.
  \end{itemize}
  \remarque Soit $a,b,c\in\R$ tels qu'au moins l'un des deux réels $a,b$ est non nul.
  Alors, l'ensemble d'équation
  \[ax+by+c=0\]
  est une droite orthogonale au vecteur de coordonnées $(a,b)$.
\remarque Soit $a,b,c\in\R$. Alors l'ensemble d'équation
  \[x^2+y^2+ax+by+c=0\]
  est soit un cercle, soit un point, soit l'ensemble vide.
\end{remarques}

\begin{definition}[utile=-3]
On note $\U$ l'ensemble des nombres complexes de module $1$.
\end{definition}

\begin{remarqueUnique}
\remarque L'identification entre $\C$ et le plan complexe nous amène à identifier $\U$
  avec le cercle de centre $O$ et de rayon 1, appelé \emph{cercle trigonométrique}.
\end{remarqueUnique}

\begin{proposition}
Soit $z$ un nombre complexe. Alors
\[\abs{z}^2=z \conj{z}.\]
De plus, $\abs{z}=0$ si et seulement si $z=0$.
\end{proposition}
\begin{preuve}
Soit $z\in\C$. Il existe $x,y\in\R$ tels que $z=x+\ii y$. On a donc
\begin{eqnarray*}
\abs{z}=0
&\ssi& \abs{z}^2=0\\
&\ssi& \underbrace{x^2}_{\geq 0}+\underbrace{y^2}_{\geq 0}=0\\
&\ssi& x^2=0 \et y^2=0\\
&\ssi& x=0 \et y=0\\
&\ssi& z=0.
\end{eqnarray*}
\end{preuve}

\begin{remarqueUnique}
\remarque Afin d'exploiter cette identité, on cherchera souvent à travailler avec le carré des modules.
% \remarque Si $\mathcal{C}$ est le cercle de centre $a\in\C$ et de rayon $r>0$, alors
%   \begin{eqnarray*}
%   \forall z\in\C\qsep z\in\mathcal{C}
%   &\ssi& \abs{z-a}=r\\
%   &\ssi& \abs{z-a}^2=r^2\\
%   &\ssi& (z-a)\conj{(z-a)}=r^2\\
%   &\ssi& z\conj{z}-\conj{a}z-a\conj{z}+\p{a\conj{a}-r^2}=0.
%   \end{eqnarray*}
%   Réciproquement, si $a\in\C$ et $b\in\R$, l'ensemble d'équation $z\conj{z}-\conj{a}z-a\conj{z}+b=0$ est soit un cercle de centre $a$, soit réduit au point $a$, soit l'ensemble vide.
%   \begin{sol}
% Soit $a\in\C$ et $b\in\R$.
% \begin{eqnarray*}
% \forall z\in\C\qsep z\conj{z}-\conj{a}z-a\conj{z}+b=0
% &\ssi& z\conj{z}-\conj{a}z-a\conj{z}+a\conj{a}=a\conj{a}-b\\
% &\ssi& (z-a)\conj{(z-a)}=a\conj{a}-b\\
% &\ssi& \abs{z-a}^2=\underbrace{a\conj{a}-b}_{\eqdef c\in\R}
% \end{eqnarray*}
% \begin{itemize}
% \item Si $c>0$, on pose $r\defeq\sqrt{c}$. Alors
%   \[\forall z\in\C\qsep z\conj{z}-\conj{a}z-a\conj{z}+b=0 \quad\ssi\quad \abs{z-a}^2=r^2 \quad\ssi\quad \abs{z-a}=r\]
%   donc l'ensemble cherché est le cercle de centre $a$ et de rayon $r$.
% \item Si $c=0$, alors
%   \[\forall z\in\C\qsep z\conj{z}-\conj{a}z-a\conj{z}+b=0 \quad\ssi\quad \abs{z-a}=0 \quad\ssi\quad z=a\]
%   donc l'ensemble cherché est réduit au point $a$.
% \item Si $c<0$, l'ensemble est vide.
% \end{itemize}
%   \end{sol}
\end{remarqueUnique}

% \begin{exoUnique}
% \exo Donner une condition nécessaire et suffisante sur $z\in\C\setminus\ens{\ii}$ pour que
%   \[\frac{z+2}{1+\ii z}\]
%   soit réel.
%   \begin{sol}
%   Soit $z\in\C$. Alors
%   \begin{eqnarray*}
% \abs{z+\ii}=\abs{z-\ii}
% &\ssi& \abs{z+\ii}^2=\abs{z-\ii}^2\\
% &\ssi& (z+\ii)(\conj{z}-\ii)=(z-\ii)(\conj{z}+\ii)\\
% &\ssi& z=\conj{z} \ssi z\in\R
%   \end{eqnarray*}
%   Donc $z$ est solution de l'équation $\abs{z+\ii}=\abs{z-\ii}$ si et seulement si $z$ est réel.
%   \end{sol}
% \end{exoUnique}

\begin{proposition}[utile=-3]
  Soit $z,z'\in\C$. Alors
  \[\abs{z z'}=\abs{z}\abs{z'} \et \abs{\conj{z}}=\abs{z}.\]
\end{proposition}

\begin{preuve}
Soit $z,z'\in\C$. Alors
\[|z z'|^2=z z'\conj{z z'}=z z'\conj{z}\ \conj{z}'=z\conj{z}z'\conj{z}'=|z|^2|z'|^2=\p{|z||z'|}^2.\]
Ainsi, $\abs{z z'}$ et $\abs{z}\abs{z'}$ sont deux réels positifs qui ont même carré, ils sont donc égaux. De plus $|\conj{z}|^2=\conj{z}\ \conj{\conj{z}}=\conj{z}z=|z|^2$. Puisque $\abs{\conj{z}}$ et $\abs{z}$ sont deux réels positifs ayant même carrés, ils sont égaux. Donc $\abs{\conj{z}}=\abs{z}$.
\end{preuve}

% \subsection{Inverse}

\begin{proposition}
  Si $z$ et $z'$ sont deux nombres complexes tels que $z z'=0$, alors
  $z=0$ ou $z'=0$. On dit que $\C$ est \emph{intègre}.
\end{proposition}

\begin{preuve}
Soit $z,z'\in\C$ tels que $zz'=0$. Alors $|zz'|=0$, donc $|z||z'|=0$. Puisque $\R$ est intègre, l'un des deux modules est nul. D'après la proposition précédente, on en déduit que $z=0$ ou $z'=0$.

\end{preuve}
\begin{proposition}
  Soit $z$ un nombre complexe non nul. Alors il existe un unique nombre
  complexe $z'$ tel que $zz'=1$. On note ce nombre $z^{-1}$ ou $1/z$. De plus
  \[\frac{1}{z}=\frac{\conj{z}}{\abs{z}^2}.\]
\end{proposition}

\begin{preuve}
Soit $z$ un nombre complexe non nul.
\begin{itemize}
\item \emph{Unicité}~: Soit $z_1,z_2\in\C$ tels que $z z_1=1$ et $z z_2=1$. Alors $z z_1=z z_2$, donc $z(z_1-z_2)=0$. Puisque $z$ est non nul et que $\C$ est intègre, on en déduit que $z_1-z_2=0$, donc $z_1=z_2$.
\item \emph{Existence}~: On pose
\[z'= \conj{z} \cdot \frac{1}{\abs{z}^2}.\]
Alors
\[z z'=z\conj{z}\cdot\frac{1}{\abs{z}^2} =1.\]
\end{itemize}
\end{preuve}

\begin{remarques}
\remarque Pour obtenir $1/z$ sous forme cartésienne, il suffit de multiplier son
  numérateur et son dénominateur par $\conj{z}$.
\remarque Si $z\in\U$, alors $1/z=\conj{z}$. Pour inverser un nombre complexe de module 1, il suffit donc de le conjuguer.
\end{remarques}

\begin{exoUnique}
\exo Calculer l'inverse de $1+\ii$.
  \begin{sol}
  On a
  \[\frac{1}{1+\ii}=\frac{1-\ii}{(1+\ii)(1-\ii)}=\frac{1-\ii}{\abs{1+\ii}^2}=\frac{1-\ii}{2}=\frac{1}{2}-\frac{1}{2}\ii.\]
  \end{sol}
\end{exoUnique}

\begin{proposition}
  Soit $z$ un nombre complexe non nul. Alors
  $$\conj{\p{\frac{1}{z}}}=\frac{1}{\conj{z}} \quad \text{et} \quad
    \abs{\frac{1}{z}}=\frac{1}{\abs{z}}.$$
\end{proposition}

\begin{preuve}
Soit $z$ un nombre complexe non nul. D'une part
\[\conj{z}\conj{\p{\frac{1}{z}}}=\conj{z\cdot\frac{1}{z}}=\conj{1}=1\]
d'où le premier résultat. D'autre part
\[\abs{\frac{1}{z}}\abs{z}=\abs{\frac{1}{z}\cdot z}=\abs{1}=1\]
d'où le deuxième résultat.
\end{preuve}

\begin{exoUnique}
\exo Soit $a,b\in\C$ tels que $\abs{a}<1$ et $\abs{b}<1$. Montrer que
  \[\abs{\frac{a-b}{1-\conj{a}b}}<1.\]
\begin{sol}
Montrons d'abord que le dénominateur est non nul. On a
\[\abs{\conj{a}b}=\abs{\conj{a}}\abs{b}=\abs{a}\abs{b}<1\]
donc $\conj{a}b\neq 1$. De plus
 \begin{eqnarray*}
 \abs{\frac{a-b}{1-\conj{a}b}}<1&\Longleftrightarrow &|a-b|<|1-\conj{a}b|\\
 &\Longleftrightarrow & |a-b|^2<|1-\conj{a}b|^2\\
 &\Longleftrightarrow & (a-b)\conj{(a-b)}<(1-\conj{a}b)\conj{(1-\conj{a}b)}\\
 &\Longleftrightarrow & \ldots \\
 &\Longleftrightarrow & \abs{a}^2+\abs{b}^2<1+\abs{a}^2\abs{b}^2 \\
 &\Longleftrightarrow & (\abs{a}^2-1)(\abs{b}^2-1)>0
 \end{eqnarray*}
  Le résultat à droite de l'équivalence étant vrai grâce aux hypothèses, le résultat au départ l'est.
  \end{sol}
\end{exoUnique}

\begin{proposition}[utile=-3]
Soit $z$ un nombre complexe. Alors
\[\Re(z)=\frac{z+\conj{z}}{2} \et
  \Im(z)=\frac{z-\conj{z}}{2\ii}.\]
En particulier
\begin{itemize}
\item $z$ est réel si et seulement si $\conj{z}=z$.
\item $z$ est imaginaire pur si et seulement si $\conj{z}=-z$.
\end{itemize}
\end{proposition}
\begin{preuve}
Soit $z$ un nombre complexe. Il existe donc $x,y\in\R$ tels que $z=x+\ii y$. Alors $\conj{z}=x-\ii y$, donc
\[z+\conj{z}=2x \et z-\conj{z}=2\ii y.\]
\end{preuve}

\begin{remarqueUnique}
\remarque En pratique, pour montrer qu'un nombre complexe est réel, une
  bonne méthode est de montrer qu'il est égal à son conjugué. La méthode
  consistant à montrer que sa partie imaginaire est nulle est à proscrire.
\end{remarqueUnique}

\begin{exos}
\exo Soit $a$ et $b$ deux nombres complexes de module 1 tels que $ab\neq -1$. Montrer que
  \[\frac{a+b}{1+ab}\]
  est un nombre réel.
\begin{sol}
Soit $a$ et $b$ deux nombres complexes de module 1 tels que $ab\neq -1$. Alors
\begin{eqnarray*}
\conj{\p{\frac{a+b}{1+ab}}}
&=& \frac{\conj{a}+\conj{b}}{1+\conj{a}\conj{b}}\\
&=& \frac{\frac{1}{a}+\frac{1}{b}}{1+\frac{1}{ab}} \quad \text{car $\abs{a}=\abs{b}=1$.}\\
&=& \frac{b+a}{ab+1} = \frac{a+b}{1+ab}.
\end{eqnarray*}
Donc $(a+b)/(1+ab)$ est réel.
\end{sol}
\exo Donner une condition nécessaire et suffisante sur $z\in\C\setminus\ens{\ii}$ pour que
  \[\frac{z+2}{1+\ii z}\]
  soit réel.
  \begin{sol}
Cette expression a un sens pour tout $z\in\C\setminus\ens{\ii}$. Soit donc un tel $z$.
\begin{eqnarray*}
 \frac{z+2}{1+\ii z} \in \R&\Longleftrightarrow &\conj{\frac{z+2}{1+\ii z}}=\frac{z+2}{1+\ii z}\\
 &\Longleftrightarrow & (\conj{z}+2)(1+iz)=(z+2)(1-i\conj{z})\\
 &\Longleftrightarrow & \ldots \\
 &\Longleftrightarrow & z\conj{z}-\p{-1-\frac{1}{2}i}z-\p{-1+\frac{1}{2}i}\conj{z}=0 
 \end{eqnarray*}
Avec la remarque de la page précédente, on peut conclure que le lieu cherché est le cercle de centre $-1+\ii/2$ et de rayon $\sqrt{5/4}$ privé de $\ii$. Il passe par l'origine et par $z=-2$. 
  \end{sol}
\end{exos}



\subsection{Inégalité triangulaire}


\begin{proposition}[utile=-3]
Soit $a\in\C$. Alors
\[\Re(a)\leq\abs{\Re(a)}\leq\abs{a} \et
  \Im(a)\leq\abs{\Im(a)}\leq\abs{a}.\]
De plus, $\Re(a)=\abs{a}$ si et seulement si $a$ est réel positif.
\end{proposition}

% \begin{exoUnique}
% \exo Résoudre sur $\C$ le système
%   \[\begin{cases}
%     \abs{1+z}\leq 1 &\\
%     \abs{1-z}\leq 1. &
%     \end{cases}\]
%   \begin{sol}
% On raisonne par analyse-synthèse. Commençons par l'analyse et donnons-nous une solution $z\in\C$ du système. Alors $\abs{1+z}\leq 1$ et $\abs{1-z}\leq 1$ donc, en élevant au carré, $1+\abs{z}^{2}+2\Re{z}\leq 1$ et $1+\abs{z}^{2}-2\Re{z}\leq 1$. En ajoutant ces deux inégalités, on obtient $ \abs{z}^{2}\leq 0 $ donc $z=0$. La synthèse est immédiate car $z=0$ est bien solution de ce système. En conclusion l'ensemble des solutions est
% \[\mathcal{S}=\ens{0}.\]
% \end{sol}
% \end{exoUnique}

\begin{proposition}[utile=3, nom=Inégalité triangulaire]
Soit $a$ et $b$ deux nombres complexes. Alors
\[\abs{a+b}\leq\abs{a}+\abs{b}.\]
De plus, l'égalité a lieu si et seulement si  $a$ et $b$ sont positivement
liés, c'est-à-dire lorsque $a=0$ ou lorsqu'il existe $\lambda\in\RP$ tel que
$b=\lambda a$.
\end{proposition}

\begin{preuve}
Soit $a$ et $b$ deux nombres complexes. Alors
\begin{eqnarray*}
\abs{a+b}\leq\abs{a}+\abs{b}
&\ssi& \abs{a+b}^2\leq\p{\abs{a}+\abs{b}}^2, \quad\text{car $\abs{a+b}$ et $\abs{a}+\abs{b}$ sont positifs}\\
&\ssi& \p{a+b}\conj{\p{a+b}}\leq\p{\abs{a}+\abs{b}}^2\\
&\ssi& \p{a+b}\p{\conj{a}+\conj{b}}\leq\p{\abs{a}+\abs{b}}^2\\
&\ssi& a\conj{a}+a\conj{b}+\conj{a}b+b\conj{b}\leq\abs{a}^2+2\abs{a}\abs{b}+\abs{b}^2\\
&\ssi& \abs{a}^2+2\Re\p{\conj{a}b}+\abs{b}^2\leq\abs{a}^2+2\abs{a}\abs{b}+\abs{b}^2\\
&\ssi& \Re\p{\conj{a}b}\leq \abs{\conj{a}b}.
\end{eqnarray*}
Or, d'après la proposition précédente, ce dernier point est vrai. Donc $\abs{a+b}\leq\abs{a}+\abs{b}$.\\

En reprenant les équivalences précédentes, on a
\begin{eqnarray*}
\abs{a+b}=\abs{a}+\abs{b}
&\ssi& \Re\p{\conj{a}b}= \abs{\conj{a}b}\\
&\ssi& \conj{a}b\in\RP.
\end{eqnarray*}
Montrons que ce dernier point est vrai si et seulement si $a=0$ ou qu'il existe $\lambda\in\RP$ tel que $b=\lambda a$.
\begin{itemize}
\item Supposons que $\conj{a}b\in\RP$. On pose $\mu\defeq\conj{a}b\in\RP$. Si $a\neq 0$, alors
\[b=\frac{\mu}{\conj{a}}=\frac{\mu a}{a\conj{a}}=\underbrace{\frac{\mu}{\abs{a}^2}}_{\eqdef\lambda\in\RP}a.\]
\item Réciproquement, supposons que $a=0$ ou qu'il existe $\lambda\in\RP$ tel que $b=\lambda a$. Si $a=0$, alors $\conj{a}b=0\in\RP$. S'il existe $\lambda\in\RP$ tel que $b=\lambda a$, alors $\conj{a}b=\conj{a}\lambda a=\abs{a}^2 \lambda\in\RP$.
\end{itemize}
En conclusion $\abs{a+b}=\abs{a}+\abs{b}$ si et seulement si $a=0$ ou s'il existe $\lambda\in\RP$ tel que $b=\lambda a$.
\end{preuve}

\begin{remarques}
\remarque Si $(ABC)$ est un triangle, alors $AC\leq AB+BC$. En effet, si on note $a$, $b$, $c$ les affixes respectives de $A$, $B$, $C$
\[AC=\abs{c-a}=\abs{c-b+b-a}\leq\abs{c-b}+\abs{b-a}=BC+AB.\]
Cette inégalité explique le nom d'inégalité triangulaire donné à la proposition précédente.
\remarque Attention, il est possible que $a$ et $b$ soient positivement liés sans
  qu'il existe $\lambda\in\RP$ tel que $b=\lambda a$.
\end{remarques}


%% Soit $a$ et $b$ deux nombres complexes.
%% \begin{itemize}
%% \item Montrons que~:
%% $$\abs{a+b}\leq\abs{a}+\abs{b}$$
%% Remarquons, que le résultat est immédiat lorsque $b=0$. On suppose donc
%% désormais que $b\not=0$. Soit $t\in\R$.
%% \begin{eqnarray*}
%%   \abs{a+tb}^2 &=& \p{a+tb}\conj{\p{a+tb}}\\
%%                    &=& \p{a+tb}\p{\conj{a}+t\conj{b}}\\
%%                    &=& a\conj{a}+ta\conj{b}+tb\conj{a}+
%%                        t^2b\conj{b}\\
%%                    &=& b\conj{b} t^2 + \p{a\conj{b}+
%%                        \conj{a\conj{b}}}t+a\conj{a}\\
%%                    &=& \abs{b}^2 t^2 + 2\Re\p{a\conj{b}} t
%%                        +\abs{a}^2
%% \end{eqnarray*}
%% Or $\abs{a+tb}^2\geq 0$ quelque soit $t\in\R$. On en déduit donc que le
%% discriminant du trinôme en $t$ est négatif. Puisque $\abs{b}^2\not= 0$,
%% on en déduit que le discriminant de ce trinôme est négatif. En effet, si tel
%% n'était pas le cas, le trinôme aurait deux racines distinctes et changerait
%% de signe. On a donc~:
%% $$\Delta=\p{2\Re\p{a\conj{b}}}^2-4\abs{b}^2 \abs{a}^2 \leq 0$$
%% Donc~:
%% $$\abs{\Re\p{a\conj{b}}}^2 \leq \abs{a}^2\abs{b}^2$$
%% Puisque $\abs{\Re\p{a\conj{b}}}$ et $\abs{a}\abs{b}$ sont positifs,
%% on en déduit que~:
%% $$\abs{\Re{\p{a\conj{b}}}} \leq \abs{a}\abs{b}$$
%% Calculons maintenant~:
%% \begin{eqnarray*}
%%   \abs{a+b}^2-\p{\abs{a}+\abs{b}}^2 &=&
%%       \p{a+b}\conj{\p{a+b}}-\p{\abs{a}+\abs{b}}^2\\
%%   &=& \abs{a}^2+\abs{b}^2+2\Re\p{a\conj{b}}-
%%       \p{\abs{a}^2+\abs{b}^2+2\abs{a}\abs{b}}\\
%%   &=& 2\p{\Re\p{a\conj{b}}-\abs{a}\abs{b}}
%% \end{eqnarray*}
%% Or $\abs{\Re{\p{a\conj{b}}}} \leq \abs{a}\abs{b}$, donc puisque
%% $\Re{\p{a\conj{b}}} \leq \abs{\Re{\p{a\conj{b}}}}$, on en déduit
%% que~:
%% $$\abs{a+b}^2-\p{\abs{a}+\abs{b}}^2 \leq 0$$
%% et donc~:
%% $$\abs{a+b}^2\leq\p{\abs{a}+\abs{b}}^2$$
%% Or $\abs{a+b}$ et $\abs{a}+\abs{b}$ sont tous deux positifs, donc~:
%% $$\abs{a+b}\leq\abs{a}+\abs{b}$$
%% \item Montrons désormais que cette inégalité devient une égalité si et
%% seulement si $a$ et $b$ sont positivements liés.\\
%% \begin{itemize}
%% \item Si $a$ et $b$ sont positivement liés,il existe $\lambda\in\RP$ tel
%% que $a=\lambda b$ ou $b=\lambda a$. Quitte à les échanger, on peut
%% supposer que $b=\lambda a$. Alors~:
%% \begin{eqnarray*}
%%   \abs{a+b} &=& \abs{a+\lambda a}\\
%%                 &=& \abs{(1+\lambda)a}\\
%%                 &=& \abs{1+\lambda}\abs{a}\\
%%                 &=&\p{1+\lambda}\abs{a}\quad\text{car $1+\lambda\geq 0$}\\
%%                 &=& \abs{a}+\lambda\abs{a}\\
%%                 &=& \abs{a}+\abs{\lambda}\abs{a}
%%                     \quad\text{car $\lambda\geq 0$}\\
%%                 &=& \abs{a}+\abs{\lambda a}\\
%%                 &=& \abs{a}+\abs{b}
%% \end{eqnarray*}
%% \item Réciproquement, supposons que $\abs{a+b}=\abs{a}+\abs{b}$ et
%% montrons que $a$ et $b$ sont positivement liés.\\
%% Si $b$ est nul, $a$ et $b$ sont positivement liés, car $b=0 a$. On
%% suppose donc que $b$ est non nul. En remontant les calculs qui ont permis
%% la démonstration de l'inégalité triangulaire, de l'égalité
%% $\abs{a+b}=\abs{a}+\abs{b}$, il découle que~:
%% $$\Re\p{a\conj{b}}=\abs{a}\abs{b}$$
%% Donc le discriminant du trinôme considéré plus haut est nul.Ce trinôme admet
%% donc une unique racine $\mu$. Par définition $\abs{a+\mu b}=0$, donc
%% $a=-\mu b$. On pose $\lambda=-\mu$. Il reste alors à montrer que
%% $\lambda \geq 0$. Or~:
%% $$\Re\p{a\conj{b}}=\abs{a}\abs{b}$$
%% Donc, un réinjectant dans cette égalité la relation $a=\lambda b$, on
%% trouve~:
%% $$\lambda \abs{b}^2=\abs{\lambda}\abs{b}^2$$
%% Puisque $b\not=0$, on peut simplifier par $\abs{b}^2$, et il vient
%% $\lambda=\abs{\lambda}\geq 0$.
%% \end{itemize}
%% \end{itemize}
%% \end{preuve}

\begin{proposition}[utile=-3,nom={Seconde inégalité triangulaire}]
Soit $a$ et $b$ deux nombres complexes. Alors
\[\abs{\abs{a}-\abs{b}}\leq\abs{a+b}.\]
%\leq\abs{a}+\abs{b}.\]  
\end{proposition}

\begin{preuve}
$|a|=|a+b-b|\leq |a+b|+|-b|$ ... + changement des rôles.

\end{preuve}

\begin{remarqueUnique}
% \remarque Si $ABC$ est un triangle,alors
%   \[\abs{AB-BC}\leq AC\leq AB+BC\]
\remarque La seconde inégalité triangulaire admet plusieurs variantes.
  \begin{itemize}
  \item Si on remplace $b$ par $-b$, on obtient l'inégalité
    \[\abs{\abs{a}-\abs{b}}\leq\abs{a-b}\]
    qui affirme que deux nombres complexes proches ont des modules proches.
  \item En remarquant que si $x$ est réel, $\abs{x}\geq x$, on obtient
    \[\abs{a+b}\geq\abs{a}-\abs{b}.\]
    Cette inégalité affirme que si $b$ a un module petit par rapport à celui de $a$, alors $a+b$ est éloigné de~0.
  \end{itemize}
\end{remarqueUnique}

\begin{exos}
\exo Soit $a$ et $b$ deux nombres complexes distincts. On pose $\delta\defeq\abs{a-b}$. Montrer que les disques ouverts de centre $a$ et $b$ et de rayon $\delta/2$ sont disjoints.
\exo Que peut-on dire de $\abs{z}$ si $\abs{1-z}\leq 1/4$~? Faire un dessin puis une preuve.
\begin{sol}
Par l'absurde, si $z$ est dans l'intersection des deux disques, $|a-b|\leq|a-z|+|z-b|<\delta$.

$|1|\leq |1-z|+|z|$ et $|z|\leq |z-1|+|1|$ donnent $5/4\geq \abs{z}\geq 3/4$.
\end{sol}
% \exo Soit $z_0,a,\ldots,z_n$ des nombres complexes de module 1. Montrer
%   que
%   \[\sum_{k=0}^n \frac{z_k}{2^k}\neq 0\]
%   \begin{sol}
%   Il suffit de remarquer que
%   \[\abs{\sum_{k=0}^n \frac{z_k}{2^k}}\geq
%     1-\abs{\sum_{k=1}^n \frac{z_k}{2^k}}\geq \frac{1}{2^n}\]
%   \end{sol}
\end{exos}

\begin{proposition}
Soit $a_1,\ldots,a_n\in\C$. Alors
\[\abs{\sum_{k=1}^n a_k}\leq \sum_{k=1}^n \abs{a_k}.\]
\end{proposition}


\subsection{Puissance entière, binôme de \nom{Newton}}

\begin{definition}[utile=-3]
  Soit $a\in\C$. On définit $a^n$ pour tout entier naturel $n\in\N$ en posant
  \begin{itemize}
    \item $a^0\defeq 1$,
    \item $\forall n\in\N \qsep a^{n+1}\defeq a^n a$.
  \end{itemize}
\end{definition}

\begin{remarqueUnique}
\remarque En particulier, $0^0=1$.
\end{remarqueUnique}

\begin{proposition}[utile=-3]
  Soit $a,b$ deux nombres complexes, $n$ et $m$ deux entiers
  naturels. Alors
  \[\p{ab}^n=a^n b^n, \quad a^{n+m}=a^n a^m \et 
    \p{a^n}^m=a^{nm}.\]
\end{proposition}

\begin{preuve}
Soit $a\in\C$ et $n\in\N$. Pour tout $m\in\N$, on pose
\[\mathcal{H}_m\defeq\mog a^{n+m}=a^n a^m\mfg.\]
Pour tout $m\in\N$, on pose
\[\mathcal{H}_m\defeq\mog \p{a^n}^m=a^{nm}\mfg.\]
\end{preuve}

\begin{proposition}[utile=-3]
  Soit $a\in\C$ et $n\in\N$. Alors
  \[\conj{a^n}=\conj{a}^n \et \abs{a^n}=\abs{a}^n.\]
\end{proposition}

\begin{exoUnique}
\exo Montrer que si $P(z)\defeq a_n z^n+\cdots+a_1 z+a_0$ est un polynôme à coefficients réels, l'ensemble de ses racines est stable par conjugaison.
\end{exoUnique}

\begin{sol}
Soit $\alpha$ une racine. On a $P(\alpha)=0$, il suffit alors d'écrire $0=\conj{P(\alpha)}=\ldots=P(\conj{\alpha})$ grâce aux règles de calcul de la conjugaison.
\end{sol}

\begin{definition}
  Soit $a$ un nombre complexe non nul. On étend la définition de
  $a^n$ à $n\in\Z$ en posant
  $$a^n\defeq \frac{1}{a^{-n}}$$
  lorsque $n<0$.
\end{definition}

\begin{proposition}
Soit $a,b$ deux nombres complexes non nuls, $n$ et $m$ deux entiers
relatifs. Alors
\[\p{ab}^n=a^n b^n,\quad
  a^{n+m}=a^n a^m \et
  \p{a^n}^m=a^{nm}.\]
\end{proposition}

\begin{proposition}
Soit $a$ un nombre complexe non nul et $n\in\Z$. Alors
\[\conj{a^n}=\conj{a}^n \et \abs{a^n}=\abs{a}^n.\]
\end{proposition}

\begin{definition}[nom={Division euclidienne}]
Soit $a\in\Z$ et $b\in\Ns$. Alors il existe un unique couple $\p{q,r}\in\Z^2$
tel que
\[a=qb+r \et 0\leq r<b.\]
$q$ est appelé \emph{quotient} de la division euclidienne de $a$ par $b$, $r$ son
\emph{reste}.
\end{definition}

\begin{preuve}
\begin{itemize}
\item[$\bullet$] \textbf{Unicité :} Supposons trouvés deux couples $(q_1,r_1)$ et $(q_2,r_2)$. On obtient alors $$b|q_1-q_2|=|r_2-r_1|.$$ Supposons un instant $q_1\neq q_2$, alors $$b\leq b|q_1-q_2|=|r_2-r_1|\leq b-1$$ d'où la contradiction.
\item[$\bullet$] \textbf{Existence :} Commençons par le cas $a\geq 0$ et considérons l'ensemble~:
\[A=\enstq{k\in\N}{kb\leq a}\]
Cet ensemble est non vide ($0\in A$), majoré (par $a$) donc admet un plus grand
élément $q$. Comme $q\in A$ et $q+1 \notin A$, on a $bq\leq a <b(q+1)$ d'où $0\leq a-bq<b$. Il reste à poser $r=a-bq$.\\
Si $a\leq 0$, il suffit d'appliquer ce qui précède à $-a$. Il existe donc $q\in \N$ et $r\in \N$ vérifiant \[-a=qb+r \et 0\leq r<b.\]
Donc $a=(-q)b+(-r)$ ce qui convient si $r=0$ et si $0<r<b$, alors $-b<-r<0$ donc $0<-r+b<b$ et dans ce cas $a=(-q-1)b+(-r+b)$ est de la forme recherchée.
\end{itemize}
\end{preuve}

\begin{exoUnique}
\exo On pose $\jj\defeq-\frac{1}{2}+\ii\frac{\sqrt{3}}{2}$. Calculer $\jj^3$, puis en déduire
  $\jj^{2023}$.
\end{exoUnique}
  

\begin{definition}[utile=-3]
Pour tout entier naturel $n$, on définit la \emph{factorielle} de $n$ que l'on note
$n!$ par
\begin{itemize}
\item $0!\defeq 1$,
\item $\forall n\in\N \qsep \p{n+1}!\defeq\p{n+1}n!$.
\end{itemize}
\end{definition}

\begin{remarqueUnique}
\remarque Si $n\in\Ns$
  \[n!=n\times(n-1)\times\cdots\times 2\times 1.\]
\end{remarqueUnique}

\begin{definition}[utile=-3]
Pour tout couple $\p{k,n}$ d'entiers naturels, on définit $\binom{n}{k}$ que l'on prononce \og $k$ parmi $n$ \fg, comme étant le nombre de parties à $k$ éléments d'un ensemble
à $n$ éléments.
\end{definition}

\begin{remarques}
\remarque Si $k>n$, alors $\binom{n}{k}=0$.
\remarque Si $n\in\N$, alors
\[\binom{n}{0}=1 \et \binom{n}{n}=1.\]
\end{remarques}

\begin{proposition}[utile=-3]
Soit $n\in\N$ et $k\in\intere{0}{n}$. Alors
\[\binom{n}{k}=\binom{n}{n-k}.\]
\end{proposition}

\begin{preuve}
On remarque que choisir $k$ éléments parmi $n$ revient à sélectionner les $n-k$ éléments qu'on ne choisira pas.\end{preuve}

\begin{proposition}[utile=2, nom=Relation de \nom{Pascal}]
Soit $k$ et $n$ deux entiers naturels. Alors
\[\binom{n}{k}+\binom{n}{k+1}=\binom{n+1}{k+1}.\]
\end{proposition}

\begin{preuve}
\'Evacuons tout d'abord certains cas en fonction des valeurs de $k$ : $k>n$, $k=n$.

Désormais on considère $k<n$. Démontrons ce résultat de manière combinatoire. Considérons l'ensemble $\ens{1, 2, . . . ,n+1}$ et dénombrons ses sous-ensembles à $k+1$ éléments. On distingue les ensembles qui contiennent l'élément $n+1$ et les ensembles qui ne le contiennent pas. Il y a$\binom{n}{k+1}$ ensembles qui ne contiennent pas $n+1$ (on choisit $k+1$ éléments dans $\ens{1, 2, . . . ,n}$ et il y en a $\binom{n}{k}$ qui contiennent $n+1$, ce qui fait bien le total souhaité.

\end{preuve}

\begin{remarqueUnique}
\remarque Cette formule est appelée relation de \nom{Pascal}. Elle permet de calculer
  efficacement les $\binom{n}{k}$ en construisant le triangle de \nom{Pascal}. Dans ce
  tableau contenant les $\binom{n}{k}$, où $n$ désigne la ligne et $k$ désigne la colonne,
  on commence par placer une colonne de 1 indiquant le fait que $\binom{n}{0}=1$, puis une diagonale
  de 1 indiquant le fait que $\binom{n}{n}=1$. Les coefficients au-dessus de la diagonale sont nuls
  et ne sont généralement pas représentés. Ceux en dessous de la diagonale sont complétés, ligne
  après ligne en utilisant la relation de Pascal qui affirme que chaque coefficient est la somme
  du coefficient se situant au-dessus de lui et de celui au-dessus à gauche.
  \[\begin{matrix}
    1 &   &    &    &   &  \\
    1 & 1 &    &    &   &  \\
    1 & 2 &  1 &    &   &  \\
    1 & 3 &  3 &  1 &   &  \\
    1 & 4 &  6 &  4 & 1 &  \\
    1 & 5 & 10 & 10 & 5 & 1
    \end{matrix}\]
  Ce triangle permet par exemple de lire sur la dernière ligne que $\binom{5}{1}=5$ et $\binom{5}{2}=10$.
\end{remarqueUnique}

\begin{proposition}[utile=-3]
Soit $n$ un entier naturel et $k\in\intere{0}{n}$. Alors
\[\binom{n}{k}=\frac{n!}{k!\p{n-k}!}.\]
\end{proposition}

\begin{preuve}
Utilisons  le  principe  de double-comptage, très  important  en  combinatoire.  Il consiste à établir une égalité en comptant de deux manières différentes une certaines quantité. Ici, comptons le nombre de suites à $k$ éléments différents qu'on peut créer en utilisant les $n$ éléments de $\ens{1, 2, . . . ,n}$. D'une part, nous avons $n$ choix pour le premier terme de la suite, $n-1$ choix pour le deuxième, et ainsi de suite jusqu'au $k$-ième élément pour lequel nous avons $n-k+1$ choix. Finalement, il y a en tout $n(n-1)\ldots(n-k+1) =\dfrac{n!}{(n-k)!}$ telles suites. D'autre part, pour créer une suite à $k$ éléments, on peut commencer par choisir les $k$ éléments qui vont constituer la suite ($\binom{n}{k}$ possibilités), puis les ordonner ($k!$ manières possibles de les ordonner). Il y a donc en tout $k!\binom{n}{k}$ telles suites.

Ainsi, $\dfrac{n!}{(n-k)!}=k!\binom{n}{k}$, d'où le résultat souhaité.

\end{preuve}

\begin{preuve}
On peut maintenant redémontrer les deux propositions précédentes à l'aide de la "formule" du coefficient binomial que l'on vient d'obtenir :
Laissé en exercice aux élèves.

\end{preuve}
\begin{remarques}
% \remarque Cette formule est très mauvaise pour calculer effectivement
%   $\binom{n}{k}$. Par exemple, si l'on souhaite calculer $\binom{13}{2}$ à
%   l'aide de cette formule, on est amené à calculer $13!$. Si on dispose d'un
%   ordinateur codant les entiers sur 32 bits, le calcul de $13!$ donnera un
%   résultat erroné car $13!\geq 2^{32}$. L'explosion d'\nom{Ariane} 5 lors de
%   son décollage le 4 juin 1996 est due à une erreur de ce type.
\remarque On peut simplifier l'écriture de $\binom{n}{k}$ en
  \[\binom{n}{k}=\frac{n!}{k!\p{n-k}!}=\frac{\frac{n!}{\p{n-k}!}}{k!}=
    \frac{\overbrace{n\p{n-1}\cdots\p{n-(k-1)}}^{\text{$k$ termes}}}{k!}.\]
  En particulier
  \[\binom{n}{1}=n \et \binom{n}{2}=\frac{n\p{n-1}}{2}.\]
\remarque Si $k,n\in\Ns$, on a la formule dite \og du capitaine \fg
  \[\binom{n}{k}=\frac{n}{k}\cdot\binom{n-1}{k-1}.\]
\end{remarques}

\begin{exoUnique}
\exo Soit $k,n\in\N$. Montrer que
  \[\binom{n}{k}\leq\frac{n^k}{k!}.\]
  \begin{sol}
  Soit $k,n\in\N$. Si $k\leq n$, alors
  \[\binom{n}{k}=
    \frac{\overbrace{n\p{n-1}\cdots\p{n-k+1}}^{\text{$k$ termes inférieurs ou égaux à $n$}}}{k!}\leq \frac{n^k}{k!}.\]
  Sinon, l'inégalité est triviale.
  \end{sol}
\end{exoUnique}


\begin{proposition}[utile=3, nom=Binôme de \nom{Newton}]
Soit $a$ et $b$ deux nombres complexes et $n$ un entier naturel. Alors
\[\p{a+b}^n=\sum_{k=0}^n \binom{n}{k}a^{n-k} b^k.\]
\end{proposition}
\begin{preuve}
Soit $a$ et $b$ deux nombres complexes. Montrons ce résultat par récurrence
sur $n$~:
\begin{recurrence}
test $$\p{a+b}^n=\sum_{k=0}^n a^k b^{n-k}$$
\recinit En effet~:
  $$\p{a+b}^0=1 \et a^0 b^0=1$$
\rechere On suppose que $\mathcal{H}_n$ est vraie. Montrons que
  $\mathcal{H}_{n+1}$ est vraie.
  On écrit $(a+b)^{n+1}=(a+b)(a+b)^n$ puis on applique l'H.R, on distribue, on réindexe la première somme...
\end{recurrence}
\end{preuve}

\begin{exoUnique}
\exo Soit $n\in\Ns$. Montrer que
  \[\sum_{k=0}^n  \binom{n}{k}=2^n \quad\et\quad  \sum_{k=0}^n  \p{-1}^k\binom{n}{k}=0.\]
\end{exoUnique}


\begin{proposition}[utile=-3]
  Soit $a$ et $b$ deux nombres complexes. Alors
  \begin{itemize}
  \item $a^2-b^2=\p{a-b}\p{a+b}.$
  \item Plus généralement, pour tout $n\in\Ns$
    \begin{eqnarray*}
  a^n-b^n
  &=& \p{a-b}\p{a^{n-1}+a^{n-2}b+\dots+
      ab^{n-2}+b^{n-1}}\\
  &=& \p{a-b}\p{\sum_{k=0}^{n-1} a^{n-1-k}b^k}.
  \end{eqnarray*}
  \end{itemize}
\end{proposition}

\begin{preuve}
On distribue dans le membre de droite puis on réindexe la deuxième somme.
\end{preuve}

\begin{proposition}[utile=-3]
Soit $a$ un nombre complexe et $n$ un entier naturel. Alors
\[\sum_{k=0}^n a^k=1+a+a^2+\cdots+a^n=
  \begin{cases}
  \dsp\frac{1-a^{n+1}}{1-a} & \text{si $a\neq 1$}\\
  n+1                  & \text{si $a=1$.}
  \end{cases}\]
\end{proposition}

\section{Forme trigonométrique}

\subsection{Exponentielle $\ii\theta$}

\begin{definition}[utile=-3]
Pour tout réel $\theta$, on définit l'exponentielle de $\ii\theta$ par
\[\e^{\ii\theta}\defeq\cos\theta+\ii\sin\theta.\]
\end{definition}

\begin{proposition}[utile=-3]
Soit $\theta_1$ et $\theta_2$ deux réels. Alors
\[\e^{\ii 0}=1 \qquad\text{et}\qquad \e^{\ii\p{\theta_1+\theta_2}}=\e^{\ii\theta_1}\e^{\ii\theta_2}.\]
\end{proposition}

\begin{preuve}
Immédiat avec $\cos(a+b)=\cos a\cos b-\sin a\sin b$ et $\sin(a+b)=\sin a\cos b+\cos a\sin b$.
\end{preuve}

\begin{proposition}[utile=-3]
Soit $\theta\in\R$. Alors
\[\overline{\e^{\ii\theta}}=\e^{-\ii\theta}.\]
De plus, $\e^{\ii\theta}$ est non nul et si $n\in\Z$, alors
\[\frac{1}{\e^{\ii\theta}}=\e^{-\ii\theta} \qquad\text{et}\qquad \e^{\ii n\theta}=
  \p{\e^{\ii \theta}}^n.\]  
\end{proposition}


\begin{proposition}[utile=-3, nom={Formules d'\nom{Euler} et \nom{Moivre}}]
Soit $\theta$ un réel. Alors les formules d'\nom{Euler} s'écrivent
\[\cos\theta=\frac{\e^{\ii \theta}+\e^{-\ii\theta}}{2} \et
  \sin\theta=\frac{\e^{\ii \theta}-\e^{-\ii\theta}}{2\ii}.\]
Pour $n\in\Z$, la formule de \nom{Moivre} nous donne
\[\cos \p{n\theta}+\ii\sin\p{n\theta}=\p{\cos \theta+\ii\sin \theta}^n.\]
\end{proposition}

% \exo Montrer que tout $z\in\U\setminus\ens{-1}$ il existe $a\in\R$ tel que
%   \[z=\frac{1+\ii a}{1-\ii a}.\]
%   \begin{sol}
% Soit $z\in\U\setminus\ens{-1}$. Alors
% \begin{eqnarray*}
% \forall a \in \R\qsep \frac{1+\ii a}{1-\ii a}=z
% &\ssi& a=\frac{z-1}{i(1+z)}
% \end{eqnarray*}
% Montrons que
% \[a\defeq \frac{z-1}{i(1+z)}\]
% est réel. On a
% \begin{eqnarray*}
% \conj{a}
% &=& \conj{\frac{z-1}{i(1+z)}}\\
% &=& -\frac{\conj{z}-1}{i(1+\conj{z})}\\
% &=& -\frac{\frac{1}{z}-1}{i\p{1+\frac{1}{z}}} \quad\text{car $z\in\U$}\\
% &=& \frac{1-z}{\ii\p{z+1}} = a
% \end{eqnarray*}
% Donc $a\in\R$.
%   \end{sol}
% \end{exoUnique}

\begin{proposition}[utile=-3]
\begin{itemize}
\item Soit $\theta\in\R$. Alors $\e^{\ii\theta}=1$ si et seulement si
  $\theta\equiv 0\ \cro{2\pi}$.
\item Plus précisément, étant donnés $\theta_1$ et $\theta_2\in\R$,
  $\e^{\ii\theta_1}=\e^{\ii\theta_2}$ si et seulement si
  $\theta_1\equiv\theta_2\ \cro{2\pi}$.
\end{itemize}
\end{proposition}

\begin{exoUnique}
\exo Déterminer la partie réelle de
  \[\frac{1}{1-\cos\theta-\ii\sin\theta}.\]
  \begin{sol}
  On commence par chercher le domaine de définition de l'expression. On a, pour tout $\theta\in\R$
  \begin{eqnarray*}
1-\cos\theta-\ii\sin\theta=0
&\ssi& 1-\e^{\ii \theta}=0\\
&\ssi& \e^{\ii \theta}=1\\
&\ssi& \theta\equiv 0\ \cro{2\pi}.
  \end{eqnarray*}
  Pour $\theta\in\R\setminus 2\pi\Z$, on a
  \begin{eqnarray*}
\frac{1}{1-\cos\theta-\ii\sin\theta}
&=& \frac{1}{1-\e^{\ii\theta}}\\
&=& \frac{1-\e^{-\ii\theta}}{(1-\e^{\ii\theta})(1-\e^{-\ii\theta})}\\
&=& \frac{1-\cos\theta+\ii\sin\theta}{2(1-\cos\theta)}
  \end{eqnarray*}
  donc
  \[\Re\p{\frac{1}{1-\cos\theta-\ii\sin\theta}}=\frac{1}{2}.\] 
  \end{sol}
\end{exoUnique}


\begin{proposition}[utile=-3,nom={Paramétrisation de $\U$ par \og l'exponentielle $\ii\theta$ \fg}]
L'application qui à $\theta$ associe $\e^{\ii\theta}$ est une surjection de $\R$
dans $\U$. Autrement dit~:
\begin{itemize}
\item Si $\theta\in\R$, $\e^{\ii\theta}\in\U$.
\item Pour tout $u\in\U$, il existe $\theta\in\R$ tel que $u=\e^{\ii\theta}$.
\end{itemize}
\end{proposition}

\begin{preuve}
La première partie vient du fait que $\cos^2+\sin^2=1$.

Soit $u\in \U$, $u=a+ib$. On a $|a|=|\Re(u)|\leq |u|=1$ donc $a\in [-1;1]$. On peut donc fixer $\theta\in \R$ tel que $a=\cos \theta$. Mais alors $b^2=1-\cos^2(\theta)$ donc $b=\pm \sin \theta=\sin(\pm \theta)$. Ce $\theta'$ convient alors.

On voit par la preuve que ce $\theta$ est unique à $2\pi$ près. \underline{Insister là-dessus en vue de la preuve d'après}
\end{preuve}



\subsection{Application à la trigonométrie}
\begin{applications}
\application {\it Factorisation par l'arc moitié}\\
  Étant donné un réel $\theta$
  \begin{eqnarray*}
  1+\e^{\ii\theta} &=& \e^{\ii\frac{\theta}{2}}\p{ \e^{-\ii\frac{\theta}{2}}
                                            +\e^{ \ii\frac{\theta}{2}}}\\
                &=& 2\cos\p{\frac{\theta}{2}}\e^{\ii\frac{\theta}{2}}.
  \end{eqnarray*}
  De même
  \begin{eqnarray*}
  1-\e^{\ii\theta} &=& \e^{\ii\frac{\theta}{2}}\p{ \e^{-\ii\frac{\theta}{2}}
                                            -\e^{ \ii\frac{\theta}{2}}}\\
                &=& -2\ii\sin\p{\frac{\theta}{2}}\e^{\ii\frac{\theta}{2}}.
  \end{eqnarray*}
  Plus généralement, étant donnés $\theta_1,\theta_2\in\R$
  \[\e^{\ii\theta_1}+\e^{\ii\theta_2}=\e^{\ii\frac{\theta_1+\theta_2}{2}}\p{\e^{\ii\frac{\theta_1-\theta_2}{2}}+\e^{-\ii\frac{\theta_1-\theta_2}{2}}}=
    2\cos\p{\frac{\theta_1-\theta_2}{2}}\e^{\ii\frac{\theta_1+\theta_2}{2}},\]
   \[\e^{\ii\theta_1}-\e^{\ii\theta_2}=\e^{\ii\frac{\theta_1+\theta_2}{2}}\p{\e^{\ii\frac{\theta_1-\theta_2}{2}}-\e^{-\ii\frac{\theta_1-\theta_2}{2}}}=
    2\ii\sin\p{\frac{\theta_1-\theta_2}{2}}\e^{\ii\frac{\theta_1+\theta_2}{2}}.\]   
\application {\it Calcul de sommes trigonométriques}\\
  Soit $\theta\in\R$ et $n\in\N$. Calculer
   \[C_n\defeq\sum_{k=0}^n \cos(k\theta) \quad \text{et} \quad
    S_n\defeq\sum_{k=0}^n \sin(k\theta).\]
  \begin{sol}
  On a
  \begin{eqnarray*}
  C_n+\ii S_n &=& \sum_{k=0}^n \cos k\theta+\ii\sin k\theta\\
           &=& \sum_{k=0}^n \e^{\ii k\theta}\\
           &=& \sum_{k=0}^n \p{\e^{\ii\theta}}^k
  \end{eqnarray*}
  \begin{itemize}
  \item Lorsque $\e^{\ii\theta}\neq 1$, c'est-à-dire lorsque
    $\theta\not\equiv 0\ \cro{2\pi}$, on a
    \begin{eqnarray*}
    C_n+\ii S_n &=& \frac{1-\p{\e^{\ii\theta}}^{n+1}}{1-\e^{\ii\theta}}\\
             &=& \frac{1-\e^{\ii \p{n+1}\theta}}{1-\e^{\ii\theta}}\\
             &=& \frac{\e^{\ii\frac{n+1}{2}\theta}}{\e^{\ii\frac{\theta}{2}}}
                 \frac{\e^{-\ii\frac{n+1}{2}\theta}-\e^{\ii\frac{n+1}{2}\theta}}
                      {\e^{-\ii\frac{\theta}{2}}-\e^{\ii\frac{\theta}{2}}}\\
             &=& \e^{\ii\frac{n\theta}{2}}\frac{-2\ii\sin\p{\frac{n+1}{2}\theta}}
                 {-2\ii\sin\p{\frac{\theta}{2}}}\\
             &=& \e^{\ii\frac{n\theta}{2}}\frac{\sin\p{\frac{n+1}{2}\theta}}
                 {\sin\p{\frac{\theta}{2}}}.
    \end{eqnarray*}
    En identifiant parties réelles et imaginaires, on obtient
    $$C_n=\frac{\sin\p{\frac{n+1}{2}\theta}\cos\p{\frac{n}{2}\theta}}
          {\sin\p{\frac{\theta}{2}}} \quad \text{et} \quad
      S_n=\frac{\sin\p{\frac{n+1}{2}\theta}\sin\p{\frac{n}{2}\theta}}
          {\sin\p{\frac{\theta}{2}}}.$$
  \item Dans le cas où $\theta\equiv 0\ \cro{2\pi}$, $\cos (k\theta)=1$ et
    $\sin(k\theta)=0$, donc
    $$C_n=n+1 \quad \text{et} \quad S_n=0.$$
  \end{itemize}      
  \end{sol}
\application {\it Linéarisation de $\cos^n \theta \sin^m \theta$}\\
  Étant donné deux entiers naturels $n$ et $m$, on cherche à exprimer
  $\cos^n \theta \sin^m \theta$ comme combinaison linéaire des $\cos(k\theta)$
  et $\sin(k\theta)$ pour $k\in\N$. Pour cela, on peut utiliser les formules
  d'\nom{Euler} avant de développer l'expression par la formule
  du binôme de \nom{Newton} et de regrouper les termes en utilisant à nouveau
  les formules d'\nom{Euler}. Cette opération sera utile lors
  du calcul de primitives.\\
  {\bf Exemple~:} Linéariser $\sin^6 \theta$ et $\sin\theta \cos^4 \theta$.
    \begin{sol}
    On a
    \begin{eqnarray*}
    \sin^6 \theta &=& \p{\frac{\e^{\ii\theta}-\e^{-\ii\theta}}{2\ii}}^6\\
    &=& -\frac{1}{64}\p{\e^{\ii 6\theta}-6\e^{\ii 4\theta}+15\e^{\ii 2\theta}-20
                        +15\e^{-i2\theta}-6\e^{-i4\theta}+\e^{-i6\theta}}\\
    &=& -\frac{1}{64}\p{\p{\e^{\ii 6\theta}+\e^{-i6\theta}}
                        -6\p{\e^{\ii 4\theta}+\e^{-i4\theta}}
                        +15\p{\e^{\ii 2\theta}+\e^{-i2\theta}}-20}\\
    &=& -\frac{1}{32}\p{\cos 6\theta-6\cos 4\theta+15\cos 2\theta-10}
    \end{eqnarray*}
    \begin{eqnarray*}
    \sin \theta \cos^4 \theta
    &=& \p{\frac{\e^{\ii\theta}-\e^{-\ii\theta}}{2\ii}}
        \p{\frac{\e^{\ii\theta}+\e^{-\ii\theta}}{2}}^4\\
    &=& -\frac{i}{32}\p{\e^{\ii\theta}-\e^{-\ii\theta}}
                     \p{\e^{\ii\theta}+\e^{-\ii\theta}}^4\\
    &=& -\frac{i}{32}\p{\e^{\ii\theta}-\e^{-\ii\theta}}
        \p{\e^{\ii 4\theta}+4\e^{\ii 2\theta}+6+4\e^{-i2\theta}+\e^{-i4\theta}}\\
    &=& -\frac{i}{32}\p{\e^{\ii 5\theta}+3\e^{\ii 3\theta}+2\e^{\ii\theta}-2\e^{-\ii\theta}
                        -3\e^{-i3\theta}-\e^{-i5\theta}}\\
    &=& -\frac{i}{32}\p{2\ii\sin 5\theta+6\ii\sin 3\theta+4\ii\sin \theta}\\
    &=& \frac{1}{16}\p{\sin 5\theta+3\sin 3\theta+2\sin \theta}
    \end{eqnarray*}      
    \end{sol}
\application {\it Expression  de $\cos(n\theta)$ et $\sin(n\theta)$ comme
  polynôme en $\cos \theta$ et $\sin \theta$}\\
  Pour cette opération, une méthode consiste à utiliser la formule de
  \nom{Moivre} avant de développer l'expression obtenue à l'aide du binôme de
  \nom{Newton}.\\
  {\bf Exemple~:} Exprimer $\cos(5\theta)$ comme un polynôme en $\cos\theta$. 
\end{applications}

\begin{exos}
% \exo Exprimer $\tan\p{5\theta}$ en fonction de $\tan\theta$.
% \begin{sol}
% On a
% \begin{eqnarray*}
% \cos\p{5\theta}+\ii\sin\p{5\theta}
% &=& \p{\cos\theta+\ii\sin\theta}^5\\
% &=& (\cos^{5}\theta-10\cos^{3}\theta\sin^{2}\theta+5\cos\theta\sin^{4}\theta)+\\
% & & \ii(5\cos^{4}\theta\sin\theta-10\cos^{2}\theta\sin^{3}\theta+\sin^{5}).
% \end{eqnarray*}
% Donc, en exploitant le fait que
% \[\tan (5\theta)=\frac{\sin (5\theta)}{\cos (5\theta)}\]
% et en divisant le numérateur et le dénominateur par $ \cos^{5}\theta $, on trouve
%   \[\tan\p{5\theta}=
%     \frac{5\tan\theta-10\tan^3\theta+\tan^5\theta}%
%     {1-10\tan^2\theta+5\tan^4\theta}.\]
% \end{sol}
\exo Soit $n\in\N$ et $\theta\in\R$. Calculer
  \[\sum_{k=0}^n \binom{n}{k}\sin\p{k\theta}.\]
  \begin{sol}
  On a
  \begin{eqnarray*}
 \sum_{k=0}^{n}\binom{n}{k}\e^{\ii k\theta}
 &=& \sum_{k=0}^{n}\binom{n}{k}(\e^{\ii\theta})^k\\
 &=& \p{1+\e^{\ii\theta}}^n\\
 &=& \cro{\e^{\frac{\ii\theta}{2}}\p{\e^{\frac{\ii\theta}{2}}+\e^{\frac{-\ii\theta}{2}}}}^n\\
 &=& \e^{\ii\frac{n\theta}{2}}\p{ \e^{\ii\frac{-n\theta}{2}}+\e^{\ii\frac{n\theta}{2}}}^{n}\\
 &=& \e^{\ii\frac{n\theta}{2}}\left( 2\cos\left( \frac{\theta}{2}\right) \right) ^{n}\\
 &=& 2^n \cos^n\p{\frac{\theta}{2}}\e^{\ii\frac{n\theta}{2}}.
  \end{eqnarray*}
  En prenant la partie imaginaire, on en déduit que
  \[\sum_{k=0}^n \binom{n}{k}\cos\p{k\theta}=2^n \cos^n\p{\frac{\theta}{2}}\sin\p{\frac{n\theta}{2}}.\]
  \end{sol}
\exo En exprimant $\cos(5\theta)$ comme un polynôme en $\cos\theta$,
  montrer que $\cos\p{\pi/10}$ est racine d'un polynôme à
  coefficients entiers. En déduire une expression de $\cos\p{\pi/10}$ à l'aide de radicaux.
\begin{sol}
On commence par exprimer $\cos(5\theta)$ comme polynôme en $\cos\theta$. On a
\[\cos (5\theta)=\cos \theta(16\cos^{4}\theta -20\cos^{2}\theta +5).\]
Pour $\theta=\pi/10$, en posant $x\defeq\cos(\pi/10)$, on obtient $x(16x^4-20x^2+5)=0$. Or $\theta\in\interfo{0}{\pi/2}$ donc $x>0$, donc $16x^4-20x^2+5=0$. Or
\begin{eqnarray*}
\forall u\in\R\qsep 16u^4-20u^2+5=0
&\ssi& u^2=\frac{5-\sqrt{5}}{8} \quad\text{ou}\quad  u^2=\frac{5+\sqrt{5}}{8}\\
&\ssi& u=\pm\sqrt{\frac{5-\sqrt{5}}{8}} \quad\text{ou}\quad  u=\pm\sqrt{\frac{5+\sqrt{5}}{8}}
\end{eqnarray*}
Or $x>0$ donc
\[x=\sqrt{\frac{5-\sqrt{5}}{8}} \quad\text{ou}\quad x=\sqrt{\frac{5+\sqrt{5}}{8}}.\]
Puisque $\pi/10\in[0,\pi/6]$, on en déduit que $x>\sqrt{3/4}$. Or, si on évalue $16u^4-20u^2+5$ en $3/4$, on obtient $-19/16<0$, donc $3/4$ est entre les deux racines donc
\[\frac{5-\sqrt{5}}{8}<\frac{3}{4}.\]
En conclusion
\[\cos\p{\frac{\pi}{10}}=\sqrt{\frac{5+\sqrt{5}}{8}}.\]
\end{sol}
\end{exos}



\subsection{Forme trigonométrique}

\begin{definition}
Soit $z\in\Cs$. On appelle \emph{forme trigonométrique} de $z$ toute écriture
\[z=r \e^{\ii\theta}\]
où $r\in\RPs$ et $\theta\in\R$.
\end{definition}

\begin{remarques}
\remarque Si $z=r\e^{\ii\theta}$ est une forme trigonométrique, alors $r=\abs{z}$.
\remarque Tout nombre complexe non nul admet une forme trigonométrique. En pratique,
pour la déterminer, on force la factorisation de $z$ par $\abs{z}$ et on écrit le second terme sous la forme $\e^{\ii\theta}$.
\remarque Il existe deux moyens de représenter un même nombre
  complexe~: la forme cartésienne et la forme trigonométrique. La première
  est particulièrement adaptée aux calculs de sommes, tandis que la seconde
  est particulièrement adaptée aux calculs de produits.
  % Nous verrons cependant, que lorsqu'ils ont même module, il existe un moyen
  % simple pour mettre sous forme trigonométrique la somme de deux nombres
  % complexes écrits sous forme trigonométrique.
\end{remarques}

\begin{exos}
  \exo Mettre $-2-2\sqrt{3}\ii$ sous forme trigonométrique. %Celle de $1+2\ii$~?
  \begin{sol}
  On a
  \[-2-2\sqrt{3}\ii=4\p{-\frac{1}{2}-\ii\frac{\sqrt{3}}{2}}=4\e^{-\ii\frac{2\pi}{3}}.\]
  De même
  \[1+2\ii=\sqrt{5}\underbrace{\p{\frac{1}{\sqrt{5}}+\ii\frac{2}{\sqrt{5}}}}_{\eqdef u}\]
  Puisque $(1/\sqrt{5})^2+(2/\sqrt{5})^2=1$, il existe $\theta\in\interf{0}{\pi/2}$ tel que $u=\e^{\ii\theta}$. Alors
  \[1+2\ii=\sqrt{5}\e^{\ii\theta}.\]
  \end{sol}
  \exo Mettre sous forme trigonométrique
    \[\p{\frac{1+\ii \sqrt{3}}{1-\ii}}^{20}.\]
  \begin{sol}
  \begin{eqnarray*}
  \p{\frac{1+\ii \sqrt{3}}{1-\ii}}^{20}
  &=& \p{\frac{2\e^{\ii\frac{\pi}{3}}}{\sqrt{2}\e^{-\ii\frac{\pi}{4}}}}^{20}\\
  &=& 1024\e^{-\ii\frac{\pi}{3}}\\
  &=& 512(1-\ii\sqrt{3}).
  \end{eqnarray*}
  \end{sol}
  \end{exos}



\begin{definition}[utile=-3]
On appelle \emph{argument} de $z\in\Cs$ tout réel
$\theta$ tel que
\[z=\abs{z}\e^{\ii\theta}.\]
\end{definition}

\begin{proposition}[utile=-3]
Tout nombre complexe non nul $z\in\Cs$ admet au moins un argument. Si $\theta$ est l'un de ses arguments, l'ensemble de ses arguments est
$\theta+2\pi\Z\defeq\ensim{\theta+k2\pi}{k\in\Z}$. On écrit
\[\arg(z)\equiv\theta\ \cro{2\pi}.\]
\end{proposition}

\begin{preuve}
Si $z\neq 0$, alors $z/|z| \in \U$ et on utilise la proposition précédente.
\end{preuve}

\begin{remarques}
\remarque Soit $r_1,r_2>0$ et $\theta_1,\theta_2\in\R$. Alors
  \[r_1\e^{\ii\theta_1}=r_2\e^{\ii\theta_2}\]
  si et seulement si $r_1=r_2$ et $\theta_1\equiv\theta_2\ [2\pi]$. Contrairement à la forme cartésienne, il n'y a donc pas unicité de la forme trigonométrique.
% \remarque Si $z=a+\ii b$ est non nul, alors~:
%   \begin{itemize}
%   \item si $a=0$, $\arg(z)\equiv \frac{\pi}{2}\ \cro{\pi}$. L'argument est donné par le
%     signe de $b$.
%   \item sinon, $\tan\theta=b/a$ ce qui permet de connaitre l'argument de
%     $z$ modulo $\pi$. Pour le connaitre modulo $2\pi$, il suffit de regarder
%     le signe de $a$.
%   \end{itemize}
\remarque Si $z$ est un nombre complexe non nul, il existe un unique
  $\theta\in\interof{-\pi}{\pi}$ tel que $\arg z\equiv \theta\ [2\pi]$.
  On dit que $\theta$ est l'\emph{argument principal} de $z$ et on le note
  $\Arg(z)$.
\remarque Étant donné un nombre complexe $z$, on appelle
   {\it forme trigonométrique généralisée} de $z$ toute écriture du type
    $z=\rho \e^{\ii\theta}$ où $\rho\in\R$ et $\theta\in\R$. Attention, même si $z\neq 0$, on n'a pas nécessairement
    $\arg{z}\equiv\theta\ \cro{2\pi}$. En effet~:
    \begin{itemize}
    \item {Si $\rho>0$}, alors $\rho=\abs{z}$ et
      $\arg{z}\equiv\theta\ \cro{2\pi}$.
    \item {Si $\rho<0$}, alors $\rho=-\abs{z}$ et
      $\arg{z}\equiv\theta+\pi\ \cro{2\pi}$.
    \end{itemize}
  Lorsque l'énoncé demandera explicitement
  de mettre un nombre complexe sous forme trigonométrique, c'est bien
  sous la forme $z=r \e^{\ii\theta}$ avec $r>0$ qu'il faudra le mettre.
  Cependant, lorsqu'on demandera de mettre $z$ sous forme trigonométrique pour
  conduire des calculs, une forme trigonométrique généralisée suffira le plus
  souvent.
\end{remarques}

\begin{exos}
% \exo Résoudre l'équation $z^2=\conj{z}$.
\exo Résoudre l'équation
  \[z^2=\conj{z}\]
  en utilisant la forme cartésienne, puis la forme trigonométrique de $z$.
  \begin{sol}
Avec la forme cartésienne, on a
\begin{eqnarray*}
z^2=\conj{z}
&\ssi& (x+\ii y)^2=x-\ii y\\
&\ssi& (x^2-y^2)+2\ii x y=x-\ii y\\
&\ssi& x^2-y^2=x \et 2xy=-y\\
&\ssi& x^2-y^2=x \et y(2x+1)=0\\
&\ssi& x^2-y^2=x \et \cro{y=0 \ou x=-\frac{1}{2}}\\
&\ssi& \cro{x^2-y^2=x \et y=0} \ou \cro{x^2-y^2=x \et x=-\frac{1}{2}}\\
&\ssi& z=1 \ou z=0 \ou z=-\frac{1}{2}+\ii\frac{\sqrt{3}}{2} \ou z=-\frac{1}{2}-\ii\frac{\sqrt{3}}{2}.
\end{eqnarray*}
  Avec la forme trigonométrique. On sait que 0 est solution. On cherche donc les solutions non nulles. Soit $z\in\Cs$, alors il existe $r\in\RPs$ et $\theta\in\R$ tels que $z=r\e^{\ii\theta}$. On a
\begin{eqnarray*}
z^2=\conj{z}
&\ssi& r^2 \e^{2\ii\theta} = r \e^{-\ii\theta}\\
&\ssi& r^2=r \et 2\theta\equiv -\theta\ [2\pi]\\
&\ssi& r=1 \et \theta\equiv 0\ \cro{\frac{2\pi}{3}}\\
\end{eqnarray*}
Les solutions sont donc $0$, $1$ et $\e^{\pm\ii\frac{2\pi}{3}}$.
  \end{sol}
\exo Résoudre l'équation $z^5=1/\conj{z}$.
\begin{sol}
Cette équation a pour domaine de définition $\Cs$. On cherche donc les solutions non nulles. Soit $z\in\Cs$, alors il existe $r\in\RPs$ et $\theta\in\R$ tels que $z=r\e^{\ii\theta}$. On a
\begin{eqnarray*}
z^5=\frac{1}{\conj{z}}
&\ssi& r^5 \e^{5\ii\theta} = \frac{1}{r} \e^{\ii\theta}\\
&\ssi& r^6=1 \et 5\theta\equiv \theta\ [2\pi]\\
&\ssi& r=1 \et \theta\equiv 0\ \cro{\frac{\pi}{2}}\\
&\ssi& z=1 \ou z=\ii \ou z=-1 \ou z=-\ii.
\end{eqnarray*}
  \end{sol}
\end{exos}

\begin{proposition}[utile=-3]
  Soit $z,z'\in\Cs$ et $n\in\Z$. Alors
  \[\arg(z z')\equiv\arg(z)+\arg(z')\ \cro{2\pi}, \qquad
    \arg\p{\frac{z}{z'}}\equiv\arg(z)-\arg(z')\ \cro{2\pi},\]
  \[\arg(z^n)\equiv n\arg(z)\ \cro{2\pi}, \qquad
    \arg(\conj{z})\equiv -\arg(z)\ [2\pi].\]
\end{proposition}

\begin{preuve}
Il suffit d'écrire la forme trigonométrique.
\end{preuve}

\begin{remarqueUnique}
\remarque En général,
  $\Arg(z z')\neq\Arg(z)+\Arg(z')$. Par exemple $\Arg((-1)(-1))=0$ et $\Arg(-1)+\Arg(-1)=2\pi$.
\end{remarqueUnique}

\subsection{Exponentielle complexe}

\begin{definition}[utile=-3]
Soit $z=x+\ii y$ un nombre complexe où $x$ et $y$ sont respectivement la partie
réelle et imaginaire de $z$. On appelle exponentielle de $z$ et on note $\e^z$
le nombre complexe défini par
\[\e^z\defeq\e^{x}\p{\cos y +\ii \sin y}.\]
\end{definition}

\begin{proposition}[utile=-3]
Soit $z$ et $z'$ deux nombres complexes. Alors
\[\e^{0}=1 \qquad \text{et} \qquad \e^{z+z'}=\e^{z} \e^{z'}.\]
\end{proposition}

\begin{preuve}
On écrit $a$ et $b$ sous forme cartésienne et on déroule.
\end{preuve}

\begin{proposition}[utile=-3]
Soit $z$ un nombre complexe, et $n\in\Z$ . Alors $\e^z$ est non nul,
\[\frac{1}{\e^z}=\e^{-z} \qquad \text{et} \qquad \e^{nz}=\p{\e^z}^n.\]
\end{proposition}

\begin{preuve}
Voir la première partie comme un corollaire de la proposition précédente. 
\end{preuve}

\begin{proposition}[utile=-3]
Soit $z$ un nombre complexe. Alors
\[\conj{\e^z}=\e^{\conj{z}}, \qquad  \abs{\e^z}=\e^{\Re(z)}
\qquad \text{et} \qquad \arg(\e^z)\equiv \Im(z)\ [2\pi].\]
\end{proposition}

\begin{proposition}[utile=-3]
\begin{itemize}
\item Soit $z\in\C$. Alors $\e^z=1$ si et seulement si il existe
  $k\in\Z$ tel que $z=\ii k2\pi$.
\item  Plus précisément, étant donnés $z_1$ et $z_2$ deux nombres
  complexes, $\e^{z_1}=\e^{z_2}$ si et seulement si il existe
  $k\in\Z$ tel que $z_1=z_2+\ii k2\pi$.
\end{itemize}
\end{proposition}

\begin{proposition}[utile=-3]
L'exponentielle est une surjection de $\C$ dans $\Cs$. Autrement dit~:
\begin{itemize}
\item Pour tout $z\in\C$, $\e^{z}\in\Cs$.
\item Pour tout $z\in\Cs$, il existe $z'\in\C$ tel que $\e^{z'}=z$.
\end{itemize}
\end{proposition}

\begin{preuve}
Soit $a\in \Cs$ que l'on écrit $a=re^{i\theta}$ avec $r>0$ et $\theta \in \R$.
\begin{eqnarray*}
\forall z=x+iy \in \C, \quad e^z=a & \Longleftrightarrow &e^z=re^{i\theta} \\
& \Longleftrightarrow & \begin{cases}
    e^x=r\\
    y-\theta \in 2\pi\Z
    \end{cases}\\
& \Longleftrightarrow & \begin{cases}
    x=\ln(r)\\
    \exists k \in \Z, y=\theta+2k\pi
    \end{cases}
\end{eqnarray*}
$a$ admet donc un seul $x$ mais une infinité de $y$ donc une infinité d'antécédents qui sont exactements les $\set{\ln(r)+i(\theta+2k\pi), k\in \Z}$.


\end{preuve}

\begin{remarqueUnique}
\remarque Si $z\in\Cs$, nous venons de voir qu'il existe $z'\in\C$ tel que
  $e^{z'}=z$. Cependant, $z'$ n'est pas unique, ce qui nous empêche de définir
  le logarithme de $z$. Par contre, on peut montrer qu'il existe un unique $z'\in\C$ tel que $\e^{z'}=z$ et
  $\Im(z')\in\interof{-\pi}{\pi}$. Ce nombre est appelé logarithme
  principal de $z$ et noté $\Ln(z)$. De plus $\Ln(z)=\ln\abs{z}+\ii\Arg(z)$. C'est le logarithme calculé par les logiciels de calcul formel ainsi que vos calculatrices. Malheureusement, l'identité $\Ln\p{z z'}=\Ln(z)+\Ln(z')$ est fausse;
  elle n'est vraie que modulo $\ii2\pi$. C'est pourquoi, nous n'emploierons
  jamais de logarithme avec les nombres complexes.
\end{remarqueUnique}

\begin{exoUnique}
\exo Résoudre sur $\C$ l'équation $\e^z=\sqrt{3}+3\ii$.
  \begin{sol}
  $\sqrt{3}+3i=2\sqrt{3}\p{\dfrac{1}{2}+\dfrac{\sqrt 3}{2}i}=2\sqrt{3}e^{i\frac{\pi}{3}}$ donc on trouve $\ln\p{2\sqrt{3}}+\ii\frac{\pi}{3}+\ii k2\pi$.
  \end{sol}
\end{exoUnique}

\section{Racines d'un nombre complexe}
\subsection{L'équation du second degré}

\begin{definition}[utile=-3]
Soit $a$ un nombre complexe. On appelle \emph{racine} de $a$ tout nombre complexe $z$
tel que
\[z^2=a.\]
\end{definition}

\begin{remarques}
\remarque Si $a$ est un réel positif, les racines de $a$ sont $\sqrt{a}$ et $-\sqrt{a}$.
  Si $a$ est un réel négatif, ses racines sont $\ii\sqrt{\abs{a}}$ et
  $-\ii\sqrt{\abs{a}}$.
\remarque Si $a\in\C$, on parlera de racine de $a$, mais on n'écrira jamais $\sqrt{a}$. Cette notation est réservée aux réels positifs.
\end{remarques}

\begin{proposition}[utile=-3]
Soit $a$ un nombre complexe non nul. Alors $a$ admet exactement deux
racines distinctes opposées l'une à l'autre.
\end{proposition}

\begin{remarqueUnique}
\remarque En pratique, pour trouver les racines d'un nombre complexe $a$, on procède ainsi
  \begin{itemize}
    \item {\it Si $a$ s'exprime facilement sous forme trigonométrique.}
      On connait donc $r>0$ et $\theta$ tels que $a=r\e^{\ii\theta}$. Alors
      les racines de $a$ sont $\sqrt{r}\e^{\ii\frac{\theta}{2}}$
      et $-\sqrt{r}\e^{\ii\frac{\theta}{2}}$.
    \item {\it Si $a$ est sous forme cartésienne et qu'il n'est pas
      simple de le mettre sous forme trigonométrique.}
      On recherche les racines de $a$ sous la forme $z\defeq x+\ii y$ en
      effectuant une analyse~: on suppose que $z$ est une racine de $a$
      et on exploite le fait que $\abs{z}^2=\abs{a}$ et que $z^2$ et $a$
      ont même partie réelle. On obtient donc
      \begin{align*}
        x^2+y^2&=\abs{a},\\
        x^2-y^2&=\Re(a).
      \end{align*}
      En résolvant ce système linéaire en $x^2$ et $y^2$, on obtient $4$
      couples $(x,y)$ de solutions parmi lesquelles se trouvent les racines de 
      $a$. Un argument de signe sur les parties imaginaires de $z^2$ et
      $a$ permet d'éliminer deux candidats. La proposition précédente
      nous assure que les deux candidats restants sont bien des racines de
      $a$.
  \end{itemize}
\end{remarqueUnique}

\begin{exos}
\exo Calculer les racines carrées de $1+\ii\sqrt{3}$.
  \begin{sol}
  On trouve après mise sous forme trigonométrique
  \[\pm\sqrt{2}e^{i\frac{\pi}{6}}=\pm\frac{\sqrt{6}+\ii\sqrt{2}}{2}.\]
  \end{sol}
\exo Calculer les racines de $1+\ii$ de deux manières différentes.
  En déduire une expression avec des radicaux emboîtés de
  $\cos(\pi/8)$, $\sin(\pi/8)$ et $\tan(\pi/8)$.
  \begin{sol}
  Avec la forme cartésienne, on doit résoudre $x^2+y^2=\sqrt{2}$ et $x^2-y^2=1$. On trouve 4 couples mais on doit avoir $xy\geq 0$ ce qui fait qu'on garde 
  \[\sqrt{\frac{1+\sqrt{2}}{2}}+i\sqrt{\frac{-1+\sqrt{2}}{2}} \et
    -\sqrt{\frac{1+\sqrt{2}}{2}}-i\sqrt{\frac{-1+\sqrt{2}}{2}}\]
  et par la forme trigonométrique, on trouve $\pm 2^{\frac{1}{4}}e^{i\frac{\pi}{8}}$. Donc
  \[\cos\frac{\pi}{8}=\frac{1}{2^{\frac{1}{4}}}\sqrt{\frac{1+\sqrt{2}}{2}}=
    \frac{\sqrt{2+\sqrt{2}}}{2} \qquad
    \sin\frac{\pi}{8}=\frac{1}{2^{\frac{1}{4}}}\sqrt{\frac{-1+\sqrt{2}}{2}}=
    \frac{\sqrt{2-\sqrt{2}}}{2}\]
  \[\tan\frac{\pi}{8}=\sqrt{2}-1\]
  \end{sol}
\end{exos}

\begin{proposition}[utile=-3]
Soit $a,b,c\in\C$ avec $a\neq 0$.
% \begin{itemize}
%\item 
On considère l'équation
  \[(E) \qquad az^2+bz+c=0.\]
  On appelle discriminant de $(E)$ le nombre complexe $\Delta=b^2-4ac$.
  \begin{itemize}
  \item Si $\Delta\not=0$, le trinôme admet deux racines distinctes
    \[z_1\defeq\frac{-b+\delta}{2a} \qquad \text{et} \qquad
      z_2\defeq\frac{-b-\delta}{2a}.\]
    où $\delta$ est une racine carrée de $\Delta$.
  \item Si $\Delta=0$, le trinôme admet une seule racine, appelée racine
    double
    \[z_0\defeq-\frac{b}{2a}.\]
  \end{itemize}
% \item Lorsque l'équation s'écrit sous la forme~:
%   \[az^2+2bz+c=0\]
%   on utilise le discriminant réduit $\Delta'=b^2-ac$. Dans ce cas~:
%   \begin{itemize}
%   \item Si $\Delta'=0$, le trinôme admet une et une seule racine appelée racine
%     double~:
%     \[z_0=-\frac{b}{a}\]
%   \item Si $\Delta'\not=0$, le trinôme admet exactement deux racines distinctes~:
%     \[a=\frac{-b+\delta'}{a} \qquad \text{et} \qquad
%       b=\frac{-b-\delta'}{a}\]
%     où $\delta'$ est une racine carrée de $\Delta'$.
%   \end{itemize}
% \end{itemize}
\end{proposition}

\begin{preuve}
$\displaystyle z^2+\frac{b}{a}z+\frac{c}{a}=\p{z+\frac{b}{2a}}^2-\p{\frac{\delta}{2a}}^2=\p{z+\frac{b+\delta}{2a}}\p{z+\frac{b-\delta}{2a}}=0$.
\end{preuve}

\begin{remarqueUnique}
\remarque Soit $a,b,c\in\R$ avec $a\neq 0$. On considère l'équation
  \[(E) \qquad az^2+bz+c=0.\]
  \begin{itemize}
  \item Si $\Delta>0$, le trinôme admet deux racines réelles distinctes
    \[z_1\defeq\frac{-b+\sqrt{\Delta}}{2a} \qquad \text{et} \qquad
      z_2\defeq\frac{-b-\sqrt{\Delta}}{2a}.\]
  \item Si $\Delta=0$, le trinôme admet une seule racine réelle, appelée racine
    double
    \[z_0\defeq-\frac{b}{2a}.\]
  \item Si $\Delta<0$, le trinôme admet deux racines complexes conjuguées
    \[z_1\defeq\frac{-b+\ii\sqrt{\abs{\Delta}}}{2a} \qquad \text{et} \qquad
      z_2\defeq\frac{-b-\ii\sqrt{\abs{\Delta}}}{2a}.\]
  \end{itemize}     
\end{remarqueUnique}

\begin{exoUnique}
\exo Soit $\theta\in\R$. Résoudre sur $\C$ l'équation
  $z^2-2\cos(\theta)z+1=0$.
  \begin{sol}
  Les racines sont $\e^{\ii\theta}$ et $\e^{-\ii\theta}$.
  \end{sol}
\end{exoUnique}

\begin{proposition}
Soit $a,b,c\in\C$ avec $a\not=0$ et $z_1,z_2$ deux nombres complexes. Alors $z_1$
et $z_2$ sont les deux racines, éventuellement égales, de l'équation
$az^2+bz+c=0$ si et seulement si
\[z_1+z_2=-\frac{b}{a} \qquad \text{et} \qquad z_1 z_2=\frac{c}{a}.\]
\end{proposition}

\begin{preuve}
$\displaystyle z^2+\frac{b}{a}z+\frac{c}{a}=(z-z_1)(z-z_2)$.
\end{preuve}
\begin{remarqueUnique}
\remarque Si $P,S\in\C$, les solutions du système
  \[\syslin{z_1+z_2&=&S\hfill\cr
             z_1 z_2&=&P,\hfill}\]
  sont $\p{\omega_1,\omega_2}$ et $\p{\omega_2,\omega_1}$ où $\omega_1$
  et $\omega_2$ sont les racines du trinôme $z^2-Sz+P=0$.
\end{remarqueUnique}
\begin{exos}
\exo Déterminer les solutions de l'équation $3z^2-5z+2=0$.
\exo Résoudre sur $\C$ le système
  \[\syslin{u^2+v^2&=&3-2\ii\hfill\cr
                 uv&=&3+\ii.\hfill}\]
  \begin{sol}
  On résout le système                 
  \[\syslin{a+b&=&3-2\ii\hfill\cr
             ab&=&8+6\ii\hfill}\]
  On obtient l'équation du second degré $z^2-\p{3-2\ii}z+\p{8+6\ii}=0$ dont le
  discriminant est $-27-36\ii$. $3-6\ii$ en est une racine donc les racines
  du trinôme sont $3-4\ii$ et $2\ii$. Donc $a=3-4\ii$ et $b=2\ii$, ou le contraire.
  Les racines de $3-4\ii$ sont $\pm\p{2-\ii}$ et les racines de $2\ii$ sont
  $\pm\p{1+\ii}$. Ce qui nous donne au final 4 solutions dans le premier cas.
  On en élimine 2 en regardant le signe de la partie réelle de $uv$. Il
  reste $\p{u,v}=\p{2-\ii,1+\ii}$ et $\p{u,v}=\p{-2+\ii,-1-\ii}$. On rajoute bien sûr
  les deux autres solutions.
  \end{sol}
\end{exos}

\subsection{Racines $n$-ièmes}

\begin{definition}[utile=-3]
Étant donné $n\in\Ns$ et $a\in\C$, on appelle \emph{racine $n$-ième} de $a$ tout
nombre complexe $z$ tel que $z^n=a$. Les racines $n$-ièmes de 1 sont appelées
\emph{racines $n$-ièmes de l'unité} et l'ensemble de ces racines est noté $\U[n]$.
\end{definition}

\begin{remarqueUnique}
\remarque Les racines $n$-ièmes de 1 sont de module 1. Autrement dit, $\U[n]\subset\U$.
\end{remarqueUnique}


% \subsection{Division euclidienne}

\begin{proposition}[utile=-3]
Soit $n\in\Ns$. Il existe exactement $n$ racines $n$-ièmes de l'unité. En posant $\omega\defeq\e^{\ii\frac{2\pi}{n}}$, ce sont
\[1,\omega,\ldots,\omega^{n-1}.\]
\end{proposition}

\begin{preuve}
Soit $z\in\R$. On peut fixer $\theta\in\R$ tel que $z=e^{i\theta}$.
\begin{eqnarray*}
z^n=1 &\Longleftrightarrow& e^{in\theta}=1 \text{ d'après Moivre}\\
&\Longleftrightarrow& \exists k\in \Z, n\theta=2k\pi \\
&\Longleftrightarrow& \exists k\in \Z, \theta=\dfrac{2k\pi}{n}
\end{eqnarray*}
On a montré ainsi que $\U[n]=\ens{e^{\frac{2ik\pi}{n}}, k\in \Z}$.

Soit $k\in \Z$. \'Ecrivons la division euclidienne de $k$ par $n$. $\exists ! (q,r) \in \Z^2 \text{ tel que } k=qn+r \text{ et } 0\leq r<n$. On a alors :
\[e^{\frac{2ik\pi}{n}}=e^{\frac{2i(qn+r)\pi}{n}}=e^{2iq\pi}e^{\frac{2ir\pi}{n}}=e^{\frac{2ir\pi}{n}},\]
ce qui montre que $\U[n]\subset\ens{e^{\frac{2ik\pi}{n}}, 0\leq k\leq n-1}$. L'autre inclusion étant claire, on a bien l'égalité.

Il reste à voir qu'elles sont bien toutes distinctes (et qu'on en a donc exactement $n$). Pour cela, supposons qu'il existe $k$ et $\ell$ dans $\llbracket0,n-1\rrbracket$ tels que $e^{\frac{2ik\pi}{n}} =e^{\frac{2i\ell\pi}{n}}$. Ainsi, $e^{\frac{2i(k-\ell)\pi}{n}}=1$ et donc il existe $m\in\Z$ tel que $\dfrac{2(k-\ell)\pi}{n}=2m\pi$. Mais alors, $n$ divise $k-\ell$. Or $-(n-1)\leq k-\ell \leq n-1$ donc $k-l=0$.
\end{preuve}

\begin{remarques}
\remarque Lorsque l’on doit calculer sur des racines $n$-ièmes, il est souvent plus efficace de les manipuler via leur propriété ($z^n = 1$) plutôt que par leur description ($z=\omega^k$). On ne se rabat sur la description que lorsque la relation $z^n = 1$ ne suffit pas, ou en toute fin de calcul.
\remarque Dans le cas où $n=3$, $\omega$ est noté $\jj$. Les racines
  3-ièmes de l'unité sont donc $1,\jj,\jj^2$. Lorsqu'on travaille avec le nombre
  complexe $\jj$, on exploite les relations
  \[\jj^3=1, \qquad 1+\jj+\jj^2=0 \et \frac{1}{\jj}=\conj{\jj}=\jj^2.\]
% \remarque Attention, si $n\geq 2$, l'égalité $a^n=b^n$ n'implique pas $a=b$.
\remarque Les racines $n$-ièmes de l'unité forment un polygone régulier à $n$ côtés.
\pgfplotsset{
    standard/.style={
        axis x line=middle,
        axis y line=middle,
        enlarge x limits=0.15,
        enlarge y limits=0.15,
        every axis x label/.style={at={(current axis.right of origin)},anchor=north west},
        every axis y label/.style={at={(current axis.above origin)},anchor=north east}
    }
}
\begin{figure}[H]
\centering
\begin{tikzpicture}
\begin{axis}[standard,axis equal,mark=none,
  ymax=1,ymin=-1,
  domain=0:3,samples=4,disabledatascaling,
  xtick=0,xticklabels=0,
  ytick=0,yticklabels=0,colormap={slategraywhite}{rgb255=(0,0,0) rgb255=(0,0,0)}]
\addplot[scatter,color=black]({cos(2*pi*deg(x)/3)},{sin(2*pi*deg(x)/3)});
\draw[help lines] (0,0) circle (1);
\node[above right] at (1,0) {$1$};
\node[above left] at (-0.5,0.8660) {$\jj$};
\node[below left] at (-0.5,-0.8660) {$\jj^2$};
\end{axis}
\end{tikzpicture}
$\qquad$
\begin{tikzpicture}
\begin{axis}[standard,axis equal,mark=none,
  ymax=1,ymin=-1,
  domain=0:7,samples=8,disabledatascaling,
  xtick=0,xticklabels=0,
  ytick=0,yticklabels=0,colormap={slategraywhite}{rgb255=(0,0,0) rgb255=(0,0,0)}]
\addplot[scatter,color=black]({cos(2*pi*deg(x)/7)},{sin(2*pi*deg(x)/7)});
\draw[help lines] (0,0) circle (1);
\node[above right] at (1,0) {$1$};
\node[above right] at (0.62349,0.781831) {$\omega$};
\node[above] at (-0.2225,0.9749) {$\omega^2$};
\node[below right] at (0.62349,-0.781831) {$\omega^{n-1}$};
\node[below] at (-0.2225,-0.9749) {$\omega^{n-2}$};
\end{axis}
\end{tikzpicture}
\end{figure}
% \remarque Si $\omega\defeq\e^{\ii 2\pi/n}$, alors~:
%   $\forall k\in\Z\qsep \omega^k=1 \quad\ssi\quad k\equiv 0\ [n].$
\end{remarques}

\begin{exos}
% \exo Soit $f$ la fonction d'expression
%   \[f(z)\defeq\frac{z^{12}+3z^7-4z^4+6z+1}{z^6+z^4+z^2-1}.\]
%   Calculer $f(\ii)$, $f(\jj)$ et $f(\jj^2)$.
\exo Que dire de deux nombres complexes $a,b\in\C$ tels que $a^3=b^3$~?

\begin{sol}
Ou bien ils sont nuls, ou bien $a/b \in \ens{1,j,j^2}$.
\end{sol}
\exo Soit $n\in\Ns$. Résoudre sur $\C$ l'équation
  \[\p{z+1}^n=\p{z-1}^n.\]
  \begin{sol}
  Déjà, $z\neq 1$.
  \begin{eqnarray*}
  \p{\frac{z+1}{z-1}}^n=1 &\Longleftrightarrow & \exists k\in \llbracket 0,n-1\rrbracket\text{ tel que } \frac{z+1}{z-1}= e^{\frac{2ik\pi}{n}}\\
  &\Longleftrightarrow & z(1-e^{\frac{2ik\pi}{n}})=-1-e^{\frac{2ik\pi}{n}}\\
  &\Longleftrightarrow & z=\frac{1+e^{\frac{2ik\pi}{n}}}{e^{\frac{2ik\pi}{n}}-1} \text{ en reprenant les équivalences avec }k\neq 0
  \end{eqnarray*}
  On trouve $z=-\ii\cotan\p{\frac{k\pi}{n}}$ pour $k\in\intere{1}{n-1}$ en passant finalement à l'angle moitié.
  \end{sol}

\exo Calculer
  \[\sum_{z\in\U[n]} \abs{z-1}.\]
  \begin{sol}
\begin{eqnarray*}
  \sum_{z\in\U[n]} \abs{z-1}&=& \sum_{k=0}^{n-1} \abs{e^{\frac{2ik\pi}{n}}-1}\\
  &=& \sum_{k=0}^{n-1} \abs{e^{\frac{ik\pi}{n}}\p{e^{\frac{ik\pi}{n}}-e^{\frac{-ik\pi}{n}}}}\\
  &=& \sum_{k=0}^{n-1}2\abs{\sin\p{\frac{k\pi}{n}}}\\
  &=& 2\sum_{k=0}^{n-1}\sin\p{\frac{k\pi}{n}}\\
  &=& 2\sum_{k=0}^{n-1}\Im\p{e^{\frac{ik\pi}{n}}}\\
  &=& 2\Im\p{\sum_{k=0}^{n-1}\p{e^{\frac{i\pi}{n}}}^k}\\
  &=& 2\Im\p{\frac{1-e^{i\pi}}{1-e^{i\pi/n}}}\\
  &=& 2\frac{2}{-e^{\frac{i\pi}{2n}}2i\sin(\frac{\pi}{2n})}\\
  &=& 2\frac{ie^{\frac{-i\pi}{2n}}}{\sin(\frac{\pi}{2n})}\\
  &=& 2\cotan\p{\frac{\pi}{2n}}
\end{eqnarray*}
  On peut vérifier avec $n=4$ et le résultat obtenu dans l'exercice 12 :  $2\cotan\p{\frac{\pi}{8}}=\dfrac{2}{\sqrt{2}-1}$.
  \end{sol}
% \exo Résoudre sur $\C$ l'équation $z^2=27\conj{z}$.
%   \begin{sol}
%   On trouve $z=0,27,27j,27j^2$.
%   \end{sol}
\end{exos}

\begin{proposition}[utile=-3]
Soit $n\geq 2$ et $\zeta$ une racine $n$-ième de l'unité, différente de $1$.
Alors
\[1+\zeta+\dots+\zeta^{n-1}=0.\]
En particulier, la somme des racines $n$-ièmes de l'unité est nulle.
\end{proposition}

\begin{preuve}
Somme géométrique et/ou développement du polynôme factorisé.
\end{preuve}

% \begin{exos}
% \exo Calculer
%   \[\cos\frac{\pi}{11}+\cos\frac{3\pi}{11}+\cos\frac{5\pi}{11}+
%     \cos\frac{7\pi}{11}+\cos\frac{9\pi}{11}\]
% \end{exos}

\begin{proposition}[utile=-3]
Soit $a\in\Cs$ et $n\in\Ns$. Alors $a$ admet exactement $n$ racines $n$-ièmes. En posant $\omega\defeq\e^{\ii\frac{2\pi}{n}}$, si $z_0$ est une racine $n$-ième de $a$, les racines $n$-ièmes de $a$ sont
\[z_0,\omega z_0,\ldots,\omega^{n-1}z_0.\]
\end{proposition}

\begin{preuve}
$z_0\neq 0$ et $z/z_0 \in \U[n]$.

\end{preuve}

\begin{remarques}
\remarque Si $a=r\e^{\ii\theta}$ est sous forme trigonométrique, alors 
  \[z_0=\sqrt[n]{r}\e^{\ii\frac{\theta}{n}}\]
  est une racine $n$-ième de $a$.
\remarque Si $n\geq 2$, la somme des racines $n$-ièmes d'un nombre complexe est
  nulle.
\end{remarques}

\begin{exos}
\exo Résoudre sur $\C$ l'équation
  \[27\p{z-1}^6+\p{z+1}^6=0.\]
  \begin{sol}
  Comme $z\neq 1$, cela revient à $\displaystyle \p{\frac{z+1}{z-1}}^6=-27$. $z_0=\sqrt{3}e^{i\pi/6}$ en est une solution. D'où $\frac{z+1}{z-1}=\sqrt{3}e^{i\pi/6}e^{2ik\pi/6}$.
  On trouve
  \[z=\frac{\sqrt{3}\e^{\ii\frac{(1+2k)\pi}{6}}+1}{\sqrt{3}\e^{\ii\frac{(1+2k)\pi}{6}}-1}\]
   pour $k\in\intere{0}{5}$
  % On trouve
  % \[z=-\frac{1+i\sqrt{3}\omega^k}{1-i\sqrt{3}\omega^k}(faute)=
  %   \frac{1-i\sqrt{3}\cos\p{\frac{k\pi}{3}}}{2+\sqrt{3}\sin\p{\frac{k\pi}{3}}}
  %   (juste)\]
  % avec $k\in\intere{0}{5}$ et $\omega=e^{i\frac{\pi}{3}}$.
  \end{sol}
\exo En considérant les racines 7-ièmes de $-1$, montrer que
  \[\cos\frac{\pi}{7}-\cos\frac{2\pi}{7}+\cos\frac{3\pi}{7}=\frac{1}{2}.\]
  \begin{sol}
  On résout $z^7=-1$ en prenant $z_0=e^{i\pi/7}$. On exploite le fait que la somme des racines est nulle en en prenant la partie réelle. On regroupe les cosinus identiques et on conclut en remarquant que $\cos(5\pi/7)=-\cos(2\pi/7)$.
  \end{sol}
\end{exos}


\section{Nombres complexes et géométrie plane}

\subsection{Le plan complexe}
\begin{definition}
Soit $\mathcal{P}$ le plan euclidien orienté et $\mathcal{R}=\p{O,\ve{e_1},\ve{e_2}}$ un repère orthonormé direct.
\begin{itemize}
\item Si $M$ est un point du plan de coordonnées $(x,y)\in\R^2$
  \[\ve{OM}=x \ve{e_1}+y \ve{e_2}\]
  on appelle \emph{affixe} de $M$ le nombre complexe $x+\ii y$.
\item Si $\ve{u}$ est un vecteur de coordonnées $(x,y)\in\R^2$
  \[\ve{u}=x \ve{e_1}+y \ve{e_2}\]
  on appelle \emph{affixe} de $\ve{u}$ le nombre complexe $x+\ii y$.
\end{itemize}
\end{definition}

\begin{remarqueUnique}
\remarque Si $M$ est un point du plan, son affixe est l'affixe du vecteur $\ve{OM}$.
\end{remarqueUnique}

\begin{proposition}
\begin{itemize}
\item Soit $A$ et $B$ deux points du plan d'affixes respectives $a$ et $b\in\C$. Alors $\ve{AB}$ a pour affixe $b-a$.
\item Soit $\ve{u}$ et $\ve{v}$ deux vecteurs d'affixes respectives $u$ et $v\in\C$ et $\lambda,\mu\in\R$. Alors $\lambda\ve{u}+\mu\ve{v}$ a pour affixe $\lambda u+\mu v$.
\end{itemize}
\end{proposition}

% \begin{definition}
% Soit $A_1,\ldots,A_n$ des points du plan d'affixes respectifs $a_1,\ldots,a_n\in\C$ et $\lambda_1,\ldots,\lambda_n\in\R$ tels que
% $\lambda_1+\cdots+\lambda_n\neq 0$. On appelle \emph{barycentre} des points pondérés $(A_1,\lambda_1),\ldots,(A_n,\lambda_n)$ le point d'affixe
% \[z\defeq\frac{\lambda_1 a_1+\cdots+\lambda_n a_n}{\lambda_1+\cdots+\lambda_n}.\]
% \end{definition}

% \begin{remarques}
% \remarque Le milieu de $[AB]$ est le barycentre du système $(A,1), (B,1)$. En particulier, si $a$ et $b$ sont les affixes respectifs de $A$ et $B$, l'affixe du milieu de $[AB]$ est $(a+b)/2$.
% \remarque Si $A$ et $B$ sont deux points du plan, le segment $[AB]$ est l'ensemble des barycentres du système $(A,\lambda)$, $(B,1-\lambda)$ pour $\lambda\in\interf{0}{1}$. La droite $(AB)$ est l'ensemble des barycentres du système $(A,\lambda)$, $(B,1-\lambda)$ pour $\lambda\in\R$.
% \remarque Si $A_1,\ldots,A_n$ sont des points du plan, $\lambda_1,\ldots,\lambda_n\in\R$ sont tels que
% $\lambda_1+\cdots+\lambda_n\neq 0$ et $\lambda\in\Rs$, le barycentre du système $(A_1,\lambda \lambda_1),\ldots,(A_n,\lambda \lambda_n)$ est égal au barycentre du système $(A_1,\lambda_1),\ldots,(A_n,\lambda_n)$.
% \remarque On dit qu'un point $M$ est dans l'\emph{enveloppe convexe} de $A_1,\ldots,A_n$ lorsqu'il existe $\lambda_1,\ldots,\lambda_n\in\RP$ tels que $M$ est le barycentre du système $(A_1,\lambda_1),\ldots,(A_n,\lambda_n)$.
% \end{remarques}


\begin{proposition}[utile=-3]
\begin{itemize}
\item Soit $a$ et $b$ deux nombres complexes. Alors $\abs{a-b}$ est la distance entre les points d'affixes $a$ et $b$.
\item Soit $u$ un nombre complexe. Alors $\abs{u}$ est la norme du vecteur
$\ve{u}$.
\end{itemize}
\end{proposition}

\begin{proposition}[utile=-3]
Soit $A,B,C$ trois points deux à deux distincts d'affixes respectives $a,b,c$. Alors
\[\abs{\frac{c-a}{b-a}} = \frac{AC}{AB} \quad\text{et}\quad   \arg\p{\frac{c-a}{b-a}}\equiv\p{\overrightarrow{AB}, \overrightarrow{AC}}\  [2 \pi].\]
\end{proposition}

\begin{preuve}
Le premier point est immédiat Le second découle de l'interprétation géométrique de l'argument. En notant $\overrightarrow{i}$ le vecteur d'affixe $1$ et en posant $\theta=\arg(z)$, on a $z=|z|(\cos(\theta)+i\sin(\theta))$ donc $\p{\overrightarrow{i}, \overrightarrow{u}})\equiv \arg(z)[2\pi]$. De même, $\p{\overrightarrow{i}, \overrightarrow{u'}})\equiv \arg(z')[2\pi]$. D'où :


\begin{eqnarray*}
\p{\overrightarrow{u}, \overrightarrow{u'}}=& \p{\overrightarrow{i}, \overrightarrow{u'}}-\p{\overrightarrow{i}, \overrightarrow{u}}&[2\pi]\\
=&  \arg(z')-\arg(z)   &[2\pi]\\
=& \arg\p{\frac{z'}{z}} &[2\pi]
\end{eqnarray*}
\end{preuve}

\begin{proposition}[utile=-3]
Soit $A,B,C$ trois points deux à deux distincts d'affixes respectives $a,b,c$. Alors
\begin{itemize}
\item $A,B,C$ sont alignés si et seulement si
  \[\frac{c-a}{b-a}\in\R.\]
\item $(AB)$ et $(AC)$ sont orthogonales si et seulement si
  \[\frac{c-a}{b-a}\in\ii\R.\]
\end{itemize}
\end{proposition}

\begin{remarques}
\remarque Soit $A$, $B$, $C$ trois points d'affixes respectives $a$, $b$, $c\in\C$. Si $A$, $B$, $C$ sont deux à deux distincts, alors
  \begin{eqnarray*}
  \text{$A$, $B$, $C$ sont alignés}
  &\ssi& \frac{c-a}{b-a}\in\R\\
  &\ssi& \frac{c-a}{b-a}=\conj{\p{\frac{c-a}{b-a}}}\\
  &\ssi& (c-a)\conj{(b-a)}=\conj{(c-a)}(b-a).
  \end{eqnarray*}
  On vérifie facilement que même si $A$, $B$, $C$ ne sont pas deux à deux distincts
  \[\text{$A$, $B$, $C$ sont alignés} \quad\ssi\quad (c-a)\conj{(b-a)}=\conj{(c-a)}(b-a).\]
\remarque La proposition précédente étant essentiellement utilisée de cette manière, on pourra tolérer exceptionnellement de l'appliquer, même si $A$, $B$, $C$ ne sont pas deux à deux distincts. Ce genre de \og division par zéro \fg est parfois tolérée en géométrie. Bien entendu, dans tout autre domaine des mathématiques, ces horreurs ne seront pas tolérées.
\end{remarques}

\begin{exoUnique}
\exo Soit $ABCD$ un quadrilatère non croisé. On construit $A_1$ extérieur
au quadrilatère tel que le triangle $BA_1C$ est isocèle et
rectangle en $A_1$. De même pour $B_1,C_1,D_1$. Montrer que les
segments $[A_1C_1]$ et $[D_1B_1]$ ont même longueur et sont
orthogonaux.
\begin{sol}
On suppose que $(ABCD)$ est dans le sens direct. Alors
\[a_1=c+\frac{1}{2}(1+\ii)(b-c), \qquad
  b_1=d+\frac{1}{2}(1+\ii)(c-d),\]
\[c_1=a+\frac{1}{2}(1+\ii)(d-a), \qquad
  d_1=b+\frac{1}{2}(1+\ii)(a-b)\]
donc $a_1-c_1=\ii(d_1-y)$.

Sol. Victor :
$c-a_1=i(b-a_1)$ donc $a_1=\frac{1}{1-i}(c-ib)$. De même $b_1=\frac{1}{1-i}(d-ic)$, $c_1=\frac{1}{1-i}(a-id)$ et $d_1=\frac{1}{1-i}(b-ia)$. D'où $a_1-c_1=i(b_1-d_1)$.
\end{sol}
\end{exoUnique}

\subsection{Les similitudes directes}

\begin{definition}
Soit $\ve{u}$ un vecteur. On appelle \emph{translation} de vecteur $\ve{u}$ l'application qui au point $M$ associe l'unique point $M'$ tel que
\[\ve{MM'}=\ve{u}.\]
\end{definition}

\begin{proposition}
Soit $\ve{u}$ un vecteur d'affixe $u\in\C$. La translation de vecteur $\ve{u}$ transforme le point $M$ d'affixe $z$ en le point $M'$ d'affixe
\[z'=z+u.\]
\end{proposition}

\begin{remarqueUnique}
\remarque Une translation conserve les distances et les angles.
\end{remarqueUnique}

\begin{definition}
Soit $\Omega$ un point du plan et $\rho\in\Rs$. On appelle \emph{homothétie} de centre $\Omega$ et de rapport $\rho$ l'application qui au point $M$ associe l'unique point $M'$ tel que
\[\ve{\Omega M'}=\rho\ve{\Omega M}.\]
\end{definition}

\begin{proposition}
Soit $\Omega$ un point du plan d'affixe $\omega\in\C$ et $\rho\in\Rs$. L'homothétie de centre $\Omega$ et de rapport $\rho$ transforme le point $M$ d'affixe $z$ en le point $M'$ d'affixe
\[z'=\rho(z-\omega)+\omega.\]
\end{proposition}

\begin{remarqueUnique}
\remarque Une homothétie de rapport $\rho\in\Rs$ multiplie les distances par $\abs{\rho}$
	et conserve les angles.
\end{remarqueUnique}

\begin{definition}
Soit $\Omega$ un point du plan et $\theta\in\R$. On appelle \emph{rotation} de centre $\Omega$ et d'angle $\theta$ l'application qui au point $M$ associe
\begin{itemize}
\item $\Omega$ si $M=\Omega$.
\item l'unique point $M'$ tel que
  \[\Omega M'=\Omega M \et \p{\ve{\Omega M}, \ve{\Omega M'}}\equiv\theta\,\cro{2\pi}\]
  sinon.
\end{itemize}
\end{definition}

\begin{proposition}
Soit $\Omega$ un point du plan d'affixe $\omega\in\C$ et $\theta\in\R$. La rotation de centre $\Omega$ et d'angle $\theta$ transforme le point $M$ d'affixe $z$ en le point $M'$ d'affixe
\[z'=\e^{\ii\theta}(z-\omega)+\omega.\]
\end{proposition}

\begin{remarqueUnique}
\remarque Une rotation conserve les distances et les angles.
\end{remarqueUnique}

\begin{exoUnique}
\exo Quelle est l'expression en notation complexe des transformations suivantes~?
\begin{itemize}
\item \textbf{a.} La symétrie centrale de centre $0$.
\item \textbf{b.} L'homothétie de centre $0$ et de rapport $2$.
\item \textbf{c.} L'homothétie de centre $2$ et de rapport $1/2$.
\item \textbf{d.} La composée des deux dernières transformations.
\item \textbf{e.} La rotation de centre $0$ et d'angle $\pi/2$.
\item \textbf{f.} La rotation de centre $1+\ii$ et d'angle $\pi/2$.
\item \textbf{g.} La composée des deux dernières transformations.
\item \textbf{h.} La symétrie orthogonale d'axe $(Ox)$.
\item \textbf{i.} La symétrie orthogonale dont l'axe $\mathcal{D}_\theta$ fait un angle $\theta$ avec l'axe $(Ox)$.
\end{itemize}
\begin{sol}
\begin{itemize}
\item $f(z)=-z$.
\item $h_1(z)=2z$.
\item $h_2(z)=(1/2)(z-2)+2$.
\item $[h_1\circ h_2](z)=2+z$, $[h_2\circ h_1](z)=1+z$.
\item $r_1(z)=\ii z$.
\item $r_2(z)=\ii(z-(1+\ii))+(1+\ii)$.
\item $[r_1\circ r_2](z)=2\ii-z=-(z-\ii)+\ii$ (symétrie de centre $\ii$ ou rotation de centre $\ii$ et d'angle $\pi$), $[r_2\circ r_1](z)=2-z=-(z-1)+1$ (symétrie de centre $1$).
\item $f(z)=\conj{z}$.
\item $f(z)=\e^{2\ii\theta}\conj{z}$.
\end{itemize}
\end{sol}
\end{exoUnique}

\begin{definition}[utile=-3]
Soit $\Omega$ un point du plan, $r\in\RPs$ et $\theta\in\R$. On appelle \emph{similitude} de centre $\Omega$, de rapport $r$ et d'angle $\theta$ la composée (commutative) de l'homothétie de centre $\Omega$ et de rapport $r$ et de la rotation de centre $\Omega$ et d'angle $\theta$.
\end{definition}

\begin{proposition}[utile=-3]
Soit $\Omega$ un point du plan d'affixe $\omega$, $r\in\RPs$ et $\theta\in\R$. La similitude de centre $\Omega$, de rapport $r$ et d'angle $\theta$ transforme le point $M$ d'affixe $z$ en le point $M'$ d'affixe
\[z'=r\e^{\ii\theta}(z-\omega)+\omega.\]
\end{proposition}

\begin{remarques}
\remarque Si $\rho<0$, l'homothétie de centre $\Omega$ et de rapport $\rho$ est
  une similitude de centre $\Omega$, de rapport $\abs{\rho}$ et d'angle $\pi$.
\remarque Une similitude de rapport $r>0$ et d'angle $\theta$ multiplie les distances par
  $r$ et conserve les angles.
\end{remarques}

Dans la suite, on confondra un point et son affixe, un vecteur et son affixe. On identifie ainsi le plan à $\C$.

\begin{definition}
On appelle \emph{similitude directe} toute application $f$ de $\C$ dans $\C$ telle qu'il existe $a\in\Cs$ et $b\in\C$ tels que
\[\forall z\in\C\qsep f(z)=az+b.\]
\end{definition}

\begin{remarqueUnique}
\remarque Les translations et les similitudes de centre $\Omega$ de rapport $r\in\RPs$ et d'angle $\theta\in\R$ sont des similitudes directes. Nous allons voir que ce sont les seules.
\end{remarqueUnique}

\begin{proposition}
La composée de deux similitudes directes est une similitude directe.
\end{proposition}

\begin{proposition}
Soit $f$ une similitude directe et $a\in\Cs$, $b\in\C$ tels que
\[\forall z\in\C\qsep f(z)=az+b.\]
\begin{itemize}
\item Si $a=1$, $f$ est la translation de vecteur $b$.
\item Sinon, il existe $r\in\RPs$ et $\theta\in\R$ tels que $a=r\e^{\ii\theta}$. $f$ admet un unique point fixe $\omega$ et
\[\forall z\in\C\qsep f(z)=r\e^{\ii\theta}(z-\omega)+\omega.\]
Autrement dit $f$ est la similitude de centre $\omega$, de rapport $r$ et d'angle $\theta$.
\end{itemize}
\end{proposition}

\begin{preuve}
Soit $a\in\Cs$, $b\in\C$ et $f$ l'application de $\C$ dans $\C$ définie par
\[\forall z\in\C\qsep f(z)\defeq az+b.\]
Alors $f$ est une similitude. En effet~:
\begin{itemize}
\item Si $a=1$, alors $f(z)=z+b$ donc $f$ est la translation du vecteur d'affixe $b$.
\item Sinon, $f$ admet un unique point fixe. En effet
\begin{eqnarray*}
\forall z\in\C\qsep f(z)=z
&\ssi& az+b = z\\
&\ssi& (1-a)z=b\\
&\ssi& z=\frac{b}{1-a}.
\end{eqnarray*}
On pose $\omega\defeq b/(1-a)$. En mettant $a$ sous forme trigonométrique $a= r\e^{\ii\theta}$, on vérifie que
\[\forall z\in\C\qsep f(z)=r\e^{\ii\theta}\p{z-\omega}+\omega.\]
Donc $f$ est la similitude de centre d'affixe $\omega$, de rapport $r$ et d'angle $\theta$.
\end{itemize}
\end{preuve}

\begin{exoUnique}
\exo {\`A} quelle transformation géométrique correspond la fonction $f : z \mapsto
(3-\ii)+2\ii z$~?
\begin{sol}
C'est la similitude de centre $1+\ii$ d'angle $\pi/2$ et de rapport 2.
\end{sol}
\end{exoUnique}


%END_BOOK

\end{document}