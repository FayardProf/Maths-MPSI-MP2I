\documentclass[book]{magnolia}

\magtex{tex_driver={pdftex},
        tex_packages={float,caption,titling,epigraph,minitoc,slashbox,tabularx,cancel,pgfplots,xypic,nicefrac},
        tex_pstricks={pstricks,pst-plot}}
\magfiche{document_nom={Cours de Sup},
          auteur_nom={François Fayard},
          auteur_mail={fayard.prof@gmail.com}}
\magcours{cours_matiere={maths},
          cours_niveau={mpsi},
          cours_chapitre_numero={11},
          cours_chapitre={Dérivation}}
% \magmisenpage{misenpage_presentation={tikzvelvia},
%           misenpage_format={presentation},
%           misenpage_nbcolonnes={1},
%           misenpage_preuve={non},
%           misenpage_sol={non}}
\magmisenpage{}
\maglieudiff{}
\magprocess

\usepackage[scale=3]{ccicons}
\title{{\Huge\bf Cours de Mathématiques}\\\vspace{1cm}
%       \textbf{\Huge 2022--2023}\\
       \textbf{\Huge Maths Sup -- Tome III}\\\vspace{1cm}
      %  \textsc{L. Bouyge-Bono, F. Fayard, V. Lambert}\\\vspace{1cm}
       \textsc{F. Fayard}\\\vspace{1cm}
       \includegraphics[width=8cm]{../../Commun/Images/lazos-bde-2024.png}\\%\vspace{1cm}
       \includegraphics[width=8cm]{../../Commun/Images/lazos.png}}
% \author{{\sc François Fayard}, {\sc Victor Lambert}}
% \date{3 Août 2020 -- 1a20aa}

% \pretitle{%
%   \begin{center}
%   \includegraphics[width=12cm]{/Users/fayard/Desktop/lazos.png}\\[\bigskipamount]
% }
% \posttitle{\end{center}}

\mtcsettitle{minitoc}{}
\mtcsettitle{secttoc}{Table des matières}
\mtcsettitle{parttoc}{Table des matières}

\usetikzlibrary{positioning}
% \lstset{%
%   frame            = tb,    % draw frame at top and bottom of code block
%   tabsize          = 1,     % tab space width
%   numbers          = none,  % display line numbers on the left
%   framesep         = 3pt,   % expand outward
%   framerule        = 0.4pt, % expand outward 
%   commentstyle     = \color{green},      % comment color
%   keywordstyle     = \color{blue},       % keyword color
%   stringstyle      = \color{blue},    % string color
%   backgroundcolor  = \color{colorLazoBlue1Light}, % backgroundcolor color
%   showstringspaces = false,              % do not mark spaces in strings
%   basicstyle       = \ttfamily,
%   breaklines       = true
% }

\input{vc.tex}

\begin{document}

\maketitle

La version de ce document est la \textsc{\GITAbrHash}.\\

Merci à tous les élèves des lycées Janson de Sailly, du Parc et des Lazaristes pour leurs remarques et corrections. Je remercie
particulièrement \nom{Samuel Auroy}, \nom{Antonin Barbier}, \nom{Amin Belfkira}, \nom{Martin Bot}, \nom{Alexandre Brousse}, \nom{Élodie Brun}, \nom{Damien Callendrier}, \nom{Lauren Calvosa}, \nom{Sylvain Crosnier}, \nom{Enguerrand De Jaegere}, \nom{Thibaud De Valicourt}, \nom{Victor Déru}, \nom{Raphaël Des Boscs}, \nom{Grégoire Dhimoïla}, \nom{Léo Duhamel-Callot}, \nom{Sacha Evrard}, \nom{Axel Faou}, \nom{Titouan Francheteau}, \nom{Hélène Ghaleb}, \nom{Cédric Holocher},
\nom{Maxime Joubert}, \nom{Maxime Lombard}, \nom{Mira Maamri}, \nom{Gauthier Malandrin}, \nom{Pierre-Antoine Nguyen},
\nom{Hilaire Oudinot}, \nom{Eliott Pradeleix}, \nom{Yann-Ellie Ravon}, \nom{Sixtine Reynaud}, \nom{Vivien Thienot},
\nom{Carole Vacherand}, \nom{Camille Vialet}, \nom{Paul Vilars} et \nom{Antonin Villepontoux}.\\

Je tiens enfin à remercier mes anciens professeurs et
collègues qui ont eu une influence sur la rédaction de ce document~: \nom{Walter Appel}, \nom{Bruno Arsac}, \nom{Jean-Pierre Barani}, \nom{Vincent Bayle}, \nom{Christophe Bertault}, \nom{Laurence Bouyge}, \nom{Gilles Chaffard}, \nom{Alain Chillès}, \nom{Denis Choimet}, \nom{Vincent Clapiès}, \nom{Gérard Esposito}, \nom{Stéphane Gonnord}, \nom{Victor Lambert}, %
%\nom{Pierre-Louis Lions},
\nom{Frédéric Morlot}, \nom{Franz Ridde}, \nom{Emmanuel Roblet} et \nom{Alain Troesch}.\\
%, \nom{Cédric Villani}.\\
\vfill


\begin{center}
  \ccbysa\\
  \vspace{2ex}
  This work is licensed under a Creative Commons\\
  Attribution-ShareAlike 4.0 International License.\\
  \url{https://creativecommons.org/licenses/by-sa/4.0/legalcode.fr}\\
  \vspace{2ex}
  La dernière version de ce document ainsi que\\
  les sources \LaTeX{} sont disponibles à l'adresse\\
  \url{https://github.com/FayardProf/Maths-MPSI-MP2I}
  \end{center}
  \vspace{2ex}
  \begin{center}
  \textbf{Vous êtes autorisés à~:}
  \end{center}
  \vspace{2ex}
  \begin{itemize}
  \item \textbf{Partager}~: copier, distribuer et communiquer le matériel par tous les moyens et sous tous formats.
  \item \textbf{Adapter}~: remixer, transformer et créer à partir du matériel
  pour toute utilisation, y compris commerciale.
  \end{itemize}
  \vspace{2ex}
  \begin{center}
  \textbf{Selon les conditions suivantes~:}
  \end{center}
  \vspace{2ex}
  \begin{itemize}
    \item \textbf{Attribution}~: Vous devez créditer l'œuvre, intégrer un lien vers la licence et indiquer si des modifications ont été effectuées à l'œuvre. Vous devez indiquer ces informations par tous les moyens raisonnables, sans toutefois suggérer que l'offrant vous soutient ou soutient la façon dont vous avez utilisé son œuvre.
    \item \textbf{Partage dans les mêmes conditions}~: Dans le cas où vous effectuez un remix, que vous transformez, ou créez à partir du matériel composant l'œuvre originale, vous devez diffuser l'œuvre modifiée dans les même conditions, c'est à dire avec la même licence avec laquelle l'œuvre originale a été diffusée.
    \item \textbf{Pas de restrictions complémentaires}~: Vous n'êtes pas autorisé à appliquer des conditions légales ou des mesures techniques qui restreindraient légalement autrui à utiliser l'œuvre dans les conditions décrites par la licence.
    \end{itemize}

\setcounter{chapter}{10}

\tableofcontents

% \part{Sup}

\chapter{Groupes}
\setcounter{numeroexercicecours}{1}
\input{cours-groupes}
\section{Exercices}
\setcounter{numeroexercice}{1}
\input{exos-groupes}


\chapter{Limites et continuité}
\setcounter{numeroexercicecours}{1}
\input{cours-continuite_limites}
\section{Exercices}
\setcounter{numeroexercice}{1}
\input{exos-continuite_limites}

\chapter{Anneaux, corps, polynômes}
\setcounter{numeroexercicecours}{1}
\input{cours-corps_polynomes}
\section{Exercices}
\setcounter{numeroexercice}{1}
\input{exos-corps_polynomes}

\chapter{Dénombrement}
\setcounter{numeroexercicecours}{1}
\input{cours-denombrement}
\section{Exercices}
\setcounter{numeroexercice}{1}
\input{exos-denombrement}

\chapter{Dimension finie}
\setcounter{numeroexercicecours}{1}
\input{cours-dimension_finie}
\section{Exercices}
\setcounter{numeroexercice}{1}
\input{exos-dimension_finie}



\end{document}
