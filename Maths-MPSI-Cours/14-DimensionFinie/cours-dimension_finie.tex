\documentclass{magnolia}

\magtex{tex_driver={pdftex},
        tex_packages={xypic}}
\magfiche{document_nom={Cours sur la dimension finie},
          auteur_nom={François Fayard},
          auteur_mail={fayard.prof@gmail.com}}
\magcours{cours_matiere={maths},
          cours_niveau={mpsi},
          cours_chapitre_numero={13},
          cours_chapitre={Dimension finie}}
\magmisenpage{misenpage_presentation={tikzvelvia},
          misenpage_format={a4},
          misenpage_nbcolonnes={1},
          misenpage_preuve={non},
          misenpage_sol={non}}
\maglieudiff{}
\magprocess

\begin{document}

%BEGIN_BOOK
\magtoc

\section{Famille libre, famille génératrice, base}
\subsection{Famille libre}

\begin{definition}[utile=-3]
On dit qu'une famille $(x_1,\ldots,x_p)\in E^p$ est \emph{libre} lorsque quels que
soient $\lambda_1,\ldots,\lambda_p\in\K$
\[\lambda_1 x_1+\cdots+\lambda_p x_p=0 \quad\implique\quad
  \lambda_1=\cdots=\lambda_p=0.\]
Sinon, on dit qu'elle est \emph{liée}. Dans ce cas, toute relation du type
$\lambda_1 x_1+\cdots+\lambda_p x_p=0$ où les $\lambda_k$ ne sont pas tous nuls
est appelée \emph{relation de liaison}.
\end{definition}

\begin{sol}
Faire le lien tout de suite entre liberté et identification
\end{sol}

\begin{exos}
\exo Donner une condition nécessaire et suffisante sur
  $\lambda\in\R$ pour que la famille $(\p{1,2,0}, \p{2,1,1}, \p{3,3,\lambda})$
  soit une famille libre de $\R^3$.
  \begin{sol} On résout $a(1,2,0)+b(2,1,1)+c(3,3,\lambda)=(0,0,0)$ à l'aide du pivot de Gauss.
  Si $\lambda\neq 1$, le système est de Cramer (après résolution avec la méthode du pivot de Gauss). Sinon, c'est équivalent à : il existe $t\in \R$ tel que $a+2b+3c=0$, $b+c=0$ et $c=t$ d'où $a=-t$, $b=-t$ et $c=t$ donc il existe une solution non nulle (même une infinité).
  \end{sol}
\exo Montrer que sur $\mathcal{F}\p{\R,\R}$, la famille $\sin,\cos$
  est libre. Que dire si on lui adjoint la fonction
  $x\mapsto\sin(x+\pi/4)$~?
  \begin{sol}
  $\forall x \in \R$, $\sin\p{x+\pi/4}=\sin(x)\cos(\pi/4)+\cos(x)\sin(\pi/4)$ ce qui donne une relation entre les fonctions.
  \end{sol} 
% \exo Si $a_0,\ldots,a_n\in\K$ sont $n+1$ scalaires deux à deux distincts,
%   La famille $P_0,\ldots,P_n\in\polyK$ définie par
%   \[\forall k\in\intere{0}{n} \quad P_k=\prod_{\substack{i=0\\i\neq k}}^n
%     \frac{X-a_i}{a_k-a_i}\]
%   est libre.
\exo Sur $\mathcal{F}\p{\R,\R}$, considéré comme un \Rev, montrer que si
  $\alpha_1,\cdots,\alpha_p\in\R$ sont tels que $\alpha_1<\cdots<\alpha_p$, la famille des fonctions d'expressions
  $\e^{\alpha_1 x},\e^{\alpha_2 x},\ldots,\e^{\alpha_p x}$ est libre.
  \begin{sol}
  Pour faire plusieurs méthodes, je rajoute l'exo avec les $\alpha_i$ réels deux à deux distincts.
  Pour les complexes, on le prouve par récurrence sur $p$. Pour l'hérédité, on dérive l'égalité et on enlève $\alpha_{p+1}$ fois la combinaison linéaire nulle à ce qu'on vient d'obtenir, ce qui permet d'annuler le terme le plus haut et donc d'utiliser l'HR sur ce qui reste. On conclut grâce aux $\alpha_i$ distincts deux à deux.
  \end{sol}
\exo Montrer que la famille $1,\sqrt{2},\sqrt{3}$ est libre dans $\R$
  considéré comme un $\Q$-espace vectoriel.
  \begin{sol}
  On élève $a+b\sqrt{2}=-c\sqrt{3}$ au carré :
  $a^2+2b^2+ab\sqrt{2}=3c^2$ donc si $ab\neq 0$, $\sqrt{2}\in \Q$ donc $a=0$ ou $b=0$. Ensuite, selon le cas, on recommence le même genre de méthodes.
  \end{sol}
\end{exos}

\begin{remarques}
\remarque Lorsque $(x_1,\ldots,x_n)$ est libre, on dit aussi que $x_1,\ldots,x_n$ sont
  \emph{linéairement indépendants}.
\remarque Si $(x_1,\ldots,x_p)$ est une famille libre d'éléments de $E$ alors,
  quels que soient $\lambda_1,\ldots,\lambda_p,\mu_1,\ldots,\mu_p\in\K$ tels que
  \[\lambda_1 x_1+\cdots+\lambda_p x_p=\mu_1 x_1+\cdots+\mu_p x_p\]
  on a $\lambda_1=\mu_1,\ldots,\lambda_p=\mu_p$. La liberté d'une famille permet
  donc \og d'identifier \fg les coefficients d'une combinaison linéaire.
\remarque[utile=-3] Une famille composée d'un unique vecteur est libre si et seulement si
  ce vecteur est non nul.
\remarque Une famille $(x,y)\in E^2$ formée de deux vecteurs est liée si et
  seulement si ces vecteurs sont \emph{colinéaires}, c'est-à-dire si et seulement si il existe
  $\lambda\in\K$ tel que $x=\lambda y$ ou $y=\lambda x$. Plus généralement, une famille est
  liée si et seulement si il existe un vecteur qui est combinaison linéaire des
  autres.  
\remarque[utile=-3] Une famille libre reste libre lorsqu'on effectue une permutation
  de ses vecteurs ou lorsqu'on retire certains de ses vecteurs. En particulier, une famille
  libre ne contient ni vecteur nul, ni doublon, ni vecteurs colinéaires.
\remarque Si $(x_1,\ldots,x_p)$ est
une famille libre et si $x\in E\setminus\vect\p{x_1,...,x_p}$, alors la famille
$(x_1,\ldots,x_p,x)$ est libre.
\remarque[utile=-1] Sur $\C$, considéré comme un $\R$-espace vectoriel, la famille
  $(1,\ii)$ est libre. Cependant, si $\C$ est considéré comme un \Cev, elle est liée. La notion
  de liberté dépend donc du corps considéré.
% \remarque[utile=-3] Sur $\C$, si $z\in\C\setminus\R$, la famille $(1,z)$ est $\R$-libre.
% \begin{sol}
% Si $a,b\in \R$, $a+bz=0$ conduit à $a=-bz \in \C\setminus\R$ si $b\neq 0$.
% \end{sol}
% \remarque L'image d'une famille liée par une application linéaire est une famille
%   liée. En particulier, si l'image d'une famille par une application
%   linéaire est libre, alors cette famille est libre.
% \remarque Pour tout $n\in\N$, la famille $1,X,\ldots,X^n$ est une famille
%   libre de $\polyK$.
\end{remarques}

\begin{proposition}
L'image d'une famille libre par une application linéaire injective est libre.
\end{proposition}

\subsection{Famille génératrice}

\begin{definition}[utile=-3]
On dit qu'une famille $(x_1,\ldots,x_p)\in E^p$ est \emph{génératrice} de $E$ lorsque, quel que soit
$x\in E$, il existe $\lambda_1,\ldots,\lambda_p\in\K$ tels que
\[x=\lambda_1 x_1+\cdots+\lambda_p x_p.\]
Autrement dit, la famille $(x_1,\ldots,x_p)$ est génératrice si et seulement si
$\vect\p{x_1,\ldots,x_p}=E$.
\end{definition}

\begin{remarques}
% \remarque Afin de montrer que $(x_1,\ldots,x_p)$ est une famille génératrice de $E$
  % il ne faut pas oublier de montrer que $x_1,\ldots,x_p$ sont des éléments de $E$.
\remarque Une famille génératrice reste génératrice lorsqu'on effectue une
  permutation de ses vecteurs ou lorsqu'on lui rajoute d'autres vecteurs.
\remarque Si $(x_1,\ldots,x_p)$ est une famille génératrice et si
  $x_p\in\vect\p{x_1,\ldots,x_{p-1}}$, alors la famille $(x_1,\ldots,x_{p-1})$ est
  génératrice.
\end{remarques}


\begin{exos}
  \exo Soit $\lambda\in\R$. Donner une condition nécessaire et suffisante sur
    $\lambda$ pour que la famille $((1,2,0), (2,1,1), (3,3,\lambda))$
    soit une famille génératrice de $\R^3$.
    \begin{sol}
    Il faut $\lambda\neq 1$.
    \end{sol}
  \exo Soit $\K$ un corps de caractéristique nulle, $n\in\N$ et $\alpha\in\K$. Montrer que la famille
    $(1,X-\alpha,(X-\alpha)^2,\ldots,(X-\alpha)^n)$ est une famille génératrice de $\polyK[n]$.
  \end{exos}


\begin{proposition}[utile=-2]
Soit $f$ et $g$ deux applications linéaires de $E$ dans $F$. On suppose que
$(x_1,\ldots,x_p)$ est une famille génératrice de $E$ telle que
\[\forall k\in\intere{1}{p} \qsep f\p{x_k}=g\p{x_k}.\]
Alors $f=g$.
\end{proposition}
  
\begin{exoUnique}
\exo Montrer la formule de \nom{Simpson}.
  \[\forall P\in\polyR[3] \qsep \integ{0}{1}{P(t)}{t}=
    \frac{P\p{0}+4P\p{1/2}+P\p{1}}{6}.\]
    \begin{sol}
    Les termes de gauche et de droite sont les images d'applications linéaires qui sont égales en $1, X, X^2$ et $X^3$ d'où le résultat.
    \end{sol}
\end{exoUnique}

\begin{proposition}[utile=-2]
Soit $f$ une application linéaire de $E$ dans $F$. On suppose que chaque
élément d'une famille génératrice $(y_1,\ldots,y_p)$ de $F$ admet
au moins un antécédent par $f$. Alors $f$ est surjective.
\end{proposition}

\begin{proposition}[utile=-2]
Soit $f$ une application linéaire de $E$ dans $F$. Alors l'image
$(f\p{x_1},\ldots,f\p{x_p})$ d'une famille génératrice $(x_1,\ldots,x_p)$ de $E$ est
une famille génératrice de $\im f$.
En particulier, si $f$ est surjective, l'image par $f$
d'une famille génératrice de $E$ est une famille génératrice de $F$.
\end{proposition}

\subsection{Base}

\begin{definition}[utile=-3]
On dit qu'une famille $(e_1,\ldots,e_n)\in E^n$ est une \emph{base} de $E$ lorsque,
quel que soit $x\in E$, il existe un unique $n$-uplet $\p{\lambda_1,\ldots,\lambda_n}\in\K^n$ tel que
\[x=\lambda_1 e_1+\cdots+\lambda_n e_n.\]
Autrement dit, la famille $(e_1,\ldots,e_n)$ est une base de $E$ si et seulement si elle est libre et génératrice.
\end{definition}

\begin{remarques}
\remarque Si $\mathcal{B}\defeq(e_1,\ldots,e_n)$ est une base de $E$ et $x\in E$, les
  $\lambda_1,\ldots,\lambda_n\in\K$ tels que $x=\lambda_1 e_1+\cdots+\lambda_n e_n$ sont appelées
  \emph{coordonnées} de $x$ relativement à la base $\mathcal{B}$.
\remarque[utile=-3] Une base reste une base lorsqu'on effectue une permutation de ses
  vecteurs.
\end{remarques}

\begin{definition}[utile=-3]
\begin{itemize}
\item Si $n\in\N$, la famille $(e_1,\ldots,e_n)$ de $\K^n$ définie par
  \[\forall k\in\intere{1}{n} \qsep e_k\defeq \p{0,\ldots,0,1,0,\ldots,0}\]
  est une base de $\K^n$, appelée \emph{base canonique}.
\item Si $n\in\N$, la famille $(1,X,\ldots,X^n)$ est une base de $\polyK[n]$.
  On l'appelle \emph{base canonique} de $\polyK[n]$.
\item Si $p,q\in\N$, la famille $(E_{i,j})_{1\leq i\leq q, 1\leq j\leq p}$ des
  matrices élémentaires est une base de $\mat{q,p}{\K}$ appelée
  \emph{base canonique}.
\end{itemize}
\end{definition}

% \begin{proposition}
% Soit $\mathcal{F}\defeq(e_1,\ldots,e_n)$ une famille d'éléments de $E$. Alors, les assertions suivantes
% sont équivalentes~:
% \begin{itemize}
% \item $\mathcal{F}$ est une base de $E$.
% \item $\mathcal{F}$ est \emph{libre maximale}, c'est-à-dire qu'elle est libre et que
%   l'ajout de tout élément la rend liée.
% \item $\mathcal{F}$ est \emph{génératrice minimale}, c'est-à-dire qu'elle est
%   génératrice et que le retrait de tout élément lui fait perdre son caractère
%   générateur.
% \end{itemize}
% \end{proposition}

\begin{proposition}[utile=3]
Soit $E$ et $F$ deux \Kevs tels que $E$ admette une base $(e_1,\ldots,e_n)$.
Alors, quel que soit $(y_1,\ldots,y_n)\in F^n$, il existe une unique application
linéaire $f$ de $E$ dans $F$ telle que
\[\forall k\in\intere{1}{n} \qsep f\p{e_k}=y_k.\]
De plus
\begin{itemize}
\item $f$ est injective si et seulement si $(y_1,\ldots,y_n)$ est libre.
\item $f$ est surjective si et seulement si $(y_1,\ldots,y_n)$ est génératrice dans $F$.
\item $f$ est un isomorphisme si et seulement si $(y_1,\ldots,y_n)$ est une base de $F$.
\end{itemize}
\end{proposition}

\begin{preuve}
Pour la première partie du théorème, on démontre l'unicité à l'aide de la proposition 1.2.
Pour l'existence, on définit alors $f$ en tout $x\in E$ en le décomposant sur $(e_1,\ldots,e_n)$. On montre ensuite que cette application est bien linéaire.\\


Pour la seconde partie du théorème :
On a déjà montré les implications gauche droite des deux premiers points. 
\begin{itemize}
\item[$\bullet$] On suppose que $(y_1,\ldots,y_n)$ est libre. Montrons que $f$ est injective. On prend $x\in \ker f$ et on démontre qu'il est nul en  le décomposant sur la base $(e_1,\ldots,e_n)$.
\item[$\bullet$] On suppose maintenant que $(y_1,\ldots,y_n)$ est génératrice et on démontre que $f$ est surjective.
\item[$\bullet$] On obtient le troisième point en combinant les deux premiers.
\end{itemize}
\end{preuve}

% \begin{remarques}
% \remarque[utile=3] En reprenant les notations du théorème,
%   $f$ est injective si et seulement si $y_1,\ldots,y_n$
%   est libre. De même, $f$ est surjective si et seulement si $y_1,\ldots,y_n$ est
%   une famille génératrice de $F$. Enfin, $f$ est bijective si et seulement si
%   $y_1,\ldots,y_n$ est une base de $F$.
% \remarque Soit $p,q\in\Ns$. Un application $f$ de $\K^p$ dans $\K^q$ est
%   linéaire si et seulement si il existe une famille de scalaires
%   $\p{a_{i,j}}_{1\leq i\leq q,1\leq j\leq p}$ telle que, pour tout
%   $\p{x_1,\ldots,x_p}\in\K^p$
%   \[f\p{x_1,\ldots,x_p}=\p{\sum_{k=1}^p a_{1,k} x_k,\ldots,
%                            \sum_{k=1}^p a_{q,k} x_k}\]
%   En particulier, si $n\in\Ns$, une application $f$ de $\K^n$ dans $\K$ est une
%   forme linéaire si et seulement si il existe $a_1,\ldots,a_n\in\K$ tels
%   que, pour tout $\p{x_1,\ldots,x_n}\in\K^n$
%   \[f\p{x_1,\ldots,x_n}=a_1 x_1+\cdots+a_n x_n\]
% \end{remarques}

\begin{exoUnique}
\exo On pose $E\defeq\R^3$. Montrer qu'il existe $u\in\mathcal{L}(E)$ tel que
  $u^3=\id$ et $u\neq \id$.
  \begin{sol}
  Suffit de bien la définir sur la base canonique en prenant par exemple $u(e_1)=e_2$, $u(e_2)=e_3$ et $u(e_3)=e_1$.
  \end{sol}
\end{exoUnique}

\begin{proposition}
Soit $A$ et $B$ deux sous-espaces vectoriels de $E$, $\mathcal{A}\defeq(a_1,\ldots,a_p)$ une famille d'éléments de $A$ et
$\mathcal{B}\defeq(b_1,\ldots,b_q)$ une famille d'éléments de $B$. On pose $\mathcal{E}\defeq(a_1,\ldots,a_p,b_1,\ldots,b_q)$.
\begin{itemize}
\item Si $A$ et $B$ sont en somme directe et $\mathcal{A}$ et $\mathcal{B}$ sont libres, alors $\mathcal{E}$ est libre.
\item Si $A+B=E$ et $\mathcal{A}$ et $\mathcal{B}$ sont respectivement génératrices de $A$ et $B$, alors
  $\mathcal{E}$ est génératrice de $E$.
\item Si $A\oplus B=E$ et $\mathcal{A}$ et $\mathcal{B}$ sont respectivement des bases de $A$ et $B$, alors
  $\mathcal{E}$ est une base de $E$. On dit qu'une telle base est \emph{adaptée} à la
  décomposition $E=A\oplus B$.
\end{itemize}
\end{proposition}

\begin{definition}
Soit $E$ un \Kev admettant une base $\mathcal{B}\defeq(e_1,\ldots,e_n)$. Pour tout $k\in\intere{1}{n}$,
on définit $e_k^\star\in E^\star$ comme étant l'application de $E$ dans $\K$ qui à $x$ associe la
$k$-ième coordonnée de $x$ relativement à la base $\mathcal{B}$. Les applications $e_k^\star$ sont
appelées \emph{applications coordonnées} de la base $\mathcal{B}$.
\end{definition}

\begin{remarques}
\remarque Si $\mathcal{B}\defeq(e_1,\ldots,e_n)$ est une base de $E$ et $x\in E$,
  alors $(e_1^\star(x),\ldots,e_n^\star(x))\in\K^n$ est la famille des coordonnées de $x$
  relativement à la base $\mathcal{B}$. Autrement dit
  \[\forall x\in E\qsep x=\sum_{k=1}^n e_k^{\star}(x)e_k.\]
\remarque En particulier, si $\mathcal{B}\defeq(e_1,\ldots,e_n)$ est la base canonique
  de $\K^n$ et $x\defeq(x_1,\ldots,x_n)\in\K^n$, alors
  \[\forall k\in\intere{1}{n}\qsep e_k^\star(x)=x_k.\]
\end{remarques}

\begin{definition}
Soit $E$ un \Kev admettant une base $\mathcal{B}\defeq(e_1,\ldots,e_n)$. Alors $(e_1^\star,\ldots,e_n^\star)$
est une base de $E^\star$, appelée \emph{base duale} de $\mathcal{B}$.
\end{definition}

\begin{remarqueUnique}
\remarque Dans $E\defeq \K^n$, si $\mathcal{B}\defeq(e_1,\ldots,e_n)$ est la base canonique
  de $E$ et $\phi\in E^*$, il existe $a_1,\ldots,a_n\in\K$ tels que
  \[\phi=a_1 e_1^*+\cdots+a_n e_n^*.\]
  Pour tout $x=(x_1,\ldots,x_n)\in\K^n$, on a donc $\phi(x)=a_1 x_1+\cdots+a_n x_n$.
\end{remarqueUnique}

\subsection{Cas des familles infinies}

\begin{definition}
  On dit qu'une famille $(\lambda_i)_{i\in I}$ d'éléments de $\K$ est \emph{presque nulle} lorsqu'il existe une partie finie
  $J\defeq\ens{i_1,\ldots,i_n}$ de $I$ telle que
  \[\forall i\in I\setminus J\qsep \lambda_i=0.\]
  On note $\K^{(I)}$ l'ensemble des familles presque nulles d'éléments de $\K$ indexées par $I$.
\end{definition}

\begin{remarques}
\remarque Si $I$ est fini, toute famille d'éléments de $\K$ indexée par $I$ est presque
  nulle. Cette notion n'a donc d'intérêt que lorsque $I$ est infini.
\remarque Si $(\lambda_i)_{i\in I}$ est une famille d'éléments de $\K$, on appelle
  \emph{support} de cette famille l'ensemble
  \[J\defeq\enstq{i\in I}{\lambda_i \neq 0}.\]
  Les familles presque nulles sont donc les familles à support fini.
\remarque Une suite $(\lambda_n)_{n\in\N}$ est presque nulle si et seulement si elle est
  nulle à partir d'un certain rang. En particulier, la famille des coefficients d'un
  polynôme est presque nulle.
\end{remarques}

\begin{definition}
Soit $(x_i)_{i\in I}$ est une famille d'éléments de $E$ et $(\lambda_i)_{i\in I}$ une
famille presque nulle d'éléments de $\K$. Soit $J\defeq\ens{i_1,\ldots,i_n}$ une partie
de $I$ en dehors de laquelle $\lambda_i$ est nul. On définit alors
\[\sum_{i\in I} \lambda_i x_i \defeq \sum_{k=1}^n \lambda_{i_k} x_{i_k}.\]
\end{definition}

\begin{remarques}
\remarque Si $P\in\polyK$ et $(a_k)_{k\in\N}$ est la famille de ses coefficients, alors
  \[P=\sum_{k\in\N} a_k X^k.\]
\remarque Soit $(\lambda_i)_{i\in I}$ et $(\mu_i)_{i\in I}$ deux familles presque nulles d'éléments de $\K$. On pose
  \[x\defeq\sum_{i\in I} \lambda_i x_i \quad\et\quad y\defeq\sum_{i\in I} \mu_i x_i.\]
  Alors, quels que soient $\lambda$ et $\mu\in\K$, $(\lambda\lambda_i+\mu\mu_i)_{i\in I}$ est une famille
  presque nulle et
  \[\lambda x+\mu y=\sum_{i\in I} (\lambda \lambda_i+\mu \mu_i)x_i.\]
\end{remarques}


\begin{proposition}
Soit $(x_i)_{i\in I}$ une famille d'éléments de $E$. Alors
\[\vect\ensim{x_i}{i\in I}=\ensim{\sum_{i\in I} \lambda_i x_i}{(\lambda_i)_{i\in I} \in \K^{(I)}}.\]
\end{proposition}

\begin{definition}[utile=-3]
On dit qu'une famille $(x_i)_{i\in I}$ d'éléments de $E$ est \emph{libre} lorsque, quelle que soit la famille
presque nulle $(\lambda_i)_{i\in I}$
\[\sum_{i\in I} \lambda_i x_i=0 \quad\implique\quad \forall i\in I\qsep \lambda_i=0.\]
Sinon, on dit qu'elle est \emph{liée}. Dans ce cas, toute relation du type
$\sum_{i\in I} \lambda_i x_i=0$ où les $\lambda_i$ ne sont pas tous nuls est appelée \emph{relation de liaison}.
\end{definition}

\begin{remarques}
\remarque On dit qu'une partie $A$ de $E$ est libre lorsque la famille de ses éléments est
  libre.
\remarque Une famille $(x_i)_{i\in I}$ d'éléments de $E$ est libre si et seulement si
  toute sous-famille finie de $(x_i)_{i\in I}$ est libre, c'est-à-dire si et seulement si
  pour tout $i_1,\ldots,i_n\in I$ deux à deux distincts et $\lambda_1,\ldots,\lambda_n\in\K$
  tels que
  \[\lambda_1 x_{i_1}+\cdots+\lambda_n x_{i_n}=0\]
  on a $\lambda_1=\cdots=\lambda_n=0$.
\remarque Une famille $(x_i)_{i\in\N}$ est libre si et seulement si pour tout $n\in\N$
  et $\lambda_0,\ldots,\lambda_n\in\K$ tels que
  \[\lambda_0 x_0+\cdots+\lambda_n x_n=0\]
  on a $\lambda_0=\cdots=\lambda_n=0$.
\end{remarques}

\begin{exoUnique}
\exo Soit $(P_n)_{n\in\N}$ une famille d'éléments de $\polyK$ telle que
\[\forall n\in\N\qsep \deg(P_n)=n.\]
Montrer qu'elle est libre.
\end{exoUnique}

\begin{sol}
On peut prendre comme exemple la famille $(x\mapsto |x-a|)_{a\in \R}$.
\end{sol}

\begin{definition}[utile=-3]
On dit qu'une famille $(x_i)_{i\in I}$ d'éléments de $E$ est \emph{génératrice} de $E$ lorsque, quel que soit
$x\in E$, il existe une famille presque nulle $(\lambda_i)_{i\in I}$ telle que
\[x=\sum_{i\in I} \lambda_i x_i.\]
Autrement dit, la famille $(x_i)_{i\in I}$ est génératrice si et seulement si $\vect\ensim{x_i}{i\in I}=E$.
\end{definition}

\begin{remarques}
\remarque On dit qu'une partie $A$ de $E$ est génératrice lorsque la famille de ses éléments est génératrice.
\remarque Une famille $(x_i)_{i\in I}$ d'éléments de $E$ est génératrice de $E$ si et
seulement si pour tout $x\in E$, il existe $i_1,\ldots,i_n\in I$ et $\lambda_1,\ldots,\lambda_n\in\K$ tels que
\[x=\sum_{k=1}^n \lambda_k x_{i_k}.\]
\remarque Comme dans le cas des familles finies, si $f,g\in\mathcal{L}(E,F)$ coïncident sur une famille génératrice de
$E$, alors $f=g$. De même, si $f\in\mathcal{L}(E,F)$ et que chaque élément d'une famille génératrice de $F$ admet un antécédent
par $f$, alors $f$ est surjective.
\remarque Enfin, l'image par une application linéaire $f\in\mathcal{L}(E, F)$ d'une famille génératrice de $E$ est une famille
génératrice de $\im f$.
\end{remarques}

\begin{definition}[utile=-3]
On dit qu'une famille $(e_i)_{i\in I}$ d'éléments de $E$ est une \emph{base} de $E$ lorsque, quel que soit $x\in E$, il existe une
unique famille presque nulle $(\lambda_i)_{i\in I}$ telle que
\[x=\sum_{i\in I} \lambda_i e_i.\]
Autrement dit, la famille $(e_i)_{i\in I}$ est une base de $E$ si et seulement si elle est libre et génératrice.
\end{definition}

\begin{remarques}
\remarque Si $(e_i)_{i\in I}$ est une base de $E$ et $x\in E$, la
  famille $(\lambda_i)_{i\in I}$ telle que
  $x=\sum_{i\in I} \lambda_i e_i$ est appelée famille des
  \emph{coordonnées} de $x$ dans la base $(e_i)_{i\in I}$.
% \remarque Comme pour les familles finies, une famille d'éléments de $E$ est une base
%   si et seulement si elle est libre maximale ou génératrice minimale.
\remarque Si $(e_i)_{i\in I}$ est une base de $E$ et $(y_i)_{i\in I}$ est une famille
d'éléments de $F$, alors il existe une unique application linéaire $f$ de $E$ dans $F$ telle que
\[\forall i\in I\qsep f(e_i)=y_i.\]
De plus, $f$ est injective si et seulement si $(y_i)_{i\in I}$ est libre, $f$ est surjective si et seulement si
$(y_i)_{i\in I}$ est génératrice de $F$ et $f$ est un isomorphisme si et seulement si $(y_i)_{i\in I}$ est une base de $F$.
\remarque La proposition sur les sommes de sous-espaces vectoriels
  et les familles s'énonce de manière similaire avec des familles
  infinies.
\end{remarques}

% \begin{exempleUnique}
% \exemple La famille $(X^k)_{k\in\N}$ est une base de $\polyK$.
% \end{exempleUnique}

\begin{definition}
La famille $(X^n)_{n\in\N}$ est une base de $\polyK$, appelée \emph{base canonique}.
\end{definition}

\begin{remarqueUnique}
\remarque Si $P\in\polyK$ alors la famille 
  $(a_k)_{k\in\N}$ des coefficients de $P$ est la famille des coordonnées de  $P$ dans
  la base canonique.
\end{remarqueUnique}

\section{Dimension}

\subsection{Espace vectoriel de dimension finie}
\begin{definition}[utile=-3]
On dit qu'un \Kev $E$ est de \emph{dimension finie} lorsqu'il admet une famille
génératrice finie. Dans le cas contraire, on dit que $E$ est de \emph{dimension
infinie}.
\end{definition}

\begin{remarqueUnique}
\remarque $\K^n$ et $\polyK[n]$ sont de dimension finie. $\polyK$ est
  de dimension infinie.
\end{remarqueUnique}


\begin{proposition}[utile=2,nom={Lemme de \nom{Steinitz}}]
Soit $E$ un \Kev de dimension finie et $(g_1,\ldots,g_q)$ une famille génératrice de $E$.
\begin{itemize}
\item Toute famille libre de $E$ possède au plus $q$ éléments.
\item Il est possible de compléter toute famille libre $(e_1,\ldots,e_p)$ en une base
$(e_1,\ldots,e_p,g_{i_{p+1}},\ldots,g_{i_n})$ de $E$ où $g_{i_{p+1}},
\ldots,g_{i_n}$ sont des éléments de la famille $(g_1,\ldots,g_q)$.
\end{itemize}

\end{proposition}

\begin{preuve}
Soit $\mathcal{F}$ l'ensemble des familles libres de $E$ formées des éléments
$(e_1,\ldots,e_m)$ et de certains des éléments de $(g_1,\ldots,g_p)$. Cet ensemble
est non vide car il contient la famille $(e_1,\ldots,e_m)$. De plus les familles
de $\mathcal{F}$ sont toutes de cardinal inférieur à $m+p$. Il existe
donc une famille de $\mathcal{F}$ de cardinal maximal. Notons
$(e_1,\ldots,e_m,g_{i_{m+1}},\ldots,g_{i_n})$ une telle famille et montrons que c'est
une base de $E$
\begin{itemize}
\item Cette famille est libre par définition de $\mathcal{F}$.
\item Montrons qu'elle est génératrice.
  \begin{itemize}
  \item Montrons d'abord que~:
    \[\forall k\in\intere{1}{p} \quad
      g_k\in\vect(e_1,\ldots,e_m,g_{i_{m+1}},\ldots,g_{i_n})\]
    Soit $k\in\intere{1}{p}$. Si $g_k$ est un des $g_{i_{m+1}},\ldots,g_{i_n}$, le
    résultat est évident. \\
    Sinon, la famille
    $(e_1,\ldots,e_m,g_{i_{m+1}},\ldots,g_{i_n},g_k)$ possède $n+1$ éléments. Puisque
    les familles de $\mathcal{F}$ ont toutes au plus $n$ éléments, on en déduit
    que cette famille n'est pas dans $\mathcal{F}$. Elle est donc liée. Il existe
    donc $\lambda_1,\ldots,\lambda_{n+1} \in\K$ non tous nuls tels que~:
    \[\lambda_1 e_1+\cdots+\lambda_m e_m + \lambda_{m+1} g_{i_{m+1}}+\cdots+
      \lambda_n g_{i_n} + \lambda_{n+1} g_k=0\]
    Montrons que $\lambda_{n+1}\neq 0$. On raisonne par l'absurde et on suppose que
    $\lambda_{n+1}=0$. Alors~:
    \[\lambda_1 e_1+\cdots+\lambda_m e_m + \lambda_{m+1} g_{i_{m+1}}+\cdots+
      \lambda_n g_{i_n}=0\]
    Puisque la famille $(e_1,\ldots,e_m,g_{i_{m+1}},\ldots,g_{i_n})$ est libre, on en
    déduit que $\lambda_1=\cdots=\lambda_n=0$. Donc les $\lambda_k$
    ($1\leq k\leq n+1$) sont tous nuls. C'est absurde, donc $\lambda_{n+1}\neq 0$.
    On en déduit donc que~:
    \[g_k=\p{-\frac{\lambda_1}{\lambda_{n+1}}} e_1+\cdots+
          \p{-\frac{\lambda_m}{\lambda_{n+1}}} e_m +
          \p{-\frac{\lambda_{m+1}}{\lambda_{n+1}}} g_{i_{m+1}}+\cdots+
          \p{-\frac{\lambda_n}{\lambda_{n+1}}} g_{i_n}\]
    Donc $g_k\in\vect\ens{e_1,\ldots,e_m,g_{i_{m+1}},\ldots,g_{i_n}}$
  \item Montrons désormais que la famille $(e_1,\ldots,e_m,g_{i_{m+1}},\ldots,g_{i_n})$
    est génératrice, c'est-à-dire que~:
    \[E=\vect(e_1,\ldots,e_m,g_{i_{m+1}},\ldots,g_{i_n})\]
    Puisque
    \[\forall k\in\intere{1}{p} \quad
      g_k\in\vect(e_1,\ldots,e_m,g_{i_{m+1}},\ldots,g_{i_n})\]
    on en déduit que~:
    \[\vect(g_1,\ldots,g_p) \subset
      \vect(e_1,\ldots,e_m,g_{i_{m+1}},\ldots,g_{i_n})\]
    Or la famille $(g_1,\ldots,g_p)$ est génératrice donc
    $\vect(g_1,\ldots,g_p)=E$. Donc~:
    \[E\subset  \vect(e_1,\ldots,e_m,g_{i_{m+1}},\ldots,g_{i_n})\]
    L'inclusion réciproque étant immédiate, on en déduit que~:
    \[E= \vect(e_1,\ldots,e_m,g_{i_{m+1}},\ldots,g_{i_n})\]
    et donc que $(e_1,\ldots,e_m,g_{i_{m+1}},\ldots,g_{i_n})$ est génératrice.
  \end{itemize}
\end{itemize}
En conclusion $(e_1,\ldots,e_m,g_{i_{m+1}},\ldots,g_{i_n})$ est une base de $E$.  
\end{preuve}
\begin{preuve}
Montrons ce résultat par récurrence sur $n$~:
\begin{center}
$\mathcal{H}_n$ : \flqq\ 
\parbox[t]{0.75\linewidth}{%
Soit $E$ un \Kev et $(g_1,\ldots,g_n)$ une famille génératrice de $E$. Alors
toute famille d'éléments de $E$ possédant plus de $n+1$ éléments est liée.
\frqq}
\end{center}
\begin{itemize}
\item $\mathcal{H}_0$ est vraie.\\
  En effet, soit $E$ un \Kev engendré par une famille vide. Alors $E=\ens{0}$.
  Montrons que toute famille qui possède au moins 1 élément est liée. En effet
  si $x_1\in E$ alors $x_1=0$ donc $1\cdot x_1=0$ est une relation de liaison.
\item $\mathcal{H}_n\implique\mathcal{H}_{n+1}$.\\
  On suppose que $\mathcal{H}_n$ est vraie. Montrons que $\mathcal{H}_{n+1}$ est
 vraie. En effet, soit $E$ un \Kev engendré par une famille $(g_1,\ldots,g_{n+1})$.
  Montrons que toute famille qui possède au moins $n+2$ éléments est liée. Soit
  $(x_1,\ldots,x_{n+2})$ une famille de $n+2$ éléments. Puisque
  $(g_1,\ldots,g_{n+1})$ est génératrice il existe une famille de scalaires
  $\p{\lambda_{i,j}}$ avec $i\in\intere{1}{n+2}$ et $j\in\intere{1}{n+1}$ telle
  que~:
  \[\syslin{x_1&=&\lambda_{1,1}g_1&+\lambda_{1,2}g_2&+\cdots&
                  +\lambda_{1,n+1}g_{n+1}\cr
            x_2&=&\lambda_{2,1}g_1&+\lambda_{2,2}g_2&+\cdots&
                  +\lambda_{2,n+1}g_{n+1}\cr
               &\hfill\vdots\hfill&&&&\cr
            x_{n+2}&=&\lambda_{n+2,1}g_1&+\lambda_{n+2,2}g_2&+\cdots&
                  +\lambda_{n+2,n+1}g_{n+1}}\]
  Deux cas se présentent~:
  \begin{itemize}
  \item Si $\forall i\in\intere{1}{n+2} \quad \lambda_{i,1}=0$.\\
    Dans ce cas la famille $(x_1,\ldots,x_{n+2})$ est une famille de $n+2$
    vecteurs de l'espace $F=\vect\p{g_2,\ldots,g_{n+1}}$ engendré par $n$
    éléments. Puisque $\mathcal{H}_n$ est vraie, on en déduit que cette famille
    est liée.
  \item Sinon, il existe $i\in\intere{1}{n+2}$ tel que $\lambda_{i,1}\neq 0$.
    Quitte à changer l'ordre des $x_i$ on peut supposer que
    $\lambda_{1,1}\neq 0$. Alors, en utilisant $\lambda_{1,1}$ comme pivot~:
    \[L_2\gets L_2-\frac{\lambda_{2,1}}{\lambda_{1,1}}L_1 \quad \ldots \quad
      L_{n+2}\gets L_{n+2}-\frac{\lambda_{n+2,1}}{\lambda_{1,1}}L_1\]
    \[\syslin{x_1&=&\lambda_{1,1}g_1&+\lambda_{1,2}g_2&+\cdots&
                  +\lambda_{1,n+1}g_{n+1}\cr
              x_2-\frac{\lambda_{2,1}}{\lambda_{1,1}}x_1
                &=&&\p{?}g_2&+\cdots&+\p{?}g_{n+1}\cr
               &\hfill\vdots\hfill&&&&\cr
              x_{n+2}-\frac{\lambda_{n+2,1}}{\lambda_{1,1}}x_1
                &=&&\p{?}g_2&+\cdots&+\p{?}g_{n+1}}\]
    où les $\p{?}$ sont des scalaires dont la valeur exacte est inutile. On en
    déduit que la famille de $n+1$ vecteurs~:
    \[x_2-\frac{\lambda_{2,1}}{\lambda_{1,1}}x_1,\ldots,
      x_{n+2}-\frac{\lambda_{n+2,1}}{\lambda_{1,1}}x_1\]
    appartient à l'espace vectoriel $F=\vect\p{g_2,\ldots,g_{n+1}}$ engendré
    par $n$ vecteurs. Puisque $\mathcal{H}_n$ est vraie, on en déduit que cette
    famille est liée. Il existe donc $\lambda_2,\ldots,\lambda_{n+2}\in\K$ non
    tous nuls tels que~:
    \[\lambda_2\p{x_2-\frac{\lambda_{2,1}}{\lambda_{1,1}}x_1}+\cdots+
      \lambda_{n+2}\p{x_{n+2}-\frac{\lambda_{n+2,1}}{\lambda_{1,1}}x_1}=0\]
    Donc~:
    \[\p{?}x_1+\lambda_2 x_2+\cdots+\lambda_{n+2}x_{n+2}=0\]
    qui est une relation de liaison puisque les $\lambda_k$ ne sont pas tous
    nuls. En conclusion, la famille $(x_1,\ldots,x_{n+2})$ est liée.
  \end{itemize}
   Donc $\mathcal{H}_{n+1}$ est vraie.
\end{itemize}
Par récurrence sur $n$ on en déduit donc que $\mathcal{H}_n$ est vraie pour
tout $n\in\N$.
\end{preuve}

\begin{remarqueUnique}
\remarque[utile=-2] Pour montrer qu'un espace vectoriel est de dimension infinie, il
  suffit donc de trouver des familles libres possédant autant d'éléments que l'on
  souhaite.
\end{remarqueUnique}

\begin{exoUnique}
\exo \`A l'aide d'éléments de la famille $(1, X, X^2)$, compléter la famille
  $(X^2-1,X^2+1)$ en une base de $\polyR[2]$.
\begin{sol}
Il suffit de rajouter un élément de la base canonique (par exemple) qui n'est pas dans son $\vect$. $X$ fonctionne. On vérifie que cela forme une base.
\end{sol}
\end{exoUnique}



\begin{theoreme}[nom={Théorèmes de la base incomplète et de la base extraite}]
Soit $E$ un \Kev de dimension finie.
\begin{itemize}
\item Toute famille libre se complète en une base finie de $E$.
\item Il est possible d'extraire une base finie de toute famille génératrice de $E$.
\end{itemize}
\end{theoreme}

\begin{preuve}
Comme $E$ est de dimension finie, il existe une famille génératrice (définition) donc on peut appliquer le théorème de la base incomplète.

Pour le deuxième point, on complète à partir de la famille vide.
\end{preuve}

% \begin{remarques}
% \remarque Algorithme de construction d'une base~:
% \begin{itemize}
% \item On commence par une famille $e_1,\ldots,e_m$
% \item Si cette famille est génératrice, on arrête. C'est une base.
% \item Sinon, on y ajoute un vecteur quelconque de $g_1,\ldots,g_p$ qui n'est pas
%   dans $\vect(F)$. On retourne à l'étape 2.
% \end{itemize}
% \end{remarques}


\begin{proposition}[utile=-3]
Tout espace vectoriel de dimension finie admet une base finie.
\end{proposition}

\begin{preuve}
C'est une autre lecture du deuxième point de la proposition précédente.
\end{preuve}

\begin{remarques}
\remarque[utile=-3] On en déduit qu'un espace vectoriel est de dimension finie si et seulement si il
  admet une base finie.
\remarque On peut montrer que tout espace vectoriel admet une base mais ce théorème est
  peu utile, hors programme, difficile à montrer et fait appel à l'axiome du choix.
\end{remarques}

\subsection{Dimension d'un espace vectoriel}


\begin{definition}[utile=-3]
Soit $E$ un \Kev de dimension finie. Alors toutes les bases de $E$ sont finies et ont le
même nombre d'éléments; on appelle \emph{dimension} de $E$ et on note $\dim E$ cet
entier.
\end{definition}
\begin{preuve}
Soit $E$ un \Kev de dimension finie, $\mathcal{B}_1=(e_1,\ldots,e_{n_1})$ et
$\mathcal{B}_2=(f_1,\ldots,f_{n_2})$ deux bases de $E$. Montrons que
$n_1=n_2$.\\
Puisque $E$ est engendré par $\mathcal{B}_2$ et que $\mathcal{B}_1$ est une
famille libre de $E$ le théorème précédent montre que $n_1\leq n_2$. De même on
montre que $n_2\leq n_1$, donc $n_1=n_2$.
\end{preuve}

\begin{remarques}
\remarque[utile=-2] Un espace vectoriel $E$ est de dimension nulle si et seulement
  si $E=\ens{0}$. On appelle \emph{droite vectorielle} tout espace vectoriel de dimension 1,
  c'est-à-dire tout espace vectoriel $E$ tel qu'il existe $x\in E\setminus\ens{0}$ tel
  que $E=\K x$. On appelle \emph{plan vectoriel} tout espace vectoriel de dimension 2.
\remarque[utile=-1] Considéré comme un \Rev, $\C$ est de dimension 2. Cependant, $\C$
  est de dimension 1 lorsqu'on le considère comme un \Cev. La dimension
  est donc une notion qui dépend du corps.
\end{remarques}

\begin{exos}
\exo Dans $\R^3$, donner une base du sous-espace vectoriel d'équation $x+y+z=0$.
\exo Dans $\polyR[2]$, donner une base puis la dimension du sous-espace vectoriel
  \[F=\enstq{P\in\polyR[2]}{\integ{0}{1}{P(t)}{t}=0}.\]
  
  \begin{sol}
  $\vect(X-1/2,X^2-1/3)$ puis on montre que c'est une famille libre.
  \end{sol}
\exo Si $E$ est un \Kev de dimension $n$, montrer qu'il existe $u\in\mathcal{L}(E)$
  tel que $u^n=0$ et $u^{n-1}\neq 0$.
  \begin{sol}
  On définit $u$ par $u(e_k)=e_{k-1}$ si $k\geq 2$ et $u(e_1)=0$.
  \end{sol}
\end{exos}

\begin{proposition}[utile=-3]
\begin{itemize}
\item Si $n\in\N$, $\K^n$ est un \Kev de dimension $n$.
\item Si $n\in\N$, $\polyK[n]$ est un \Kev de dimension $n+1$.
\item Si $p,q\in\N$, $\mat{q,p}{\K}$ est un \Kev de dimension $pq$. 
\end{itemize}
\end{proposition}

\begin{remarques}
\remarque En particulier, $\mat{n}{\K}$ est de dimension $n^2$.
\remarque $\mathcal{T}_n(\K)$ et $\mathcal{S}_n\p{\K}$ sont de dimension $n(n+1)/2$. $\mathcal{A}_n\p{\K}$ est de dimension $n(n-1)/2$.
\end{remarques}

\begin{proposition}[utile=-3]
Soit $E$ un \Kev de dimension $n$ et $(x_1,\ldots,x_p)$ une famille de $p$
éléments de $E$.
\begin{itemize}
\item Si $(x_1,\ldots,x_p)$ est libre, alors $p\leq n$.
\item Si $(x_1,\ldots,x_p)$ est génératrice, alors $p\geq n$.
\end{itemize}
\end{proposition}
\begin{preuve}
Soit $\mathcal{B}=(e_1,\ldots,e_n)$ une base de $E$ et $(x_1,\ldots,x_p)$ une
famille de $E$. Si $(x_1,\ldots,x_p)$ est libre, alors $p\leq n$ car
$\mathcal{B}$ est génératrice. Si $(x_1,\ldots,x_p)$ est génératrice, alors
$n\leq p$ car $\mathcal{B}$ est libre.
\end{preuve}

\begin{exoUnique}
\exo Soit $E$ un \Kev de dimension $n$ et $f$ un endomorphisme nilpotent
  de $E$. En notant $m$ le plus petit entier tel que $f^{m}=0$, montrer
  qu'il existe $x\in E$ tel que la famille $(x,f(x),\ldots,f^{m-1}(x))$ est
  libre. Que peut-on en déduire sur $m$~?
  
  \begin{sol}
  On prend $x\in E$ tel que $f^{m_0-1}(x)\neq 0$. Montrons que la famille $(x,f(x),\ldots,f^{m_0-1}(x))$ est libre. On considère une CL nulle de cette famille et on la suppose par l'absurde non triviale. On prend le plus petit indice tel que $\lambda_i\neq 0$ et on compose par la bonne puissance de $f$ qui annule tous les autres et laisse tout seul l'indice $i_0$ pour aboutir à une contradiction.
  \end{sol}
\end{exoUnique}

\begin{proposition}[utile=2]
Soit $E$ et $F$ deux \Kevs. On suppose que $E$ est de dimension finie. Alors
$E$ et $F$ sont isomorphes si et seulement si $F$ est de dimension finie et
$\dim E=\dim F$.
En particulier, si $E$ est de dimension $n$, $E$ est isomorphe à $\K^n$.
\end{proposition}
\begin{preuve}
$\Longrightarrow$
Supposons que $E$ et $F$ soient isomorphes et montrons que $F$ est de dimension
finie et que $\dim F=\dim E$. Soit $(e_1,\ldots,e_n)$ une base de $E$ et $f$
un isomorphisme de $E$ dans $F$. Alors la famille $(f\p{e_1},\ldots,f\p{e_n})$
est une base de $F$. En particulier $F$ est de dimension finie et
$\dim F=n=\dim E$.\\
$\Longleftarrow$
Réciproquement, supposons que $F$ soit de dimension finie et que
$\dim F=\dim E$. Soit $(e_1,\ldots,e_n)$ une base de $E$ et $(f_1,\ldots,f_n)$ une
base de $F$. Alors, il existe une unique application linéaire $f$ de $E$ dans
$F$ telle que~:
\[\forall k\in\intere{1}{n} \quad f\p{e_k}=f_k\]
Puisque $(f\p{e_1},\ldots,f\p{e_n})$ est une base de $F$, on en déduit que $f$
est un isomorphisme.
\end{preuve}

\begin{exos}
\exo Soit $a_0,\ldots,a_{p-1}\in\K$ et $E$ l'ensemble des suites $(u_n)$ à valeurs dans
$\K$ telles que
\[\forall n\in\N\qsep u_{n+p}=a_{p-1}u_{n+p-1}+\cdots+a_0 u_n.\]
Montrer que
\[\dspappli{\phi}{E}{\K^p}{(u_n)}{(u_0,\ldots,u_{p-1})}\]
est un isomorphisme. En déduire que $E$ est de dimension finie et que
$\dim E=p$.
\exo Soit $a,b,c\in\R$ avec $a\neq 0$ et $E$ l'ensemble des solutions de l'équation
  différentielle
  \[\forall t\in\R\qsep ay''(t)+by'(t)+cy(t)=0.\]
  Montrer que si $t_0\in\R$, alors
  \[\dspappli{\phi}{E}{\R^2}{y}{(y(t_0),y'(t_0))}\]
  est un isomorphisme. En déduire que $E$ est un \Rev de dimension 2.
\end{exos}

\subsection{Existence et unicité en dimension finie}

\begin{proposition}[utile=3]
Soit $E$ un \Kev de dimension $n$. Alors
\begin{itemize}
\item Toute famille libre de $E$ comportant $n$ éléments est une base de $E$.
\item Toute famille génératrice de $E$ comportant $n$ éléments est une base de
  $E$.
\end{itemize}
Autrement dit, si la famille $\mathcal{F}$ est composée de $n$ éléments,
\[\text{$\mathcal{F}$ est libre} \quad\ssi\quad \text{$\mathcal{F}$ est une base}
  \quad\ssi\quad \text{$\mathcal{F}$ est génératrice.}\]
\end{proposition}
\begin{preuve}
Soit $(e_1,\ldots,e_n)$ une famille libre à $n$ éléments de $E$. D'après le
théorème de la base incomplète, on peut la compléter en une base
$\mathcal{B}=(e_1,\ldots,e_n,\ldots,e_{n+k})$ de $E$. Puisque $\dim E=n$ on en
déduit que $n+k=n$ donc $k=0$. En conclusion $\mathcal{B}=(e_1,\ldots,e_n)$ donc
$(e_1,\ldots,e_n)$ est une base de $E$.\\
Soit $(e_1,\ldots,e_n)$ une famille génératrice à $n$ éléments de $E$. En partant
de la famille vide, on peut d'après le théorème de la base à incomplète la
compléter avec des éléments de la famille génératrice $(e_1,\ldots,e_n)$ pour
en faire une base de $E$. Comme cette base a $n$ éléments, on en déduit que
c'est la famille $(e_1,\ldots,e_n)$.
\end{preuve}

\begin{proposition}
Soit $\mathcal{B}\defeq (P_0,\ldots,P_n)$ une famille de polynômes de degrés
\emph{échelonnés}, c'est-à-dire telle que
\[\forall k\in\intere{0}{n} \qsep \deg P_k=k.\]
Alors $\mathcal{B}$ est une base de $\polyK[n]$.
\end{proposition}

% \begin{remarqueUnique}
% \remarque[utile=3] On dit qu'une famille de polynômes $(P_0,\ldots,P_n)$ est
%   \emph{échelonnée} lorsque
%   Une telle famille est une base de $\polyK[n]$.
% %  En particulier
% %   si $\alpha\in\K$, la famille $1$, $(X-\alpha)$,
% %   $(X-\alpha)^2,\ldots,(X-\alpha)^n$ est une base de $\polyK[n]$.
% \end{remarqueUnique}

\begin{exos}
% \exo La famille $\p{0,1,1}$, $\p{1,0,1}$, $\p{1,1,0}$ est libre. C'est donc
%   une base de $\R^3$.
\exo Montrer que l'application de $\polyC[n+1]$ dans $\polyC[n]$ qui à $P$
  associe $P\p{X+1}-P(X)$ est surjective.
  \begin{sol}
  On montre que $\phi(X^k)$ est de degré $k-1$ donc on a $(\phi(X),\ldots,\phi(X^{n+1}))$ qui forme une famille de degrés échelonnés de $\polyC[n]$ donc ça en est une base, en particulier c'est générateur, d'où le résultat.
  \end{sol}
% \exo La famille $P_0,\ldots,P_n$ définie par $P_k=X^{n-k}(1-X)^k$ est une
%   base de $\polyK[n]$.
%\exo Base de Lagrange.
% \exo Soit $n\in\Ns$ et $E$ l'espace des fonctions affines par morceaux sur
%   $\interf{k/n}{(k+1)/n}$. Alors~:
%   \begin{itemize}
%   \item $E$ est de dimension $n+1$~: Construire l'isomorphisme
%     $f\mapsto \p{f(0),\ldots,f(1)}$.
%   \item La famille : $x\mapsto\abs{x-k/n}$ est libre
%   \item Comme elle comporte $n+1$ éléments, c'est une base de $E$.
%   \end{itemize}
\exo Quels sont les sous-espaces vectoriels de $\polyR$ stables par
  dérivation~?
  \begin{sol}
  $\polyK$ et $\polyK[n]$. On fait une analyse-synthèse. On introduit l'ensemble des degrés d'éléments de $E\setminus\set{0}$ et on étudie le cas où cet ensemble est majoré ou non.
  \end{sol}
\end{exos}

\begin{proposition}[utile=3]
Soit $E$ et $F$ deux \Kevs de dimension finie et $f$ une application linéaire
de $E$ dans $F$.
\begin{itemize}
\item Si $f$ est injective et $\dim E=\dim F$, alors $f$ est un isomorphisme.
\item Si $f$ est surjective et $\dim E=\dim F$, alors $f$ est un isomorphisme.
\end{itemize}
Autrement dit, si $\dim E=\dim F$
\[\text{$f$ est injective} \quad\ssi\quad \text{$f$ est bijective}
   \quad\ssi\quad \text{$f$ est surjective.}\]
\end{proposition}
\begin{preuve}
Supposons que $f$ soit injective et que $\dim E=\dim F=n$. Soit
$(e_1,\ldots,e_n)$ une base de $E$. Puisque $f$ est injective on en déduit que
$(f\p{e_1},\ldots,f\p{e_n})$ est libre dans $F$. Or $\dim F=n$ donc cette
famille est une base de $F$. Donc $f$ est un isomorphisme.\\
Supposons que $f$ soit surjective et que $\dim E=\dim F=n$. Soit
$(e_1,\ldots,e_n)$ une base de $E$. Puisque $f$ est surjective
$(f\p{e_1},\ldots,f\p{e_n})$ est génératrice dans $F$. Puisque $\dim F=n$ on
en déduit que c'est une base de $F$. Donc $f$ est un isomorphisme.
\end{preuve}

\begin{remarques}
\remarque[utile=-3] Si $f$ est un endomorphisme d'un \Kev de dimension
  finie $E$, pour montrer que c'est un automorphisme, il suffit de montrer
  qu'il est injectif (ou surjectif).
\remarque[utile=-1] Ce théorème est faux si $E$ et $F$ ne sont pas de même dimension
  ou si ils sont de dimension infinie. Par exemple, les applications linéaires
  \[\dspappli{f}{\R}{\R^2}{x}{\p{x,0}} \et
    \dspappli{g}{\polyK}{\polyK}{P}{XP}\]
  sont injectives mais pas surjectives. De même, les applications linéaires
  \[\dspappli{f}{\R^2}{\R}{\p{x,y}}{x+y} \et
    \dspappli{f}{\polyK}{\polyK}{P}{P'}\]
  sont surjectives mais ne sont pas injectives.
\end{remarques}

\begin{proposition}
Soit $E$ un \Kev de dimension finie et $f\in\Endo{E}$. Alors
\begin{itemize}
\item $f$ est inversible si et seulement si $f$ est inversible à gauche, c'est-à-dire
  si et seulement si il existe $g\in\mathcal{L}(E)$ tel que $g\circ f=\id_E$. Si tel
  est le cas, $g=f^{-1}$ et en particulier $f\circ g=\id_E$.
\item $f$ est inversible si et seulement si $f$ est inversible à droite, c'est-à-dire
  si et seulement si il existe $g\in\mathcal{L}(E)$ tel que $f\circ g=\id_E$. Si tel
  est le cas, $g=f^{-1}$ et en particulier $g\circ f=\id_E$.
\end{itemize}
\end{proposition}


\subsection{Dimension d'un sous-espace vectoriel}
\begin{proposition}[utile=3]
Soit $E$ un \Kev de dimension finie et $A$ un sous-espace vectoriel de $E$.
Alors
\begin{itemize}
\item $A$ est de dimension finie et $\dim A\leq \dim E$.
\item $A=E$ si et seulement si $\dim A=\dim E$.
\end{itemize}
\end{proposition}
\begin{preuve}
Soit $E$ un \Kev de dimension finie $n$ et $A$ un sous-espace vectoriel de $E$.
Montrons que $A$ est de dimension finie. Soit $\mathcal{A}$ l'ensemble des
familles libres de $A$. Puisque toute famille libre de $A$ est une famille
libre de $E$ ces familles ont au plus $n$ éléments. Il existe donc une
famille libre de $A$ de cardinal maximum. Soit $(e_1,\ldots,e_m)$ une telle
famille. Montrons que c'est une base de $A$.
\begin{itemize}
\item Cette famille est libre par définition.
\item Montrons que cette famille est génératrice. Soit $x$ un élément de $A$.
  Puisque $(e_1,\ldots,e_m,x)$ possède $m+1$ éléments, elle n'est pas libre.
  Il existe donc $\lambda_1,\ldots,\lambda_m,\lambda\in\K$ non tous nuls tels
  que~:
  \[\lambda_1 e_1+\cdots+\lambda_m e_m +\lambda x=0\]
  Alors $\lambda\neq 0$. En effet si tel était le cas on aurait
  $\lambda_1 e_1+\cdots+\lambda_m e_m=0$ donc tous les $\lambda_k$ seraient
  nuls car la famille $(e_1,\ldots,e_m)$ est libre. Ce serait absurde car
  la famille $(\lambda_1,\ldots,\lambda_m,\lambda)$ comporte au moins un élément
  non nul. Donc $\lambda\neq 0$ et~:
  \[x=\p{-\frac{\lambda_1}{\lambda}}e_1+\cdots+
      \p{-\frac{\lambda_m}{\lambda}}e_m\]
  En conclusion, la famille $(e_1,\ldots,e_m)$ est génératrice.
\end{itemize}
Donc la famille $(e_1,\ldots,e_m)$ est une base de $A$. $A$ est donc de dimension
finie. De plus $m\leq n$ par définition de $(e_1,\ldots,e_m)$. Donc $A$ est de
dimension finie et $\dim A\leq \dim E$.\\
Montrons que $A=E$ si et seulement si $\dim A=\dim E$. Si $E=A$ alors il est
évident que $E$ et $A$ sont de même dimension. Réciproquement supposons que
$E$ et $A$ soient de même dimension $n$. Soit $(e_1,\ldots,e_n)$ une base de
$A$. Puisque cette famille est libre dans $E$ et possède $n$ éléments, c'est
une base de $E$. Donc $E=A$.
\end{preuve}

\begin{remarqueUnique}
\remarque Si $E$ est un \Kev de dimension $n$ et $A$ un sous-espace vectoriel de $E$
  de dimension $p$, toute base $\mathcal{B}\defeq(e_1,\ldots,e_n)$ de $E$
  telle que $(e_1,\ldots,e_p)$ est une base de $A$ est appelée \emph{base adaptée}
  au sous-espace vectoriel $A$.
\end{remarqueUnique}

\begin{exoUnique}
\exo Soit $E$ un \Kev de dimension finie, $u\in\gl{}{E}$ et $F$ un
  sous-espace vectoriel de $E$. Montrer que si $F$ est stable par $u$,
  alors $u(F)=F$ et $F$ est stable par $u^{-1}$.

\end{exoUnique}

  \begin{sol}
  On sait que $\forall x \in F$, $u(x)\in F$ donc $u(F)\subset F$. Réciproquement, on pose $\dspappli{\phi}{F}{F}{x}{u(x)}$. $\phi$ est bien défini grâce à la stabilité. Elle est linéaire. On veut montrer qu'elle est surjective. Comme $F$ est de dimension finie, c'est équivalent de démontrer que $F$ est injective. On prend $x\in \ker\phi$. Alors $\phi(x)=u(x)=0$. Donc $x\in \ker u=\set{0}$ car $u$ est un isomorphisme. Ainsi, $u$ est bien injective donc surjective. Donc $\forall y \in F$, il existe $x\in F$ tel que $u(x)=\phi(x)=y$.\\
  Montrons que $F$ est stable par $u^{-1}$. Soit $y\in F$, il existe $x\in F$ tel que $u(x)=y$ d'après la première partie. Ainsi, $u^{-1}(y)=u^{-1}(u(x))=x \in F$ ce qui montre bien la stabilité.
  \end{sol}



\begin{proposition}[utile=2]
Dans un \Kev de dimension finie, tout sous-espace vectoriel admet un
supplémentaire.
\end{proposition}
\begin{preuve}
Puisque $A$ est un sous-espace vectoriel de $E$, il est de dimension finie.
Il admet donc une base $(e_1,\ldots,e_m)$. Cette famille est libre dans $E$, donc
d'après le théorème de la base incomplète on en déduit qu'on peut la
compléter en une base $(e_1,\ldots,e_m,\ldots,e_{m+k})$ de $E$. On pose
$B=\vect\p{e_{m+1},\ldots,e_{m+k}}$. Montrons que $B$ est un supplémentaire de
$A$ dans $E$.
\begin{itemize}
\item La somme est directe, c'est-à-dire que $A\cap B=\ens{0}$.\\
  En effet, soit $x\in A\cap B$. Puisque $x\in A$, il existe
  $\lambda_1,\ldots,\lambda_m\in\K$ tels que $x=\lambda_1 e_1+\cdots+
  \lambda_m e_m$. Comme de plus $x\in B$, il existe $\lambda_{m+1},\ldots,
  \lambda{m+k} e_{m+k}$ tels que $x=\lambda_{m+1} e_{m+1}+\cdots+
  \lambda_{m+k}e_{m+k}$. Donc~:
  \[\lambda_1 e_1+\cdots+\lambda_m e_m -\lambda_{m+1} e_{m+1}-\cdots-
    \lambda_{m+k}e_{m+k}=0\]
  Puisque la famille $(e_1,\ldots,e_{m+k})$ est libre, on en déduit que tous les
  $\lambda_k$ sont nuls. En conclusion $x=0$. Donc la somme est directe.
\item $E=A+B$.\\
  En effet, soit $x\in E$. Alors, puisque $e_1,\ldots,e_{m+k}$ est génératrice
  dans $E$, il existe $\lambda_1,\ldots,\lambda_{m+k}$ tels que~:
  \[x=\underbrace{\lambda_1 e_1+\cdots+\lambda_m e_m}_{\in A} +
      \underbrace{\lambda_{m+1} e_{m+1}+\cdots+\lambda_{m+k} e_{m+k}}_{\in B}\]
  Donc $x\in A+B$.
\end{itemize}
En conclusion $E=A\oplus B$, donc $B$ est un supplémentaire de $F$ dans $E$.
\end{preuve}

\begin{exoUnique}
\exo Trouver un supplémentaire dans $\R^3$ de l'espace vectoriel engendré par $\p{1,1,1}$.
  \begin{sol}
  On complète $\p{1,1,1}$ en une base avec des vecteurs d'une base (la canonique par exemple).
  \end{sol}
% \exo Soit $E$ un \Kev de dimension finie et $f,g\in\Endo{E}$. Montrer
%   qu'il existe $h\in\Endo{E}$ tel que $g=h\circ f$ si et seulement si
%   $\ker f\subset \ker g$. 
\end{exoUnique}

\begin{remarqueUnique}
% \remarque Si $E$ est un \Kev de dimension $n$, $A$ un sous-espace vectoriel de $E$
%   de dimension $p$ et $B$ un supplémentaire de $A$ de dimension $q$ (où $p+q=n$),
%   toute base
%   $\mathcal{B}\defeq(a_1,\ldots,a_p,b_1,\ldots,b_q)$ de $E$ telle que
%   $(a_1,\ldots,a_p)$ est une base de $A$ et $(b_1,\ldots,b_q)$ est une base de $B$ est
%   appelée \emph{base adaptée} à la décomposition $E=A\oplus B$.
\remarque Si $E$ est de dimension infinie, on peut montrer que tout sous-espace vectoriel
  de $E$ admet un supplémentaire, mais ce théorème est peu utile, hors programme, difficile à
  démontrer et fait appel à l'axiome du choix.
\end{remarqueUnique}

\subsection{Notion de rang}

\begin{definition}[utile=-3]
Soit $f$ une application linéaire de $E$ dans $F$. Lorsque $\im f$ est de
dimension finie, on appelle \emph{rang} de $f$ et on note $\rg f$ la dimension de
$\im f$.
\end{definition}

\begin{remarques}
\remarque[utile=-2] Si $F$ est de dimension finie, $\im f$ est de dimension finie et
  $\rg f\leq \dim F$. De plus, cette inégalité est une égalité si et seulement
  si $f$ est surjective.
\remarque Si $E$ est de dimension finie, $\im f$ est
  de dimension finie et $\rg f\leq \dim E$. De plus, cette inégalité est une
  égalité si et seulement si $f$ est injective.
\remarque Si $A$ est un sous-espace
  vectoriel de $E$, alors $\dim f(A)\leq \dim A$.
  \begin{sol}
  La deuxième partie de la remarque est moins triviale. Si $E$ est de dimension finie, $E$ admet une base $(e_1,\ldots,e_n)$ et $(f(e_1),\ldots,f(e_n))$ est une famille génératrice de $\im f$. Donc $\im f$ est de dimension finie et $\rg f\leq n=\dim E$.
  \end{sol}
\remarque[utile=-3] Si $u\in\lin{E}{F}$, alors $\rg u=0$ si et seulement si $u=0$.
\end{remarques}

\begin{exoUnique}
\exo Calculer le rang de l'application de $\polyK[n]$ dans lui-même
  qui à $P$ associe $P\p{X+1}-P(X)$.
\end{exoUnique}

  \begin{sol}
  On a déjà vu que $\phi$ est à valeurs dans $\polyK[n-1]$ et que l'image de $(X,X^2,\ldots,X^n)$ formait une famille échelonnée donc était libre. Ainsi, $\im \phi\subset \polyK[n-1]$ et $\dim( \im \phi)\geq n$ donc c'est une égalité et par égalité des dimensions + inclusion, les deux espaces sont égaux. Ainsi, le rang vaut $n$.
  \end{sol}


\begin{proposition}[utile=2]
Soit $E$ et $F$ deux \Kevs de même dimension $n$ et $f$ une application
linéaire de $E$ dans $F$. Alors $f$ est un isomorphisme si et seulement si
$\rg f=n$.
\end{proposition}
\begin{preuve}
Supposons que $f$ soit un isomorphisme. Alors $f$ est surjective donc
$\im f=F$. Donc $\rg f=\dim\im f=\dim F=n$.\\
Réciproquement, supposons que $\rg f=n$. Alors puisque $\dim F=n$, on en
déduit que $\im f=F$. Donc $f$ est surjective. Puisque $E$ et $F$ ont même
dimension, on en déduit que $f$ est un isomorphisme.
\end{preuve}

\begin{proposition}
\begin{itemize}
\item Soit $E, F, G$ des $\K$-espaces vectoriels de dimension finie, $f\in\lin{E}{F}$ et
  $g\in\lin{F}{G}$. Alors
  \[\rg(g\circ f)\leq\rg(g) \quad\et\quad \rg(g\circ f)\leq\rg(f).\]
\item On ne change pas le rang d'une application linéaire si on la compose à gauche ou à
  droite par un isomorphisme.
\end{itemize}
\end{proposition}

\begin{definition}[utile=-3]
On appelle \emph{rang} d'une famille $(x_1,\ldots,x_p)\in E^p$ et on note
$\rg\p{x_1,\ldots,x_p}$ la dimension de $\vect\p{x_1,\ldots,x_p}$.
\end{definition}

\begin{remarques}
\remarque Si $(x_1,\ldots,x_p)$ est une famille d'éléments d'un \Kev $E$ de
  dimension $n$, alors son rang est inférieur ou égal à $n$ et à $p$.
\remarque 
  Si $u\in\mathcal{L}(E,F)$, alors
  $\rg(u(x_1),\ldots,u(x_p))\leq\rg(x_1,\ldots,x_p)$. De plus, si $u$ est injective, alors
  \[\rg(u(x_1),\ldots,u(x_p))=\rg(x_1,\ldots,x_p).\]
\end{remarques}

\begin{sol}
On pose $G=\vect(x_1,\ldots,x_p)$ et $H=\vect(f(x_1),\ldots,f(x_p))$ et on définit $\dspappli{\phi}{G}{H}{x}{f(x)}$. $\phi$ est bien définie et est linéaire. Grâce à l'injectivité de $f$, $\phi$ est injective également. On montre ensuite que $g$ est surjective. Donc $G$ et $H$ ont même dimension.
\end{sol}
  


\section{Calcul de dimension et de rang, hyperplan}

\subsection{Somme de deux sous-espaces vectoriels}

% \begin{proposition}
% Soit $E$ un \Kev et $A$ et $B$ deux sous-espaces vectoriels de dimensions
% finies, supplémentaires dans $E$. Si $a_1,\ldots,a_{n_A}$ est une base de $A$ et
% $b_1,\ldots,b_{n_B}$ est une base de $B$, alors $a_1,\ldots,a_{n_A},b_1,\ldots,
% b_{n_B}$ est une base de $E$. Un telle base est dite adaptée à la décomposition
% $E=A \oplus B$.
% \end{proposition}

\begin{proposition}[utile=2]
Soit $E$ un \Kev de dimension finie et $A$, $B$ deux sous-espaces
vectoriels supplémentaires de $E$. Alors
\[\dim E=\dim A+\dim B.\]
\end{proposition}

\begin{preuve}
On concatène les bases.
\end{preuve}

% \begin{remarques}
% \remarque[utile=-2] Plus généralement, on a démontré que si $E$ est un \Kev admettant deux
%   sous-espaces vectoriels $A$ et $B$ supplémentaires de dimensions finies, alors
%   $E$ est de dimension finie et $\dim E=\dim A+\dim B$.
% \end{remarques}

\begin{proposition}[nom={Formule de \nom{Grassmann}}]
Soit $E$ un \Kev de dimension finie et $A$, $B$ deux sous-espaces vectoriels
de $E$. Alors
\[\dim\p{A+B}=\dim A+\dim B-\dim\p{A\cap B}.\]
\end{proposition}

\begin{preuve}
On pose $F=A+B$.
On pose $B'$ un supplémentaire de $A\cap B$ dans $B$. Montrons que $A\oplus B'=F$.
\begin{itemize}
\item[$\bullet$] Soit $x\in A\cap B'$. Comme $B'\subset B$, $x\in A\cap B\cap B'=\set{0}$ d'après la définition de $B'$. Donc $A\cap B'=\set{0}$ (l'autre inclusion étant triviale).
\item[$\bullet$] Montrons que $A+B'=F$. On a bien $A+B'\subset F$. Maintenant, considérons $x\in F=A+B$. $x$ s'écrit $x=a+b=a+(a'+b')=(a+a')+b' \in A+B'$.
\end{itemize}
Donc $F=A\oplus B'$. Donc $\dim(A+B)=\dim F=\dim A+\dim B'$. Or, $B=(A\cap B)\oplus B'$ donc $\dim B=\dim(A\cap B)+\dim B'$ donc $$\dim\p{A+B}=\dim A+\dim B-\dim\p{A\cap B}.$$
\end{preuve}

\begin{remarqueUnique}
\remarque En particulier, $\dim(A+B)\leq\dim A+\dim B$ et cette inégalité est
  une égalité si et seulement si $A$ et $B$ sont en somme directe.
\end{remarqueUnique}
\begin{exos}
\exo Dans $\R^3$, montrer que l'intersection de deux plans vectoriels est soit un
  plan soit une droite.
  \begin{sol}
  $\dim(A+B)=4-\dim(A\cap B)$. Or $A\subset A+B\subset R^3$ donc $\dim(A+B)\in \set{2,3}$, ce qui donne un plan ou une droite pour $A\cap B$.
  \end{sol}
\end{exos}

\begin{proposition}[utile=3]
Soit $E$ un \Kev de dimension finie et $A$ et $B$ deux sous-espaces vectoriels
de $E$.
\begin{itemize}
\item Si $A\cap B=\ens{0}$ et $\dim E=\dim A+\dim B$, alors $E=A\oplus B$.
\item Si $E=A+B$ et $\dim E=\dim A+\dim B$, alors $E=A\oplus B$.
\end{itemize}
Autrement dit, si $\dim E=\dim A+\dim B$
\[\text{$A$ et $B$ sont en somme directe} \quad\ssi\quad
  E=A\oplus B \quad\ssi\quad E=A+B.\]

\end{proposition}

\begin{preuve}
Immédiat pour les deux points avec l'application de la formule de Grassmann.
\end{preuve}


\subsection{Produit d'espaces vectoriels, espace $\lin{E}{F}$}

% \begin{proposition}
% Soit $E$ et $F$ deux \Kevs de dimensions finies. Soit
% $\mathcal{B}_E=x_1,\ldots,x_{n_E}$ une base de $E$ et
% $\mathcal{B}_F=y_1,\ldots,y_{n_F}$ une base de $F$. Alors
% $\p{x_1,0_F},\ldots,\p{x_{n_E},0_F},\p{0_E,y_1},\ldots,\p{0_E,y_{n_F}}$ est une
% base
% de $E\times F$.
% \end{proposition}

\begin{proposition}[utile=-2]
Soit $E$ et $F$ deux \Kevs de dimension finie. Alors $E\times F$ est de
dimension finie et
\[\dim\p{E\times F}=\dim E+\dim F.\]
En particulier, si $n\in\N$, $E^n$ est de dimension finie et
\[\dim\p{E^n}=n\dim E.\]
\end{proposition}

\begin{preuve}
Avec $(e_1,\ldots,e_n)$ une base de $E$ et $(f_1,\ldots,f_p)$ une base de $F$, on montre que \\$((e_1,0),\ldots,(e_n,0),(0,f_1),\ldots,(0,f_p))$ est une base de $E\times F$ en montrant son caractère libre et générateur.
\end{preuve}

\begin{proposition}[utile=-3]
Soit $E$ et $F$ deux \Kevs de dimension finie. Alors $\lin{E}{F}$ est de
dimension finie et
\[\dim\lin{E}{F}=\dim E\cdot\dim F.\]
En particulier, $\Endo{E}$ est de dimension finie et
\[\dim\Endo{E}=\p{\dim E}^2.\]
\end{proposition}

\begin{preuve}
Soit $(e_1,\ldots,e_n)$ une base de $E$. On pose $$\dspappli{\phi}{\lin{E}{F}}{F^n}{f}{(f(e_1),\ldots,f(e_n)}.$$
On montre que $\phi$ est linéaire. De plus, c'est une bijection car pour tout $(y_1,\ldots,y_n)\in F^n$, il existe un unique $f\in \lin{E}{F}$ tel que $\forall k \in \intere{1}{n}, f(e_k)=y_k$, i.e $\phi(f)=(y_1,\ldots,y_n)$.
Ainsi, $$\dim(\lin{E}{F})=\dim(F^n)=n\dim(F)=\dim(E)\dim(F).$$
\end{preuve}

\begin{proposition}[utile=-1]
Si $E$ est un \Kev de dimension finie, $E^\star$ est de dimension finie et
\[\dim E^\star=\dim E.\]
\end{proposition}


% \begin{definition}
% Soit $E$ un \Kev de dimension $n$ et $\mathcal{B}=e_1,\ldots,e_n$ une base de
% $E$. Alors, pour tout $j\in\intere{1}{n}$, il existe une unique forme linéaire
% $e_j^\star$ telle que~:
% \[\forall k\in\intere{1}{n} \quad e_j^\star\p{e_k}=\delta_{j,k}=
%   \begin{cases}
%   0 &\text{si $k\neq j$}\\
%   1 &\text{si $k=j$}
%   \end{cases}\]
% La forme linéaire $e_j^\star$ associe à tout élément $x\in E$ sa $j$-ième
% coordonnée relativement à la base $\mathcal{B}$. De plus, la famille
% $e_1^\star,\ldots,e_n^\star$ est une base de $E^\star$ notée $\mathcal{B}^\star$
% et appelée base duale de la base $\mathcal{B}$.
% \end{definition}


\subsection{Théorème du rang}

% \begin{proposition}
% Soit $E$, $F$ et $G$ trois \Kevs de dimensions finies.
% \begin{itemize}
% \item Soit $g$ une application linéaire surjective de $E$ dans $F$ et
%   $f$ une application linéaire de $F$ dans $G$. Alors~:
%   \[\rg\p{f\circ g}=\rg f\]
% \item Soit $g$ une application linéaire injective de $F$ dans $G$ et $f$
%   une application linéaire de $E$ dans $F$. Alors~:
%   \[\rg\p{g \circ f}=\rg f\]
% \end{itemize}
% En particulier, on ne change pas le rang d'une application linéaire si on la
% compose à droite ou à gauche par un isomorphisme.
% \end{proposition}

% \begin{proposition}[utile=1]
% On ne change pas le rang d'une application linéaire si on la compose à droite ou à
% gauche par un isomorphisme.
% \end{proposition}


% \begin{preuve}
% Soit $f\in \lin{E}{F}$ avec $E$ et $F$ de dimension finie.
% \begin{itemize}
% \item[$\bullet$] Soit $G$ un $\Kev$ de dimension finie et $g\in \lin{F}{G}$ un isomorphisme. Montrons que $\rg(g\circ f)=\rg (f)$. On pose $$\dspappli{\phi}{\im f}{\im(g\circ f)}{y}{g(y)}.$$
% Tout d'abord, $\phi$ est bien définie : soit $y\in \im f$. Alors, il existe $x\in E$ tel que $y=f(x)$, alors $\phi(y)=g(y)=g(f(x))=(g\circ f)(x)\in \im(g\circ f).$ De plus, $\phi$ est linéaire (évident), on montre qu'elle est injective et surjective.
% Ainsi, c'est un isomorphisme, on a donc égalité entre les dimensions de l'espace de départ et d'arrivée, ce qui correspond au résultat souhaité.

% Remarque : seul l'injectivité de $g$ a servi.
% \item[$\bullet$] Pour la droite, on montre directement que $\im(f\circ u)=\im(f)$ avec $u$ un isomorphisme (on peut remarquer qu'on utilise en fait seulement le caractère surjectif).
% \end{itemize}

% \end{preuve}


% \begin{remarqueUnique}
% \remarque[utile=-1] Plus généralement, on a démontré qu'on ne changeait pas le rang d'une
%   application linéaire en la composant par la droite par une application surjective
%   ou en la composant par la gauche par une application injective.  
% \end{remarqueUnique}

\begin{theoreme}[utile=3, nom=Théorème du rang]
Soit $E$ un \Kev de dimension finie, $F$ un \Kev et $f$ une application
linéaire de $E$ dans $F$. Alors $\im f$ est de dimension finie et
\[\dim E=\dim\ker f+\dim\im f.\]
Autrement dit
\[\rg f=\dim E-\dim\ker f.\]
\end{theoreme}
\begin{preuve}
Puisque $E$ est de dimension finie, $\ker f$ admet un supplémentaire $G$ dans
$E$. Soit $\phi$ l'application~:
\[\dspappli{\phi}{G}{\im f}{x}{f(x)}\]
Alors~:
\begin{itemize}
\item $\phi$ est bien à valeurs dans $\im f$.
\item $\phi$ est {\it injective}.\\
  En effet, montrons que $\ker\phi=\ens{0}$. Soit
  $x\in\ker\phi$. Alors $x\in G$ et $\phi(x)=0$. Donc $f(x)=0$,
  c'est-à-dire $x\in\ker f$. Donc $x\in\ker f\cap G$. Or $G$ est un
  supplémentaire de $\ker f$ dans $E$ donc $x=0$.
\item $\phi$ est {\it surjective}.\\
  En effet, si on se donne $y\in\im f$, il existe
  $x\in E$ tel que $y=f(x)$. Puisque $E=\ker f+G$, il existe $a\in\ker f$ et
  $b\in G$ tels que $x=a+b$. Alors
  \[f(x)=f\p{a+b}=\underbrace{f(a)}_{=0}+f(b)=f(b)\]
  Donc $y=f(b)=\phi(b)$. Donc $y\in\im\phi$.
\end{itemize}
En conclusion $\phi$ est un {\it isomorphisme} de $G$ dans $\im f$. Puisque $G$
est de dimension finie, il en est de même pour $\im f$ et $\dim\im f=\dim G$.
Or, puisque $E=\ker f\oplus G$, on a $\dim E=\dim \ker f+\dim G$. Donc~:
\[\dim E=\dim\ker f+\dim\im f\]
Puisque par définition $\rg f=\dim\im f$, il vient~:
\[\rg f=\dim E-\dim\ker f\]
\end{preuve}

\begin{remarqueUnique}
% \application Si $u_1,\ldots,u_p$ est une famille de vecteurs de $\K^n$, alors
%   l'application~:
%   \[\dspappli{f}{\K^p}{\K^n}{\p{\lambda_1,\ldots,\lambda_p}}{\lambda_1 u_1+
%     \cdots+\lambda_p u_p}\]
%   est linéaire et $\im f=\vect\p{u_1,\ldots,u_p}$. On en déduit donc que
%   $\rg\p{u_1,\ldots,u_p}=\rg f$. Le théorème du rang nous donne~:
%   \[\rg f=\dim \K^p-\dim\ker f=p-\dim\ker f\]
%   On est donc ramené à calculer $\ker f$, c'est-à-dire résoudre un système
%   linéaire à $n$ équations et $p$ inconnues.\\
%   Par exemple, calculons le rang de la famille $u_1$,$u_2$,$u_3$,$u_4$ avec~:
%   \[u_1=\p{1,2,3} \quad u_2=\p{4,5,6} \quad u_3=\p{7,8,9} \quad
%     u_4=\p{-2,-1,0}\]
%   Soient $\lambda_1,\ldots,\lambda_4\in\R$. Alors~:
%   \begin{eqnarray*}
%   &    &\p{\lambda_1,\lambda_2,\lambda_3,\lambda_4}\in\ker f\\
%   &\ssi& \lambda_1 u_1+\lambda_2 u_2+\lambda_3 u_3+\lambda_4 u_4=0\\
%   &\ssi& \syslin{\lambda_1&+4\lambda_2&+7\lambda_3&-2\lambda_4&=&0\cr
%                  2\lambda_1&+5\lambda_2&+8\lambda_3&-\lambda_4&=&0\cr
%                  3\lambda_1&+6\lambda_2&+9\lambda_3& &=&0}\\
%   &\ssi& \syslin{\lambda_1&+4\lambda_2&+7\lambda_3&-2\lambda_4&=&0&\cr
%    &-3\lambda_2&-6\lambda_3&+3\lambda_4&=&0&\quad L_2\gets L_2-2L_1\cr
%    &-6\lambda_2&-12\lambda_3&+6\lambda_4 &=&0&\quad L_3\gets L_3-3L_1}\\
%   &\ssi& \syslin{\lambda_1&+4\lambda_2&+7\lambda_3&-2\lambda_4&=&0&\cr
%    &\lambda_2&+2\lambda_3&-\lambda_4&=&0&\quad L_2\gets L_2/\p{-3}\cr
%    &\lambda_2&+2\lambda_3&-\lambda_4 &=&0&\quad L_3\gets L_3/\p{-6}}\\
%   &\ssi& \syslin{\lambda_1&+4\lambda_2&+7\lambda_3&-2\lambda_4&=&0\cr
%                             &\lambda_2&+2\lambda_3&-\lambda_4&=&0}\\
%   &\ssi& \exists t_1,t_2\in\R \quad 
%          \syslin{\lambda_1&+4\lambda_2&+7\lambda_3&-2\lambda_4&=&0\cr
%                              &\lambda_2&+2\lambda_3&-\lambda_4&=&0\cr
%                              &         &\lambda_3&            &=&t_1\cr
%                              &         &         &\lambda_4   &=&t_2}\\
%   &\ssi& \exists t_1,t_2\in\R \quad
%          \syslin{\lambda_1&=&t_1&-2t_2 \cr
%                  \lambda_2&=&-2t_1&+t_3 \cr
%                  \lambda_3&=&t_1&    \cr
%                  \lambda_4&=&   &t_2 }\\
%   &\ssi& \exists t_1,t_2\in\R \quad \p{\lambda_1,\lambda_2,\lambda_3,\lambda_4}
%          =t_1\p{1,-2,1,0}+t_2\p{-2,1,0,1}
%   \end{eqnarray*}
%   Remarquons que les vecteurs $e_1=\p{1,-2,1,0}$ et $e_2=\p{-2,1,0,1}$ forment
%   une famille libre. En effet si $\mu_1$ et $\mu_2$ sont deux réels tels que
%   $\mu_1 e_1+\mu_2 e_2=0$, alors la troisième composante de cette égalité
%   (respectivement la quatrième) nous donne $\mu_1=0$ (respectivement
%   $\mu_2=0$). Donc $e_1,e_2$ est une base de $\ker f$, donc $\ker f$ est
%   de dimension 2. En conclusion~:
%   \[\rg\p{u_1,u_2,u_3,u_4}=\dim \R^4-\dim\ker f=4-2=2\]
%   En analysant cette méthode, il est possible de montrer que la famille
%   $e_1,\ldots,e_k$ est toujours libre (la méthode utilisée dans cet exemple se
%   généralise aisément). C'est donc une base de $\ker f$,
%   donc $\dim \ker f=k$ et $\rg\p{u_1,\ldots,u_p}=p-k$. Puisque $k$ est
%   le nombre d'inconnues $p$ moins le nombre d'équations restantes $r$ lorsque
%   le système linéaire d'origine est triangularisé, on en déduit que
%   $\rg\p{u_1,\ldots,u_p}=p-\p{p-r}=r$. On aurait donc pu s'arrêter à cet
%   instant et dire que le rang du système était $r$.\\
%   Nous reviendrons sur cette méthode lors du calcul du rang d'une matrice par
%   la méthode du pivot de Gauss.
\remarque Le théorème du rang permet de retrouver le fait que si $f$ est une
  application linéaire d'un \Kev $E$ de dimension finie dans un \Kev $F$ de
  même dimension, alors il suffit de montrer que $f$ est injective ou 
  surjective pour montrer que $f$ est un isomorphisme.
%   En effet, si $f$ est injective $\ker f=\ens{0}$ donc $\dim\ker f=0$. D'après
%   le théorème du rang, $\rg f=\dim E=\dim F$. Donc $\im f$ est de même
%   dimension que $F$ donc $\im f=F$. En conclusion $f$ est surjective. C'est
%   donc un isomorphisme.\\
%   De la même manière, si $f$ est surjective $\im f=F$ donc $\rg f=n$. D'après
%   le théorème du rang $\dim E=\dim\ker f+\rg f$ donc $n=\dim\ker f+n$
%   donc $\dim\ker f=0$. On en déduit que $\ker f=\ens{0}$ puis que $f$ est
%   injective. En conclusion, $f$ est un isomorphisme.
\end{remarqueUnique}


\begin{exos}
\exo Soit $n\in\N$ et
  \[\dspappli{\phi}{\polyR[n]}{\polyR[n]}{P}{P(X+1)-P(X)}\]
  Déterminer le noyau de $\phi$, son rang, puis son image.
\exo Soit $E$ et $F$ deux \Kevs de dimension finie, $f\in\lin{E}{F}$ et $A$
  un sous-espace vectoriel de $E$. Montrer que
  \[\dim\cro{f(A)}=\dim A-\dim\p{A\cap\ker f}\]
  \begin{sol}
  Il suffit de l'appliquer à la restriction de $f$ à $A$ au départ.
  \end{sol}
%%\exo Si $E$ est de dimension finie et $f\in\Endo{E}$, alors
%  \[\dim\p{\im f}=\dim\p{\im f^2}+\dim\p{\ker f\cap\im f}\]
%   En effet, soit $\phi$ l'application~:
%   \[\dspappli{\phi}{\im f}{\im f^2}{x}{f(x)}\]
%   Alors on montre aisément que $\phi$ est bien à valeurs dans $\im f^2$,
%   qu'elle est surjective et que $\ker\phi=\ker f\cap\im f$. En appliquant le
%   théorème du rang à $\phi$, il vient~:
%   \[\dim \im f=\dim\p{\ker f\cap\im f}+\dim\im f^2\]
\exo Soit $E$ un \Kev de dimension finie et $g\in\Endo{E}$. Calculer le rang de
  \[\dspappli{\phi}{\Endo{E}}{\Endo{E}}{f}{g\circ f}\]
  \begin{sol}
  Déjà, on montre que $\phi$ est linéaire. Il nous faut déterminer $\dim(\ker\phi))$ puis appliquer le théorème du rang. 
  $$\ker \phi=\set{f\in \Endo{E} \text{ tel que } \im(f)\subset \ker g}.$$ Pour avoir sa dimension, on définit :
  $$\dspappli{\psi}{\lin{E}{\ker g}}{\ker \phi}{f}{\dspappli{\psi(f)}{E}{E}{x}{f(x)}}.$$
  On montre que $\psi$ est bien définie, linéaire, injective, surjective. On en déduit donc $$\dim E\times\dim(\ker g)=\dim(\lin{E}{\ker g})=\dim(\ker\phi)$$ et ainsi $$\rg(\phi)=(\dim E)^2-\dim E\times\dim(\ker g)=\dim E \times \rg(g).$$
  \end{sol}
\exo Soit $E,F$ et $G$ des espaces vectoriels de dimension finie,
  $f\in\lin{F}{E}$ et $g\in\lin{G}{E}$. Montrer qu'il existe une application
  linéaire $h$ de $F$ dans $G$ telle que $f=g\circ h$ si et seulement si
  $\im f\subset\im g$.  
  \begin{sol}
  Gauche-droite ok. Droite-gauche :
  Soit $G'$ un supplémentaire de $\ker g$ dans $G$. On définit :
  $$\dspappli{\phi}{G'}{\im g}{x}{g(x)}$$ alors $\phi$ est un isomorphisme (cf. preuve précédente et remarque). On pose alors $$\dspappli{h}{F}{G}{x}{\phi^{-1}(f(x))}.$$ On montre que $h$ est bien définie, linéaire et vérifie $f=g\circ h$.
  \end{sol}
\end{exos}

\subsection{Hyperplan}

% \begin{definition}
% Soit $E$ un \Kev. On appelle \emph{hyperplan} de $E$ tout noyau d'une forme linéaire non
% nulle.
% \end{definition}

% \begin{proposition}
% Soit $E$ un \Kev.
% \begin{itemize}
% \item Si $H$ est un hyperplan de $E$ et $D$ est une droite vectorielle non contenue dans
%   $H$, alors
%   \[E=H\oplus D.\]
% \item Si $D$ est une droite vectorielle, tout supplémentaire de $D$ est un hyperplan.
% \end{itemize}
% \end{proposition}

% \begin{proposition}[utile=1]
% Soit $H$ un hyperplan de $E$ et $\phi_0$ une forme linéaire telle que $H=\ker \phi_0$.
% Alors l'ensemble des formes linéaires de $E$ dont le noyau est $H$ est
% \[\Ks\phi_0=\ensim{\lambda\phi_0}{\lambda\in\K^*}.\]  
% \end{proposition}

% \begin{preuve}
% Soit $E$ un \Kev de dimension finie, $H$ un hyperplan de $E$ et $\phi_0$
% une forme linéaire telle que $H=\ker \phi_0$.

% Montrons que $H=\ker\phi\Longleftrightarrow \exists \lambda\in \Ks \text{ tel que } \phi=\lambda\phi_0$.
% Le sens droite-gauche est aisée. Réciproquement, soit $\phi\in E^\star$ telle que $H=\ker\phi$. Considérons $(e_1,\ldots,e_{n-1})$ une base de $H$, qu'on complète avec $e_n$ en une base de $E$. Comme $e_n\notin H$, $\phi_0(e_n)\neq 0$. On pose $\displaystyle \lambda=\frac{\phi(e_n)}{\phi_0(e_n)}$. On montre que $\phi=\lambda\phi_0$ en montrant qu'elles coïncident sur une base (la base $(e_1,\ldots,e_n)$).
% \end{preuve}


\begin{proposition}[utile=1]
Soit $E$ un \Kev de dimension $n$. Un sous-espace vectoriel $H$ de $E$ est
un hyperplan si et seulement si $\dim H=n-1$.
\end{proposition}

\begin{preuve}
Pour montrer l'équivalence entre les deux définitions. La remontée vient du théorème du rang.
Pour la descente, on prend un supplémentaire de $H$ qui est donc de dimension $1$ et on définit la forme linéaire qui à $x$ associe la coordonnée sur une base de ce supplémentaire. On vérifie que la fonction ainsi définie est bien linéaire et que son noyau est bien $H$.
\end{preuve}


\begin{exos}
\exo Dans $\R^3$, donner la dimension du sous-espace vectoriel d'équation $3x+2y-z=0$.
\exo Dans $\polyR[n]$, donner la dimension du sous-espace vectoriel
  \[H\defeq\enstq{P\in\polyR[n]}{\integ{0}{1}{P(t)}{t}=0}.\]
\exo Soit $E$ un \Kev de dimension finie et $\phi_1$, $\phi_2$ deux formes
  linéaires sur $E$. Calculer le rang de
  \[\dspappli{f}{E}{\K^2}{x}{\p{\phi_1(x),\phi_2(x)}}\]
  en fonction de $\phi_1$ et $\phi_2$.
\exo Quelle est la dimension de l'intersection de deux hyperplans~?
\begin{sol}
Soient $H_1$ et $H_2$ deux hyperplans de $E$. On a $\dim(H_1\cap H_2)=\dim H_1+ \dim H_2-\dim(H_1+H_2)\geq 2n-2-n=n-2$. De plus, comme $H_1\cap H_2 \subset H_1$, on a $\dim(H_1\cap H_2)\leq n-1$ donc la dimension vaut $n-1$ ou $n-2$. Supposons que ce soit $n-1$, alors par inclusion et égalité des dimensions, on a $H_1\cap H_2=H_1=H_2$. Réciproquement, si $H_1=H_2$, on a bien la dimension de l'intersection qui vaut $n-1$. Et on a alors la dimension qui vaut $n-2$ si et seulement s'ils sont distincts.
\end{sol}
\end{exos}

\begin{proposition}[utile=1]
Soit $E$ un \Kev de dimension $n$ et $\mathcal{B}\defeq(e_1,\ldots,e_n)$ une base de
$E$. 
\begin{itemize}
\item Si $a_1,\ldots,a_n\in\K$ ne sont pas tous nuls, l'ensemble
  $H$ d'équation
  \[a_1 e_1^\star(x)+\cdots+a_n e_n^\star(x)=0\]
  est un hyperplan de $E$.
\item Réciproquement, si $H$ est un hyperplan de $E$, il existe
  $a_1,\ldots,a_n\in\K$ non tous nuls tels que
  \[a_1 e_1^\star(x)+\cdots+a_n e_n^\star(x)=0\]
  est une équation de $H$. De plus, si $b_1,\ldots,b_n\in\K$, alors
  \[b_1 e_1^\star(x)+\cdots+b_n e_n^\star(x)=0\]
  est une équation de $H$ si et seulement si il existe $\alpha\in\K^*$ tel que
  \[\forall k\in\intere{1}{n} \qsep b_k=\alpha a_k.\]
\end{itemize}
\end{proposition}

\begin{preuve}
En considérant $(e_1,\ldots,e_n)$ une base de $E$. on définit $$\dspappli{\phi}{E}{\K}{x=\sum_{i=1}^nx_k e_k}{\sum_{i=1}^n\lambda_k x_k}.$$
On montre que $\phi$ est linéaire et non nulle (pour la non nullité, comme les $\lambda_i$ sont non tous nuls, on a $\phi(e_{i_0})=\lambda_{i_0}\neq 0$ pour un certain $i_0$).


Réciproquement, soit $H$ un hyperplan de $E$. Alors $H$ est le noyau d'une forme linéaire non nulle $\phi$. En posant $\lambda_k=\phi(e_k)$, ils sont non tous nuls (sinon, $\phi=0$). Et si $x=\displaystyle \sum_{k=1}^nx_ke_k$, on a $x\in H\Longleftrightarrow \phi(x)=0\Longleftrightarrow \displaystyle \sum_{k=1}^n\lambda_k x_k=0$, ce qui fournit bien une équation de $H$.


Soit $\mu_1,\ldots, mu_n \in \K$. On définit $$\dspappli{\psi}{E}{\K}{x=\sum_{i=1}^nx_k e_k}{\sum_{i=1}^n\mu_k x_k}.$$ L'ensemble d'équation $\sum_{i=1}^n\mu_k x_k=0$ est $\ker \psi$.
Et \begin{eqnarray*}
H=\ker \psi &\Longleftrightarrow & \exists \alpha\in \Ks, \psi=\lambda \phi\\
&\Longleftrightarrow & \exists \alpha\in \Ks, \forall k \in \intere{1}{n}, \psi(e_k)=\alpha\phi(e_k)\\
&\Longleftrightarrow & \exists \alpha\in \Ks, \forall k \in \intere{1}{n}, u_k=\alpha\lambda_k.
\end{eqnarray*}
\end{preuve}

\begin{proposition}[utile=1]
Soit $E$ un \Kev de dimension $n$.
\begin{itemize}
\item L'intersection de $p$ hyperplans est de dimension supérieure ou égale à $n-p$.
\item Si $p\in \intere{0}{n}$, tout sous-espace vectoriel de dimension $n-p$ est l'intersection de $p$ hyperplans.
\end{itemize}
\end{proposition}

\begin{preuve}
\begin{itemize}
\item Montrons par récurrence sur $p$ que l'intersection de $p$ hyperplans est de dimension supérieure ou égale à $n-p$.
La propriété est vraie pour $p=1$. Supposons la vraie pour $p$. D'après la formule de Grassmann $\dim(H_1\cap\ldots\cap H_{p+1})=\dim(H_1\cap\ldots\cap H_{p})+\dim H_{p+1}-\dim(H_1\cap\ldots\cap H_{p}+H_{p+1})\geq \dim(H_1\cap\ldots\cap H_{p})+n-1-n\underbrace{\geq}_{HR}n-p-1=n-(p+1).$
\item Montrons par récurrence sur $p\in \intere{1}{n}$ que tout sous-espace vectoriel de dimension $n-p$ est l'intersection de $p$ hyperplans. Vrai pour $p=1$. Supposons la propriété vraie pour $p\in \intere{1}{n-1}$ et montrons-là pour $p+1$. Soit $F$ un sous-espace vectoriel de dimension $n-(p+1)$. $F\neq E$ donc il existe $a\in E\setminus F$. Posons $D=\vect(a)$ et $G=F+D$. $a$ n'appartenant pas à $F$, $F$ et $D$ sont en somme directe et $G$ est donc de dimension $n-p$. $G$ est donc d'après l'hypothèse de récurrence l'intersection de $p$ hyperplans $H_1,\ldots, H_p$. Soit $G'$ un supplémentaire de $G$ dans $E$. On a $\dim(G)=p$. Posons $H_{p+1}=F+G'$. Cette somme est directe car si $x\in F\cap G'$, $x\in G\cap G'=\set{0}$. Ainsi, $\dim(H_{p+1})=\dim F+\dim G'=n-(p+1)+p=n-1$, donc $H_{p+1}$ est un hyperplan. Montrons pour finir que $F=G\cap H_{p+1}$, ainsi, nous aurons $F=H_1\cap\ldots\cap H_p\cap H_{p+1}$, ce qui correspond au résultat souhaité.

$F$ est contenu dans $G$ et dans $H_{p+1}$ par construction de ces deux sous-espaces. Ainsi, $F\subset G\cap H_{p+1}$. Réciproquement, soit $x\in G\cap H_{p+1}$. Comme $x\in H_{p+1}=F+G'$, il s'écrit $x=x_1+x_2$ avec $(x_1,x_2)\in F\times G'$. $x$ et $x_1$ sont deux éléments de $G$ donc $x-x_1=x_2\in G$. Mais $x_2\in G'$. Donc $x_2\in G\cap G'=\set{0}$, d'où $x_2=0$ et finalement, $x=x_1\in F$, ce qui achève de démontrer $G\cap H_{p+1}\subset F$.
\end{itemize}

\end{preuve}

\section{Sous-espace affine}


Soit $E$ un \Kev. Nous avons vu que les éléments de $E$ sont naturellement des vecteurs.
Dans cette partie, nous allons voir les éléments de $E$ comme des points. Afin de marquer
cette différence de point de vue, nous utiliserons dans cette partie les lettres
$a$, $b$, $c$, $d$ du début de l'alphabet pour désigner les éléments de $E$ que nous
considérerons comme des points et les lettres $x$, $y$, $z$ de la fin de l'alphabet pour
désigner les éléments de $E$ que nous considérerons comme des vecteurs.
\vspace{2ex}
\begin{definition}
Soit $a,b\in E$. On définit le vecteur $\ve{ab}\in E$ par
\[\ve{ab}\defeq b-a.\]
\end{definition}

\begin{proposition}
\begin{itemize}
\item Quel que soit $a\in E$, l'application
  \[\dspappli{\phi}{E}{E}{b}{\ve{ab}}\]
  est bijective.
\item Quels que soient $a, b, c\in E$
  \[\ve{ab}+\ve{bc}=\ve{ac}.\]
\end{itemize}
\end{proposition}

\begin{remarqueUnique}
\remarque Soit $a\in E$ et $x\in E$. Alors $b\defeq a+x$ est l'unique élément de $E$ tel
  que $\ve{ab}=x$.
\end{remarqueUnique}

\begin{definition}
Soit $x\in E$. On appelle translation de vecteur $x$ l'application
\[\dspappli{\tau_x}{E}{E}{a}{a+x.}\]
\end{definition}

\begin{remarques}
\remarque Si $x,y\in E$, alors $\tau_x \circ \tau_y=\tau_{x+y}$. 
\remarque Si $x\in E$, l'application $\tau_x$ est linéaire si et seulement si $x=0$.
  Dans ce cas, $\tau_x=\id$.
\end{remarques}



\begin{definition}
On dit qu'une partie $\mathcal{F}$ de $E$ est un \emph{sous-espace affine} de $E$
lorsqu'il existe $a\in E$ et un sous-espace vectoriel $F$ de $E$ tel que
\[\mathcal{F}=\ensim{a+x}{x\in F}.\]
On écrit alors $\mathcal{F}=a+F$. De plus $a\in\mathcal{F}$ et
\[F=\ensim{\ve{bc}}{b, c\in\mathcal{F}}.\]
On dit que $F$ est la \emph{direction} de $\mathcal{F}$.
\end{definition}

\begin{remarques}
\remarque En particulier, un sous-espace affine est non vide.
\remarque Si $\mathcal{F}$ est un sous-espace affine de direction $F$, alors, quel que
  soit $a\in\mathcal{F}$, $\mathcal{F}=a+F$.
\remarque Un sous-espace vectoriel est un sous-espace affine. Réciproquement, un sous-espace
  affine est un sous-espace vectoriel si et seulement si il contient 0.
\end{remarques}

\begin{exos}
\exo Déterminer l'ensemble des solutions dans $\R^3$ de
  \[\syslin{x&+2y&+3z&=1&\cr
           2x&+3y&+4z&=1&\cr
           3x&+4y&+5z&=1&}\]
  et l'exprimer comme un sous-espace affine de $\R^3$.
\begin{sol}
$$\mathcal{S}=SP+\vect((1,-2,1)).$$ C'est une droite affine de $\R^3$.
\end{sol}
\exo Résoudre sur $\R$ l'équation différentielle suivante et exprimer son ensemble de
  solutions comme un sous-espace affine de $\mathcal{D}(\R,\R)$.
  \[y'+2y=x^2-2x+3.\]

\begin{sol}
$y(x)=ce^{-2x}+(2x^2-6x+9)/4$
\end{sol}
\end{exos}

\begin{proposition}
Soit $u\in\lin{E}{F}$ et $b\in F$. Alors l'ensemble
\[\mathcal{S}\defeq\enstq{a\in E}{u(a)=b}\]
de solutions de l'équation $u(a)=b$ est
\begin{itemize}
\item Soit vide. 
\item Soit un sous-espace affine de $E$ de direction $\ker u$.
\end{itemize}
\end{proposition}

\begin{remarqueUnique}
\remarque En particulier, si $a_0\in E$ est une solution de l'équation $u(a)=b$, l'ensemble
  des solutions de cette équation est
  \[a_0 + \ker u = \ensim{a_0+x}{x\in\ker u}.\]
\end{remarqueUnique}

\begin{proposition}
Soit $(\mathcal{F}_i)_{i\in I}$ une famille de sous-espaces affines de $E$ et
$(F_i)_{i\in I}$ la famille de leurs directions respectives. On pose
\[\mathcal{F}\defeq\bigcap_{i\in I} \mathcal{F}_i \quad\et\quad F\defeq\bigcap_{i\in I} F_i.\]
Alors
\begin{itemize}
\item Soit $\mathcal{F}$ est vide. Dans ce cas, ce n'est pas un sous-espace affine.
\item Soit $\mathcal{F}$ est non vide. Dans ce cas, c'est un sous-espace affine de
  $E$ de direction $F$.
\end{itemize}
\end{proposition}

% \begin{exoUnique}
% \exo Soit $\mathcal{A}$ et $\mathcal{B}$ deus sous-espaces affines de $E$ de directions
%   respectives $A$ et $B$. On suppose que $A$ et $B$ sont supplémentaires dans $E$. Montrer
%   que $\mathcal{A}\cap\mathcal{B}$ contient excatement un point.
% \end{exoUnique}


  
% \begin{definition}
% Soit $\mathcal{F}_1$ et $\mathcal{F}_2$ deux sous-espaces affines de
% $E$ de directions respectives $F_1$ et $F_2$. On dit que
% \begin{itemize}
% \item $\mathcal{F}_1$ est parallèle à $\mathcal{F}_2$ lorsque
%   $F_1\subset F_2$.
% \item $\mathcal{F}_1$ et $\mathcal{F}_2$ sont parallèles lorsque
%   $F_1=F_2$.
% \end{itemize}
% \end{definition}

% \begin{remarqueUnique}
% \item Une droite peut être parallèle à un plan, mais pas l'inverse.
% \end{remarqueUnique}

% \begin{proposition}
% Soit $\mathcal{F}_1$ et $\mathcal{F}_2$ deux sous-espaces affines de
% $\mathcal{E}$.
% \begin{itemize}
% \item Si $\mathcal{F}_1$ est parallèle à $\mathcal{F}_2$, alors
%   $\mathcal{F}_1\cap\mathcal{F}_2=\emptyset$ ou
%   $\mathcal{F}_1\subset\mathcal{F}_2$.
% \item Si $\mathcal{F}_1$ et $\mathcal{F}_2$ sont parallèles
%   $\mathcal{F}_1\cap\mathcal{F}_2=\emptyset$ ou 
%   $\mathcal{F}_1=\mathcal{F}_2$.
% \end{itemize}
% \end{proposition}

% \begin{preuve}
% \begin{itemize}
% \item Si $\mathcal{F}_1$ est parallèle à $\mathcal{F}_2$ alors ou bien 
%   $\mathcal{F}_1\cap\mathcal{F}_2=\emptyset$ ou bien il existe $A \in \mathcal{F}_1\cap\mathcal{F}_2$ auquel cas $\mathcal{F}_1=A+\ve{\mathcal{F}_1} \subset A+\ve{\mathcal{F}_2}=\mathcal{F}_2$.
% \item D'après le point 1 et la définition d'être tous deux parallèles.
% \end{itemize}
% \end{preuve}

% \begin{proposition}
% Soit $\mathcal{F}_1$ et $\mathcal{F}_2$ deux sous-espaces affines de
% $\mathcal{E}$. Alors soit $\mathcal{F}_1\cap\mathcal{F}_2=\emptyset$, soit 
% $\mathcal{F}_1\cap\mathcal{F}_2$ est un sous-espace affine de direction
% $\ve{\mathcal{F}_1}\cap \ve{\mathcal{F}_2}$.
% \end{proposition}

% \begin{preuve}
% Supposons $\mathcal{F}_1\cap\mathcal{F}_2\neq \emptyset$. On peut alors fixer $A$ dans l'intersection. Montrons que $$\mathcal{F}_1\cap\mathcal{F}_2=A+\ve{\mathcal{F}_1}\cap\ve{\mathcal{F}_2}.$$
% On raisonne par double inclusion :
% \begin{itemize}
% \item[$\bullet$] Soit $B\in \mathcal{F}_1\cap\mathcal{F}_2$, comme il est dans chaque sous espace-affine, on peut écrire $$B=A+\ve{u} \text{ avec } \ve{u} \in \ve{\mathcal{F}_1} \et B=A+\ve{v} \text{ avec } \ve{v} \in \ve{\mathcal{F}_2}.$$
% Donc $\ve{u}=\ve{v}$ et $B$ s'écrit donc comme $A$+ un élément de $\ve{\mathcal{F}_1}\cap\ve{\mathcal{F}_2}$.
% \item[$\bullet$] La réciproque se voit encore plus clairement à la main.
% \end{itemize}
% \end{preuve}


% \begin{proposition}
% Soit $\mathcal{F}$ et $\mathcal{G}$ deux sous-espaces affines de $E$ de directions
% respectives $F$ et $G$. Si $E=F\oplus G$, alors $\mathcal{F}\cap\mathcal{G}$ est réduit à
% un unique point.
% \end{proposition}

% \begin{preuve}
% Commençons par montrer que $\mathcal{F}\cap\mathcal{G}$ est non vide. On considère $A\in \mathcal{F}$ et $B\in \mathcal{G}$. $\ve{AB}$ en tant qu'élément de $E$ s'écrit $\ve{u}+\ve{v}$ avec $(\ve{u},\ve{v})\in \ve{\mathcal{F}}\times\ve{\mathcal{G}}$.\\
% Posons alors $C=A+\ve{u} \in \mathcal{F}$. On a aussi $C=B-\ve{AB}+\ve{u}=B-\ve{v} \in \ve{\mathcal{G}}$. Ainsi, $\mathcal{F}\cap\mathcal{G}\neq \emptyset$.\\

% Mais alors, d'après la proposition précédente, $\mathcal{F}\cap\mathcal{G}$ est donc un sous-espace affine de $E$, de direction $\ve{\mathcal{F}}\cap\ve{\mathcal{G}}={\ve{0}}$ donc il s'agit d'un singleton.
% \end{preuve}
%END_BOOK

\end{document}

