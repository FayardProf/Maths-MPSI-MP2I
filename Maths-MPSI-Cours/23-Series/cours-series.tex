\documentclass{magnolia}

\magtex{tex_driver={pdftex},
        tex_packages={epigraph,xypic}}
\magfiche{document_nom={Cours sur les séries},
          auteur_nom={François Fayard},
          auteur_mail={fayard.prof@gmail.com}}
\magcours{cours_matiere={maths},
          cours_niveau={mpsi},
          cours_chapitre_numero={21},
          cours_chapitre={Séries}}
\magmisenpage{misenpage_presentation={tikzvelvia},
          misenpage_format={a4},
          misenpage_nbcolonnes={1},
          misenpage_preuve={non},
          misenpage_sol={non}}
\magmisenpage{}
\maglieudiff{}
\magprocess

\begin{document}

%BEGIN_BOOK


\magtoc
\vspace{2ex}
Dans ce chapitre, $\K$ désignera l'un des corps $\R$ ou $\C$.

\section{Série}
\subsection{Série}

\begin{definition}
Soit $(u_n)$ une suite d'éléments de $\K$. On appelle \emph{série de terme général $u_n$}
et on note $\sum u_n$ la suite $(S_n)$ définie par
\[\forall n\in\N\qsep S_n\defeq \sum_{k=0}^n u_k.\]
Le terme $S_n$ est appelé \emph{somme partielle d'indice $n$} de la série.
\end{definition}

% \begin{remarqueUnique}
% \remarque Soit $(v_n)$ une suite à valeurs dans $\RouC$. On définit la suite $(u_n)$ par
%   \[\forall n\in\N\qsep u_n\defeq v_{n+1} - v_n.\]
%   Alors
%   \[\forall n\in\N\qsep v_n=v_0 + \sum_{k=0}^{n-1} u_k.\]
%   L'étude de la suite $(v_n)$ se ramène donc à l'étude de la série $\sum u_n$. On dit que
%   la suite $(u_n)$ est la suite \emph{dérivée} de la suite $(v_n)$. Il sera parfois utile
%   d'exprimer une suite à l'aide d'une série afin d'appliquer des techniques propres à ces
%   dernières.
% \end{remarqueUnique}

\begin{definition}
On dit qu'une série $\sum u_n$ \emph{converge} lorsque la suite de ses sommes partielles
converge. Si c'est le cas, sa limite $l\in\K$ est appelée \emph{somme} de la série. On
la note
\[\sum_{n=0}^{+\infty} u_n.\]
Dans le cas contraire, on dit qu'elle \emph{diverge}.
\end{definition}

\begin{remarques}
\remarque Si on change un nombre fini de termes de la suite $(u_n)$, on ne change pas
  la nature de la série $\sum u_n$. Par contre, si elle converge, cela peut changer sa somme.
\remarque Soit $\sum u_n$ une série convergente. Alors, quel que soit $n\in\N$
  \[\sum_{k=0}^{+\infty} u_k=\sum_{k=0}^n u_k + \sum_{k=n+1}^{+\infty} u_k.\]
\end{remarques}

% \begin{remarques}
% \remarque Lorsque la série $\sum u_n$ converge vers $l$, on définit la suite $(r_n)$ par
%   \[\forall n\in\N\qsep r_n\defeq l-s_n.\]
%   Quel que soit $n\in\N$, on a alors
%   \[\sum_{k=n+1}^m u_k \tendvers{m}{+\infty} r_n\]
%   et on note
%   \[r_n=\sum_{k=n+1}^{+\infty} u_k.\]
% \end{remarques}

\begin{exoUnique}
\exo Montrer que la série
  \[\sum_{n\geq 1} \frac{1}{n(n+1)}\]
  converge et calculer sa somme.
\end{exoUnique}


\begin{definition}
Soit $\sum u_n$ une série convergente. On définit la suite $(R_n)$ par
\begin{eqnarray*}
\forall n\in\N\qsep R_n &\defeq& \sum_{k=0}^{+\infty} u_k - \sum_{k=0}^n u_k\\
&=& \sum_{k=n+1}^{+\infty} u_k.
\end{eqnarray*}
Le terme $R_n$ est appelé \emph{reste d'indice $n$} de la série.
\end{definition}

\begin{remarqueUnique}
\remarque La suite $(R_n)$ des restes converge vers 0.
\end{remarqueUnique}

\begin{proposition}
Soit $\sum u_n$ une série. Si elle est convergente, alors
\[u_n\tendvers{n}{+\infty}0.\]
Par contraposée, si la suite $(u_n)$ ne converge pas vers 0, la série $\sum u_n$ est
divergente. On dit qu'elle diverge \emph{grossièrement}.
\end{proposition}

\begin{remarqueUnique}
\remarque Il est possible qu'une série diverge sans diverger grossièrement.
  Par exemple, si $(u_n)$ est la suite définie par
  \[\forall n\in\N\qsep u_n\defeq\sqrt{n+1}-\sqrt{n}\]
  alors, la série associée diverge alors que la suite $(u_n)$ converge vers 0.
\end{remarqueUnique}

\begin{proposition}
La suite $(u_n)$ et la série $\sum (u_{n+1}-u_n)$ sont de même nature.
\end{proposition}

\begin{remarqueUnique}
\remarque Si $(u_n)$ est une suite, la suite de terme général $u_{n+1}-u_n$ est appelée
  \emph{dérivée} de la suite $(u_n)$. Par sommation télescopique
  \[\forall n\in\N\qsep u_n=u_0 + \sum_{k=0}^{n-1} (u_{k+1}-u_k).\]
  L'étude de la suite $(u_n)$ se ramène donc à l'étude de la série
  $\sum (u_{n+1}-u_n)$.
\end{remarqueUnique}

\begin{proposition}
Soit $\sum u_n$ et $\sum v_n$ deux séries convergentes et
$\lambda,\mu\in\K$. Alors, la série \mbox{$\sum (\lambda u_n+\mu v_n)$} est convergente et
\[\sum_{n=0}^{+\infty} \p{\lambda u_n+\mu v_n}=\lambda\sum_{n=0}^{+\infty} u_n
  +\mu\sum_{n=0}^{+\infty} v_n.\]
\end{proposition}

\begin{remarques}
\remarque Attention, il est possible que la série $\sum (\lambda u_n+\mu v_n)$ soit convergente
  sans que les séries $\sum u_n$ et $\sum v_n$ le soient. Avant d'écrire
  \[\sum_{n=0}^{+\infty} \p{\lambda u_n+\mu v_n}=\lambda\sum_{n=0}^{+\infty} u_n
    +\mu\sum_{n=0}^{+\infty} v_n\]
  il faudra donc toujours vérifier que les séries $\sum u_n$ et $\sum v_n$ soient convergentes.
  Un tel oubli pourrait conduire à écrire des horreurs comme
  \begin{eqnarray*}
  0 = \sum_{n=0}^{+\infty} 0 &=& \sum_{n=0}^{+\infty} (1+(-1))\\
  &=& \sum_{n=0}^{+\infty} 1 + \sum_{n=0}^{+\infty} (-1).
  \end{eqnarray*}
  La dernière expression n'a en effet aucun sens car les deux séries sont grossièrement divergentes.
\remarque Si $\sum u_n$ est convergente et $\sum v_n$ est divergente, alors
  $\sum (u_n+v_n)$ est divergente.
\end{remarques}


% \begin{exoUnique}
% \exo Déterminer les $z\in\C$ pour lesquels la série
%   \[\sum_{n\in\N} z^n\]
%   converge. Lorsque c'est le cas, calculer sa somme.
% \end{exoUnique}

\begin{proposition}
Soit $z\in\C$. Alors la série
\[\sum z^n\]
converge si et seulement si $\abs{z}<1$. Si tel est le cas, sa somme est
\[\sum_{n=0}^{+\infty} z^n = \frac{1}{1-z}.\]
\end{proposition}

\begin{exoUnique}
\exo Soit $(F_n)$ la suite de Fibonacci définie par
  \[F_0\defeq 0, \quad F_1\defeq 1, \et \forall n\in\N \qsep F_{n+2}\defeq F_{n+1}+F_n.\]
  Démontrer l'existence puis calculer
  \[\sum_{n=0}^{+\infty} \frac{F_n}{2^n}.\]
\begin{sol}
On trouve 2.
\end{sol}
\end{exoUnique}

% \begin{exoUnique}
% \exo Pour tout $x\in\intero{-1}{1}$, calculer
%   \[\sum (n+1)x^n.\]
%   On pourra conjecturer le résultat avant de le prouver rigoureusement.
% \end{exoUnique}

\begin{proposition}
Soit $z\in\C$. Alors la série
\[\sum \frac{z^n}{n!}\]
est convergente et
\[\sum_{n=0}^{+\infty} \frac{z^n}{n!}=\e^z.\]
\end{proposition}

\begin{exoUnique}
\exo Établir l'existence et calculer
    \[\sum_{n=0}^{+\infty} \frac{n^2+1}{n!}.\]
\begin{sol}
On trouve $3\e$.
\end{sol}
\end{exoUnique}



\subsection{Série à termes positifs}

\begin{definition}
On dit qu'une série réelle $\sum u_n$ est à termes positifs lorsque
\[\forall n\in\N\qsep u_n\geq 0.\]
\end{definition}

\begin{remarques}
\remarque La suite des sommes partielles d'une série à termes positifs est croissante.
  Réciproquement, si $(u_n)$ est une suite croissante, sa suite dérivée $(u_{n+1}-u_n)$ est
  une suite à termes positifs.
\remarque Puisque la convergence d'une série ne dépend pas de ses premiers termes, les
  théorèmes de convergence sur les séries à termes positifs s'appliquent même si la série
  est à termes positifs à partir d'un certain rang. Bien entendu, des théorèmes similaires
  aux théorèmes que nous allons énoncer existent pour les séries à termes négatifs.
  Les théorèmes que nous allons énoncer dans cette section sont donc utiles pour étudier
  les séries de signe constant à partir d'un certain rang.
\end{remarques}

\begin{proposition}
Une série à termes positifs converge si et seulement si la suite de ses sommes partielles
est majorée.
\end{proposition}

\begin{proposition}[nom={Série de \nom{Riemann}}]
Soit $\alpha\in\R$. Alors, la série
\[\sum \frac{1}{n^\alpha}\]
est convergente si et seulement si $\alpha > 1$.
\end{proposition}

\begin{remarqueUnique}
\remarque En particulier, la série harmonique $(H_n)$ définie par
  \[\forall n\in\N\qsep H_n\defeq\sum_{k=1}^n \frac{1}{k}\]
  diverge. Une comparaison série intégrale permet de montrer que
  $H_n\equi{n}{+\infty} \ln n$.
\end{remarqueUnique}

\begin{exos}
\exo Montrer que la série
  \[\sum \frac{1}{n^2}\]
  converge et donner un équivalent de son reste.
\exo Prouver la divergence et donner un équivalent des sommes partielles de la série
  \[\sum \frac{1}{n \ln n}.\]
\end{exos}

\begin{proposition}
Soit $\sum u_n$ et $\sum v_n$ deux séries à termes positifs telles que
\[\forall n\in\N\qsep u_n\leq v_n.\]
\begin{itemize}
\item Si la série $\sum v_n$ converge, alors il en est de même pour $\sum u_n$.
\item Si la série $\sum u_n$ diverge, alors il en est de même pour $\sum v_n$.
\end{itemize}
\end{proposition}

\begin{exos}
\exo Donner la nature des séries
  \[\sum\frac{\sin^2 n}{n^3},\qquad\sum\frac{\cos^2\p{\frac{n\pi}{3}}}{\sqrt{n}}.\]
\exo Soit $r\in\RP$. Déterminer la nature de la série
  \[\sum \frac{r^n}{n}\]
  en fonction de $r$.
\end{exos}

\begin{proposition}
Soit $\sum u_n$ et $\sum v_n$ deux séries à termes positifs. On suppose que
\[u_n=\grando{n}{+\infty}{v_n}\]
et que la série $\sum v_n$ est convergente. Alors la série $\sum u_n$ est convergente.
\end{proposition}

\begin{remarqueUnique}
\remarque En particulier, pour des séries à termes positifs, si
\[u_n=\petito{n}{+\infty}{v_n}\]
et si $\sum v_n$ est convergente, alors $\sum u_n$ est convergente. 
\end{remarqueUnique}

\begin{exoUnique}
\exo Déterminer la nature de la série
  \[\sum \frac{\ln n}{n\sqrt{n}}.\]
\end{exoUnique}

\begin{proposition}
Soit $\sum u_n$ et $\sum v_n$ deux séries réelles. On suppose que $\sum v_n$ est à termes
positifs et que
\[u_n\equi{n}{+\infty} v_n.\]
Alors la série $\sum u_n$ est à termes positifs à partir d'un certain rang et les deux séries
sont de même nature.
\end{proposition}

\begin{exos}
\exo Établir la nature des séries suivantes
  \[\sum \frac{1}{3n+1}, \qquad\sum \tan\p{\frac{1}{n}}, \qquad \sum \frac{1}{2^n-n}.\]
\exo Donner la nature des séries
  \[\sum \ln\p{\tan\frac{\pi n}{4n+1}},\qquad \sum \cro{\p{\tanh n}^{\frac{1}{n}}-1}.\]
\exo Montrer qu'il existe $\gamma\in\R$ tel que
  \[\sum_{k=1}^n \frac 1k = \ln n+ \gamma + \petito{n}{+\infty}{1}.\]
  La constante $\gamma\approx 0.577$ est appelée constante d'\nom{Euler}.
\end{exos}


\subsection{Série absolument convergente}

\begin{definition}
Soit $\sum u_n$ une série d'éléments de $\K$. Si la série à termes positifs
\[\sum \abs{u_n}\]
converge, alors la série $\sum u_n$ converge. On dit dans ce cas que la série
$\sum u_n$ est \emph{absolument convergente}.
\end{definition}

\begin{remarqueUnique}
\remarque Une série convergente qui n'est pas absolument convergente est appelée série
  \emph{semi-convergente}.
\end{remarqueUnique}

\begin{exoUnique}
\exo Montrer que la série
  \[\sum\frac{\sin n}{n\sqrt{n}}\]
  est convergente.
\end{exoUnique}


\begin{proposition}
Soit $\sum u_n$ une série d'éléments de $\K$ et $\sum v_n$ une série à termes positifs
telle que
\[u_n=\grando{n}{+\infty}{v_n}.\]
Si $\sum v_n$ est convergente, alors $\sum u_n$ est absolument convergente, donc convergente.
\end{proposition}

\begin{remarqueUnique}
\remarque En particulier, si $(v_n)$ est une suite positive, si
  \[u_n=\petito{n}{+\infty}{v_n}\]
  et si $\sum v_n$ est convergente, alors $\sum u_n$ est absolument convergente, donc
  convergente.
\end{remarqueUnique}

\begin{proposition}[nom={Règle de d'\nom{Alembert}}]
Soit $\sum u_n$ une série d'éléments de $\K$ ne s'annulant pas. On suppose que
\[\abs{\frac{u_{n+1}}{u_n}}\tendvers{n}{+\infty} \omega\in\RP\cup\ens{+\infty}.\]
Alors
\begin{itemize}
\item Si $\omega<1$, la série $\sum u_n$ est absolument convergente.
\item Si $\omega>1$, la série $\sum u_n$ est grossièrement divergente.
\end{itemize}
\end{proposition}

\begin{remarqueUnique}
\remarque Si $\omega=1$, la règle de d'\nom{Alembert} ne permet pas de conclure. Dans ce cas, il peut être
  intéressant d'effectuer une comparaison avec une série de \nom{Riemann}.
\end{remarqueUnique}

\begin{exos}
\exo Déterminer la nature de la série
  \[\sum \frac{n^4}{3^n}.\]
\exo Soit $z\in\C$. Retrouver le fait que la série
  \[\sum \frac{z^n}{n!}\]
  est convergente.
  \exo Soit $a,b\in\R$. Donner une condition nécessaire et suffisante sur $a$ et $b$ pour que la série
  \[\sum \cro{\frac{n^2+2}{n^2+2n+1}-\p{a+\frac{b}{n}}}\]
  soit convergente.
\end{exos}

\subsection{Série semi-convergente}

\begin{theoreme}[nom={Théorème des séries alternées}]
Soit $(u_n)$ une suite à termes positifs, décroissante et convergeant vers 0.
Alors la série
\[\sum (-1)^n u_n\]
converge. De plus, si $(R_n)$ est la suite des restes définie par
\[\forall n\in\N\qsep R_n\defeq\sum_{k=n+1}^{+\infty} (-1)^k u_k\]
alors, pour tout $n\in\N$, $R_n$ est du signe de $(-1)^{n+1}$ et $\abs{R_n}\leq u_{n+1}$.
\end{theoreme}

\begin{remarques}
\remarque La série
  \[\sum \frac{(-1)^n}{\sqrt{n}}\]
  est convergente, mais n'est pas absolument convergente.
\remarque Les séries alternées permettent de construire des suites $(u_n)$ et $(v_n)$
  qui sont équivalentes en $+\infty$ mais qui ne sont pas de même nature. Par exemple,
  si on définit les suites $(u_n)$ et $(v_n)$ par
  \[\forall n\in\Ns\qsep u_n\defeq\frac{(-1)^n}{\sqrt{n}} \et
    v_n\defeq\frac{(-1)^n}{\sqrt{n}}+\frac{1}{n}\]
  alors $(u_n)$ et $(v_n)$ sont équivalentes en $+\infty$ bien que $\sum u_n$ soit convergente
  et que $\sum v_n$ soit divergente.
\end{remarques}

\begin{exos}
\exo Soit $\alpha\in\R$. Donner une condition nécessaire et suffisante sur $\alpha$ pour que
  la série
  \[\sum \frac{(-1)^n}{n^\alpha}\]
  soit convergente.
\exo Soit $\alpha>0$. Discuter, selon $\alpha$, de la nature de la série
  \[\sum \frac{(-1)^n}{n^\alpha + (-1)^n}.\]
\end{exos}

% \begin{remarqueUnique}
% \remarque Il arrive que l'on doive prouver la convergence de séries de la forme
%   $\sum a_n u_n$ où
%   \begin{itemize}
%   \item La suite $(u_n)$ est une suite positive, décroissante et convergeant vers 0.
%   \item La série de terme général $(a_n)$ est bornée.
%   \end{itemize}
%   Pour prouver la convergence d'une telle série on applique ce qu'on appelle une
%   \emph{transformée d'\nom{Abel}} qui est l'équivalent discret d'une intégration par
%   parties. On définit la suite $(A_n)$ par
%   \[\forall n\in\N\qsep A_n\defeq\sum_{k=0}^{n-1} a_k.\]
%   Alors
%   \begin{eqnarray*}
%   \forall n\in\N\qsep
%   \sum_{k=0}^n a_k u_k
%   &=& \sum_{k=0}^n (A_{k+1} - A_k)u_k\\
%   &=& \sum_{k=0}^n A_{k+1} u_k  - \sum_{k=0}^n A_k u_k\\
%   &=& \sum_{k=1}^{n+1} A_k u_{k-1}  - \sum_{k=0}^n A_k u_k\\
%   &=& \p{A_{n+1} u_n - A_0 u_0} - \sum_{k=1}^n A_k (u_{k}-u_{k-1}).
%   \end{eqnarray*}
%   On montre ensuite que $A_{n+1} u_n$ tend vers 0 lorsque $n$ tend vers $+\infty$ et
%   que la série $\sum A_n(u_n - u_{n-1})$ est absolument convergente, ce qui permet de prouver
%   la convergence de $\sum a_n u_n$.
% \end{remarqueUnique}

% \begin{exoUnique}
% \exo Soit $\theta\in\R\setminus 2\pi\Z$ et $\alpha\in\RPs$. Montrer que la série
%   \[\sum \frac{\cos(n\theta)}{n^\alpha}\]
%   est convergente.
% \end{exoUnique}


% \subsection{Développement décimal d'un réel}

% \begin{proposition}[utile=1, nom=Développement décimal propre d'un réel]
% Soit $x$ un réel positif. Alors il existe un unique $a_0\in\N$ et une unique suite
% $(a_n)_{n \geq 1}$ d'éléments de $\intere{0}{9}$, non stationnaire égale à $9$, telle que 
% $$x=\sum\limits_{n=0}^{+\infty} \frac{a_n}{10^n}.$$
% \end{proposition}


% Le but de cette partie est de donner un sens aux écritures suivantes :
% $$\pi = 3,14159265358979\dots \qquad - \frac{7}{11} = -0,6363636363 \dots $$


% \begin{center}
% \begin{tabular}{lr}
% \begin{minipage}{6cm}
% \begin{eqnarray*}
% x & = & 3,14159265358979\dots \\
% 10^3 x & = & 3141,59265358979\dots \\
% \ent{10^3 x} & = & 3141 \\
% \frac{\ent{10^3 x}}{10^3} &= & 3,141
% \end{eqnarray*}
% \end{minipage}

% &
% \begin{minipage}{6cm}
% \begin{eqnarray*}
% x & = & 3,14159265358979\dots \\
% 10^4 x & = & 31415,9265358979\dots \\
% \ent{10^4x} & = & 31415 \\
% \frac{\ent{10^4x}}{10^4} &= & 3,1415
% \end{eqnarray*}
% \end{minipage}
% \end{tabular}
% \end{center}


% On obtient donc une suite $S_n = \frac{\ent{10^n x}}{10^n}$ qui converge vers $x$ et les chiffres $a_n$ qui sont définis par 
% $$a_0 = \ent{x} \text{ et } a_n = \ent{10^n x} - 10 \ent{10^{n-1}x} \text{ pour } n \geq 1$$
% de sorte que 
% $$x = \sum\limits_{n=0}^{\infty} \frac{a_n}{10^n}$$

% Pour que ceci corresponde au développement décimal usuel, il faut prendre $x \geq 0$. Si $x<0$, $\ent{x}$ ne donne pas l'entier avant la virgule.
% La convention, pour $x<0$, est de prendre le développement décimal de $-x$ puis de mettre un signe moins devant.
% $a_0$ n'est pas un chiffre mais un entier quelconque.

% Par contre, si $n \geq 1$ alors $a_n \in \intere{0}{9}$.
% Malheureusement, il n'y a pas unicité d'une telle écriture. Par exemple,
% $$1=1,000 \dots = 0,999\dots$$

% Pour avoir l'unicité, il faut imposer que la suite $(a_n)$ n'est pas stationnaire égale à $9$ i.e. 
% $$\forall N \in \N \qsep \exists n \geq N \qsep a_n \neq 9.$$
% On dit alors que le développement décimal est propre. Montrons que l'on a bien alors l'unicité et l'existence de l'écriture décimale.

% \begin{proposition}[utile=1, nom=Développement décimal propre d'un réel]
% Soit $x$ un réel positif. Alors il existe un unique $a_0 \in \N$ et une unique suite $(a_n)_{n \geq 1}$ d'entiers entre $0$ et $9$ non stationnaire égale à $9$ telle que 
% $$x=\sum\limits_{n=0}^{\infty} \frac{a_n}{10^n}.$$

% \end{proposition}



%END_BOOK
\end{document}