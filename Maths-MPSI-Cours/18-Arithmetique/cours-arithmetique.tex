\documentclass{magnolia}

\magtex{tex_driver={pdftex},
        tex_packages={xypic,epigraph}}
\magfiche{document_nom={Cours d'arithmétique},
          auteur_nom={François Fayard},
          auteur_mail={fayard.prof@gmail.com}}
\magcours{cours_matiere={maths},
          cours_niveau={mpsi},
          cours_chapitre_numero={11},
          cours_chapitre={Arithmétique}}
\magmisenpage{}
\maglieudiff{}
\magprocess

\begin{document}

%BEGIN_BOOK
\setlength\epigraphwidth{.6\textwidth}
\epigraph{\og La Mathématique est la reine des sciences et l'Arithmétique est
la reine des mathématiques. \fg}{--- \textsc{Carl Friedrich Gauss (1777--1855)}}
\hidemesometimes{
\bigskip
\hfill\includegraphics[width=0.5\textwidth]{../../Commun/Images/maths-cours-factoring_the_time.png}}

\magtoc

\section{Divisibilité, division euclidienne}

\subsection{Relation de divisibilité}
\begin{definition}
Soit $a,b\in\Z$. On dit que $a$ \emph{divise} $b$ lorsqu'il existe $k\in\Z$ tel que
$b=ka$.  
\end{definition}

\begin{remarques}
\remarque Soit $a,b\in\Z$ tels que $a|b$. Alors $-a|b$, $a|-b$ et $-a|-b$.
  Autrement dit, lorsqu'on parle de divisibilité, le signe n'est pas
  significatif.
\remarque Soit $a\in\Z$. Alors $a|1$ si et seulement si $a=\pm 1$.
\begin{sol}
Soit $a\in\Z$ tel que $a|1$, alors il existe $k\in \Z$ tel que $1=ka$. Supposons un instant que $a\notin\set{-1;1}$. Ainsi, comme on a également $a\neq 0$, $|a|\geq 2$. Donc $|k|=\dfrac{1}{|a|}\leq \dfrac{1}{2}$ d'où $k=0$, ce qui fournit une contradiction.\\
Réciproquement, $1=1\cdot 1$ et $1=(-1)\cdot(-1)$ donc $-1$ et $1$ divisent bien $1$.
\end{sol}
\remarque Soit $a,b,c\in\Z$. Si $ac|bc$ et $c\neq 0$, alors $a|b$
\begin{sol}
Si $ac|bc$, il existe $k\in \Z$ tel que $bc=kac$ donc $(b-ka)c=0$ et $c\neq 0$ donc $b=ka$, i.e. $a|b$.
\end{sol}
\end{remarques}

\begin{proposition}
La relation de divisibilité
\begin{itemize}
\item est réflexive~: $\forall a\in\Z \qsep a|a$.
\item est transitive~: $\forall a,b,c\in\Z \qsep \cro{a|b \et b|c}\quad\implique\quad a|c$.
\item n'est pas antisymétrique. Cependant
  \[\forall a,b\in\Z \qsep \cro{a|b \et b|a}\quad\ssi\quad a=\pm b.\]
\end{itemize}
\end{proposition}

\begin{remarques}
\remarque La relation de divisibilité n'étant pas antisymétrique sur $\Z$, ce
  n'est pas une relation d'ordre. Cependant, si $a,b\in\N$, on a
  \[\cro{a|b \et b|a}\quad\ssi\quad a=b.\]
  En particulier, la relation de divisibilité est une relation d'ordre sur $\N$.
\remarque Quel que soit $n\in\Z$,  $1 | n$ et $n | 0$. En particulier, pour la relation de divisibilité, $\N$ admet 1 pour plus petit élément et 0 pour plus grand élément.
\remarque Soit $a,b\in\N$. Si $a|b$ et $b\neq 0$, alors $a\leq b$.
\end{remarques}

\begin{proposition}
Soit $a,b,c\in\Z$ et $k_1,k_2\in\Z$. Alors
\[\cro{a|b \et a|c}\quad\implique\quad a|\p{k_1b+k_2c}.\]
\end{proposition}

\begin{exos}
\exo Soit $a,b,c\in\Z$. Les assertions suivantes sont-elles vraies~?
  \begin{itemize}
  \item Si $a$ divise $b+c$, alors $a$ divise $b$ et $c$.
  \item Si $a$ et $b$ divisent $c$, alors $ab$ divise $c$.
  \end{itemize}
\begin{sol}
NON. $2\mid6=3+3$ mais $2\nmid 3$
NON. $2\mid 6$ et $2\mid6$ mais $4\nmid6$.
\end{sol}
\exo Déterminer les entiers $n\in\N$ tels que $n|n+8$.
\begin{sol}
Analyse : Si $n\mid n+8$, $n$ divise $n$ et $n+8$ donc $n$ divise $n+8-n=8$, d'où $n\in \set{1,2,4,8}$.
Synthèse : OK.
\end{sol}
\end{exos}

\subsection{Congruence, division euclidienne}

\begin{definition}
Soit $a,b\in\Z$ et $m\in\Ns$. On dit que $a$ est \emph{congru} à $b$ modulo $m$ et on
note
\[a\equiv b\quad[m]\]
lorsque $m|\p{a-b}$, c'est-à-dire lorsqu'il existe $k\in\Z$ tel que
$a=b+km$.
\end{definition}

\begin{remarqueUnique}
\remarque Si $m\in\Ns$, la relation binaire $\mathcal{R}$ définie sur $\Z$ par
  \[\forall a,b\in\Z \quad a\mathcal{R}b \quad\ssi\quad a\equiv b\ [m]\]
  est une relation d'équivalence.
\end{remarqueUnique}

\begin{proposition}
Soit $a_1,a_2,b_1,b_2\in\Z$ et $m\in\Ns$ tels que
\[a_1\equiv b_1 \quad [m] \et a_2\equiv b_2 \quad [m].\]
Alors, si $k_1,k_2\in\Z$ et $k\in\N$
\[k_1a_1+k_2a_2\equiv k_1b_1+k_2b_2 \quad [m] \qquad
  a_1a_2\equiv b_1b_2 \quad [m] \qquad\text{et}\qquad
  a_1^k\equiv b_1^k \quad [m].\]
\end{proposition}

\begin{preuve}
Tout se montre très bien "à la main". Pour les puissances, il s'agit de faire une récurrence et dans l'hérédité on a en fait besoin du cas $k=2$ obtenu dans la propriété précédente.
\end{preuve}

\begin{remarqueUnique}
\remarque Soit $m,n\in\Ns$ et $a,b\in\Z$. Alors
  $a\equiv b\ [m] \ssi an\equiv bn\ [mn]$.
\end{remarqueUnique}

\begin{exos}
\exo Montrer que pour tout $n\in\N$, 11 divise $3^{n+3}-4^{4n+2}$.
\begin{sol}
On veut montrer que $3^{n+3}-4^{4n+2} \equiv 0 \quad [11]$.
Or, $3^3\equiv 5 \quad [11]$, $4^2\equiv 5 \quad [11]$ et donc $4^4=(4^2)^2\equiv 3 \quad [11]$ d'où :
$$3^{n+3}-4^{4n+2} \equiv 5\times 3^n-3^n\times 5 \equiv 0 \quad [11].$$
\end{sol}
\exo Trouver les $n\in\N$ tels que $10^n+5^n+1$ est un multiple de 3.
  \begin{sol} Soit $n\in\N$.
  \begin{eqnarray*}
  3\mid 10^n+5^n+1 &\Longleftrightarrow& 10^n+5^n+1 \equiv 0 \quad [3]\\
  &\Longleftrightarrow& 1^n+(-1)^n+1 \equiv 0 \quad [3]\\
  &\Longleftrightarrow& (-1)^n \equiv -2 \quad [3]\\
  &\Longleftrightarrow& (-1)^n \equiv 1 \quad [3]\\
  &\Longleftrightarrow& n \text{ est pair}.
  \end{eqnarray*}
  On peut montrer la dernière équivalence par disjonction de cas.
  \end{sol}
\exo Montrer qu'un entier est divisible par 3 si et seulement si la somme
  de ses chiffres est divisible par 3. De même, montrer qu'un entier est
  divisible par 9 si et seulement si la somme de ses chiffres est divisible
  par 9. Enfin, montrer qu'un nombre est divisible par 11 si et seulement
  si la somme alternée de ses chiffres est divisible par 11.
  \begin{sol}
  Si $a=\sum_{k=0}^n 10^k a_k$, alors
  \begin{itemize}
  \item $a$ est divisible par 3 si et seulement si $\sum_{k=0}^n a_k$ est
    divisible par 3.
  \item $a$ est divisible par 9 si et seulement si $\sum_{k=0}^n a_k$ est
    divisible par 9.
  \item $a$ est divisible par 11 si et seulement si
    \[\sum_{\substack{k=0\\k\equiv 0\,[2]}}^n a_k-
       \sum_{\substack{k=0\\k\equiv 1\,[2]}}^n a_k\] est divisible par 11 car $$\sum_{k=0}^n 10^k a_k\equiv \sum_{k=0}^n (-1)^k a_k \quad [11].$$
  \end{itemize}
  \end{sol}
\end{exos}

\begin{proposition}
Soit $a\in\Z$ et $b\in\Ns$. Alors il existe un unique couple $\p{q,r}\in\Z^2$
tel que
\[a=qb+r \et 0\leq r<b.\]
$q$ est appelé \emph{quotient} de la division euclidienne de $a$ par $b$, $r$ son
\emph{reste}.
\end{proposition}

\begin{preuve}
\begin{itemize}
\item[$\bullet$] \textbf{Unicité :} Supposons trouvés deux couples $(q_1,r_1)$ et $(q_2,r_2)$. On obtient alors $$b|q_1-q_2|=|r_2-r_1|.$$ Supposons un instant $q_1\neq q_2$, alors $$b\leq b|q_1-q_2|=|r_2-r_1|\leq b-1$$ d'où la contradiction.
\item[$\bullet$] \textbf{Existence :} Commençons par le cas $a\geq 0$ et considérons l'ensemble~:
\[A=\enstq{k\in\N}{kb\leq a}\]
Cet ensemble est non vide ($0\in A$), majoré (par $a$) donc admet un plus grand
élément $q$. Comme $q\in A$ et $q+1 \notin A$, on a $bq\leq a <b(q+1)$ d'où $0\leq a-bq<b$. Il reste à poser $r=a-bq$.\\
Si $a\leq 0$, il suffit d'appliquer ce qui précède à $-a$. Il existe donc $q\in \N$ et $r\in \N$ vérifiant \[-a=qb+r \et 0\leq r<b.\]
Donc $a=(-q)b+(-r)$ ce qui convient si $r=0$ et si $0<r<b$, alors $-b<-r<0$ donc $0<-r+b<b$ et dans ce cas $a=(-q-1)b+(-r+b)$ est de la forme recherchée.
\end{itemize}

\end{preuve}

\begin{remarques}
\remarque Si $a\in\Z$ et $b\in\Ns$, le reste de la division euclidienne de $a$
  par $b$ est l'unique élément $r$ de $\intere{0}{b-1}$ tel que
  $a\equiv r\ [b]$.
\remarque Les langages Python et OCaml possèdent tous les deux une division entière
  et un opérateur \og modulo \fg. Si $a\in\Z$ et $b\in\Ns$, la division entière
  s'obtient avec \verb_a // b_ en Python et avec \verb_a / b_ en OCaml. L'opérateur
  \og modulo \fg s'obtient quant à lui avec \verb_a % b_ en Python et
  \verb_a mod b_ en OCaml. Si on note $q$ la division entière de $a$ par $b$ et $r$
  le résultat de l'opérateur modulo, on aura toujours $a=qb+r$. En Python, ce sont respectivement le quotient
  et le reste de la division euclidienne de $a$ par $b$. C'est aussi le cas en OCaml lorsque $a\geq 0$. 
  Cependant, lorsque $a<0$, ce n'est
  plus le cas. Par exemple, la division entière de $-7$ par $2$ renvoie $-3$ alors que le quotient de
  la division euclidienne de $-7$ par $2$ est $-4$.
  % Cependant, $q$ et $r$
  % ne sont pas toujours le quotient et le reste de la division euclidienne de $a$ par $b$.
  % En pratique, on ne s'intéressera qu'au cas où $a\in\Z$ et $b\in\Ns$, car aucun programmeur
  % sensé ne va tenter de faire une division entière par un nombre $b$ strictement négatif. Cependant,
  % Python et OCaml n'ont pas le même comportement, même lorsque $b>0$.
  % \begin{itemize}
  % \item En Python, si $a\in\Z$ et $b\in\Ns$, $q$ et $r$ sont respectivement le quotient et le
  %   reste de la division euclidienne de $a$ par $b$.
  % \item En OCaml, si $a\in\N$ et $b\in\Ns$, $q$ et $r$ sont respectivement le quotient et le
  %   reste de la division euclidienne de $a$ par $b$. Mais, si $a<0$ et $b>0$, ce n'est plus
	% le cas. Dans ce cas, le résultat renvoyé par \verb_a / b_ est \verb_-((-a) / b)_. Par exemple, si
	% $a=-5$ et $b=2$, alors les quotient et reste de la division euclidienne de $a$ par
	% $b$ sont respectivement $-3$ et $1$, mais \verb_a / b_ renvoie $2$ et \verb_a mod b_
	% renvoie $-1$.
  % \end{itemize}
\remarque Si $a\in\Z$ et $b\in\Zs$, on montre qu'il existe un unique couple $\p{q,r}\in\Z^2$
  tel que $a=qb+r$ et $0\leq r<\abs{b}$. On peut donc ainsi étendre la définition
  de la division euclidienne au cas où $b\in\Zs$. Mais en pratique, on effectuera
  toujours des divisions euclidiennes par des entiers strictement positifs.
\end{remarques}



\begin{exoUnique}
\exo Déterminer le reste de la division euclidienne de $4852^{203}$ par 5.
  \begin{sol}
  \begin{eqnarray*}
  4852^{203}&\equiv& 2^{203} \quad [5]\\
  &\equiv& 2^{4*50+3} \quad [5]\\
  &\equiv& (2^{4})^{50}\times 2^3 \quad [5]\\
  &\equiv& 1^{50}\times 3 \quad [5]\\
  &\equiv& 3 \quad [5]
  \end{eqnarray*}
  \end{sol}
\end{exoUnique}

% \subsection{Idéaux de $\Z$}

% \begin{definition}
% On dit qu'une partie $\mathcal{I}$ de $\Z$ est un idéal lorsque~:
% \begin{itemize}
% \item $0\in\mathcal{I}$
% \item $\forall a,b\in\mathcal{I} \quad \forall k_1,k_2\in\Z \quad
%        k_1 a+k_2 b\in\mathcal{I}$
% \end{itemize}
% \end{definition}

% %% Remarque :
% %% 1) Les idéaux de Z sont les sous-groupes de (Z,+)

% \begin{definition}
% Si $a\in\Z$
% \[a\Z=\enstq{ka}{k\in\Z}\]
% est un idéal de $\Z$. Tout idéal de ce type est appelé idéal principal.
% \end{definition}

% \begin{proposition}
% L'intersection d'une famille d'idéaux est un idéal.
% \end{proposition}

%% Exemple :
%% 1) 4Z n 6Z = 12Z
%%    - 12Z est inclu dans 4Z n 6Z
%%    - réciproquement si n est un multiple de 4 et de 6. Alors n=6k. Comme
%%      6k-4k=2k, on a n-4k=2k. Or n est un mutiple de 4, donc n=4p.
%%      Donc 4p-4k=2k. Donc 2p-2k=k. Donc k=2k'. Donc n=12k'

% \begin{definition}
% $\quad$
% \begin{itemize}
% \item Soit $\mathcal{I}_1,\ldots,\mathcal{I}_n$ une famille d'idéaux.
%   Alors il existe un plus petit idéal $\mathcal{I}$ contenant
%   $\mathcal{I}_1\cup\dots\cup\mathcal{I}_n$; on l'appelle idéal engendré par
%   $\mathcal{I}_1,\ldots,\mathcal{I}_n$ et on le note
%   $\mathcal{I}_1+\cdots+\mathcal{I}_n$.
% \item Soit $a_1,\ldots,a_n\in\Z$. Alors~:
%   \[a_1\Z+\cdots+a_n\Z=
%     \enstq{k_1 a_1+\cdots+k_n a_n}{k_1,\ldots,k_n\in\Z}\]
% \end{itemize}
% \end{definition}

%% Exemple :
%% 1) L'idéal engendré par 4Z et 6Z est 2Z

% \begin{theoreme}
% Tout idéal de $\Z$ est principal; on dit que $\Z$ est un anneau principal.
% \end{theoreme}

% \begin{proposition}
% Soit $\mathcal{I}$ un idéal de $\Z$. Alors, il existe un unique $n\in\N$ tel
% que $\mathcal{I}=n\Z$.
% \end{proposition}

%% Exemple :
%% 1) Ordre d'un élément dans un groupe.
%%    Soit x dans G. On considère l'application phi : Z -> G qui à n associe
%%    x^n. C'est un morphime de groupe.
%%    On dit que x est d'ordre fini, lorsque phi n'est pas injective,
%%    c'est-à-dire lorsque Ker phi n'est pas réduit à {0}. Si tel est le cas,
%%    il existe un unique n\in\Ns tel que ker phi=nZ. On l'appelle ordre de 
%%    x et on lle note omega(x)
%%    Remarquons que si G est fini, tout élément est d'ordre fini
%%    
%%    - e est l'unique élément d'ordre 1
%%    - -1 est d'ordre 2, j d'ordre 3 et i d'ordre 4

\section{Pgcd, ppcm}

\subsection{Plus grand commun diviseur}

\begin{definition}
Soit $a,b\in\Z$. Il existe un unique entier positif $p$ tel que
\begin{itemize}
\item $p|a \et p|b$
\item $\forall q\in\Z \qsep \cro{q|a \et q|b} \implique q|p$
\end{itemize}
On l'appelle $\pgcd$ (\emph{plus grand commun diviseur}) de $a$ et de $b$ et on le note
$\pgcd\p{a,b}$ ou $a\wedge b$.
\end{definition}

\begin{preuve}
\begin{itemize}
\item[$\bullet$] \textbf{Unicité :} On suppose qu'il y en a deux $p_1$ et $p_2$. D'après (ii), $p_1 \mid p_2$ et $p_2\mid p_1$ avec $p_1, p_2$ dans $\N$, donc $p_1=p_2$.
\item[$\bullet$] \textbf{Existence :} Puisque la définition de divisibilité est invariante par changement de signe et qu'ici les rôles de $a$ et $b$ sont symétriques, on peut supposer $0\leq b\leq a$.\\
Idée :On va faire une récurrence forte sur $n=a+b$ en remarquant que le $\pgcd$ de $a,b$ est
égal à au $\pgcd$ de $a-b,b$ (en prenant bien sur $b\leq a$ et en traitant
à part le cas où $b=0$).\\
Définissons alors l'assertion $\mathcal{H}_n$ : "Soit $a,b\in \N$ tels que $a\geq b$ et $a+b=n$. Alors, il existe $p\in\N$ tel que \begin{itemize}
\item $p|a \et p|b$
\item $\forall q\in\Z \quad \cro{q|a \et q|b} \implique q|p$
\end{itemize}".
$\mathcal{H}_0$ : $a+b=0$ donc $a=b=0$ et $p=0$ convient.\\
Soit $n\in \N$. Supposons $\mathcal{H}_i$ vraie pour $i\in \set{0,\ldots,n}$. Montrons que $\mathcal{H}_{n+1}$ est vraie. Soit donc $a,b\in \N$ tels que $a\geq b$ et $a+b=n+1$.\\
Si $b=0$. On pose $p=a$. Alors $p\mid a$ et $p\mid 0=b$. De plus, si $q\in \Z$ est tel que $q|a \et q|b$ alors $q|a=p$.\\
Sinon, posons $a'=a-b\geq 0$. Alors $a'+b=a-b+b=a+b-b=n+1-b<n+1$. \underline{Si $a'\geq b$} on peut appliquer $\mathcal{H}_{a'+b}$. Il existe $p\in\N$ tel que :
\begin{itemize}
\item $p|a' \et p|b$
\item $\forall q\in\Z \quad \cro{q|a' \et q|b} \implique q|p$
\end{itemize}
Montrons que $p$ convient. $p|a'+b=a$ et $p|b$. Soit $q\in \Z$ tel que $q|a \et q|b$, $q|a-b=a'$ donc $q|a' \et q|b$ donc $q|p$. Ainsi, $p$ convient.\\
\underline{Si $a'<b$}, on a $0\leq a'<b$. On pose $a''=b$ et $b''=a'$. Alors $a''+b''=a'+b<n+1$. Donc il existe $p\in \N$ tel que :
\begin{itemize}
\item $p|a'' \et p|b''$
\item $\forall q\in\Z \quad \cro{q|a'' \et q|b''} \implique q|p$
\end{itemize}
De la même façon, $p$ convient.
Donc $\mathcal{H}_{n+1}$ est vraie.
\end{itemize}
 
\end{preuve}

\begin{remarques}
\remarque Si $a,b\in\Z$, les diviseurs de $a$ et de $b$ sont les diviseurs
  de $a\wedge b$.
\remarque Soit $a,b\in\N$. Pour la relation d'ordre de divisibilité sur $\N$, l'ensemble
  des diviseurs de $a$ et de $b$ n'est rien d'autre que
  l'ensemble des minorants de $\ens{a,b}$. La définition précédente montre donc que
  cet ensemble admet un plus grand élément (au sens de la divisibilité) qui est
  $a\wedge b$. Autrement dit, au sens de la divisibilité, l'ensemble $\ens{a,b}$ admet
  une borne inférieure qui est $a\wedge b$.
\remarque Soit $a,b\in\N$. Si l'un des deux entiers est non nul,
  le $\pgcd$ de $a$ et de $b$ est le plus grand (au sens de l'ordre)
  diviseur commun de $a$ et $b$.
\end{remarques}

\begin{proposition}
\begin{eqnarray*}
\forall a\in\Z, & & a\wedge 0=\abs{a}\\
\forall a\in\Z, & & a\wedge 1=1\\
\forall a,b\in\Z, & & a\wedge b=0 \quad\ssi\quad\cro{a=0 \et b=0}
\end{eqnarray*}
\end{proposition}

\begin{proposition}
\begin{eqnarray*}
\forall a,b\in\Z, & & a\wedge b=b\wedge a\\
\forall a,b\in\Z, & & a\wedge b=\p{-a}\wedge b=a\wedge\p{-b}=
  \p{-a}\wedge\p{-b}=\abs{a}\wedge\abs{b}\\
\forall a,b,k\in\Z, & & \p{ka}\wedge\p{kb}=\abs{k}\p{a\wedge b}
\end{eqnarray*}
\end{proposition}

\begin{preuve}
Par symétrie de la définition en $a$ et $b$, on a le premier point. Cette définition ne faisant intervenir que la relation de divisibilité qui est invariante par changement de signe, on a le deuxième point.\\
Pour le troisième point, montrons que $\p{ka}\wedge\p{kb} \et \abs{k}\p{a\wedge b}$ se divisent l'un l'autre. Comme ce sont des entiers naturels, on pourra conclure qu'ils sont égaux.\\
Si $k=0$, cela donne $0=0$. Sinon :
\begin{itemize}
\item On a $a\wedge b\mid a$ donc $k(a\wedge b)\mid ka$ donc $\abs{k}\p{a\wedge b}\mid ka$. De même $\abs{k}\p{a\wedge b} \mid kb$ donc $\abs{k}\p{a\wedge b}\mid ka$ et $\abs{k}\p{a\wedge b}\mid kb$, ce qui implique que $\abs{k}\p{a\wedge b}\mid \p{ka}\wedge\p{kb}$.
\item $k\mid ka$ et $k\mid kb$ donc $k\mid \p{ka}\wedge\p{kb}$. Il existe donc $u\in \Z$ tel que $\p{ka}\wedge\p{kb}=ku$. Montrons que $u\mid a\wedge b$.\\
$ku=\p{ka}\wedge\p{kb}\mid ka$. Or $k\neq 0$ donc $u\mid a$. De même, $u\mid b$ donc $u\mid a\wedge b$. Ainsi, $ku\mid k(a\wedge b)$, i.e. $\p{ka}\wedge\p{kb}\mid k(a\wedge b)$. Finalement $\p{ka}\wedge\p{kb}\mid \abs{k}(a\wedge b)$.
\end{itemize}
\end{preuve}

\begin{definition}
Soit $a_1,\ldots,a_n\in\Z$. Il existe un unique entier positif $p$ tel que
\begin{itemize}
\item $\forall i\in\intere{1}{n} \qsep p|a_i$
\item $\forall q\in\Z \qsep \cro{\forall i\in\intere{1}{n} \qsep q|a_i} \implique q|p$
\end{itemize}
On l'appelle $\pgcd$ (plus grand commun diviseur) de la famille $(a_1,\ldots,a_n)$ et on le note
$\pgcd\p{a_1,\ldots,a_n}$ ou $a_1\wedge\cdots\wedge a_n$.
\end{definition}

\begin{preuve}
Pour l'existence, on le montre par récurrence sur $n$ et la clé est de montrer dans l'hérédité que $p=(a_1\wedge \ldots \wedge a_n)\wedge a_{n+1}$ (qui existe d'après l'hypothèse de récurrence) convient.
\begin{itemize}
\item Déjà, par définition, $p\mid a_1\wedge \ldots \wedge a_n$ donc chaque $a_i$ pour $i\in \set{1,\ldots,n}$ et $p\mid a_{n+1}$.
\item Soit maintenant $q\in\Z$ tel que$\forall i\in\intere{1}{n+1} \quad q|a_i$. Montrons que $q\mid p$.\\
$q\mid a_i, \forall i\in\intere{1}{n}$ donc $q\mid a_1\wedge \ldots \wedge a_n$. De plus, $q\mid a_{n+1}$ donc $q\mid p$.
\end{itemize}
\end{preuve}

\begin{remarqueUnique}
\remarque Le $\pgcd$ d'une famille d'entiers $(a_1,\ldots,a_n)$ ne dépend pas de l'ordre de
  ces derniers.
\end{remarqueUnique}

\begin{proposition}
Soit $a_1,\ldots,a_n\in\Z$ et $p\in\intere{1}{n-1}$. Alors
\[a_1\wedge\cdots\wedge a_n=(a_1\wedge\cdots\wedge a_p)\wedge(a_{p+1}\wedge\cdots\wedge a_n).\]
\end{proposition}


\subsection{Algorithme d'\nom{Euclide}}

\begin{proposition}
Soit $a,b,k\in\Z$. Alors
\[a\wedge b=a\wedge\p{b+ka}=\p{a+kb}\wedge b.\]
En particulier, si $b\in\Ns$ et $r$ est le reste de la division euclidienne de
$a$ par $b$, on a
\[a\wedge b=b\wedge r.\]
\end{proposition}

\begin{preuve}
Posons $p=a\wedge b$ et montrons que $p=a\wedge\p{b+ka}$.
\begin{itemize}
\item $p\mid a \et p\mid b$ donc $p\mid b+ka$. Ainsi, $p\mid a \et p\mid b+ka$.
\item Soit maintenant $q\in\Z$ tel que $q\mid a \et q\mid b+ka$. Montrons que $q\mid p$.\\
$q\mid a \et q\mid b+ka$ donc $q\mid 1\times (b+ka)+(-k)\times a=b$. Ainsi, $q\mid a \et q\mid b$ donc $q\mid a\wedge b=p$.
\end{itemize}
\end{preuve}


% Soit $a,b\in\N$, pour calculer $a\wedge b$, on utilise l'algorithme
% suivant, appelé {\bf algorithme d'Euclide}.
% \begin{itemize}
% \item On pose $r_0=a$ et $r_1=b$.
% \item Si $r_1=0$, la réponse est $r_0$. Sinon, on effectue la division
%   euclidienne de $r_0$ par $r_1$. Il existe donc $q_1\in\N$ et
%   $r_2\in\intere{0}{r_1-1}$ tels que $r_0=q_1r_1+r_2$.
% \item Si $r_2=0$, la réponse est $r_1$. Sinon, on effectue la division
%   euclidienne de $r_1$ par $r_2$. Il existe donc $q_2\in\N$ et
%   $r_3\in\intere{0}{r_2-1}$ tels que $r_1=q_2r_2+r_3$.
% \item Si $r_3=0$, etc.
% \item Supposons que $r_{n+1}$ soit défini. Si $r_{n+1}=0$, la réponse est
%   $r_n$. Sinon, on effectue la division euclidienne de $r_n$ par
%   $r_{n+1}$. Il existe donc $q_{n+1}\in\N$ et $r_{n+2}\in\intere{0}{r_{n+1}-1}$
%   tels que $r_n=q_{n+1}r_{n+1}+r_{n+2}$.
% \end{itemize}
% Démontrons la validité de cet algorithme~:
% \begin{itemize}
% \item {\bf L'algorithme s'arrête~:}\\
%   En effet, si tel n'était pas le cas, on aurait
%   ainsi construit une suite $(r_n)_{n\in\N}$ strictement décroissante (à partir
%   du rang 1) d'entiers naturels, ce qui n'existe pas.
% \item {\bf L'algorithme donne la bonne réponse~:}\\ En effet, comme
%   $r_n=q_{n+1}r_{n+1}+r_{n+2}$, on en déduit que
%   \begin{eqnarray*}
%   r_n\wedge r_{n+1}
%   &=& (q_{n+1}r_{n+1}+r_{n+2})\wedge r_{n+1}\\
%   &=& r_{n+2}\wedge r_{n+1} \qquad \text{(lemme d'Euclide)}\\
%   &=& r_{n+1}\wedge r_{n+2}
%   \end{eqnarray*}
%   On en déduit facilement par récurrence finie que, pour tout entier $n$,
%   $r_n\wedge r_{n+1}=r_0\wedge r_1$ $(=a\wedge b)$. Lorsque l'algorithme
%   s'arrête, $r_{n+1}=0$ donc $a\wedge b=r_n\wedge r_{n+1}=r_n\wedge 0=r_n$.
%   L'algorithme donne donc la bonne réponse.
% \end{itemize}
% \medskip

\begin{exos}
\exo Calculer $105\wedge 147$.
  \begin{sol}
  $105\wedge 147=147\wedge 105=105\wedge42=42\wedge21=21\wedge0=21$
  \end{sol}
\exo Soit $n\in\N$. Calculer $(3n+1)\wedge(2n)$, puis
  $(n^4-1)\wedge(n^6-1)$.
  \begin{sol}
  On a
  \[(3n+1,2n)=(n+1,2n)=(n+1,-2)=(n+1,2)=
    \begin{cases}
    2 &\text{si n est impair}\\
    1 &\text{si n est pair}
    \end{cases}\]
  Puis
  \begin{eqnarray*}
  (n^4-1)\wedge(n^6-1)&=&\left[(n^2-1)(n^2+1)\right]\wedge \left[(n^2-1)(n^4+n^2+1)\right]\\
  &=&|n^2-1|\left[(n^2+1)\wedge (n^4+n^2+1)\right]\\
  &=&|n^2-1|\left[(n^2+1)\wedge (n^4+n^2+1-n^2(n^2+1))\right]\\
  &=&|n^2-1|\left[(n^2+1)\wedge 1\right]\\
  &=&|n^2-1|.
  \end{eqnarray*} 
  \end{sol}
\exo Soit $(F_n)$ la suite, appelée suite de Fibonacci, définie par
  \[F_0\defeq 0, \quad F_1\defeq 1, \et \forall n\in\N \qsep F_{n+2}\defeq F_{n+1}+F_n.\]
  Montrer que pour tout $n\in\N$, $F_n\wedge F_{n+1}=1$.
  \begin{sol}
  Alors, quel que soit $n$ dans $\N$ le $\pgcd$ de $F_n$ et de $F_{n+1}$ est
  égal à 1. On le montre par récurrence sur $n$.
  $$F_{n+1}\wedge F_{n+2}=F_{n+1}\wedge (F_{n}+F_{n+1})=F_{n+1}\wedge (F_{n}+F_{n+1}-F_{n+1})=F_{n+1}\wedge F_{n}=1.$$\\
  Donc deux termes consécutifs de la suite 0,1,1,2,3,5,8,13,21, ont un pgcd
  égal à 1.\\
  Leonardo Pisano (Leonardo de Pise) appelé aussi Leonardo Fibonacci
  Publié en 1202. Il avait lu le travail de Al-Khwarizmi, 825 (Persan)
  (le mot \og algèbre \fg vient du titre de son oeuvre majeure : Kitab al-jabr
  wa'l-muqabala) (Rules of restoring and equating)
  Son livre contient le problème suivant : Combien de couples de lapins
  peuvent naître d'un couple de lapin en un an~? Pour résoudre ce problème, on
  sait que~:
  \begin{itemize}
  \item Chaque couple donne naissance a un couple chaque mois.
  \item Chaque couple devient fertile à l'âge d'un mois.
  \item Les lapins ne meurent jamais
  \end{itemize}
  Avant le travail de Fibonacci, la suite $(F_n)$ a déjà été étudiée par les
  Indiens. Ils se demandaient combien de rythmes de n temps il était possible
  de faire avec des noires et des blanches. Ce nombre est $F_{n+1}$ .   
  \end{sol}
\end{exos}

\begin{remarqueUnique}
\remarque Si $a,b\in\N$, l'algorithme suivant, appelé algorithme d'\nom{Euclide}, calcule le $\pgcd$ de $a$ et $b$.
\begin{pythoncode}
def pgcd(a, b):
    """pgcd(a: int, b: int) -> int"""
    while b != 0:
        a, b = b, a % b
    return a
\end{pythoncode}
% \remarque Remarquons qu'il est inutile de supposer que $b$ soit inférieur à $a$
%   pour que l'algorithme fonctionne. En effet, si $a<b$, alors la division
%   euclidienne de $a$ par $b$ donne $a=0\cdot b+a$. La première étape de l'algorithme échange donc $a$ et $b$.  
\end{remarqueUnique}

\subsection{Relation de \nom{Bézout}}

\begin{proposition}
Soit $a,b\in\Z$. Alors il existe $u,v\in\Z$ tels que
\[ua+vb=a\wedge b.\]
\end{proposition}

\begin{preuve}
Quitte à changer $a$ en $-a$ et $b$ en $-b$, on peut les supposer positifs. Définissons $\mathcal{H}_n$ : "quels que soient $a$ et $b$ dans $\N$ tels que $a+b=n$, il existe $u,v\in \Z$ tels que $ua+vb=a\wedge b$."
\begin{itemize}
\item[$\bullet$] $\mathcal{H}_0$ est vraie : $a+b=0$ donc $a=0$ et $b=0$. Posons $u=v=0$.
\item[$\bullet$] Soit $n\in \N$. Supposons $\mathcal{H}_i$ vraie pour $i\in \set{0,\ldots,n}$. Montrons que $\mathcal{H}_{n+1}$ est vraie. Soit donc $a,b\in \N$ tels que $a\geq b$ et $a+b=n+1$. Quitte à échanger $a$ et $b$, on peut supposer $a\geq b$.\\
Si $b=0$, alors $a\wedge b=a\wedge 0=a$. On a alors $a\wedge b=a=1\times a+0\times b$.\\
Si $b\neq 0$, alors on pose $a'=a-b\geq 0$ et $b'=b$. On a $a'+b'=a-b+b=a+b-b=n+1-b<n+1$. Donc  $\mathcal{H}_{a'+b'}$ est vraie. Il existe donc $u,v\in \Z$ tels que $ua'+vb'=a'\wedge b'=(a-b)\wedge b=a\wedge b$ et $ua'+vb'=u(a-b)+vb=ua+(v-u)b$. Donc $\mathcal{H}_{n+1}$ est vraie.
\end{itemize}
\end{preuve}

\begin{exos}
\exo Trouver une relation de \nom{Bézout} pour $105$ et $147$.
  \begin{sol}
  $$147=105+42$$
  $$105=2*42+21$$
  $$42=21*2+0$$ donc en remontant
  $$21=105-2*(147-105)=3\times 105-2\times 147.$$
  \end{sol}
\exo Soit $n,m\in\Ns$. Montrer que
  \[\U[n]\cap\U[m]=\U[n\wedge m].\]
  \begin{sol}
  On va procéder par double-inclusion :
  \begin{itemize}
\item[$\bullet$] Inclusion droite-gauche facile.
\item[$\bullet$] Pour l'inclusion gauche-droite, on prend $z$ dans $\U[n]\cap\U[m]$. D'après \nom{Bézout}, il existe $u,v\in \Z$ tels que $n\wedge m=un+vm$. Reste plus qu'à élever à la puissance...
\end{itemize}
  \end{sol}
\end{exos}

\begin{definition}
Soit $a,b\in\Z$. On dit que $a$ et $b$ sont \emph{premiers entre eux} lorsque
$a\wedge b=1$.
\end{definition}

\begin{remarqueUnique}
\remarque Soit $a,b\in\Z$. Puisque $a\wedge b$ est un diviseur commun à $a$ et
  $b$, il existe $a',b'\in\Z$ tels que $a=a'(a\wedge b)$ et $b=b'(a\wedge b)$.
  Si $(a,b)\neq (0,0)$, alors $a'$ et $b'$ sont premiers entre eux.
  \begin{sol}
  $$a\wedge b=(a'(a\wedge b))\wedge (b'(a\wedge b))=(a\wedge b)(a'\wedge b')$$
  \end{sol}
\end{remarqueUnique}

\begin{exoUnique}
\exo Soit $a,b\in\Z$ deux entiers premiers entre eux. Calculer
  $(a-b)\wedge(a+b)$.
  \begin{sol}
  Si $d$ divise $(a-b)\wedge(a+b)$, il divise la somme et la différence donc il divise $2a$ et $2b$ donc leur pgcd, i.e. $d\mid 2(a\wedge b)=2$. Ainsi, c'est $1$ ou $2$. C'est $2$ ssi $2\mid a-b$ et $2\mid a+b$, c'est-à-dire s'ils ont même parité. Or, étant premiers entre eux, ils ne peuvent être tous les deux pairs. Ainsi, on trouve 2 si $a$ et $b$ sont impairs et 1 sinon.
  \end{sol}
\end{exoUnique}

\begin{proposition}
Soit $a,b\in\Z$. Alors $a$ et $b$ sont premiers entre eux si et seulement si il
existe $u,v\in\Z$ tels que
\[ua+vb=1.\]
\end{proposition}

\begin{preuve}
Sens gauche-droite : relation de \nom{Bézout}.\\
Pour le sens droite-gauche, $a\wedge b$ divise la gauche donc la droite, donc $1$, donc c'est $1$.
\end{preuve}

\begin{exoUnique}
  \exo Soit $n\in\Ns$. On se place dans le groupe $(\U[n],\times)$ et on pose $\omega\defeq\e^{\ii\frac{2\pi}{n}}$. Montrer que si $k\in\Z$, le groupe engendré par
    $\omega^k$ est égal à $\U[n]$ si et seulement si $k$ et $n$ sont premiers
    entre eux.
  \end{exoUnique}
  
\begin{proposition}
$\quad$
\begin{itemize}
\item Soit $a,b,c\in\Z$ tels que $a\wedge b=1$ et $a\wedge c=1$. Alors
  $a\wedge \p{bc}=1$.
\item Plus généralement, si $a\in\Z$ est premier avec chaque élément d'une
  famille d'entiers $b_1,\ldots,b_n\in\Z$, alors $a$ est premier avec leur
  produit.
\item Soit $a,b\in\Z$ deux entiers premiers entre eux et $m,n\in\N$. Alors
  $a^m\wedge b^n=1$.
\end{itemize}
\end{proposition}

\begin{preuve}
\begin{itemize}
\item On fait le produit des deux relations de \nom{Bézout}.
\item On récurre.
\item On utilise deux fois la propriété précédente en utilisant la commutativité du pgcd.
\end{itemize}
\end{preuve}

\begin{exos}
\exo Soit $a,b\in\Z$ deux entiers premiers entre eux. Montrer que $a+b$ et
  $ab$ sont premiers entre eux.
  \begin{sol}
  Si $d$ divise les deux, $d\mid ab-a(a+b)=-a^2$ donc $d\mid a^2$. De même, $d\mid b^2$. Or $a^2\wedge b^2=1$ donc $d\mid 1$.
  \end{sol}
\exo Résoudre sur $\Z$ l'équation $2n\equiv 7\ [9]$.
\begin{sol}
Idée : la relation de Bezout donne l'inverse d'un nombre modulo $n$ (s'il existe, cf. chapitre suivant).\\
$2\wedge 9=1$ et on dispose de la relation $1\times 9+(-4)\times 2=1$ qu'on peut lire $2\times (-4)\equiv 1 \quad [9]$. Ainsi,

$$2n\equiv 7\quad [9] \Longleftrightarrow 5\times 2n\equiv 5\times 7\quad [9] \Longleftrightarrow n \equiv -1 \quad [9]$$
Donc c'est $\set{-1+9k, k\in \Z}$.
\end{sol}
\end{exos}

\begin{definition}
Soit $n\in\Ns$. On dit que $a\in\Z$ est inversible modulo $n$ lorsqu'il existe $b\in\Z$
tel que $ab\equiv 1\ [n]$.
\end{definition}

\begin{proposition}
Soit $n\in\Ns$. Alors $a\in\Z$ est inversible modulo $n$ si et seulement si $a\wedge n=1$.
\end{proposition}


\begin{proposition}
Soit $a_1,\ldots,a_n\in\Z$. Alors il existe $u_1,\ldots,u_n\in\Z$ tels que
\[u_1 a_1+\cdots+u_n a_n = a_1\wedge \ldots\wedge a_n.\]
\end{proposition}


\begin{definition}
Soit $a_1,\ldots,a_n\in\Z$.
\begin{itemize}
\item On dit que $a_1,\ldots,a_n$ sont \emph{deux à deux premiers entre eux} lorsque
  \[\forall i,j\in\intere{1}{n} \quad i\neq j \implique a_i\wedge a_j=1.\]
\item On dit que $a_1,\ldots,a_n$ sont \emph{premiers entre eux dans leur ensemble} lorsque
  \[a_1\wedge\cdots\wedge a_n=1.\]
\end{itemize}
\end{definition}

\begin{remarqueUnique}
\remarque Si les entiers $a_1,\ldots,a_n$ sont deux à deux premiers entre eux, alors
  ils sont premiers entre eux dans leur ensemble. Cependant, la réciproque est fausse.
  Par exemple, $a_1=2$, $a_2=3$ et $a_3=6$
  sont premiers entre eux dans leur ensemble, mais ne sont pas deux à deux premiers
  entre eux.
\end{remarqueUnique}

\begin{proposition}
Soit $a_1,\ldots,a_n\in\Z$. Alors $a_1,\ldots,a_n$ sont premiers entre eux dans leur
ensemble si et seulement si il existe $u_1,\ldots,u_n\in\Z$ tels que
\[u_1 a_1+\cdots+u_n a_n=1.\]
\end{proposition}

\begin{preuve}
Sens gauche-droite : On récurre en montrant en fait le résultat plus général avec l'hypothèse de récurrence "relation de Bezout avec $n$ éléments dans le cas général". Ainsi, comme $a_1\wedge \ldots \wedge a_n \wedge a_{n+1}=(a_1\wedge \ldots \wedge a_n) \wedge a_{n+1}$ le cas $n=2$ nous dit qu'on a $$ua_{n+1}+v(a_1\wedge \ldots \wedge a_n)=a_1\wedge \ldots \wedge a_n \wedge a_{n+1}$$ et il nous reste à écrire $a_1\wedge \ldots \wedge a_n$ grâce à l'HR.\\
Sens droite-gauche : idem antépénultième proposition.
\end{preuve}

\begin{exoUnique}
\exo Trouver les solutions entières de l'équation
  \[a^2+b^2=3c^2.\]
\end{exoUnique}

\begin{sol}
\begin{itemize}
\item[$\bullet$] \textbf{Méthode 1 :} On procède par analyse. Posons $p=a\wedge b \wedge c$. Si $p=0$, comme $p\mid a$, $p\mid b$ et $p\mid c$, cela donne $a=b=c=0$. Sinon, avec $a=pa'$, $b=pb'$ et $c=pc'$, on a $a'\wedge b' \wedge c'=1$ et $a'^2+b'^2=3c'^2$. Ainsi, $a'^2+b'^2\equiv 0 \quad [3]$. Un tableau des carrés mod $3$ montre que nécessairement $a'\equiv 0 \quad [3]$ et $b'\equiv 0 \quad [3]$. On a donc $a'=3a''$ et $b'=3b''$ et en injectant cela dans l'équation on en déduit que $3\mid c'^2$. D'où $c'^2\equiv 0 \quad [3]$ et d'après le tableau fait précédemment, $c'\equiv 0 \quad [3]$. Ceci est impossible car $3$ divise alors $a'\wedge b' \wedge c'=1$. Finalement, $p=0$ et $a=b=c=0$.
Synthèse triviale.
\item[$\bullet$] \textbf{Méthode 2 :}
Supposons par l'absurde qu'il existe une solution non nulle (on voit rapidement dans ce cas qu'ils sont tous non nuls): On regarde tout cela modulo 4. Les carrés modulo 4 étant $0$ ou $1$ on a forcément $c^2 \equiv 0 \quad [4]$ et ensuite les deux autres aussi. Donc $a$, $b$ et $c$ sont pairs. Si on divise le triplet par $2$ on a donc une nouvelle solution, et ceci indéfiniment, d'où la contradiction.
\end{itemize}
\end{sol}

\subsection{Lemme de \nom{Gauss}}

\begin{theoreme}
Soit $a,b,c\in\Z$. Alors
\[\cro{a|bc \et a\wedge b=1} \quad\implique\quad a|c.\]  
\end{theoreme}

\begin{preuve}
$bc=wa$ et $ua+vb=1$ donne $uac+vbc=c$ donc $a(uc+vw)=c$ donc $a\mid c$.
\end{preuve} 

\begin{exos}
\exo Soient $a,b,c\in\Z$ tels que $a\wedge c=1$. Montrer que
  \[\p{ab}\wedge c=b\wedge c.\]
  \begin{sol}
  Soit $a,b,c\in\Z$ tels que $a\wedge c=1$. Montrons que
  $\p{ab}\wedge c=b\wedge c$. Ces deux nombres étant positifs, il suffit de
  montrer que $\p{ab}\wedge c|b\wedge c$ et $b\wedge c|\p{ab}\wedge c$.
  \begin{itemize}
  \item $b\wedge c|\p{ab}\wedge c$.\\
    En effet $b\wedge c|b$ donc $b\wedge c|ab$. De plus $b\wedge c|c$ donc
    $b\wedge c|\p{ab}\wedge c$.
  \item $\p{ab}\wedge c|b\wedge c$.\\
    En effet $\p{ab}\wedge c|c$. De plus $\p{ab}\wedge c|ab$. Comme
    \[\p{\p{ab}\wedge c}\wedge a=\p{ab}\wedge\p{c\wedge a}=\p{ab}\wedge 1=1\]
    on en déduit, d'après le lemme de Gauss, que $\p{ab}\wedge c|b$. Donc
    $\p{ab}\wedge c|b\wedge c$.
  \end{itemize}
  \end{sol}
\exo Résoudre l'équation $105u+147v=21$ dans $\Z$.
  \begin{sol}
  Analyse : Soit $(u,v) \in \Z^2$ tel que $105u+147v=21$. On a alors $5u+7v=1=5u_0+7v_0$ avec $u_0=3$ et $v_0=-2$. Donc $$5(u-u_0)=-7(v_0-v)$$ ce qui implique que $5\mid 7(v_0-v)$. Or, $5\wedge 7=1$ donc $5\mid v-v_0$ d'après le lemme de Gauss. Ainsi, il existe $k\in \Z$ tel que $v=-2+5k$, ce qui donne en réinjectant $u=3-7k$.\\
  Synthèse : il suffit de l'écrire.\\
  Finalement, on trouve $\set{(3-7k,-2+5k), k\in\Z}.$
  \end{sol}
\exo Trois comètes passent régulièrement dans le ciel \nom{Shadok}. La
  première, la comète Gabu, passe tous les 10 jours depuis
  le deuxième jour d'existence de leur planète. La seconde, la comète Zomeu
  passe tous les 21 jours depuis le cinquième jour d'existence de leur planète.
  Enfin, la comète Gibi passe tous les 6 jours depuis le troisième jour
  d'existence de leur planète. Est-il possible d'admirer les comètes Gabu et
  Zomeu le même jour dans le ciel Shadok~? Si oui, lesquels~? Même question
  pour les comètes Gabu et Gibi.
  \begin{sol}
  \begin{itemize}
  \item[$\bullet$]
  On souhaite résoudre l'équation 
  $$\begin{cases}n\equiv 2\quad [10]\\
  n\equiv 5\quad [21]\end{cases}$$
  \underline{Analyse :}
  Supposons trouvé une telle solution, alors il existe $u,v \in \Z$ tels que $n=2+10u=5+21v$ donc $10u-21v=3$. On cherche une SP en partant de l'équation de Bezout $10\times (-2)+21\times 1=1$ on a $u_0=-6$ et $v_0=-3$ qui sont solutions. Ainsi, $10(u-u_0)=21(v-v_0)$ donc $10 \mid 21(v-v_0)$. Or, $10\wedge 21=1$ donc d'après le Lemme de Gauss $10\mid v-v_0$, i.e il existe $k\in \Z$ tel que $v=v_0+10k=-3+10k$ (on en déduit $u=-6+21k$). Ainsi,
  $$n=5+21(-3+10k)=-58+210k.$$\\
  \underline{Synthèse :} On vérifie que $n=-58+210k$ vérifie bien l'équation initiale.
  \item[$\bullet$]On souhaite résoudre l'équation 
  $$\begin{cases}n\equiv 2\quad [10]\\
  n\equiv 3\quad [6]\end{cases}$$
  \underline{Analyse :}
  Supposons trouvé une telle solution, alors il existe $u,v \in \Z$ tels que $n=2+10u=3+6v$ donc $10u-6v=1$, ce qui est absurde (modulo 2 pour le voir). Donc il n'est pas possible d'admirer Gabu et Gibi le même jour. 
  \end{itemize}
  \end{sol}
\exo Soit $(G,\star)$ un groupe et $x\in G$ un élément d'ordre fini $n\in\Ns$.
    Étant donné $k\in\Z$, calculer l'ordre de $x^k$.
  %   \begin{sol}
  %   $\forall a \in \Z, \p{w^k}^a=1\Longleftrightarrow \omega^{ka}=1 \Longleftrightarrow n\mid ka$ (cf. prop. précédente).
  %   \'Ecrivons alors $n=(k\wedge n)n'$ et $k=(k\wedge n)k'$ ce qui impose (classique maintenant) $k'\wedge n'=1$. Reprenons les équivalences précédentes :
  %   $$\forall a \in \Z, \p{w^k}^a=1\Longleftrightarrow (k\wedge n)n'\mid ka=(k\wedge n)k' \Longleftrightarrow n'\mid k'a \underbrace{\Longleftrightarrow}_{Gauss} n'\mid a \Longleftrightarrow \dfrac{n}{k\wedge n}\mid a.$$ Donc l'ordre de $\omega^k$ est $\dfrac{n}{k\wedge n}$.
    
  %   En particulier, l'ordre de $\omega^k$ est $n$ ssi $k\wedge n=1$.
    
  %   \end{sol}
\end{exos}

\begin{remarqueUnique}
\remarque Soit $a,b,c\in\Z$ tels que $(a,b)\neq (0,0)$. On cherche les
  solutions entières de l'équation
  \[(E) \qquad ua+vb=c\]
  \begin{itemize}
  \item Si $a\wedge b$ ne divise pas $c$, il n'y a aucune solution.
  \item Sinon, il existe $c'\in\Z$ tel que $c=c'(a\wedge b)$.
    En utilisant l'algorithme d'\nom{Euclide}, on trouve $u_0',v_0'\in\Z$ tels que
    $u_0' a+v_0' b=a\wedge b$. On a donc $(c'u_0')a+(c'v_0')b=c$ ce qui nous
    donne une solution particulière à l'équation $(E)$. Soit $a',b'\in\Z$ tels
    que $a=a'(a\wedge b)$ et $b=b'(a\wedge b)$. Alors $a'$ et $b'$ sont premiers
    entre eux. On a alors~:
    \begin{eqnarray*}
    \forall u,v\in\Z \quad ua+vb=c
    &\ssi& ua+vb=(c'u_0')a+(c'v_0')b\\
    &\ssi& (u-c'u_0')a=(c'v_0'-v)b\\
    &\ssi& (u-c'u_0')a'=(c'v_0'-v)b' \quad (E')
    \end{eqnarray*}
    Si le couple $(u,v)$ est solution de $(E')$, on en déduit que $b'$ divise
    $(u-c'u_0')a'$. Or $a'$ et $b'$ sont premiers entre eux, donc d'après le
    lemme de Gauss, $b'$ divise $u-c'u_0'$. Il existe donc $k\in\Z$ tel que
    $u=c'u_0'+kb'$. En reportant cette égalité dans $(E')$, on trouve
    $v=c'v_0'-ka'$. Réciproquement, on vérifie que de tels $u$ et $v$ sont
    bien solution de $(E')$. L'ensemble des solutions de $(E)$ est donc
    \[\mathcal{S}=\enstq{(c'u_0'+kb',c'v_0'-ka')}{k\in\Z}\]
  \end{itemize}
\end{remarqueUnique}


\begin{proposition}
Soit $r\in\Q$.
\begin{itemize}
\item Alors, il existe un unique couple $\p{a,b}\in\Z\times\Ns$ tel que
  \[r=\frac{a}{b} \et a\wedge b=1.\]
  Cette écriture est appelée \emph{forme irréductible} de $r$.
\item De plus, si $p\in\Z$ et $q\in\Zs$, $r=p/q$ si et seulement si il existe
  $k\in\Zs$ tel que $p=ka$ et $q=kb$.
\end{itemize}
\end{proposition}

\begin{preuve}
\begin{itemize}
\item \begin{itemize}
\item[$\bullet$] \textbf{Existence :} On part de $r=\dfrac{p}{q}$ et avec $s=p\wedge q$ on simplifie et on la rend irréductible.
\item[$\bullet$] \textbf{Unicité :} On suppose trouvé deux couples puis grâce à $a_1b_2=a_2b_1$ on démontre grâce au lemme de Gauss que $b_1\mid b_2$ puis par symétrie $b_2\mid b_1$ donc $b_1=b_2$ ce qui conduit aussi à $a_1=a_2$.
\end{itemize}
\item Soit $r=\dfrac{a}{b}$ avec $a\wedge b=1$. On procède alors par analyse-synthèse pour déterminer les autres formes $r=\dfrac{p}{q}$. $pb=qa$ + Gauss donne $b\mid q$ donc $q=kb$ qu'on reporte pour avoir $p=ka$.
\end{itemize}

\end{preuve}

\begin{remarqueUnique}
\remarque Si $P(x)=a_n x^n+\cdots+a_1 x+a_0$ est un polynôme à coefficients
  entiers et $r=p/q$ est une racine rationnelle de $P$, mise sous
  forme irréductible, alors $q|a_n$ et $p|a_0$. On a ainsi un moyen de trouver
  toutes les racines rationnelles d'un polynôme à coefficients entiers.
\end{remarqueUnique}

\begin{sol}
Si $x$ est une racine rationnelle de $P$, il existe $p\in\Z$ et $q\in\Ns$ premiers entre eux tels que $x=p/q$. Puisque $P(x)=0$, on en déduit que
\[a_n\p{\frac{p}{q}}^n+\cdots+a_1\cdot\frac{p}{q}+a_0=0.\]
En multipliant par $q^n$, on obtient
\[a_n p^n + a_{n-1}p^{n-1}q+\cdots+a_1 p q^{n-1}+a_0 q^n=0.\]
On en déduit que
\[a_n p^n = -q\p{a_{n-1}p^{n-1}+\cdots+a_1 p q^{n-2}+a_0 q^{n-1}}\]
et donc que $q$ divise $a_n p^n$. Or $q$ et $p$ sont premiers entre eux donc $q$ et $p^n$ sont premiers entre eux. D'après le lemme de {\sc Gauss}, on en déduit que $q$ divise $a_n$. De même, on montre que $p$ divise $a_0$. Comme il existe un nombre fini de diviseurs d'un entier, les racines rationnelles sont donc à chercher parmi un nombre fini d'éléments.
\end{sol}

\begin{exos}
\exo Rechercher les racines rationnelles de $P(x)=2x^3+x^2+x-1$. En déduire
  une factorisation de ce polynôme.
  \begin{sol}
  $1/2$ est racine. On peut donc factoriser le polynôme en
  $(2x-1)(x^2+x+1)$.    
  \end{sol}
\exo Soit $n\in\N$. Montrer que $\sqrt{n}$ est soit  entier, soit
  irrationnel.
  \begin{sol}
  On pose $r=\sqrt{n}$ et on suppose que $r=\dfrac{a}{b}\in \Q$ avec $a\wedge b=1$. $r$ est racine de $X^2-n$ donc $b\mid 1$, donc $b=\pm 1$ et ainsi, $r\in \Z$. Ainsi, $r\in \Z$ ou $r\in \R\setminus\Q$.
  \end{sol} 
%\exo Déterminer les triplets $(a,b,c)\in\Ns^3$ tels que
%  \[\frac{1}{a}+\frac{1}{b}=\frac{1}{c}\]
%  \begin{sol}
%  On trouve $a=\alpha (p+k)k$, $b=\alpha p(p+k)$ et $c=\alpha pk$ avec
%  $\alpha\in\Ns$ et $p,k\in\Ns$ premiers entre eux.
%  \end{sol}
\end{exos}


\begin{proposition}
$\quad$
\begin{itemize}
\item Soit $a,b,c\in\Z$. On suppose que $a|c$, $b|c$ et $a\wedge b=1$.
  Alors $ab|c$.
\item Plus généralement si $a\in\Z$ est divisé par chaque élément d'une famille
  $b_1,\ldots,b_n\in\Z$ d'entiers deux à deux premiers entre eux, alors il
  est divisé par leur produit.
\end{itemize}
\end{proposition}

\begin{preuve}
$a\mid c$ donc $c=ua$, mais alors $b\mid ua$ et $b\wedge a=1$ donc $b\mid u$. En écrivant $u=vb$, on a $c=vba$ donc $ab\mid c$.\\
Récurrence sur $n$ pour la deuxième propriété.
\end{preuve}

\subsection{Plus petit commun multiple}

\begin{definition}
Soit $a,b\in\Z$. Il existe un unique entier positif $p$ tel que
\begin{itemize}
\item $a|p \et b|p$
\item $\forall q\in\Z \qsep \cro{a|q \et b|q} \implique p|q$
\end{itemize}
On l'appelle $\ppcm$ (\emph{plus petit commun multiple}) de $a$ et de $b$ et on le note
$\ppcm\p{a,b}$ ou $a\vee b$.
\end{definition}

\begin{preuve}
\begin{itemize}
\item[$\bullet$] \textbf{Unicité :} On suppose qu'il y en a $2$. On obtient directement en appliquant les propriétés que $p_1\mid p_2$ et $p_2\mid p_1$ donc $p_1=p_2$.
\item[$\bullet$] \textbf{Existence :} Quitte à changer $a$ en $-a$ et $b$ en $-b$, on peut les supposer positifs. Si l'un des deux est nul, on montre que $p=0$ convient. Sinon, $a>0$ et $b>0$. Posons alors $$X=\set{n\in \Ns \text{ tel que } a\mid n \et b\mid n}.$$
$X\neq \emptyset$ car $ab\in X$. Donc $X$ admet un ppe qu'on note $p$. La première propriété est donc vérifiée car $p\in X$. Soit alors $q\in \Z$ tel que $a|q \et b|q$. Montrons que $p\mid q$. On effectue la DE de $q$ par $p\neq 0$ : $q=sp+r$ avec $s\in \Z$ et $r\in \intere{0}{p-1}$. $a$ divise $p$ et $q$ donc $r$. De même pour $b$. Donc $r=0$ par définition de $p$ (ppe de $X$), ce qui signifie bien que $p\mid q$.
\end{itemize}

\end{preuve}

\begin{remarques}
  \remarque Si $a,b\in\Z$, les multiples de $a$ et de $b$ sont les multiples
    de $a\vee b$.
  \remarque Soit $a,b\in\N$. Pour la relation d'ordre de divisibilité sur $\N$, l'ensemble
    des multiples de $a$ et de $b$ n'est rien d'autre que
    l'ensemble des majorants de $\ens{a,b}$. La définition précédente montre donc que
    cet ensemble admet un plus petit élément (au sens de la divisibilité) qui est
    $a\vee b$. Autrement dit, au sens de la divisibilité, l'ensemble $\ens{a,b}$ admet
    une borne supérieure qui est $a\vee b$.
  \end{remarques}


\begin{proposition}
\begin{eqnarray*}
\forall a\in\Z, & & a\vee 0=0\\
\forall a\in\Z, & & a\vee 1=\abs{a}\\
\forall a,b\in\Z, & & a\vee b=0 \ssi \cro{a=0 \ou b=0}
\end{eqnarray*}
\end{proposition}

\begin{preuve}
Pour le troisième point, pour gauche implique droit, si $a\vee b=0$, d'après (ii), comme $a\mid ab$ et $b\mid ab$, $0\mid ab$ donc $ab=0$ d'où $a=0 \ou b=0$.
\end{preuve}

\begin{remarqueUnique}
  \remarque Si $a,b\in\Ns$, 
    $a\vee b$ est, au sens de l'ordre, le plus petit
    multiple commun strictement positif de $a$ et $b$.
\end{remarqueUnique}

\begin{proposition}
\begin{eqnarray*}
\forall a,b\in\Z, & & a\vee b=b\vee a\\
\forall a,b\in\Z, & &a\vee b=\p{-a}\vee b=a\vee\p{-b}=
  \p{-a}\vee\p{-b}=\abs{a}\vee\abs{b}\\
\forall a,b,k\in\Z, & & \p{ka}\vee\p{kb}=\abs{k}\p{a\vee b}
\end{eqnarray*}
\end{proposition}

\begin{preuve}
Montrons que $\forall a,b,k\in\Z  \p{ka}\vee\p{kb}=\abs{k}\p{a\vee b}$. Si $k=0$, OK. Sinon, on va montrer qu'ils se divisent l'un l'autre.
\begin{itemize}
\item[$\bullet$] $a\mid a\vee b$ donc $ka\mid k(a\vee b)$ donc $ka\mid |k|(a\vee b)$. De même pour $kb$ donc $\p{ka}\vee\p{kb}\mid \abs{k}\p{a\vee b}$.
\item[$\bullet$] $\p{ka}\vee\p{kb}$ étant un multiple de $ka$ c'est un multiple de $k$. On peut l'écrire $\p{ka}\vee\p{kb}=ku$. Montrons que $a\vee b \mid u$. $ka\mid ku$ donc $a\mid u$. De même, $b\mid u$. Donc $a\vee b \mid u$. On peut donc écrire $u=v(a\vee b)$. Ainsi, on a montré que $\p{ka}\vee\p{kb}=kv(a\vee b)$ donc $k(a\vee b) \mid \p{ka}\vee\p{kb}$ donc $|k|(a\vee b) \mid \p{ka}\vee\p{kb}$.
\end{itemize}
Comme ils sont positifs et se divisent l'un l'autre, ils sont égaux.
\end{preuve}

\begin{proposition}
Soit $a,b\in\Z$.
\begin{itemize}
\item Si $a\wedge b=1$, alors
  \[a\vee b=\abs{ab}.\]
\item De manière générale
  \[\p{a\wedge b}\p{a\vee b}=\abs{ab}.\]
\end{itemize}
\end{proposition}

\begin{preuve}
\begin{itemize}
\item Posons $p=\abs{ab}$ et montrons qu'il vérifie les propriétés du ppcm. La première est évidente. Pour la deuxième, considérons $q\in \Z$ tel que $a\mid q$ et $b\mid q$. Comme $a\wedge b=1$ d'après la proposition 2.10, $p\mid q$.
\item Pour le cas général, on évacue déjà le cas où $a\vee b=0$ facilement.\\
Sinon, on écrit $a=(a\wedge b)a'$ et $b=(a\wedge b)b'$ d'où on tire (déjà fait plein de fois) $a'\wedge b'=1$. Ainsi, d'après le premier cas $a'\vee b'=|a'b'|$. Finalement $$\p{a\wedge b}\p{a\vee b}=(a\wedge b)(a'\wedge b')(a\wedge b)(a'\vee b')=(a\wedge b)^2\times 1 \times |a'b'|=|(a\wedge b)a'(a\wedge b)b'|=|ab|.$$
\end{itemize}
\end{preuve}

\begin{remarqueUnique}
\remarque On peut définir $a\vee b \vee c$ mais attention, en général, $(a\wedge b \wedge c)(a\vee b \vee c)\neq |abc| $.
\end{remarqueUnique}

\begin{sol}
$a=2$, $b=3$, $c=6$.
\end{sol}

\begin{exoUnique}
\exo Résoudre dans $\Z$ l'équation $a\vee b=a+b-1$.
  \begin{sol}
  Analyse : Si $a=0$, $b=1$ et si $b=0$, $a=1$. On suppose maintenant $a$ et $b$ non nuls. $a\vee b\geq 0$ donc $a$ et $b$ ne peuvent être tous deux négatifs. Supposons par exemple $a\geq 0$. Or, comme $a\vee b\neq 0$, $a+b-1=a\vee b\geq a\geq 0$ donc $b-1\geq 0$. Donc $b\geq 1$ et en particulier ils sont tous deux positifs.\\
  $a\wedge b$ divise $a\vee b$, $a$ et $b$ donc $1$ donc c'est $1$. Ainsi, $a\vee b=ab=a+b-1$ d'où $(a-1)(b-1)=0$ donc $a=1$ ou $b=1$. Ainsi, $(a,b)$ est du type $(1,k)$ ou $(k,1)$ avec $k\in\N$.\\
  
  Synthèse : vérifions que les couples $(1,k)$ et $(k,1)$ avec $k\in\N$ fonctionnent.
  \end{sol}
\end{exoUnique}

% \subsection{Pgcd et ppcm d'une famille d'entiers}

% \begin{proposition}
% Soit $a,b,c\in\Z$. Alors~:
% \[\p{a\wedge b}\wedge c=a\wedge\p{b\wedge c}\]
% Cet entier est appelé $\pgcd$ de $a,b,c$ et est noté $\pgcd\p{a,b,c}$ ou
% $a\wedge b\wedge c$.
% \end{proposition}

% \begin{remarques}
% \remarque Si $a,b,c\in\Z$, $a\wedge b\wedge c$ est l'unique entier naturel
%   $p$ tel que~:
%   \begin{itemize}
%   \item $p|a \et p|b \et p|c$
%   \item $\forall q\in\Z \quad \cro{q|a \et q|b \et q|c} \implique q|p$
%   \end{itemize}
% \remarque On dit que $a,b,c\in\Z$ sont premiers entre eux dans leur ensemble
%   lorsque $a\wedge b\wedge c=1$. Si $a,b,c$ sont deux à deux premiers entre
%   eux, alors ils sont premiers entre eux dans leur ensemble. Cependant
%   la réciproque est fausse.
% \remarque Si $a,b,c\in\Z$, il existe $a',b',c'\in\Z$ tels que
%   $a=a'(a\wedge b\wedge c)$, $b=b'(a\wedge b\wedge c)$ et
%   $c=c' (a\wedge b\wedge c)$. Si $(a,b,c)\neq(0,0,0)$, alors $a'$, $b'$ et $c'$
%   sont premiers entre eux dans leur ensemble.
% \end{remarques}



% \begin{proposition}
% Soit $a,b,c\in\Z$. Alors~:
% \[\p{a\vee b}\vee c=a\vee\p{b\vee c}\]
% Cet entier est appelé $\ppcm$ de $a,b,c$ et est noté $\ppcm\p{a,b,c}$ ou
% $a\vee b\vee c$. 
% \end{proposition}

% \begin{remarqueUnique}
% \remarque Si $a,b,c\in\Z$, $a\vee b\vee c$ est l'unique entier naturel
%   $p$ tel que~:
%   \begin{itemize}
%   \item $a|p \et b|p \et c|p$
%   \item $\forall q\in\Z \quad \cro{a|q \et b|q \et c|q} \implique p|q$
%   \end{itemize}
% % \remarque On définit de même le $\ppcm$ d'une famille de $n$ entiers.
% \end{remarqueUnique}

\section{Nombres premiers}
\subsection{Nombres premiers}

\begin{definition}
On dit qu'un entier $p\geq 2$ est \emph{premier} lorsque ses seuls diviseurs
positifs sont 1 et $p$. On note $\mathcal{P}$ l'ensemble des nombres premiers.
\end{definition}

\begin{remarques}
\remarque Par convention, 1 n'est pas un nombre premier.
\remarque Un nombre $p\geq 2$ n'est pas premier si et seulement si il existe $a,b\geq 2$ tel que $p=ab$.
\remarque Soit $p$ un entier supérieur ou égal à 2. Pour montrer que $p$ est
  premier, il suffit de montrer que $k$ ne divise pas $p$ pour tout entier
  $k$ compris (au sens large) entre 2 et $\sqrt{p}$.
\end{remarques}

\begin{exos}
\exo Pour tout $n\in\N$, on définit le $n$-ième nombre de Mersenne comme
  $M_n=2^n-1$. Montrer que si $M_n$ est premier, alors $n$ est premier. La
  réciproque est-elle vraie~?
  \begin{sol}
  On raisonne par contraposée et on montre que si $n$ n'est pas premier alors $M_n$ n'est pas premier. Soit donc $n\in \N$ tel qu'il existe $a,b\geq 2$ tel que $n=ab$. On a alors :
  $$M_n=2^{ab}-1=(2^a)^b-1^b=(2^a-1)\sum_{k=0}^{b-1}(2^{a})^k1^{b-k-1}=\underbrace{(2^a-1)}_{\geq 3\geq 2}\underbrace{\sum_{k=0}^{b-1}2^{ak}}_{\geq \sum_{k=0}^{b-1}1=b\geq 2 }.$$ Donc $M_n$ n'est pas premier.
  La réciproque est fausse: $2^{11}-1=2047=23\times 89$.
  \end{sol}
\exo Soit $p$ un nombre premier supérieur ou égal à 5. Montrer que
  $24|p^2-1$.
  \begin{sol}
  $p$ est impair. $p=2n+1$ donc $p^2-1=(p-1)(p+1)=2n(2n+2)=4n(n+1)$.
  $n$ et $n+1$ étant consécutifs l'un des deux est pair donc $2\mid n(n+1)$. Montrons que $3\mid n(n+1)$. Pour cela, regardons $p$ modulo $3$. On sait que $p\not\equiv 0 \quad [3]$.
  \begin{itemize}
  \item[$\bullet$]Si $p\equiv 1 \quad [3]$, $2n\equiv 0 \quad [3]$ donc $3\mid 2n$ et $3\wedge 2=1$ donc d'après le lemme de Gauss, $3\mid n$ donc $3\mid n(n+1)$.
  \item[$\bullet$]Si $p\equiv 2 \quad [3]$, $2n+2\equiv 0 \quad [3]$ donc $3\mid 2n+2=2(n+1)$ et $3\wedge 2=1$ donc d'après le lemme de Gauss, $3\mid n+1$ donc $3\mid n(n+1)$.
  \end{itemize}
  \end{sol}
\exo Soit $n\in\Ns$. Montrer qu'il existe $n$ nombres consécutifs non
  premiers
  \begin{sol}
  Par exemple $(n+1)!+2,\ldots,(n+1)!+(n+1)$.
  \end{sol}
\end{exos}

\begin{proposition}
Soit $p$ un nombre premier et $n\in\Z$. Alors $p|n$ ou $p\wedge n=1$.
\end{proposition}

\begin{preuve}
$p\wedge n \mid p$. Or $p$ est premier donc $p\wedge n=1$ ou $p\wedge n=p$ auquel cas $p\mid n$.
\end{preuve}

\begin{exoUnique}
\exo Soit $p$ un nombre premier.
\begin{questions}
\question Montrer que pour tout $k\in\intere{1}{p-1}$, $p$ divise $\binom{p}{k}$.
\question Montrer que
  \[\forall a,b\in\Z\qsep (a+b)^p\equiv a^p + b^p\ \ [p].\]
\end{questions}
\end{exoUnique}
\begin{sol}
Déjà, on a $$p\mid k!\binom{p}{k}=\underbrace{p(p-1)\ldots (p-k+1)}_{k \text{ termes}}.$$
Or, $p\wedge i=1, \forall i \in \intere{1}{k}$, donc $p\wedge k!$ et donc d'après le lemme de Gauss, $p\mid \binom{p}{k}$.
\end{sol}

\begin{proposition}[nom={Petit théorème de \nom{Fermat}}]
Soit $p$ un nombre premier et $m\in\Z$ un entier qui n'est pas un multiple de
$p$. Alors
\[m^{p-1}\equiv 1\ [p].\]
\end{proposition}

\begin{preuve}
Soit $p$ un nombre premier.
\begin{itemize}
\item[$\bullet$] On commence par montrer que pour tout $(a,b)\in \Z^2$, $(a+b)^p\equiv a^p+b^p \quad [p]$ en développant avec le binôme de Newton et en utilisant l'exercice précédent.
\item[$\bullet$] On démontre ensuite par récurrence à l'aide de ce qu'on vient de démontrer que $\forall m \in \N$, $m^p\equiv m \quad [p]$.
\item[$\bullet$] On étend cela à $\Z$. Si $m\leq 0\in \Z$, $0\equiv (-m+m)^p\equiv (-m)^p+m^p \equiv -m+m^p \quad [p]$ donc $m^p\equiv m \quad [p]$.
\item[$\bullet$] Soit alors $m\in \Z$ tel que $p\nmid m$, on a alors $p\wedge m=1$ car $p$ est premier. D'après ce qu'on vient de démontrer, $p\mid m^p-m=m(m^{p-1}-1)$ et $p\wedge m=1$ donc d'après le lemme de Gauss, $p\mid m^{p-1}-1$, i.e. $m^{p-1}\equiv 1\ [p]$.
\end{itemize}
\end{preuve}

% \begin{remarqueUnique}
% \remarque Cette proposition porte le nom de \og petit théorème de Fermat\fg.
% \end{remarqueUnique}


\begin{proposition}
Soit $p$ un nombre premier.
\begin{itemize}
\item Si $a,b\in\Z$, alors
  \[p|ab \quad\ssi\quad \cro{p|a \ou p|b}.\]
\item Plus généralement, $p$ divise un produit si et seulement si il divise
  un de ses facteurs.
\end{itemize}
\end{proposition}

\begin{proposition}
Tout entier supérieur ou égal à $2$ admet un diviseur premier.  
\end{proposition}

\begin{preuve}
Soit $n\in \N$ tel que $n\geq 2$.
On pose $$X=\set{k\in \N, k\geq 2 \et k\mid n}.$$
Comme $n\in X$, $X$ est une partie non vide de $\N$ et admet donc un ppe que l'on note $p$. Montrons que $p$ est premier. Pour cela, considérons $k$ un diviseur de $p$ dans $\N$.

Ou bien il est plus grand que $2$ auquel cas il divise $p$ donc $n$ et par définition de $p$, on en déduit $p\leq k$. Comme $k$ divise $p$, on en déduit $k=p$.\\
Ou bien il est $<2$ auquel cas $k=1$ car $k=0$ est impossible puisque $p$ est non nul.

Ainsi, $p$ n'étant divisible que par $1$ et lui-même, il est premier.
\end{preuve}

\begin{remarqueUnique}
\remarque Soit $n\geq 2$. On cherche l'ensemble des nombres premiers inférieurs
  ou égaux à $n$. Pour cela, on utilise le crible d'Ératosthène~:
  \begin{itemize}
  \item On forme une table avec tous les entiers compris entre 2 et $n$.
  \item On raye tous les multiples de 2.
  \item On cherche le plus petit entier qui n'est pas rayé~: c'est 3 et il est
    premier. On raye alors tous les multiples de 3.
  \item On cherche ensuite le plus petit entier qui n'est pas rayé (c'est 5).
    Il est forcément premier car on a trouvé tous les nombres premiers
    strictement inférieurs à celui-ci et on a rayé tous leurs multiples.
    On raye alors tous les multiples de 5.
  \item On continue ainsi jusqu'à ce qu'on trouve un nombre premier dont le
    carré est strictement supérieur à $n$. Les nombres qui ne sont pas rayés
    sont les nombres premiers compris entre 2 et $n$.
  \end{itemize}
  Par exemple, si on cherche les nombres premiers inférieurs à 99, on trouve~:
  \begin{center}
    \begin{tabular}{|c|c|c|c|c|c|c|c|c|c|}
    \hline
       &    &  2 &  3 &  4 &  5 &  6 &  7 &  8 &  9\\\hline
    10 & 11 & 12 & 13 & 14 & 15 & 16 & 17 & 18 & 19\\\hline
    20 & 21 & 22 & 23 & 24 & 25 & 26 & 27 & 28 & 29\\\hline
    30 & 31 & 32 & 33 & 34 & 35 & 36 & 37 & 38 & 39\\\hline
    40 & 41 & 42 & 43 & 44 & 45 & 46 & 47 & 48 & 49\\\hline
    50 & 51 & 52 & 53 & 54 & 55 & 56 & 57 & 58 & 59\\\hline
    60 & 61 & 62 & 63 & 64 & 65 & 66 & 67 & 68 & 69\\\hline
    70 & 71 & 72 & 73 & 74 & 75 & 76 & 77 & 78 & 79\\\hline
    80 & 81 & 82 & 83 & 84 & 85 & 86 & 87 & 88 & 89\\\hline
    90 & 91 & 92 & 93 & 94 & 95 & 96 & 97 & 98 & 99\\\hline
    \end{tabular}
  \end{center}
  \begin{sol}
  On trouve
  2,3,5,7,11,13,17,19,23,29,31,37,41,43,47,53,59,61,67,71,73,79,83,89,97.
  \end{sol}
\end{remarqueUnique}

\begin{proposition}
L'ensemble $\mathcal{P}$ des nombres premiers est infini.
\end{proposition}

\begin{remarqueUnique}
\remarque Cette démonstration est due à \nom{Euclide} (325--265 avant J.C.).
\end{remarqueUnique}

\subsection{Valuation $p$-adique, décomposition en facteurs premiers}

\begin{definition}
Lorsque $n\in\Zs$ et $p$ est nombre premier, on appelle \emph{valuation
$p$-adique de $n$} et on note ${\rm Val}_p(n)$ le plus grand
$\alpha\in\N$ tel que $p^\alpha|n$.
\end{definition}

\begin{remarques}
\remarque Si $n\in\Zs$, il n'existe qu'un nombre fini de nombres premiers $p$ tels que
  ${\rm Val}_p(n)> 0$.
\remarque Soit $p$ et $q$ deux nombres premiers. Alors
  \[{\rm Val}_p(q)=\begin{cases}
    1 & \text{si $p=q$,}\\
    0 & \text{sinon.}
  \end{cases}\]
% Ce sont les nombres premiers apparaissant dans la décomposition
% de $n$ en facteurs premiers.
% \remarque Si $u\in\ens{1,-1}$ est le signe de $n$, la décomposition de $n$ en facteurs
%   premiers s'écrit
%   \[n=u \prod_{p\in\mathcal{P}} p^{{\rm Val}_p(n)}\]
\end{remarques}

\begin{proposition}
Soit $n_1,n_2\in\Zs$ et $p\in\mathcal{P}$. Alors
\[{\rm Val}_p\p{n_1 n_2}={\rm Val}_p\p{n_1}+{\rm Val}_p\p{n_2}.\]
\end{proposition}

\begin{remarques}
\remarque Plus généralement, si $p$ est un nombre premier, $n_1,\ldots,n_r\in\Zs$ et $\alpha_1,\ldots,\alpha_r\in\N$, alors
  \[{\rm Val}_p\p{\prod_{k=1}^r n_k^{\alpha_k}}=\sum_{k=1}^r \alpha_k {\rm Val}_p(n_k).\]
\remarque Si $n\in\Zs$, certains auteurs définissent la valuation $p$-adique de $n$ pour tout entier $p\geq 2$. Par exemple
  la valuation $10$-adique de $n$ est le plus grand entier $\alpha\in\N$ tel que $10^\alpha$ divise $n$, c'est-à-dire
  le nombre de 0 à la fin de l'écriture décimale de $n$. Remarquons simplement que si $p$ n'est pas premier, la propriété de
  la proposition précédente n'est plus vérifiée.
\end{remarques}

\begin{theoreme}[nom={Théorème de décomposition en nombres premiers}]
Soit $n\in\Zs$. Alors, il existe $u\in\ens{-1,1}$, $p_1,\ldots,p_r$ des nombres
premiers deux à deux distincts et $\alpha_1,\ldots,\alpha_r\in\Ns$ tels que
\[n=u\prod_{k=1}^r p_k^{\alpha_k}.\]
De plus, à permutation près des $p_k$, cette décomposition est unique.
\end{theoreme}

\begin{preuve}
Pour l'unicité, on suppose deux écritures puis on commence par montrer que  que le signe est le même, et enfin, en fixant un $k_0$ quelconque, si les puissances correspondantes sont distinctes, on divise de façon à les faire disparaitre d'un côté et la laisser strictement positive de l'autre côté et on aboutit à une absurdité car $p_{k_0}$ divise un membre mais pas l'autre.\\
Pour l'existence, on fait une récurrence sur $n\in \N$. Pour l'hérédité, ou bien $n$ est premier, ou bien il admet un diviseur premier $p$ et on applique l'HR à $n/p$.\\
On étend enfin aisément à $\Z$.
\end{preuve}

\begin{remarqueUnique}
\remarque Soit $n\in\Zs$. Si $u\in\ens{1,-1}$ est le signe de $n$, la décomposition de $n$ en facteurs
  premiers s'écrit
  \[n=u \prod_{p\in\mathcal{P}} p^{{\rm Val}_p(n)}\]
  ce produit ne comportant qu'un nombre fini de termes n'étant pas égal à 1.
\end{remarqueUnique}

\begin{proposition}
Soit $n_1,n_2\in\Zs$. Alors
\begin{itemize}
\item $n_1|n_2$ si et seulement si
  \[\forall p\in\mathcal{P} \qsep {\rm Val}_p\p{n_1}\leq {\rm Val}_p\p{n_2}.\]
\item $n_1=\pm n_2$ si et seulement si
  \[\forall p\in\mathcal{P}\qsep {\rm Val}_p\p{n_1}= {\rm Val}_p\p{n_2}.\]
\end{itemize}
\end{proposition}

\begin{preuve}
\'Ecrivons $n_1=up_1^{\alpha_1}\ldots p_r^{\alpha_r}$ et $n_2=vp_1^{\beta_1}\ldots p_r^{\beta_r}$ quitte à ce que les $\alpha_i$ ou les $\beta_i$ soient nuls, c'est-à-dire que les $p_i$ sont l'ensemble des diviseurs de $n_1$ \underline{ou} $n_2$.

\begin{itemize}
\item[$\bullet$] Si $n_1|n_2$, on peut écrire $n_2=an_1$ donc pour tout $i\in \intere{1}{r}$, $p_i^{\alpha_i}$ divise $n_2$ donc $\alpha_i\leq \beta_i$. Pour tous les autres nombres premiers, les deux valuations sont nulles.
\item[$\bullet$] Si $\forall p\in\mathcal{P} \qsep {\rm Val}_p\p{n_1}\leq {\rm Val}_p\p{n_2}$, on a $\alpha_i\leq \beta_i$ et donc $$n_2=uv\times \underbrace{v \prod_{i=1}^r p_i^{\alpha_i}}_{=n_1}\times \prod_{i=1}^r p_i^{\beta_i-\alpha_i}.$$
\end{itemize}
\end{preuve}

\begin{exoUnique}
\exo Si $n\in\N$ et $p$ est un nombre premier, montrer que
  \[{\rm Val}_p\p{n!}=\sum_{k=1}^{+\infty} \ent{\frac{n}{p^k}}\]
  En déduire le nombre de zéros à la fin de l'écriture décimale de $2023!$.
  \begin{sol}
  $n$ et $p$ sont fixés donc cette somme existe car est en fait à support fini. Il existe $k_0$ tel que $\forall k\geq k_0$, $\ent{\frac{n}{p^k}}=0$.
  Définissons $d$ par $d(a,b)=1$ si $a\mid b$, $0$ sinon.
  On a :
  $${\rm Val}_p\p{n!}={\rm Val}_p\p{\prod_{k=1}^n k}=\sum_{k=1}^n{\rm Val}_p(k)=\sum_{k=1}^n\sum_{i=1}^{+\infty}d(p^i,k)=\sum_{i=1}^{+\infty}\sum_{k=1}^nd(p^i,k)$$
  $\sum_{k=1}^nd(p^i,k)$ compte le nombre d'entiers entre $1$ et $n$ qui sont des multiples de $p^i$. Il y a $p^i, 2p^i, \ldots, mp^i$ où $m\in \N$ est tel que $mp^i\leq n <(m+1)p^i$, c'est-à-dire $m=\ent{\frac{n}{p^i}}$ d'où le résultat souhaité.\\
  En appliquant la formule obtenue, on trouve que la valuation de 5 est 500. Celle de 2 est supérieure, donc il y a 500 zéros à la fin de l'écriture décimale de $2007!$.
  \end{sol}  
\end{exoUnique}


\begin{proposition}
Soit $n_1,n_2\in\Zs$. Alors, le $\pgcd$ et le $\ppcm$ de $n_1$ et $n_2$ sont donnés par les relations
  \begin{eqnarray*}
  \forall p\in\mathcal{P} \qsep {\rm Val}_p \p{n_1\wedge n_2}&=&\min\p{{\rm Val}_p\p{n_1},{\rm Val}_p\p{n_2}}\\
  {\rm Val}_p \p{n_1\vee n_2}&=&\max\p{{\rm Val}_p\p{n_1},{\rm Val}_p\p{n_2}}.
  \end{eqnarray*}
\end{proposition}

\begin{preuve}
 Posons $m=\prod_{i=1}^r p_i^{\min(\alpha_i,\beta_i)}$ et on vérifie que cela vérifie bien les propriétés du pgcd en utilisant fortement la propriété que l'on vient de démontrer.
Pour le ppcm, même chose avec $M=\prod_{i=1}^r p_i^{\max(\alpha_i,\beta_i)}$
\end{preuve}

\begin{exoUnique}
\exo Soient $a,b\in\N$ tels que $a\wedge b=1$ et $ab$ est un carré parfait ($ab$ est le carré
  d'un entier). Montrer que $a$ et $b$ sont des carrés parfaits.
  \begin{sol}
  Il suffit d'écrire leur décomposition en nombre premier et celle du produit indexé par l'union des $p$ qui divise l'un ou l'autre. Clé : Ils n'en ont aucun en commun.
  \end{sol}
\end{exoUnique}

\subsection{Les grands problèmes d'arithmétique}

\begin{itemize}
\item {\bf Postulat de \nom{Bertrand}}\\
  Le postulat de \nom{Bertrand} affirme que si $n\in\Ns$, alors il existe un nombre
  premier $p$ tel que $n<p\leq 2n$. Cette conjecture fut énoncée par \nom{Joseph
  Bertrand} en 1845 et démontrée par \nom{Tchebychev} en 1848. Bien que ce résultat
  soit aujourd'hui un théorème, le nom de postulat lui est resté associé.
\item {\bf Théorème de la progression arithmétique}\\
  Ce théorème affirme que si $a$ et $b$ sont premiers entre aux, alors il existe
  une infinité de nombres premiers $p$ tels que $p\equiv a\ [b]$. On le doit
  à \nom{Dirichlet} (1805--1859).
\item {\bf Théorème des nombres premiers}\\
  Pour tout $n\in\Ns$, on définit $\pi_n$ comme le cardinal de l'ensemble des
  nombres premiers inférieurs ou égaux à $n$. Le théorème des nombres premiers
  affirme que
  \[\pi_n\equi{n}{+\infty}\frac{n}{\ln n}\]
  Autrement dit, si l'on choisit au hasard un entier entre 1 et $n$, la
  probabilité pour qu'il soit premier est de l'ordre de $1/(\ln n)$. Remarquons
  que cette quantité tend vers 0 lorsque $n$ tend vers $+\infty$, c'est-à-dire
  que les nombres premiers deviennent \og de plus en plus rares \fg lorsqu'on
  avance parmi les entiers naturels. Ce théorème fut conjecturé de
  manière indépendante par \nom{Gauss} et \nom{Legendre} vers 1800. Il fut démontré par
  \nom{Hadamard} et \nom{de la Vallée Poussin} en 1896.
\item {\bf Grand (ou dernier) théorème de \nom{Fermat}}\\
  Il s'énonce ainsi~:
  \medskip
  \begin{center}
  \og \parbox[t]{0.5\linewidth}{%
    Pour tout entier $n\geq 3$, il n'existe pas de triplet $(a,b,c)\in\Ns^3$
    tel que $a^n+b^n=c^n$.\fg}
  \end{center}
  \medskip
  Contrairement au petit théorème, il s'agit d'un résultat extrêmement
  difficile, dont \nom{Fermat} n'a pas publié de démonstration. \nom{Fermat} n'a même jamais
  affirmé publiquement l'avoir démontré. Il a cependant écrit dans une marge
  du livre {\sc II} des Oeuvres de \nom{Diophante}~: \og J'ai découvert une
  démonstration merveilleuse, mais je n'ai pas la place de la mettre dans la
  marge \fg. Le livre et cette annotation ont été publiés après sa mort, par son
  fils. De nombreux mathématiciens ont tenté de le prouver et sont arrivés
  à des résultats partiels, notamment
  \begin{itemize}
  \item \nom{Fermat} (1601--1665) le démontre pour $n=4$.
  \item \nom{Euler} (1707--1783) le démontre pour $n=3$.
  \item \nom{Sophie Germain} (1776--1831) apporte un résultat majeur ouvrant la
    porte à la démonstration du cas $n=5$, démontré quelques années plus tard
    par \nom{Legendre} (1752--1833).
  \item \nom{Kummer} (1810--1893) le prouve pour tout $n\in\intere{3}{99}$.
  \end{itemize}
  En 1993, \nom{Andrew Wiles} prouve un résultat sur les courbes elliptiques, résultat qui
  admet le grand théorème de \nom{Fermat} pour corolaire. La démonstration initiale
  possède une erreur mais elle sera vite réparée. La conjecture de Fermat
  devient alors le théorème de \nom{Fermat-Wiles}.
\item {\bf Nombres premiers jumeaux}\\
  On dit qu'un couple $(p,q)\in\N$ est un couple de nombres premiers jumeaux
  lorsque $q=p+2$. Par exemple $(3,5)$, $(5,7)$, $(11,13)$ sont des couples de
  nombres premiers jumeaux. On conjecture qu'il existe une infinité de nombres premiers jumeaux. Bien que l'on pense que cette conjecture est vraie,
  elle n'a jamais été démontrée.
  En janvier 2016, le plus grand couple de nombres premiers
  jumeaux connu est $2\,996\,863\,034\,895 \times 2^{1\,290\,000}\pm 1$.
\item {\bf Conjecture de \nom{Goldbach}}\\
  En 1742, \nom{Goldbach} (1690--1764) et \nom{Euler} (1707--1783) énoncent
  \medskip
  \begin{center}
  \og \parbox[t]{0.7\linewidth}{%
    Tout entier pair supérieur ou égal à 4 peut s'écrire comme la somme de
    deux nombres premiers.\fg}
  \end{center}
  \medskip
  On pense que cette conjecture est vraie, mais aucune démonstration n'en a jamais
  été faite.
\end{itemize}

% \begin{exos}
% \exo Supposons que l'on ait démontré le grand théorème de Fermat pour $n=4$ et
%   pour tout nombre premier $p\geq 3$. Montrer qu'alors le grand théorème de
%   Fermat est démontré.
% \end{exos}
%END_BOOK

\end{document}