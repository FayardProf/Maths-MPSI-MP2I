\documentclass{magnolia}

\magtex{tex_driver={pdftex}}
\magfiche{document_nom={Fonctions et procédures},
          auteur_nom={François Fayard},
          auteur_mail={fayard.prof@gmail.com}}
\magcours{cours_matiere={maths},
          cours_niveau={mpsi},
          cours_chapitre_numero={5},
          cours_chapitre={TP Python~: Sudoku}}
\magmisenpage{misenpage_presentation={tikzvelvia},
              misenpage_format={a4},
              misenpage_nbcolonnes={1},
              misenpage_preuve={non},
              misenpage_sol={non}}
\maglieudiff{lieu_lycee={Aux Lazaristes},
             lieu_classe={Sup},
             lieu_annee={2020--2021}}
             
             
\usepackage{tikz}
\usepackage{tkz-tab}


\magprocess

\begin{document}


%BEGIN_BOOK

\tikzset{math3d/.style={x={(-0.353cm,-0.353cm)},z={(0cm,1cm)},y={(1cm,0cm)}}}

\tikzset{
xmin/.store in=\xmin, xmin/.default=-3, xmin=-3,
xmax/.store in=\xmax, xmax/.default=3, xmin=3,
ymin/.store in=\ymin, ymin/.default=-3, ymin=-3,
ymax/.store in=\ymax, ymax/.default=-3, ymax=-3,
}

\usetikzlibrary{matrix}

\newcommand{\grille}{\draw[help lines] (\xmin,\ymin) grid (\xmax,\ymax);}

\newcommand{\axes}{%
\draw[->] (\xmin,0)--(\xmax,0);
\draw[->] (0,\ymin)--(0,\ymax);
\draw[->][very thick] (0,0)--(1,0);
%\draw (1 , 0) node[below] {$1$};
%\draw (0.5 , 0) node[below] {$\vec{i}$};
\draw (0.5 , 0) node[below] {$\vec{\imath}$};
\draw[->] [very thick](0,0)--(0,1);
%\draw (0,0.5) node[left] {$\vec{j}$};
\draw (0 , 0.5) node[left] {$\vec{\jmath}$};
%\draw (0,1) node[left] {$1$};
%\draw (0,0) node[below left]{$O$};
}

\newcommand{\axesbis}{%
\draw[->] (\xmin,0)--(\xmax,0)node[below]{$x$};
\draw[->] (0,\ymin)--(0,\ymax)node[left]{$y$};
}
\newcommand{\aaa}{\hphantom{\hspace{0.5cm}}\vphantom{9}}

\section{Règles du Sudoku}

Le sudoku est un jeu de logique de la famille des carrés latins. L'origine de son nom vient des deux mots japonais \og su \fg{} qui signifie chiffre et \og doku \fg{} qui signifie seul.\\

On dispose d'une grille de $9\times 9=81$ cases divisées en $3\times 3=9$ régions. Initialement, certaines cases de la grille sont préremplies par des chiffres compris entre
$1$ et $9$. Le but du jeu est de remplir les cases restantes avec des chiffres compris
entre $1$ et $9$ en respectant les contraintes suivantes:
\begin{itemize}
\item Chaque chiffre doit apparaître une seule fois dans chaque ligne.
\item Chaque chiffre doit apparaître une seule fois dans chaque colonne.
\item Chaque chiffre doit apparaître une seule fois dans chaque région.
\end{itemize}
% \bigskip
% La grille que l'on va compléter est la suivante~:

% \begin{center}
% \begin{tikzpicture}[xmin=-12,xmax=12,ymin=-12,ymax=12,scale=0.5]
% %\tkzInit[xmin=\xmin, xmax=\xmax, ymin=\ymin, ymax=\ymax]
% %\tkzGrid
% %\tkzAxeXY
% \matrix(M)[matrix of math nodes, nodes = {minimum size = 0.8cm}]at(0,0)
% {\aaa&8&\aaa&\aaa&4&1&\aaa&\aaa&9\\
% 1&6&\aaa&\aaa&\aaa&5&\aaa&\aaa&\aaa\\
% 4&\aaa&\aaa&6&\aaa&\aaa&\aaa&\aaa&\aaa\\
% \aaa&1&8&5&\aaa&\aaa&6&\aaa&\aaa\\
% \aaa&\aaa&2&\aaa&\aaa&\aaa&7&\aaa&\aaa\\
% \aaa&\aaa&5&\aaa&\aaa&9&3&8&\aaa\\
% \aaa&\aaa&\aaa&\aaa&\aaa&7&\aaa&\aaa&3\\
% \aaa&\aaa&\aaa&9&\aaa&\aaa&\aaa&7&8\\
% 9&\aaa&\aaa&1&8&\aaa&\aaa&2&\aaa\\
% };
% \foreach \j in {1,2,...,9}
% {\draw(M-1-\j.north west)--(M-9-\j.south west);}
% \foreach \i in {1,2,...,9}
% {\draw(M-\i-1.south west)--(M-\i-9.south east);}
% \draw(M-1-1.north west)--(M-1-9.north east);
% \draw[very thick](M-1-1.north west)--(M-9-1.south west);
% \draw[very thick](M-1-1.north west)--(M-1-9.north east);
% \foreach \j in{3,6,9}
% {\draw[very thick](M-1-\j.north east)--(M-9-\j.south east);}
% \foreach \i in{3,6,9}
% {\draw[very thick](M-\i-1.south west)--(M-\i-9.south east);}
% %\draw[->,>=latex](-9.5,9.4)--(-9.5,-9.4)node[below]{$i$};
% %\draw(-10,8)node{$0$};
% %\draw(-10,6)node{$1$};
% %\draw(-10,4)node{$2$};
% %\draw[->,>=latex](-11,9.4)--(-11,-9.4)node[below]{$a$};
% %\draw(-12,6)node{\Large{$0$}};
% %\draw(-12,0)node{\Large{$1$}};
% %\draw(-12,-6)node{\Large{$2$}};
% %\draw[->,>=latex](-9.4,9.6)--(9,9.6)node[right]{$j$};
% %\draw(-8,10)node{$0$};
% %\draw(-6,10)node{$1$};
% %\draw(-4,10)node{$2$};
% %\draw[->,>=latex](-9.4,11)--(9.4,11)node[right]{$b$};
% %\draw(-6,12)node{\Large{$0$}};
% %\draw(0,12)node{\Large{$1$}};
% %\draw(6,12)node{\Large{$2$}};
% \end{tikzpicture}
% \end{center}

% \section{Modélisation}
\bigskip

La grille de sudoku est modélisée par une matrice carrée de 9 par 9,
à l'aide d'une liste de listes. Pour repérer une case dans la grille, on utilise ses coordonnées $(i,j)$ comme dans le schéma ci-dessous où $0\leq i,j < 9$. Une région est, quant à elle, définie par ses coordonnées $(a,b)$ où $0\leq a,b<3$.

\begin{center}
\begin{tikzpicture}[xmin=-12,xmax=12,ymin=-12,ymax=12,scale=0.4]
%\tkzInit[xmin=\xmin, xmax=\xmax, ymin=\ymin, ymax=\ymax]
%\tkzGrid
%\tkzAxeXY
\matrix(M)[matrix of math nodes, nodes = {minimum size = 0.8cm}]at(0,0)
{\aaa&8&\aaa&\aaa&4&1&\aaa&\aaa&9\\
1&6&\aaa&\aaa&\aaa&5&\aaa&\aaa&\aaa\\
4&\aaa&\aaa&6&\aaa&\aaa&\aaa&\aaa&\aaa\\
\aaa&1&8&5&\aaa&\aaa&6&\aaa&\aaa\\
\aaa&\aaa&2&\aaa&\aaa&\aaa&7&\aaa&\aaa\\
\aaa&\aaa&5&\aaa&\aaa&9&3&8&\aaa\\
\aaa&\aaa&\aaa&\aaa&\aaa&7&\aaa&\aaa&3\\
\aaa&\aaa&\aaa&9&\aaa&\aaa&\aaa&7&8\\
9&\aaa&\aaa&1&8&\aaa&\aaa&2&\aaa\\
};
\foreach \j in {1,2,...,9}
{\draw(M-1-\j.north west)--(M-9-\j.south west);}
\foreach \i in {1,2,...,9}
{\draw(M-\i-1.south west)--(M-\i-9.south east);}
\draw(M-1-1.north west)--(M-1-9.north east);
\draw[very thick](M-1-1.north west)--(M-9-1.south west);
\draw[very thick](M-1-1.north west)--(M-1-9.north east);
\foreach \j in{3,6,9}
{\draw[very thick](M-1-\j.north east)--(M-9-\j.south east);}
\foreach \i in{3,6,9}
{\draw[very thick](M-\i-1.south west)--(M-\i-9.south east);}
\draw[->,>=latex](-9.5,9.4)--(-9.5,-9.4)node[below]{$i$};
\draw(-10,8)node{$0$};
\draw(-10,6)node{$1$};
\draw(-10,4)node{$2$};
\draw[->,>=latex](-11,9.4)--(-11,-9.4)node[below]{$a$};
\draw(-12,6)node{\Large{$0$}};
\draw(-12,0)node{\Large{$1$}};
\draw(-12,-6)node{\Large{$2$}};
\draw[->,>=latex](-9.4,9.6)--(9,9.6)node[right]{$j$};
\draw(-8,10)node{$0$};
\draw(-6,10)node{$1$};
\draw(-4,10)node{$2$};
\draw[->,>=latex](-9.4,11)--(9.4,11)node[right]{$b$};
\draw(-6,12)node{\Large{$0$}};
\draw(0,12)node{\Large{$1$}};
\draw(6,12)node{\Large{$2$}};
\end{tikzpicture}
\end{center}

\begin{enumerate}
\item Récupérez le fichier \texttt{grille.py}. Dans ce fichier, vous trouverez la matrice \texttt{g}, nous donnant un exemple de grille à traiter. Les cases vides sont symbolisées par des $0$. Copiez cette matrice dans votre script.
\item Quelle doit être la valeur de \texttt{g[3][2]}? Vérifiez sur votre machine.
\item Écrire une fonction \verb!cases_vides(g: list[list[int]]) -> list[tuple[int, int]]!  prenant pour argument une grille \verb!g! et renvoyant la liste des coordonnées $(i,j)$
des cases vides. Dans cet algorithme, le parcours de la grille se fera ligne par ligne.
\item Écrire une fonction \verb!compat_ligne(g: list[list[int]], i: int, c: int) -> bool!
   prenant pour argument une grille $g$, une ligne $i$ et un chiffre
   $c$ compris entre $1$ et $9$, et renvoyant \verb!True! s'il est possible
   d'ajouter le chiffre $c$ sur une des cases vides de la ligne $i$, tout en satisfaisant
   les contraintes de ligne du plateau. Autrement dit, cette fonction doit renvoyer
   \verb!True! si $c$ est différent de toutes les autres chiffres présents sur la ligne
   $i$, et \verb!False! sinon.
\item Écrire une fonction analogue
  \verb!compat_colonne(g: list[list[int]], j: int, c: int) -> bool! pour les colonnes,
  ainsi qu'une fonction
  \verb!compat_region(g: list[list[int]], a: int, b: int, c: int) -> bool! pour les
  régions.
\item En utilisant la division entière en Python, écrire une fonction
  \verb!region(i: int, j: int) -> tuple[int, int]! prenant pour arguments les
  coordonnées $(i,j)$ d'une case et renvoyant les coordonnées $(a,b)$ de la région
  dans laquelle elle se trouve.
\item Déduire des questions précédentes une fonction
  \begin{center}
    \verb!compat(g: list[list[int]], i: int, j: int, c: int) -> bool!
  \end{center}
  prenant pour
  arguments une grille $g$, les coordonnées $(i,j)$ d'une case vide
  de la grille et un chiffre $c$ entre 1 et 9 et renvoyant \verb!True! s'il est
  possible d'ajouter le chiffre $c$ sur la case vide  de coordonnées $(i,j)$ tout
  en respectant les contraintes du Sudoku portant sur les lignes, les colonnes et
  les régions.
\end{enumerate}


\section{Le backtracking}

Il existe de nombreuses méthodes de résolution du Sudoku. Les méthodes simulant le raisonnement humain sont efficaces mais difficiles à programmer. La méthode du \emph{retour arrière systématique} ou \emph{backtracking} est préférable~: elle teste
toutes les possibilités de remplissage. Étant donnée la puissance des ordinateurs actuels, une grille solution est généralement  donnée en quelques millisecondes. Détaillons cette
méthode pour le jeu du Sudoku.\\

On commence par déterminer la liste \verb!cv! des cases initialement vides de la grille.
Ensuite, on teste
la possibilitée de placer un 1 dans la première case vide. Si cette valeur ne rentre
pas en conflit avec les chiffres déjà présents, on passe à la case vide suivante.
Sinon, on teste si 2 convient, puis 3, etc. Si aucun chiffre entre 1 et 9 ne convient
pour remplir une case, nous sommes face à une incompatibilité et il est alors
nécessaire de revenir en arrière dans la liste de nos cases vides. On tente alors de
la remplir avec une nouvelle valeur, plus grande que celle tentée précédemment.\\

Concrètement, supposons qu'on a déjà rempli les $k$ premières cases vides de la liste
\verb!cv!. La prochaine case vide à traiter dont les coordonnées sont données par
\verb!cv[k]! contient une valeur $c$.
\begin{itemize}
\item Si $c=0$, c'est que nous venons juste de tester une nouvelle valeur pour la
  case vide d'indice $k-1$.
\item Si $c\in\intere{1}{9}$, c'est que nous venons de réaliser que cette
  valeur a conduit à une incompatibilité en tentant de remplir les cases vides
  suivantes. 
\end{itemize}
\medskip

Dans les deux cas, nous allons tenter de remplir cette case avec un chiffre
strictement supérieur à $c$~: on commence par $c+1$, puis $c+2$, jusqu'à 9.
\begin{itemize}
\item Si on trouve un chiffre compatible avec le reste de la grille, on place
  ce chiffre dans la grille, puis on passe à la case vide d'indice $k+1$.
\item Si aucun chiffre n'est compatible, c'est que nous sommes face à une incompatibilité
  et qu'il faut revenir en arrière pour tenter un nouveau chiffre. On place donc
  le chiffre 0 dans la case pour signifier qu'elle est de nouveau vide et on passe à la case
  initialement vide d'indice $k-1$.
\end{itemize}
\medskip

Si on note $n$ la longueur de la liste \verb!cv!, l'algorithme peut se terminer de deux
manières distinctes~:
\begin{itemize}
\item Si $k=n$, c'est que la grille a été entièrement remplie et donc qu'une solution a été trouvée.
\item Si $k=-1$, c'est qu'aucune solution n'a été trouvée. L'algorithme étant exhaustif,
  cela prouve qu'il n'est pas possible de compléter la grille.
\end{itemize}


\begin{enumerate}
\setcounter{enumi}{8}
\item Écrire une fonction
  \begin{center}
  \verb!prochain(g: list[list[int]], cv: list[tuple[int, int]], k: int) -> int!
  \end{center}
  qui prend une telle grille où les $k$ premières cases vides de la grille $g$ ont
  été remplies, où \verb!cv! contient la liste des coordonnées des cases
  initialement vides et qui traite comme l'on vient de voir la case
  de coordonnées \verb!cv[k]!. Cette fonction devra renvoyer l'index de la prochaine
  case à traiter, qui sera soit $k+1$ si on a trouvé un chiffre compatible, soit
  $k-1$ si une incompatibilité a été trouvée.
\item Écrire la fonction \verb!solution(g: list[list[int]]) -> bool!
  prenant en entrée une grille de Sudoku $g$ et renvoyant \verb!True! s'il est possible
  de la résoudre (et en transformant au passage la grille $g$ en une solution)
  et renvoyant \verb!False! dans le cas contraire.
\item A l'aide de la macro \verb!%timeit! de IPython, déterminer le temps de
  nécessaire à la résolution de la grille proposée en exemple.
\end{enumerate}


% \section{Partie facultative}

% \begin{enumerate}
% \item \'Ecrire une fonction \texttt{grille} d'argument une matrice \texttt{M} qui renvoie une chaîne de caractères représentant la grille de sudoku associée à \texttt{M}.

% Par exemple, en écrivant \texttt{print(grille(M))} dans le script avec la matrice \texttt{M} initiale (non remplie), on voit s'afficher après exécution:
% \begin{center}
% \includegraphics[width=7cm]{../../Commun/Images/python-tps-sudoku.png}
% \end{center}
% \item \'Ecrire une fonction \texttt{est\_un\_sudoku} d'arguments une matrice \texttt{M} (représentant une grille de sudoku remplie). Cette fonction renverra un booléen permettant de savoir si \texttt{M} remplit bien les conditions pour être un sudoku.

% \emph{Indication}: On pourra utiliser des fonctions intermédiaires.

% \end{enumerate}


%END_BOOK
 
%\end{enumerate}
\end{document}